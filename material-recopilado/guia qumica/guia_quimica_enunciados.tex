\documentclass{article}
\usepackage{fullpage}
\usepackage[utf8]{inputenc}
\usepackage[T1]{fontenc}
\usepackage{graphicx}
\usepackage[spanish]{babel}
\usepackage{amssymb}
\usepackage{amsmath}
\usepackage{cancel}
\usepackage{booktabs} 
\usepackage{url}
\usepackage{tikz}


%%%%% Comandos Personalizados %%%%%
\newcommand{\N}{\mathbb{N}}
\newcommand{\R}{\mathbb{R}}
\newcommand{\Q}{\mathbb{Q}}
\newcommand{\E}{\mathbb{E}}
\newcommand{\PP}{\mathbb{P}}
\newcommand{\la}{\leftarrow}
\newcommand{\ra}{\rightarrow}
\newcommand{\lra}{\leftrightarrow}
\newcommand{\Ra}{\Rightarrow}
\newcommand{\La}{\Leftarrow}
\newcommand{\LRa}{\Leftrightarrow}
\newcommand{\sub}{\subseteq}
\newcommand{\matro}{\mathcal{M}}

\newcommand{\twopartdef}[4]
{
	\left\{
		\begin{array}{ll}
			#1 &  \text{#2} \\
			#3 &  \text{#4}
		\end{array}
	\right.
}

%%%%%  Fin Comandos Personalizados %%%%%

 %%%%%%%%%% MODIFICAR %%%%%%%%%%
\newcommand{\alumnos}{Rodrigo Ogalde Lisboa - Felipe Estay Zamorano}
\newcommand{\departamento}{Departamento de Ingenieria Industrial y de Sistemas}
\newcommand{\ramo}{Modelos Estocasticos 2025-1}
\newcommand{\sigla}{ICS2123}
\newcommand{\titulo}{T2}
\newcommand{\semestre}{01}
\newcommand{\anio}{2025}
\newcommand{\med}{\frac{1}{2}}
\newcommand{\indep}{\mathcal{I}}
%%%%%%%%%% FIN MODIFICAR %%%%%%%%%%

\renewcommand{\thesubsection}{\alph{subsection}}


\usepackage{tikz}
\usetikzlibrary{arrows.meta}


\begin{document}
\title{Guía de Ejercicios Química}
\maketitle
\section{2016-1}
\subsection*{Pregunta 09 - 2016-1}
¿Cuál de las siguientes afirmaciones es FALSA respecto a conceptos de oxidación y reducción?

a) La Ecuación de Nernst es usada para calcular el potencial de una celda bajo condiciones de estado no estándar.

b) Celda voltaica es una celda electroquímica que adquiere la energía eléctrica a partir de reacciones redox.

c) El ánodo es el electrodo positivo de una celda electrolítica al cual se dirigen los cationes de la disolución.

d) Todas las reacciones electroquímicas implican la transferencia de electrones y por lo tanto son reacciones redox.
\vspace{0.5cm}
\subsection*{Pregunta 10 - 2016-1}
Considerando la siguiente reacción redox no balanceada:
$$
\mathrm{Fe}^{2+}+\mathrm{Cr}_2 \mathrm{O}_7^{2-} \rightarrow \mathrm{Fe}^{3+}+\mathrm{Cr}^{3+}
$$

¿Cuál de las siguientes alternativas de ecuación iónica balanceada es la correcta?

a) $6 \mathrm{Fe}^{2+}+14 \mathrm{H}^{+}+\mathrm{Cr}_2 \mathrm{O}_7^{2-} \rightarrow 6 \mathrm{Fe}^{3+}+2 \mathrm{Cr}^{3+}+7 \mathrm{H}_2 \mathrm{O}+6 e^{-}$

b) $6 \mathrm{Fe}^{2+}+\mathrm{Cr}_2 \mathrm{O}_7^{2-} \rightarrow 6 \mathrm{Fe}^{3+}+2 \mathrm{Cr}^{3+}$

c) $6 \mathrm{Fe}^{2+}+14 \mathrm{H}^{+}+\mathrm{Cr}_2 \mathrm{O}_7^{2-}+6 e^{-} \rightarrow 6 \mathrm{Fe}^{3+}+2 \mathrm{Cr}^{3+}+7 \mathrm{H}_2 \mathrm{O}$

d) $6 \mathrm{Fe}^{2+}+14 \mathrm{H}^{+}+\mathrm{Cr}_2 \mathrm{O}_7^{2-} \rightarrow 6 \mathrm{Fe}^{3+}+2 \mathrm{Cr}^{3+}+7 \mathrm{H}_2 \mathrm{O}$
\vspace{0.5cm}
\subsection*{Pregunta 11 - 2016-1}
$6,69$ moles de un gas se encuentra a $257^{\circ} \mathrm{C}$, y a una presión de $10.10 \mathrm{~atm}$. Calcular el volumen que ocupa el gas.

a) $2,93 \times 10^3 \mathrm{~L}$

b) $1,40 \times 10 \mathrm{~L}$

c) $2,9 \times 10 \mathrm{~L}$

d) $1,42 \times 10^3 \mathrm{~L}$
\vspace{0.5cm}
\section{2016-2}
\subsection*{Pregunta 07 - 2016-2}
Considerando la siguiente reacción redox no balanceada 

$$
\mathrm{Bi}(\mathrm{OH})_3+\mathrm{SnO}_2^{2-} \rightarrow \mathrm{SnO}_3^{2-}+ \mathrm{Bi}
$$

¿Cuál de las siguientes alternativas de ecuación iónica balanceada es la correcta, considerando un medio básico?

a) $\mathrm{Bi}(\mathrm{OH})_3+3 \mathrm{SnO}_2^{2-} \rightarrow \mathrm{Bi}+2 e^{-}+3 \mathrm{SnO}_3^{2-}$

b) $\mathrm{Bi}(\mathrm{OH})_3+3 \mathrm{SnO}_2^{2-}+2 e^{-} \rightarrow \mathrm{Bi}+3 \mathrm{SnO}_3^{2-}$

c) $2 \mathrm{Bi}(\mathrm{OH})_3+3 \mathrm{SnO}_2^{2-} \rightarrow 2 \mathrm{Bi}+3 \mathrm{OH}^{-}+3 \mathrm{SnO}_3^{2-}$

d) $2 \mathrm{Bi}(\mathrm{OH})_3+3 \mathrm{SnO}_2^{2-} \rightarrow 2 \mathrm{Bi}+3 \mathrm{H}_2 \mathrm{O}+3 \mathrm{SnO}_3^{2-}$
\vspace{0.5cm}
\subsection*{Pregunta 08 - 2016-2}
Considerando la siguiente reacción redox no balanceada:
$$
\mathrm{Fe}^{2+}+\mathrm{Cr}_2 \mathrm{O}_7^{2-} \rightarrow \mathrm{Fe}^{3+}+\mathrm{Cr}^{3+}
$$

¿Cuál de las siguientes alternativas de ecuación iónica balanceada es la correcta?

a) $6 \mathrm{Fe}^{2+}+14 \mathrm{H}^{+}+\mathrm{Cr}_2 \mathrm{O}_7^{2-} \rightarrow 6 \mathrm{Fe}^{3+}+2 \mathrm{Cr}^{3+}+7 \mathrm{H}_2 \mathrm{O}+6 e^{-}$

b) $6 \mathrm{Fe}^{2+}+\mathrm{Cr}_2 \mathrm{O}_7^{2-} \rightarrow 6 \mathrm{Fe}^{3+}+2 \mathrm{Cr}^{3+}$

c) $6 \mathrm{Fe}^{2+}+14 \mathrm{H}^{+}+\mathrm{Cr}_2 \mathrm{O}_7^{2-}+6 e^{-} \rightarrow 6 \mathrm{Fe}^{3+}+2 \mathrm{Cr}^{3+}+7 \mathrm{H}_2 \mathrm{O}$

d) $6 \mathrm{Fe}^{2+}+14 \mathrm{H}^{+}+\mathrm{Cr}_2 \mathrm{O}_7^{2-} \rightarrow 6 \mathrm{Fe}^{3+}+2 \mathrm{Cr}^{3+}+7 \mathrm{H}_2 \mathrm{O}$
\vspace{0.5cm}
\subsection*{Pregunta 09 - 2016-2}
El volumen de un gas desconocido es de $8.55 \mathrm{~L}$, medidos a presión atmosférica ($P_{\mathrm{atm}}=1 \mathrm{~atm}$). Considerando que el volumen del mismo gas alcanza los $6.259 \mathrm{~L}$ con temperatura constante, calcula la nueva presión del gas.

a) $1.37 \mathrm{~mmHg}$

b) $40670.9 \mathrm{~mmHg}$

c) $556.36 \mathrm{~mmHg}$

d) $1038.19 \mathrm{~mmHg}$
\vspace{0.5cm}
\subsection*{Pregunta 10 - 2016-2}
¿Cuántos moles de $\mathrm{MgCl}_2$ están presentes en $60.0 \mathrm{~mL}$ en una solución de $0.100 M$ de $\mathrm{MgCl}_2$ ?

a) $1.67 \mathrm{~mol} \mathrm{~MgCl}_2$

b) $6.00 \times 10^{-3} \mathrm{~mol} \mathrm{~MgCl}_2$

c) $6.00 \times 10^1 \mathrm{~mol} \mathrm{~MgCl}_2$

d) $1.67 \times 10^{-4} \mathrm{~mol} \mathrm{~MgCl}_2$
\vspace{0.5cm}
\section{2017-1}
\subsection*{Pregunta 07 - 2017-1}
Considerando la siguiente reacción redox no balanceada $\mathrm{Cr}_2 \mathrm{O}_7^{2-}+\mathrm{C}_2 \mathrm{O}_4^{2-} \rightarrow \mathrm{Cr}^{3+}+\mathrm{CO}_2$, ¿Cuál de las siguientes alternativas de ecuación iónica balanceada es la correcta, considerando un medio ácido?

a) $\mathrm{Cr}_2 \mathrm{O}_7^{2-}+3 \mathrm{C}_2 \mathrm{O}_4^{2-}+14 \mathrm{H}^{+}+7 e^{-} \rightarrow 2 \mathrm{Cr}^{3+}+6 \mathrm{CO}_2+7 \mathrm{H}_2 \mathrm{O}$

b) $\mathrm{Cr}_2 \mathrm{O}_7^{2-}+3 \mathrm{C}_2 \mathrm{O}_4^{2-}+14 \mathrm{H}^{+} \rightarrow 2 \mathrm{Cr}^{3+}+6 \mathrm{CO}_2+7 \mathrm{H}_2 \mathrm{O}+7 e^{-}$

c) $\mathrm{Cr}_2 \mathrm{O}_7^{2-}+3 \mathrm{C}_2 \mathrm{O}_4^{2-}+14 \mathrm{H}^{+} \rightarrow 2 \mathrm{Cr}^{3+}+6 \mathrm{CO}_2+7 \mathrm{H}_2 \mathrm{O}$

d) $\mathrm{Cr}_2 \mathrm{O}_7^{2-}+3 \mathrm{C}_2 \mathrm{O}_4^{2-}+7 \mathrm{H}^{+} \rightarrow 2 \mathrm{Cr}^{3+}+6 \mathrm{CO}_2+3 \mathrm{H}_2 \mathrm{O}$
\vspace{0.5cm}
\subsection*{Pregunta 08 - 2017-1}
Calcular el pH en el equilibrio considerando una concentración inicial de $0.020 \mathrm{M}$ $\mathrm{Ba}(\mathrm{OH}) 2$; compuesto es una base fuerte, logrando que el compuesto sea completamente ionizado.

a) $1.40$

b) $1.70$

c) $12.30$

d) $12.60$
\vspace{0.5cm}
\subsection*{Pregunta 09 - 2017-1}
¿Cuál es la geometría molecular del compuesto $\mathrm{CBr}_4$?

a) Tetraédrica

b) Angular

c) Lineal

d) Trigonal plana
\vspace{0.5cm}
\subsection*{Pregunta 10 - 2017-1}
La caliza $\left(\mathrm{CaCO}_3\right)$ se descompone utilizando calor logrando obtener cal $(\mathrm{CaO})$ y dióxido de carbono. Calcule cuantos gramos de cal se puede producir a partir de 1.0 kg de caliza.

a) $5.6 \times 10^1 \mathrm{~g}$

b) $5.6 \times 10^2 \mathrm{~g}$

c) $5.6 \times 10^6 \mathrm{~g}$

d) $5.6 \times 10^4 \mathrm{~g}$
\vspace{0.5cm}
\section{2017-2}
\subsection*{Pregunta 07 - 2017-2}
Considerando la siguiente reacción redox no balanceada:

$$
\mathrm{MnO}_4^{-}+\mathrm{Cl}^{-} \rightarrow \mathrm{Mn}^{2+}+\mathrm{Cl}_2
$$

¿Cuál de las siguientes alternativas de ecuación iónica balanceada es la CORRECTA, considerando un medio ácido?

a) $2 \mathrm{MnO}_4^{-}+16 \mathrm{H}^{+}+10 \mathrm{Cl}^{-} \rightarrow 2 \mathrm{Mn}^{2+}+8 \mathrm{H}_2 \mathrm{O}+5 \mathrm{Cl}_2+5 e^{-}$

b) $2 \mathrm{MnO}_4^{-}+8 \mathrm{H}^{+}+10 \mathrm{Cl}^{-}+5 \mathrm{e}^{-} \rightarrow 2 \mathrm{Mn}^{2+}+4 \mathrm{H}_2 \mathrm{O}+5 \mathrm{Cl}_2$

c) $2 \mathrm{MnO}_4^{-}+16 \mathrm{H}^{+}+10 \mathrm{Cl}^{-} \rightarrow 2 \mathrm{Mn}^{2+}+8 \mathrm{H}_2 \mathrm{O}+5 \mathrm{Cl}_2$

d) $2 \mathrm{MnO}_4^{-}+16 \mathrm{H}^{+}+10 \mathrm{Cl}^{-}+5 \mathrm{e}^{-} \rightarrow 2 \mathrm{Mn}^{2+}+8 \mathrm{H}_2 \mathrm{O}+5 \mathrm{Cl}_2$
\vspace{0.5cm}
\subsection*{Pregunta 08 - 2017-2}
¿Cuál de las siguientes afirmaciones es FALSA respecto a las reacciones óxido-reducción?

a) Las reacciones electroquímicas implican transferencia de electrones, pero no todas son reacciones redox.

b) Las reacciones redox implican la transferencia de electrones entre los elementos involucrados.

c) Las ecuaciones que representan procesos redox pueden equilibrarse utilizando el método de ion-electrón.

d) La corrosión de los metales, tales como la oxidación del hierro, es un fenómeno electroquímico.
\vspace{0.5cm}
\subsection*{Pregunta 09 - 2017-2}
Un recipiente de $2.5 \mathrm{~L}$ a $15^{\circ} \mathrm{C}$ contiene una mezcla de gases $\left(\mathrm{N}_2, \mathrm{He}, \mathrm{Ne}\right)$; las presiones parciales son $\mathrm{P}_{\mathrm{N} 2}=0.32 \mathrm{~atm}, \mathrm{P}_{\mathrm{He}}=0.15 \mathrm{~atm}, \mathrm{P}_{\mathrm{Ne}}=0.42 \mathrm{~atm}$. Determinar la presión total del sistema.

a) $0.89 \mathrm{~atm}$

b) $0.32 \mathrm{~atm}$

c) $1.00 \mathrm{~atm}$

d) No se puede determinar con la información entregada.
\vspace{0.5cm}
\subsection*{Pregunta 10 - 2017-2}
¿Cuántos gramos de $\mathrm{NaNO}_3$ contiene 250 mL de una solución 0.707 M?

a) 0.18 g

b) 0.24 g

c) 15 g

d) 15000 g
\vspace{0.5cm}
\section{2018-1}
\subsection*{Pregunta 07 - 2018-1}
Considerando la siguiente reacción redox no balanceada:

$$
\mathrm{Mn}^{2+}+\mathrm{H}_2 \mathrm{O}_2 \rightarrow \mathrm{MnO}_2+\mathrm{H}_2 \mathrm{O}
$$

¿Cuál de las siguientes alternativas de ecuación iónica balanceada es la correcta, considerando un medio básico?

a) $\mathrm{Mn}^{2+}+\mathrm{H}_2 \mathrm{O}_2+2 \mathrm{OH}^{-} \rightarrow \mathrm{MnO}_2+2 \mathrm{H}_2 \mathrm{O}+2 e^{-}$

b) $\mathrm{Mn}^{2+}+2 \mathrm{H}_2 \mathrm{O}_2 \rightarrow \mathrm{MnO}_2+\mathrm{H}_2 \mathrm{O}+4 \mathrm{H}^{+}$

c) $\mathrm{Mn}^{2+}+\mathrm{H}_2 \mathrm{O}_2+2 \mathrm{OH}^{-} \rightarrow \mathrm{MnO}_2+2 \mathrm{H}_2 \mathrm{O}$

d) $\mathrm{Mn}^{2+}+\mathrm{H}_2 \mathrm{O}_2+2 \mathrm{OH}^{-}+2 e^{-} \rightarrow \mathrm{MnO}_2+2 \mathrm{H}_2 \mathrm{O}$
\vspace{0.5cm}
\subsection*{Pregunta 08 - 2018-1}
En relación a las disoluciones tampón o amortiguadoras (buffer), ¿cuál de las siguientes afirmaciones es FALSA?

a) Uno de los compuestos más conocidos para ser amortiguador es el bicarbonato.

b) Una disolución amortiguadora necesita un ácido fuerte y su base conjugada o una base fuerte y su ácido conjugado.

c) Los organismos vivos necesitan un intervalo de pH relativamente pequeño en su interior por lo que utilizan disoluciones amortiguadoras para mantener un pH constante.

d) La disolución amortiguadora tiene la capacidad de resistir los cambios de pH cuando se adiciona pequeñas cantidades de ácido o base.
\vspace{0.5cm}
\subsection*{Pregunta 09 - 2018-1}
Calcular el número másico de un átomo de hierro que posee tiene 28 neutrones.

a) 2

b) 26

c) 54

d) 28
\vspace{0.5cm}
\subsection*{Pregunta 10 - 2018-1}
Considerando la siguiente reacción: $\mathrm{S}_8(I)+4 \mathrm{Cl}_2(g) \rightarrow 4 \mathrm{~S}_2 \mathrm{Cl}_2(I)$ se mezclan y calientan 4.06 g de $\mathrm{S}_8$ y 6.24 g de $\mathrm{Cl}_2$ donde en el rendimiento real es de 6.55 g de $\mathrm{S}_2 \mathrm{Cl}_2$. ¿Cuál es el porcentaje del rendimiento de la reacción?

a) $76.6 \%$

b) $55.1 \%$

c) $23.4 \%$

d) $44.9 \%$
\vspace{0.5cm}
\section{2018-2}
\subsection*{Pregunta 07 - 2018-2}
Calcular el valor más cercano de la concentración, en el equilibrio, de iones $\left[\mathrm{H}^{+}\right]$ considerando una concentración inicial de $0,20 \mathrm{M}$ de $\mathrm{C}_6 \mathrm{H}_5 \mathrm{COOH}$, un $\mathrm{K}_{\mathrm{a}}=6,5 \times 10^{-5}$ y la reacción $\mathrm{C}_6 \mathrm{H}_5 \mathrm{COOH} \leftrightarrow \mathrm{H}^{+}+$ $\mathrm{C}_6 \mathrm{H}_5 \mathrm{COO}^{-}$.

a) $1,3 \times 10^{-5} \mathrm{M}$

b) $3,6 \times 10^{-3} \mathrm{M}$

c) $2,8 \times 10^2 \mathrm{M}$

d) $7,7 \times 10^4 \mathrm{M}$
\vspace{0.5cm}
\subsection*{Pregunta 08 - 2018-2}
Comparando $\mathrm{O}_2$ en medio ácido o en medio básico, en condiciones estándar. ¿Cuál de estos medios permite que el $\mathrm{O}_2$ sea mejor agente oxidante?

a) Ambos medios dan las mismas condiciones.

b) Con la información entregada no es posible determinar la respuesta.

c) Medio básico.

d) Medio ácido.
\vspace{0.5cm}
\subsection*{Pregunta 09 - 2018-2}
Sobre líquidos, sólidos y fuerzas intermoleculares, ¿Cuál de los siguientes enunciados es FALSO?

a) Todas las substancias existen en alguna de los tres estados: gas, líquido o sólido.

b) Para cada sustancia hay una temperatura, Ilamada temperatura crítica, por encima del cual la fase gaseosa (un gas) no puede ser licuado.

c) Las relaciones entre las fases de una sola sustancia se ilustran mediante un diagrama de fases, que representan gráficamente las diferentes fases entre los estados de la materia.

d) La tensión superficial es la cantidad de energía requerida para estirar o aumentar la superficie de un líquido por una unidad de volumen.
\vspace{0.5cm}
\subsection*{Pregunta 10 - 2018-2}
Considerando la siguiente reacción,

$$
\mathrm{CaF}_2+\mathrm{H}_2 \mathrm{SO}_4 \rightarrow \mathrm{CaSO}_4+2 \mathrm{HF}
$$

$6,00 \mathrm{~kg}$ de $\mathrm{CaF}_2$ son tratados con exceso de $\mathrm{H}_2 \mathrm{SO}_4$ produciendo $2,86 \mathrm{~kg}$ de $\mathrm{HF}$.

Calcular el porcentaje de rendimiento de $\mathrm{HF}$.

a) $108 \%$

b) $0,093 \%$

c) $186 \%$

d) $93 \%$
\vspace{0.5cm}
\section{2019-1}
\subsection*{Pregunta 18 - 2019-1}
Para la reacción $2 \mathrm{NO}(\mathrm{g})+\mathrm{O}_2(\mathrm{g}) \leftrightarrow 2 \mathrm{NO}_2(\mathrm{g})$, ¿cuál de las siguientes afirmaciones es INCORRECTA?

a) Si se agrega Oxigeno el sistema reacciona consumiendo oxigeno hasta restablecer el equilibrio.

b) La constante de equilibrio no varía al agregar $\mathrm{O}_2$.

c) Si se agrega $\mathrm{NO}_2$ el sistema favorecerá el sentido inverso de la reacción hasta restablecer el equilibrio.

d) Si se agrega un catalizador se aumenta la rapidez de la reacción y cambia el valor de K.
\vspace{0.5cm}
\subsection*{Pregunta 19 - 2019-1}
Una solución que posee un compuesto desconocido, pero que es sabido que es un compuesto de base débil (B) tiene un pH igual 8,8 y la concentración inicial es de $0,35 \mathrm{~mol} \mathrm{~L}^{-1}$.

¿Cuál es $\mathrm{K}_{\mathrm{b}}$ de la base?

a) $1,89 \times 10^{-5}$

b) $1,14 \times 10^{-10}$

c) $7,18 \times 10^{-18}$

d) $4,53 \times 10^{-9}$
\vspace{0.5cm}
\subsection*{Pregunta 20 - 2019-1}
Sobre las fuerzas intermoleculares y uniones interatómicas ¿Cuál de estas afirmaciones es FALSA?

a) Todas las sustancias, en principio, pueden existir en tres estados de la materia: sólido, líquido y gas.

b) La viscosidad es una medida de la resistencia de un líquido al momento de estar estático.

c) Las fuerzas intermoleculares actúan entre las moléculas o entre moléculas e iones.

d) El puente de hidrógeno es un fuerte tipo de atracción intermolecular.
\vspace{0.5cm}
\subsection*{Pregunta 21 - 2019-1}
Un reactor de combustión se alimenta con 100 moles de butano (C4H10) y 5.000 moles de aire ( $21 \%$ en moles de O 2 ).

¿Cuál es el exceso de aire si la combustión es completa?

a) $61,6 \%$

b) $70 \%$

c) $0,006 \%$

d) $0 \%$
\vspace{0.5cm}
\section{2019-2}
\subsection*{Pregunta 18 - 2019-2}
\begin{center}
    \includegraphics[width=0.6\textwidth]{images/figura_3_p18_2019_2.png}
\end{center}

Con respecto a los sólidos iónicos como el NaCl (Figura 3), es CORRECTO decir que:

I. Los sólidos iónicos se mantienen unidos por fuerzas de atracción electrostáticas (cargas netas).

II. Los sólidos iónicos poseen orbitales atómicos ocupados por cationes y aniones, como los expuestos en la figura.

III. Los sólidos iónicos tienen átomos "nadando en un mar de electrones".

a) Sólo II

b) Sólo I y III

c) Sólo I

d) Sólo I y II
\vspace{0.5cm}
\subsection*{Pregunta 19 - 2019-2}
A $500^{\circ} \mathrm{C}$ la constante de equilibrio $(\mathrm{Kp})$, de la reacción, es $60$.

$$
\mathrm{H}_{2}(\mathrm{g})+\mathrm{I}_{2}(\mathrm{g}) \rightleftharpoons 2 \mathrm{HI}(\mathrm{g})
$$

Prediga la dirección en la que avanzará la reacción para alcanzar el equilibrio y calcule el valor más cercano al cociente de reacción si las presiones parciales de $P_{I_2}=0,888$ atm; $P_{HI}=0,592$ atm; $P_{H_2}= 0,296$ atm.

a) $\mathrm{Q}=2,25$; la reacción avanza de reactantes a productos.

b) $\mathrm{Q}=1,3$; la reacción avanza de productos a reactantes.

c) $Q=2,25$; la reacción avanza de productos a reactantes.

d) $Q=1,3$; la reacción avanza de reactantes a productos.
\vspace{0.5cm}
\subsection*{Pregunta 20 - 2019-2}
¿Cuál de las siguientes afirmaciones es CORRECTA?

a) El pH influye en la solubilidad de las sales que contienen un anión que puede sufrir hidrólisis, como el $\mathrm{BaF}_2$, ya que el F- puede reaccionar con los protones del agua en un medio ácido para dar HF. Este proceso hace que el equilibrio de solubilidad tiene que desplazarse a la izquierda y el $\mathrm{BaF}_2$ se disuelve menos.

b) Los hidróxidos alcalinotérreos que son parcialmente solubles, como el $\mathrm{Mg}(\mathrm{OH})_2$, por ejemplo. Al añadir iones hidroxilos (aumento de pH ) el equilibrio se desplaza hacia la izquierda, disminuyendo la solubilidad.

c) El compuesto $\mathrm{BaF}_2$, libera iones $\mathrm{F}^{-}$en disolución. Estos iones tienen la posibilidad de interactuar con el agua, lo que no implica una variación de la solubilidad y el pH.

d) $\mathrm{El} \mathrm{Ca}(\mathrm{OH})_2$, es parcialmente soluble por en agua. Al añadir protones (disminución del pH ) el equilibrio se desplaza hacia la izquierda, disminuyendo la solubilidad.
\vspace{0.5cm}
\subsection*{Pregunta 21 - 2019-2}
Con respecto a la solubilidad, es CORRECTO AFIRMAR que:

I. Un ejemplo de un soluto no polar con disolvente polar, podría ser cera en agua.

II. Las fuerzas intramoleculares determinan la solubilidad de las moléculas.

III. La regla general de solubilidad es que "lo similar disuelve a lo no similar".

a) Sólo I y II

b) Sólo 1

c) Sólo II

d) Sólo II y III
\vspace{0.5cm}
\subsection*{Pregunta 33 - 2019-2}
\textit{[Pregunta en revisión - El texto anterior era duplicado de 2023-2]}
\vspace{0.5cm}
\section{2023-2}
\subsection*{Pregunta 28 - 2023-2}
El átomo de plata presenta un radio atómico de 172 pm, exprese esta longitud en cm.

Nota: 1 pm equivale a $1 \times 10^{-10} \mathrm{~cm}$.

a) $1,72 \times 10^{-8} \mathrm{~cm}$

b) $1,72 \times 10^{-10} \mathrm{~cm}$

c) $1,72 \times 10^{-7} \mathrm{~cm}$

d) $0,172 \mathrm{~cm}$
\vspace{0.5cm}
\subsection*{Pregunta 29 - 2023-2}
Considere un sistema cerrado que contiene agua y que se encuentra a una presión de 1 atm. En estas condiciones, se sabe que al calentar el agua hasta una temperatura de $100^{\circ} \mathrm{C}$ ocurre el proceso de ebullición.

Al respecto, indique cuál de las siguientes afirmaciones es FALSA:

a) El calentamiento del agua líquida hace que aumente su presión de vapor.

b) A $100^{\circ} \mathrm{C}$, el agua líquida se encuentra en equilibrio con la fase gaseosa.

c) Las moléculas de agua que se encuentran en la fase gaseosa presentan energía cinética mayor que las moléculas de agua en fase líquida.

d) A nivel del mar, a $100^{\circ} \mathrm{C}$ la presión de vapor del agua líquida es menor a 1 atm.
\vspace{0.5cm}
\subsection*{Pregunta 30 - 2023-2}
A $25^{\circ} \mathrm{C}$ el tetróxido de dinitrógeno, $\mathrm{N}_2 \mathrm{O}_4$, se disocia un $24,0 \%$ formando dióxido de nitrógeno, $\mathrm{NO}_2$, según la siguiente reacción química.
$$
\mathrm{N}_2 \mathrm{O}_{4(\mathrm{~g})} \leftrightarrows 2 \mathrm{NO}_{2(\mathrm{~g})} \quad \mathrm{K}_{\mathrm{c}}=4,7 \times 10^{-3}
$$

Determine la concentración de las especies en el equilibrio a esta temperatura.

a) No se puede determinar.

b) $\left[\mathrm{N}_2 \mathrm{O}_4\right]=0,0118 \mathrm{~mol} / \mathrm{L},\left[\mathrm{NO}_2\right]=7,44 \times 10^{-3} \mathrm{~mol} / \mathrm{L}$.

c) $\left[\mathrm{N}_2 \mathrm{O}_4\right]=0,0470 \mathrm{~mol} / \mathrm{L},\left[\mathrm{NO}_2\right]=0,0149 \mathrm{~mol} / \mathrm{L}$.

d) $\left[\mathrm{N}_2 \mathrm{O}_4\right]=0,760 \mathrm{~mol} / \mathrm{L},\left[\mathrm{NO}_2\right]=0,480 \mathrm{~mol} / \mathrm{L}$.
\vspace{0.5cm}
\subsection*{Pregunta 31 - 2023-2}
Se desea preparar una disolución amortiguadora (buffer) que simule el pH sanguíneo de 7,4. Para ello se dispone de 0,1 mol de un ácido débil, HA, y su sal KA, los que se encuentran en un volumen de 200 mL . Determine el valor más cercano de la concentración de la sal KA.

Dato: $\mathrm{K}_{\mathrm{HA}}=2,5 \times 10^{-8}$

a) $0,32 \mathrm{~mol} / \mathrm{L}$.

b) $0,063 \mathrm{~mol} / \mathrm{L}$.

c) $0,95 \mathrm{~mol} / \mathrm{L}$.

d) $0,79 \mathrm{~mol} / \mathrm{L}$.
\vspace{0.5cm}
\subsection*{Pregunta 32 - 2023-2}
Considere la siguiente reacción química balanceada:
$$
3 \mathrm{Cu}+2 \mathrm{HNO}_3+6 \mathrm{H}^{+} \rightarrow 3 \mathrm{Cu}^{2+}+2 \mathrm{NO}+4 \mathrm{H}_2 \mathrm{O}
$$

Indique cuál de las siguientes afirmaciones es VERDADERA:

a) $\mathrm{HNO}_3$ es el agente reductor.

b) Cu es el agente oxidante.

c) La cantidad de electrones intercambiados es 6.

d) El estado de oxidación de nitrógeno en $\mathrm{HNO}_3$ es $7+$.
\vspace{0.5cm}
\subsection*{Pregunta 33 - 2023-2}
\begin{center}
    \includegraphics[width=0.6\textwidth]{images/pregunta_33_2023_2.png}
\end{center}

La siguiente imagen representa el electrodo de hidrógeno en condiciones estándar, que es utilizado como electrodo de referencia. Al respecto, indique cuál(es) de las siguientes afirmaciones es(son) VERDADERA(S):

I. El metal "X" corresponde a un electrodo de platino.

II. La disolución de HCl tiene $\mathrm{pH}=0$.

III. $\mathrm{H}_2$ se encuentra a una presión de 1 atm.

a) Solo II y III.

b) Solo III.

c) Solo II.

d) Todas.
\vspace{0.5cm}
\section{2024-2}
\subsection*{Pregunta 28 - 2024-2}
A una temperatura de $0^{\circ} \mathrm{C}$ y una presión de 1 atm, una masa de $0,967 \mathrm{~kg}$ de un gas ocupa un volumen de $1 \mathrm{~m}^3$. Determine la densidad del gas en $\mathrm{kg} / \mathrm{L}$ en estas condiciones.
$$
1 \mathrm{~m}^3=1.000 \mathrm{~L}
$$

a) $0,000967 \mathrm{~kg} / \mathrm{L}$

b) $0,967 \mathrm{~kg} / \mathrm{L}$

c) $967 \mathrm{~kg} / \mathrm{L}$

d) $9,67 \mathrm{~kg} / \mathrm{L}$
\vspace{0.5cm}
\subsection*{Pregunta 29 - 2024-2}
Considere el diagrama de fases del agua pura:

\begin{center}
    \includegraphics[width=0.6\textwidth]{images/pregunta_29_2024_2.png}
\end{center}

Al respecto, indique cuál de las siguientes afirmaciones es FALSA:

a) A $0,00603$ atm y $0,01^\circ C$ el agua se comporta como un fluido supercrítico.

b) A $0,00603$ atm y $1000^\circ C$ el agua se encuentra en estado gaseoso.

c) A 218 atm y $100^\circ C$ el agua se encuentra en estado líquido.

d) A una presión de 1 atm el agua se congela a $0^\circ C$ y ebulle a $100^\circ C$.
\vspace{0.5cm}
\subsection*{Pregunta 30 - 2024-2}
Considere que a una temperatura de $25^{\circ} \mathrm{C}$ una reacción química presenta $\mathrm{Kc}=14,8$ y, al iniciarse la reacción a $25^{\circ} \mathrm{C}, \mathrm{Q}=10$. En función de lo anterior, indique cuál(es) de las siguientes afirmaciones es(son) CORRECTA(S):

I. La reacción química procede hacia la formación de productos para alcanzar el equilibrio químico.

II. De acuerdo con el valor de Kc, el equilibrio químico se encuentra desplazado hacia la formación de reactantes.

III. El sistema alcanzará el equilibrio cuando $Q=14,8$.

a) I y III

b) Todas son correctas.

c) Sólo I

d) Sólo III
\vspace{0.5cm}
\subsection*{Pregunta 31 - 2024-2}
Indique cuál de las siguientes disoluciones generará una mayor cantidad de iones, si todas se encuentran disueltas en un mismo volumen de agua:

a) Cloruro de sodio, $\mathrm{NaCl}, 0,5 \mathrm{~mol} / \mathrm{L}$.

b) Glucosa, $\mathrm{C}_6 \mathrm{H}_{12} \mathrm{O}_6 1,3 \mathrm{~mol} / \mathrm{L}$.

c) Ácido clorhídrico, $\mathrm{HCl}, 1,0 \mathrm{~mol} / \mathrm{L}$.

d) Hidróxido de sodio, $\mathrm{NaOH} 0,7 \mathrm{~mol} / \mathrm{L}$.
\vspace{0.5cm}
\subsection*{Pregunta 32 - 2024-2}
Considere la siguiente reacción redox balanceada:
$$
3 \mathrm{Cu}+2 \mathrm{HNO}_3+6 \mathrm{H}^{+} \rightarrow 3 \mathrm{Cu}^{2+}+2 \mathrm{NO}+4 \mathrm{H}_2 \mathrm{O}
$$

Indique cuál de las siguientes afirmaciones es INCORRECTA:

a) Los electrones intercambiados en el proceso son 2.

b) $\mathrm{HNO}_3$ se reduce.

c) El proceso ocurre en medio ácido.

d) $\mathrm{Cu}$ se oxida.
\vspace{0.5cm}
\subsection*{Pregunta 33 - 2024-2}
Determine la constante de equilibrio para la siguiente reacción a $25^{\circ} \mathrm{C}$ :

$$
2 \mathrm{Ag}^{+}{ }_{(\mathrm{ac})}+\mathrm{Fe}_{(\mathrm{s})} \rightleftharpoons 2 \mathrm{Ag}_{(\mathrm{s})}+\mathrm{Fe}^{2+}{ }_{(\mathrm{aq})}
$$

- $\mathrm{E}^{\circ} \mathrm{Fe}^{2+} / \mathrm{Fe}=-0,447 \mathrm{~V}$
- $\mathrm{E}^{\circ} \mathrm{Ag}^{+} / \mathrm{Ag}=0,7996 \mathrm{~V}$

a) $K=1,3 \times 10^{42}$

b) $\mathrm{K}=1,2 \times 10^{21}$

c) $\mathrm{K}=1,2 \times 10^{-21}$

d) $\mathrm{K}=9,18 \times 10^5$
\vspace{0.5cm}

\newpage
\section*{Tabla de Respuestas}
\begin{center}
\begin{tabular}{|c|c|c||c|c|c|}
\hline
\textbf{Año} & \textbf{Pre.} & \textbf{Res.} & \textbf{Año} & \textbf{Pre.} & \textbf{Res.} \\
\hline
2016-1 & 09 & c & 2019-1 & 18 & d \\
2016-1 & 10 & d & 2019-1 & 19 & b \\
2016-1 & 11 & c & 2019-1 & 20 & b \\
\cline{1-3}
2016-2 & 07 & d & 2019-1 & 21 & a \\
\cline{4-6}
2016-2 & 08 & d & 2019-2 & 18 & d \\
2016-2 & 09 & d & 2019-2 & 19 & d \\
2016-2 & 10 & b & 2019-2 & 20 & b \\
\cline{1-3}
2017-1 & 07 & c & 2019-2 & 21 & b \\
\cline{4-6}
2017-1 & 08 & d & 2023-2 & 28 & a \\
2017-1 & 09 & a & 2023-2 & 29 & d \\
2017-1 & 10 & b & 2023-2 & 30 & b \\
\cline{1-3}
2017-2 & 07 & c & 2023-2 & 31 & a \\
2017-2 & 08 & a & 2023-2 & 32 & c \\
2017-2 & 09 & a & 2023-2 & 33 & d \\
\cline{4-6}
2017-2 & 10 & c & 2024-2 & 28 & a \\
\cline{1-3}
2018-1 & 07 & c & 2024-2 & 29 & a \\
2018-1 & 08 & b & 2024-2 & 30 & a \\
2018-1 & 09 & c & 2024-2 & 31 & c \\
2018-1 & 10 & a & 2024-2 & 32 & a \\
\cline{1-3}
2018-2 & 07 & b & 2024-2 & 33 & a \\
2018-2 & 08 & d & & & \\
2018-2 & 09 & d & & & \\
2018-2 & 10 & d & & & \\
\hline
\end{tabular}
\end{center}

\vfill
\begin{center}
    \small Puedes ver este repositorio en \url{https://github.com/anomvlito/respositorio-fundamentals}
\end{center}

\end{document}