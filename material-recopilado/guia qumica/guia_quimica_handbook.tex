\documentclass{article}
\usepackage{fullpage}
\usepackage{graphicx}
\usepackage[spanish]{babel}
\usepackage{amssymb}
\usepackage{amsmath}
\usepackage{cancel}
\usepackage{booktabs} 
\usepackage{tikz}

%%%%% Comandos Personalizados %%%%%
\newcommand{\N}{\mathbb{N}}
\newcommand{\R}{\mathbb{R}}
\newcommand{\Q}{\mathbb{Q}}
\newcommand{\E}{\mathbb{E}}
\newcommand{\PP}{\mathbb{P}}
\newcommand{\la}{\leftarrow}
\newcommand{\ra}{\rightarrow}
\newcommand{\lra}{\leftrightarrow}
\newcommand{\Ra}{\Rightarrow}
\newcommand{\La}{\Leftarrow}
\newcommand{\LRa}{\Leftrightarrow}
\newcommand{\sub}{\subseteq}
\newcommand{\matro}{\mathcal{M}}

\newcommand{\twopartdef}[4]
{
	\left\{
		\begin{array}{ll}
			#1 &  \text{#2} \\
			#3 &  \text{#4}
		\end{array}
	\right.
}

%%%%%  Fin Comandos Personalizados %%%%%

 %%%%%%%%%% MODIFICAR %%%%%%%%%%
\newcommand{\alumnos}{Solucionario Generado}
\newcommand{\departamento}{Departamento de Ingenieria Industrial y de Sistemas}
\newcommand{\ramo}{Química General}
\newcommand{\sigla}{QIM100}
\newcommand{\titulo}{Solucionario Guía de Ejercicios}
\newcommand{\semestre}{Recopilación}
\newcommand{\anio}{2025}
\newcommand{\med}{\frac{1}{2}}
\newcommand{\indep}{\mathcal{I}}
%%%%%%%%%% FIN MODIFICAR %%%%%%%%%%

\renewcommand{\thesubsection}{\alph{subsection}}


\usepackage{tikz}
\usetikzlibrary{arrows.meta}

\begin{document}

\title{Solucionario Guía de Ejercicios Química}
\maketitle

\section{2016-1}

\subsection*{Pregunta 09 - 2016-1}
\textbf{Enunciado:} ¿Cuál de las siguientes afirmaciones es FALSA respecto a conceptos de oxidación y reducción?

\textbf{Solución:}

Analicemos cada alternativa:

\begin{itemize}
    \item[a)] \textbf{Verdadera}. La ecuación de Nernst permite relacionar el potencial estándar de celda ($E^\circ$) con el potencial ($E$) en condiciones no estándar (concentraciones distintas a 1 M o presiones distintas a 1 atm).
    \item[b)] \textbf{Verdadera}. Una celda voltaica (o galvánica) es un dispositivo electroquímico que genera electricidad a partir de una reacción redox espontánea.
    \item[c)] \textbf{Falsa}. En cualquier celda electroquímica (sea voltaica o electrolítica), el \textbf{ánodo} es el electrodo donde ocurre la \textbf{oxidación} y hacia donde migran los \textbf{aniones}. Los \textbf{cationes} siempre migran hacia el \textbf{cátodo} (donde ocurre la reducción). Por lo tanto, afirmar que los cationes se dirigen al ánodo es incorrecto.
    \item[d)] \textbf{Verdadera}. Por definición, la electroquímica estudia la interconversión entre energía química y eléctrica, lo cual implica necesariamente transferencia de electrones (reacciones redox).
\end{itemize}

\textbf{Respuesta Correcta: c)}

\vspace{0.5cm}

\subsection*{Pregunta 10 - 2016-1}
\textbf{Enunciado:} Balancear la reacción redox $\mathrm{Fe}^{2+}+\mathrm{Cr}_2 \mathrm{O}_7^{2-} \rightarrow \mathrm{Fe}^{3+}+\mathrm{Cr}^{3+}$ (asumiendo medio ácido por la presencia de dicromato y los protones en las alternativas).

\textbf{Solución:}

Usamos el método del ion-electrón en medio ácido.

1. \textbf{Semirreacción de Oxidación:} El hierro pasa de estado +2 a +3.
$$ \mathrm{Fe}^{2+} \rightarrow \mathrm{Fe}^{3+} + 1e^- $$
Para igualar electrones más adelante, multiplicamos por 6:
$$ 6\mathrm{Fe}^{2+} \rightarrow 6\mathrm{Fe}^{3+} + 6e^- \quad \text{(1)} $$

2. \textbf{Semirreacción de Reducción:} El cromo en el dicromato ($\mathrm{Cr}_2\mathrm{O}_7^{2-}$) actúa con estado +6 y pasa a +3.
Balanceamos átomos de Cr:
$$ \mathrm{Cr}_2\mathrm{O}_7^{2-} \rightarrow 2\mathrm{Cr}^{3+} $$
Balanceamos Oxígenos agregando agua:
$$ \mathrm{Cr}_2\mathrm{O}_7^{2-} \rightarrow 2\mathrm{Cr}^{3+} + 7\mathrm{H}_2\mathrm{O} $$
Balanceamos Hidrógenos agregando protones ($H^+$):
$$ \mathrm{Cr}_2\mathrm{O}_7^{2-} + 14\mathrm{H}^+ \rightarrow 2\mathrm{Cr}^{3+} + 7\mathrm{H}_2\mathrm{O} $$
Balanceamos carga agregando electrones (Carga neta izquierda: $-2+14=+12$; derecha: $2(+3)=+6$. Faltan 6 electrones a la izquierda):
$$ \mathrm{Cr}_2\mathrm{O}_7^{2-} + 14\mathrm{H}^+ + 6e^- \rightarrow 2\mathrm{Cr}^{3+} + 7\mathrm{H}_2\mathrm{O} \quad \text{(2)} $$

3. \textbf{Suma de Semirreacciones:} Sumamos (1) y (2), cancelando los electrones:
$$ 6\mathrm{Fe}^{2+} + \mathrm{Cr}_2\mathrm{O}_7^{2-} + 14\mathrm{H}^+ + \cancel{6e^-} \rightarrow 6\mathrm{Fe}^{3+} + \cancel{6e^-} + 2\mathrm{Cr}^{3+} + 7\mathrm{H}_2\mathrm{O} $$

Ecuación final:
$$ 6\mathrm{Fe}^{2+} + 14\mathrm{H}^{+} + \mathrm{Cr}_2 \mathrm{O}_7^{2-} \rightarrow 6\mathrm{Fe}^{3+} + 2\mathrm{Cr}^{3+} + 7\mathrm{H}_2 \mathrm{O} $$

Esta ecuación coincide con la alternativa d).

\textbf{Respuesta Correcta: d)}

\vspace{0.5cm}

\subsection*{Pregunta 11 - 2016-1}
\textbf{Enunciado:} Calcular el volumen de 6,69 moles de gas a $257^{\circ} \mathrm{C}$ y 10.10 atm.

\textbf{Solución:}

Datos:
\begin{itemize}
    \item $n = 6,69$ mol
    \item $T = 257^{\circ}\mathrm{C} = 257 + 273,15 = 530,15 \text{ K}$
    \item $P = 10,10$ atm
    \item $R = 0,08206 \text{ L atm mol}^{-1} \text{ K}^{-1}$
\end{itemize}

Usamos la Ley de los Gases Ideales:
$$ PV = nRT $$
$$ V = \frac{nRT}{P} $$

Reemplazamos los valores:
$$ V = \frac{6,69 \cdot 0,08206 \cdot 530,15}{10,10} $$

Calculamos:
$$ V \approx \frac{291,1}{10,10} \approx 28,82 \text{ L} $$

La alternativa más cercana es $2,9 \times 10 \text{ L}$ (que equivale a 29 L).

\textbf{Respuesta Correcta: c)}

\vspace{0.5cm}

\vspace{0.5cm}

\section{2016-2}

\subsection*{Pregunta 07 - 2016-2}
\textbf{Enunciado:} Balancear la reacción redox $\mathrm{Bi}(\mathrm{OH})_3+\mathrm{SnO}_2^{2-} \rightarrow \mathrm{SnO}_3^{2-}+ \mathrm{Bi}$ en medio básico.

\textbf{Solución:}

1. \textbf{Semirreacción de Reducción:} El Bismuto pasa de +3 a 0.
$$ \mathrm{Bi}(\mathrm{OH})_3 \rightarrow \mathrm{Bi} $$
Balanceamos oxígenos con agua:
$$ \mathrm{Bi}(\mathrm{OH})_3 \rightarrow \mathrm{Bi} + 3\mathrm{H}_2\mathrm{O} $$
Balanceamos hidrógenos con protones:
$$ \mathrm{Bi}(\mathrm{OH})_3 + 3\mathrm{H}^+ \rightarrow \mathrm{Bi} + 3\mathrm{H}_2\mathrm{O} $$
Neutralizamos protones con $\mathrm{OH}^-$ (medio básico):
$$ \mathrm{Bi}(\mathrm{OH})_3 + 3\mathrm{H}_2\mathrm{O} \rightarrow \mathrm{Bi} + 3\mathrm{H}_2\mathrm{O} + 3\mathrm{OH}^- $$
Simplificando y balanceando carga ($3e^-$):
$$ \mathrm{Bi}(\mathrm{OH})_3 + 3e^- \rightarrow \mathrm{Bi} + 3\mathrm{OH}^- \quad \text{(1)} $$
Multiplicamos por 2 para igualar electrones con la oxidación (ver paso 2):
$$ 2\mathrm{Bi}(\mathrm{OH})_3 + 6e^- \rightarrow 2\mathrm{Bi} + 6\mathrm{OH}^- \quad \text{(1.1)} $$

2. \textbf{Semirreacción de Oxidación:} El Estaño pasa de +2 a +4.
$$ \mathrm{SnO}_2^{2-} \rightarrow \mathrm{SnO}_3^{2-} $$
Balanceamos oxígenos con agua:
$$ \mathrm{SnO}_2^{2-} + \mathrm{H}_2\mathrm{O} \rightarrow \mathrm{SnO}_3^{2-} $$
Balanceamos hidrógenos con protones:
$$ \mathrm{SnO}_2^{2-} + \mathrm{H}_2\mathrm{O} \rightarrow \mathrm{SnO}_3^{2-} + 2\mathrm{H}^+ $$
Neutralizamos protones con $\mathrm{OH}^-$:
$$ \mathrm{SnO}_2^{2-} + \mathrm{H}_2\mathrm{O} + 2\mathrm{OH}^- \rightarrow \mathrm{SnO}_3^{2-} + 2\mathrm{H}_2\mathrm{O} $$
$$ \mathrm{SnO}_2^{2-} + 2\mathrm{OH}^- \rightarrow \mathrm{SnO}_3^{2-} + \mathrm{H}_2\mathrm{O} + 2e^- \quad \text{(2)} $$
Multiplicamos por 3:
$$ 3\mathrm{SnO}_2^{2-} + 6\mathrm{OH}^- \rightarrow 3\mathrm{SnO}_3^{2-} + 3\mathrm{H}_2\mathrm{O} + 6e^- \quad \text{(2.1)} $$

3. \textbf{Suma:}
$$ 2\mathrm{Bi}(\mathrm{OH})_3 + 3\mathrm{SnO}_2^{2-} + \cancel{6\mathrm{OH}^-} + \cancel{6e^-} \rightarrow 2\mathrm{Bi} + \cancel{6\mathrm{OH}^-} + 3\mathrm{H}_2\mathrm{O} + 3\mathrm{SnO}_3^{2-}+\cancel{6e^-} $$

Recorrigiendo la suma con cuidado:
Ec 1: $2\mathrm{Bi}(\mathrm{OH})_3 + 6e^- \rightarrow 2\mathrm{Bi} + 6\mathrm{OH}^-$
Ec 2.1: $3\mathrm{SnO}_2^{2-} + 6\mathrm{OH}^- \rightarrow 3\mathrm{SnO}_3^{2-} + 3\mathrm{H}_2\mathrm{O} + 6e^-$

Suma:
$$ 2\mathrm{Bi}(\mathrm{OH})_3 + 3\mathrm{SnO}_2^{2-} \rightarrow 2\mathrm{Bi} + 3\mathrm{SnO}_3^{2-} + 3\mathrm{H}_2\mathrm{O} $$

La alternativa d) muestra: $2 \mathrm{Bi}(\mathrm{OH})_3+3 \mathrm{SnO}_2^{2-} \rightarrow 2 \mathrm{Bi}+3 \mathrm{H}_2 \mathrm{O}+3 \mathrm{SnO}_3^{2-}$
Coincide perfectamente.

\textbf{Respuesta Correcta: d)}

\vspace{0.5cm}

\subsection*{Pregunta 08 - 2016-2}
\textbf{Enunciado:} Balancear la reacción redox $\mathrm{Fe}^{2+}+\mathrm{Cr}_2 \mathrm{O}_7^{2-} \rightarrow \mathrm{Fe}^{3+}+\mathrm{Cr}^{3+}$.
\textbf{Solución:}
Este ejercicio es idéntico a la Pregunta 10 de 2016-1.
La ecuación balanceada es:
$$ 6\mathrm{Fe}^{2+} + 14\mathrm{H}^{+} + \mathrm{Cr}_2 \mathrm{O}_7^{2-} \rightarrow 6\mathrm{Fe}^{3+} + 2\mathrm{Cr}^{3+} + 7\mathrm{H}_2 \mathrm{O} $$

\textbf{Respuesta Correcta: d)}

\vspace{0.5cm}

\subsection*{Pregunta 09 - 2016-2}
\textbf{Enunciado:} Gas a $V_1=8,55$ L ($P_1=1$ atm). Se comprime a $V_2=6,259$ L a T constante. Calcular $P_2$.

\textbf{Solución:}
Usamos la Ley de Boyle ($P_1 V_1 = P_2 V_2$ a T constante).
$$ (1 \text{ atm})(8,55 \text{ L}) = P_2 (6,259 \text{ L}) $$
$$ P_2 = \frac{8,55}{6,259} \text{ atm} \approx 1,366 \text{ atm} $$

Convertimos a mmHg ($1 \text{ atm} = 760 \text{ mmHg}$):
$$ P_2 = 1,366 \times 760 \text{ mmHg} \approx 1038,18 \text{ mmHg} $$

\textbf{Respuesta Correcta: d)}

\vspace{0.5cm}

\subsection*{Pregunta 10 - 2016-2}
\textbf{Enunciado:} Moles de $\mathrm{MgCl}_2$ en 60.0 mL de solución 0.100 M.

\textbf{Solución:}
Molaridad ($M$) = $\frac{\text{moles de soluto}}{\text{volumen disolución (L)}}$
$$ n = M \times V $$
$$ n = 0,100 \text{ mol/L} \times 0,060 \text{ L} $$
$$ n = 0,006 \text{ mol} = 6,00 \times 10^{-3} \text{ mol} $$

\textbf{Respuesta Correcta: b)}

\vspace{0.5cm}

\vspace{0.5cm}

\section{2017-1}

\vspace{0.5cm}

\subsection*{Pregunta 07 - 2017-1}
\textbf{Enunciado:} Balancear la reacción redox $\mathrm{Cr}_2 \mathrm{O}_7^{2-}+\mathrm{C}_2 \mathrm{O}_4^{2-} \rightarrow \mathrm{Cr}^{3+}+\mathrm{CO}_2$ en medio ácido.

\textbf{Referencia FE Handbook:}
\begin{itemize}
    \item \textbf{Pág. 86}: Potenciales de semicelda (Reducción). Se listan las semirreacciones balanceadas para especies comunes como el Dicromato.
    \item El método de ión-electrón es un procedimiento estándar de química general.
\end{itemize}

\vspace{0.5cm}

\subsection*{Pregunta 08 - 2017-1}
\textbf{Enunciado:} Calcular pH de solución 0.020 M de $\mathrm{Ba}(\mathrm{OH})_2$ (base fuerte).

\textbf{Referencia FE Handbook:}
Consultar sección \textbf{Acids, Bases, and pH} (Pág. 86).
\begin{itemize}
    \item \textbf{Pág. 86}: Definición de pH: $\text{pH} = -\log[H^+]$.
    \item \textbf{Pág. 86}: Producto iónico del agua: $[H^+][OH^-] = 10^{-14}$.
\end{itemize}

\vspace{0.5cm}

\subsection*{Pregunta 09 - 2017-1}
\textbf{Enunciado:} Geometría molecular de $\mathrm{CBr}_4$.

\textbf{Referencia FE Handbook:}
El Handbook no contiene una tabla explícita de Geometría Molecular (VSEPR).
\begin{itemize}
    \item Se requiere conocimiento general de química sobre enlaces covalentes y teoría de repulsión de pares electrónicos de valencia.
    \item \textbf{Pág. 85-86}: Sección general de Química.
\end{itemize}

\vspace{0.5cm}

\subsection*{Pregunta 10 - 2017-1}
\textbf{Enunciado:} Caliza ($\mathrm{CaCO}_3$) se descompone a Cal ($\mathrm{CaO}$) y $\mathrm{CO}_2$. Gramos de CaO a partir de 1.0 kg de $\mathrm{CaCO}_3$.

\textbf{Referencia FE Handbook:}
\begin{itemize}
    \item \textbf{Pág. 85}: Definiciones de Masa Molar, Moles.
    \item \textbf{Pág. 153/155}: Conceptos de Estequiometría y Rendimiento (Yield) en la sección de Termodinámica/Ingeniería Química.
    \item \textbf{Tabla Periódica} (al final del Handbook, Pág. 88 o similar): Masas atómicas.
\end{itemize}

\vspace{0.5cm}

\vspace{0.5cm}

\vspace{0.5cm}

\section{2017-2}

\subsection*{Pregunta 07 - 2017-2}
\textbf{Enunciado:} Balancear la reacción redox $\mathrm{MnO}_4^{-}+\mathrm{Cl}^{-} \rightarrow \mathrm{Mn}^{2+}+\mathrm{Cl}_2$ en medio ácido.

\textbf{Solución:}

1. \textbf{Semirreacción de Reducción:} El Manganeso pasa de +7 a +2.
$$ \mathrm{MnO}_4^- \rightarrow \mathrm{Mn}^{2+} $$
Balanceamos O con agua:
$$ \mathrm{MnO}_4^- \rightarrow \mathrm{Mn}^{2+} + 4\mathrm{H}_2\mathrm{O} $$
Balanceamos H con protones:
$$ \mathrm{MnO}_4^- + 8\mathrm{H}^+ \rightarrow \mathrm{Mn}^{2+} + 4\mathrm{H}_2\mathrm{O} $$
Balanceamos carga ($+7$ izq, $+2$ der $\implies$ agregar $5e^-$ izq):
$$ \mathrm{MnO}_4^- + 8\mathrm{H}^+ + 5e^- \rightarrow \mathrm{Mn}^{2+} + 4\mathrm{H}_2\mathrm{O} \quad \text{(1)} $$
Multiplicamos por 2 para igualar electrones (10):
$$ 2\mathrm{MnO}_4^- + 16\mathrm{H}^+ + 10e^- \rightarrow 2\mathrm{Mn}^{2+} + 8\mathrm{H}_2\mathrm{O} \quad \text{(1.1)} $$

2. \textbf{Semirreacción de Oxidación:} El Cloro pasa de -1 a 0.
$$ \mathrm{Cl}^- \rightarrow \mathrm{Cl}_2 $$
Balanceamos Cl:
$$ 2\mathrm{Cl}^- \rightarrow \mathrm{Cl}_2 $$
Balanceamos carga (agregar $2e^-$ der):
$$ 2\mathrm{Cl}^- \rightarrow \mathrm{Cl}_2 + 2e^- \quad \text{(2)} $$
Multiplicamos por 5 para igualar electrones (10):
$$ 10\mathrm{Cl}^- \rightarrow 5\mathrm{Cl}_2 + 10e^- \quad \text{(2.1)} $$

3. \textbf{Suma:}
$$ 2\mathrm{MnO}_4^- + 16\mathrm{H}^+ + \cancel{10e^-} + 10\mathrm{Cl}^- \rightarrow 2\mathrm{Mn}^{2+} + 8\mathrm{H}_2\mathrm{O} + 5\mathrm{Cl}_2 + \cancel{10e^-} $$

Ecuación final:
$$ 2\mathrm{MnO}_4^- + 16\mathrm{H}^+ + 10\mathrm{Cl}^- \rightarrow 2\mathrm{Mn}^{2+} + 8\mathrm{H}_2\mathrm{O} + 5\mathrm{Cl}_2 $$

La alternativa a) muestra esta ecuación, pero incluye $+5e^-$ a la derecha, lo cual es incorrecto en una ecuación global balanceada (los electrones deben cancelarse).
La alternativa c) muestra: $2 \mathrm{MnO}_4^{-}+16 \mathrm{H}^{+}+10 \mathrm{Cl}^{-} \rightarrow 2 \mathrm{Mn}^{2+}+8 \mathrm{H}_2 \mathrm{O}+5 \mathrm{Cl}_2$. Esta es la correcta.

\textbf{Respuesta Correcta: c)}

\vspace{0.5cm}

\subsection*{Pregunta 08 - 2017-2}
\textbf{Enunciado:} ¿Cuál de las siguientes afirmaciones es FALSA respecto a las reacciones óxido-reducción?

\textbf{Referencia FE Handbook:}
Consultar sección \textbf{Electrochemistry} (Pág. 86).
\begin{itemize}
    \item \textbf{Pág. 86}: Definición de Ánodo (Oxidación) y Cátodo (Reducción).
    \item Relación con $\Delta G < 0$ para espontaneidad (aunque la fórmula $\Delta G = -nFE$ es fundamental, a veces aparece en Termodinámica o Química).
    \item \textbf{Pág. 156}: $\Delta G^\circ = -RT \ln K$ (Equilibrio).
\end{itemize}

\vspace{0.5cm}

\subsection*{Pregunta 09 - 2017-2}
\textbf{Enunciado:} Presión total de mezcla de gases: $P_{N2}=0.32$, $P_{He}=0.15$, $P_{Ne}=0.42$ atm.

\textbf{Referencia FE Handbook:}
Sección \textbf{Thermodynamics / Ideal Gas Mixtures} (Pág. 145).
\begin{itemize}
    \item \textbf{Pág. 145}: Definición de Presión Parcial.
    $$ P_i = y_i P $$
    (Ley de Dalton: $P_{total} = \Sigma P_i$).
\end{itemize}

\vspace{0.5cm}

\subsection*{Pregunta 10 - 2017-2}
\textbf{Enunciado:} Gramos de $\mathrm{NaNO}_3$ en 250 mL de solución 0.707 M.

\textbf{Referencia FE Handbook:}
\begin{itemize}
    \item \textbf{Pág. 85-86}: Conceptos básicos de moles y concentración (Molaridad denotada como $[A]$).
    \item Cálculo básico: $n = M \times V$ y $m = n \times MM$.
\end{itemize}

\vspace{0.5cm}

\vspace{0.5cm}

\vspace{0.5cm}

\section{2018-1}

\subsection*{Pregunta 07 - 2018-1}
\textbf{Enunciado:} Balancear la reacción redox $\mathrm{Mn}^{2+}+\mathrm{H}_2 \mathrm{O}_2 \rightarrow \mathrm{MnO}_2+\mathrm{H}_2 \mathrm{O}$ en medio básico.

\textbf{Solución:}

\vspace{0.5cm}

\subsection*{Pregunta 07 - 2018-1}
\textbf{Enunciado:} Balancear la reacción redox $\mathrm{Mn}^{2+}+\mathrm{H}_2 \mathrm{O}_2 \rightarrow \mathrm{MnO}_2+\mathrm{H}_2 \mathrm{O}$ en medio básico.

\textbf{Referencia FE Handbook:}
\begin{itemize}
    \item \textbf{Pág. 86}: Potenciales de semicelda / Electroquímica.
    \item Balanceo Redox (medio básico): Se balancea como en medio ácido y se neutralizan los $H^+$ con $OH^-$.
\end{itemize}

\vspace{0.5cm}

\subsection*{Pregunta 08 - 2018-1}
\textbf{Enunciado:} ¿Cuál afirmación es FALSA respecto a disoluciones amortiguadoras (buffer)?

\textbf{Referencia FE Handbook:}
Sección \textbf{Acids, Bases, and pH} (Pág. 86).
\begin{itemize}
    \item Concepto de Buffer no tiene una ecuación explícita con ese nombre ("Buffer"), pero la ecuación de Henderson-Hasselbalch (o el principio de equilibrio ácido-base) es fundamental.
    \item \textbf{Pág. 86}: $K_a = \frac{[H^+][A^-]}{[HA]}$ es la base para entender la acción amortiguadora.
\end{itemize}

\vspace{0.5cm}

\subsection*{Pregunta 09 - 2018-1}
\textbf{Enunciado:} Número másico de un átomo de hierro con 28 neutrones.

\textbf{Referencia FE Handbook:}
\begin{itemize}
    \item \textbf{Tabla Periódica} (Pág. 88): Número Atómico ($Z$) del Hierro (Fe) es 26.
    \item Definición básica: $A = Z + N$.
\end{itemize}

\vspace{0.5cm}

\subsection*{Pregunta 10 - 2018-1}
\textbf{Enunciado:} Reacción $\mathrm{S}_8 + 4\mathrm{Cl}_2 \rightarrow 4\mathrm{S}_2\mathrm{Cl}_2$. Rendimiento.

\textbf{Referencia FE Handbook:}
\begin{itemize}
    \item \textbf{Pág. 85}: Pesos Atómicos (Tabla Periódica).
    \item \textbf{Pág. 153/155}: Definiciones de Reactivo Limitante y Rendimiento (Yield).
\end{itemize}

\vspace{0.5cm}

\section{2018-2}

\subsection*{Pregunta 07 - 2018-2}
\textbf{Enunciado:} Calcular $[H^+]$ en equilibrio para ácido benzoico 0.20 M ($K_a = 6.5 \times 10^{-5}$).

\textbf{Referencia FE Handbook:}
Sección \textbf{Acids, Bases, and pH} (Pág. 86).
\begin{itemize}
    \item \textbf{Pág. 86}: Fórmula para constante de acidez $K_a$.
    $$ K_a = \frac{[H^+][A^-]}{[HA]} $$
\end{itemize}

\vspace{0.5cm}

\subsection*{Pregunta 08 - 2018-2}
\textbf{Enunciado:} ¿En qué medio (ácido o básico) es el $\mathrm{O}_2$ mejor agente oxidante?

\textbf{Referencia FE Handbook:}
Sección \textbf{Electrochemistry} (Pág. 86).
\begin{itemize}
    \item \textbf{Pág. 86}: Tabla de Potenciales de Semicelda ($E^\circ$).
    \item Comparar $E^\circ$ para la reducción de oxígeno en medio ácido vs básico. Mayor $E^\circ$ implica mayor fuerza oxidante.
\end{itemize}

\vspace{0.5cm}

\subsection*{Pregunta 09 - 2018-2}
\textbf{Enunciado:} ¿Cuál enunciado es FALSO sobre líquidos, sólidos y fuerzas intermoleculares?

\textbf{Referencia FE Handbook:}
Conceptos generales de Química y Termodinámica.
\begin{itemize}
    \item \textbf{Pág. 153}: Vapor-Liquid Equilibrium (Raoult, Henry).
    \item \textbf{Pág. 155}: Diagramas de Fase (Clapeyron).
    \item Definición de Tensión Superficial (Unidades fundamentales).
\end{itemize}

\vspace{0.5cm}

\subsection*{Pregunta 10 - 2018-2}
\textbf{Enunciado:} $6.00$ kg de $\mathrm{CaF}_2$ producen $2.86$ kg de $\mathrm{HF}$ (con exceso de $\mathrm{H}_2\mathrm{SO}_4$). Calcular \% rendimiento.

\textbf{Referencia FE Handbook:}
\begin{itemize}
    \item \textbf{Pág. 88}: Masas atómicas (Ca, F, H, S, O).
    \item \textbf{Pág. 155}: Yield (Rendimiento).
\end{itemize}

\vspace{0.5cm}

\vspace{0.5cm}

\vspace{0.5cm}

\section{2019-1}

\subsection*{Pregunta 18 - 2019-1}
\textbf{Enunciado:} Reacción $2 \mathrm{NO}(\mathrm{g})+\mathrm{O}_2(\mathrm{g}) \leftrightarrow 2 \mathrm{NO}_2(\mathrm{g})$. ¿Cuál afirmación es INCORRECTA?

\textbf{Solución:}
Analicemos:
\begin{itemize}
    \item[a)] \textbf{Verdadera}. Si aumenta la concentración de un reactivo ($O_2$), el sistema se desplaza hacia los productos para consumir parte de ese exceso (Principio de Le Chatelier).
    \item[b)] \textbf{Verdadera}. La constante de equilibrio ($K$) solo depende de la temperatura. No cambia al variar concentraciones.
    \item[c)] \textbf{Verdadera}. Agregar producto ($\mathrm{NO}_2$) desplaza el equilibrio hacia reactivos (inverso).
    \item[d)] "Si se agrega un catalizador se aumenta la rapidez de la reacción y \textbf{cambia el valor de K}". \textbf{Falsa}. Un catalizador acelera la velocidad de reacción (ambos sentidos) disminuyendo la energía de activación, pero \textbf{NO afecta el valor de la constante de equilibrio} ni la posición del equilibrio.
\end{itemize}

\textbf{Respuesta Correcta: d)}

\vspace{0.5cm}

\subsection*{Pregunta 19 - 2019-1}
\textbf{Enunciado:} Base débil (B) con pH=8,8 y $C_0=0,35$ M. Calcular $K_b$.

\textbf{Solución:}
1. Calcular pOH y $[OH^-]$:
$$ \text{pOH} = 14 - \text{pH} = 14 - 8.8 = 5.2 $$
$$ [OH^-] = 10^{-5.2} \approx 6.31 \times 10^{-6} \text{ M} $$

2. Expresión de $K_b$:
$$ \mathrm{B} + \mathrm{H}_2\mathrm{O} \leftrightarrow \mathrm{BH}^+ + \mathrm{OH}^- $$
En equilibrio calculamos: $[OH^-] \approx [\mathrm{BH}^+] \approx 6.31 \times 10^{-6}$.
$[B] \approx C_0 - [OH^-] \approx 0.35$ (dado que la disociación es muy pequeña).

$$ K_b = \frac{[\mathrm{BH}^+][OH^-]}{[B]} \approx \frac{(6.31 \times 10^{-6})^2}{0.35} $$
$$ K_b \approx \frac{39.8 \times 10^{-12}}{0.35} \approx 1.137 \times 10^{-10} $$

Redondeando: $1.14 \times 10^{-10}$.

\textbf{Respuesta Correcta: b)}

\vspace{0.5cm}

\subsection*{Pregunta 20 - 2019-1}
\textbf{Enunciado:} ¿Cuál afirmación sobre fuerzas intermoleculares es FALSA?

\textbf{Solución:}
Analicemos:
\begin{itemize}
    \item[a)] \textbf{Verdadera}. Principio general de los estados de la materia.
    \item[b)] "La viscosidad es una medida de la resistencia de un líquido al momento de estar \textbf{estático}". \textbf{Falsa}. La viscosidad se define como la resistencia de un fluido a \textbf{fluir} (movimiento). Es una propiedad dinámica.
    \item[c)] \textbf{Verdadera}.
    \item[d)] \textbf{Verdadera}. El enlace de hidrógeno es una interacción dipolo-dipolo particularmente fuerte.
\end{itemize}

\textbf{Respuesta Correcta: b)}

\vspace{0.5cm}

\subsection*{Pregunta 21 - 2019-1}
\textbf{Enunciado:} Combustión de 100 moles de Butano ($\mathrm{C}_4\mathrm{H}_{10}$) con 5.000 moles de aire (21\% $\mathrm{O}_2$). Calcular \% exceso de aire.

\textbf{Solución:}
1. Reacción balanceada:
$$ \mathrm{C}_4\mathrm{H}_{10} + \frac{13}{2}\mathrm{O}_2 \rightarrow 4\mathrm{CO}_2 + 5\mathrm{H}_2\mathrm{O} $$
O: $2\mathrm{C}_4\mathrm{H}_{10} + 13\mathrm{O}_2 \rightarrow 8\mathrm{CO}_2 + 10\mathrm{H}_2\mathrm{O}$

Estequiometría: 1 mol Butano requiere 6.5 mol $\mathrm{O}_2$.

2. Oxígeno requerido (Teórico):
$$ n_{O2,teorico} = 100 \text{ mol Butano} \times 6.5 = 650 \text{ mol } \mathrm{O}_2 $$

3. Oxígeno alimentado:
Aire contiene 21\% de $\mathrm{O}_2$.
$$ n_{O2,alim} = 5000 \text{ mol aire} \times 0.21 = 1050 \text{ mol } \mathrm{O}_2 $$

4. Exceso:
$$ \text{Exceso} = n_{O2,alim} - n_{O2,teorico} = 1050 - 650 = 400 \text{ mol } \mathrm{O}_2 $$

5. Porcentaje de Exceso:
Se calcula sobre lo teórico requerido.
$$ \% \text{ Exceso} = \frac{400}{650} \times 100 \approx 61.538\% $$

Nota: El exceso de aire es el mismo porcentaje que el exceso de oxígeno (ya que el aire es proporcional al oxígeno).
Calculando con moles de aire:
Aire teórico = $650 / 0.21 \approx 3095.2$ mol.
Aire exceso = $5000 - 3095.2 = 1904.8$ mol.
$\% = (1904.8 / 3095.2) \times 100 \approx 61.5\%$.

La alternativa a) es 61.6\%.

\vspace{0.5cm}

\subsection*{Pregunta 18 - 2019-1}
\textbf{Enunciado:} Reacción $2 \mathrm{NO}(\mathrm{g})+\mathrm{O}_2(\mathrm{g}) \leftrightarrow 2 \mathrm{NO}_2(\mathrm{g})$. ¿Cuál afirmación es INCORRECTA?

\textbf{Referencia FE Handbook:}
Sección \textbf{Thermodynamics / Chemical Reaction Equilibrium} (Pág. 156).
\begin{itemize}
    \item \textbf{Pág. 156}: Principio de Le Chatelier (implícito en $\Delta G^\circ = -RT \ln K$).
    \item La constante de equilibrio $K$ solo depende de la temperatura ($\Delta G^\circ(T)$). No cambia con concentraciones o catalizadores.
\end{itemize}

\vspace{0.5cm}

\subsection*{Pregunta 19 - 2019-1}
\textbf{Enunciado:} Base débil (B) con pH=8,8 y $C_0=0,35$ M. Calcular $K_b$.

\textbf{Referencia FE Handbook:}
Sección \textbf{Acids, Bases, and pH} (Pág. 86).
\begin{itemize}
    \item \textbf{Pág. 86}: Relación pH/pOH y constantes de disociación.
    \item \textbf{Pág. 156}: Concepto general de constante de equilibrio ($K$).
\end{itemize}

\vspace{0.5cm}

\subsection*{Pregunta 20 - 2019-1}
\textbf{Enunciado:} Fuerzas intermoleculares y viscosidad.

\textbf{Referencia FE Handbook:}
\begin{itemize}
    \item \textbf{Pág. 177}: Fluid Mechanics (Definición de Viscosidad).
    \item General Chemistry: Fuerzas intermoleculares (London, Dipolo-Dipolo, Puente Hidrógeno).
\end{itemize}

\vspace{0.5cm}

\subsection*{Pregunta 21 - 2019-1}
\textbf{Enunciado:} Combustión y exceso de aire.

\textbf{Referencia FE Handbook:}
Sección \textbf{Thermodynamics / Combustion Processes} (Pág. 153).
\begin{itemize}
    \item \textbf{Pág. 153}: Definiciones de Combustión Estequiométrica, Exceso de Aire y Relación Aire-Combustible (Air-Fuel Ratio).
\end{itemize}

\vspace{0.5cm}

\section{2019-2}

\subsection*{Pregunta 18 - 2019-2}
\textbf{Enunciado:} Sobre sólidos iónicos.

\textbf{Referencia FE Handbook:}
Sección \textbf{Materials Science} (Pág. 94) o \textbf{Chemistry} (Pág. 85).
\begin{itemize}
    \item \textbf{Pág. 94}: Crystalline Structures.
    \item Definición de enlace iónico (atracción electrostática).
\end{itemize}

\vspace{0.5cm}

\subsection*{Pregunta 19 - 2019-2}
\textbf{Enunciado:} $K_p = 60$. Predicción de dirección ($Q$ vs $K$).

\textbf{Referencia FE Handbook:}
Sección \textbf{Chemistry / Thermodynamics} (Pág. 156).
\begin{itemize}
    \item \textbf{Pág. 156}: Relación entre $\Delta G$ y el cociente de reacción. Si $Q < K$, la reacción avanza hacia productos.
\end{itemize}

\vspace{0.5cm}

\subsection*{Pregunta 20 - 2019-2}
\textbf{Enunciado:} Solubilidad y pH. Efecto ion común.

\textbf{Referencia FE Handbook:}
\begin{itemize}
    \item \textbf{Pág. 86}: Ácidos y Bases (pH).
    \item \textbf{Pág. 156}: Principio de Le Chatelier aplicado a equilibrios de solubilidad ($K_{sp}$).
\end{itemize}

\vspace{0.5cm}

\subsection*{Pregunta 21 - 2019-2}
\textbf{Enunciado:} Solubilidad (Polar/No Polar).

\textbf{Referencia FE Handbook:}
Conceptos generales de Química ("Lo similar disuelve a lo similar").
\begin{itemize}
    \item \textbf{Pág. 85}: Estructura molecular y polaridad.
\end{itemize}

\vspace{0.5cm}

\vspace{0.5cm}

\vspace{0.5cm}

\section{2016-1}

\vspace{0.5cm}

\subsection*{Pregunta 09 - 2016-1}
\textbf{Enunciado:} Afirmación FALSA sobre oxidación y reducción.

\textbf{Referencia FE Handbook:}
Consultar la sección de \textbf{Electrochemistry} y \textbf{Nernst Equation}.
\begin{itemize}
    \item \textbf{Pág. 86}: Ecuación de Nernst ($E = E^\circ - \frac{RT}{nF} \ln Q$).
    \item \textbf{Pág. 86}: Definiciones de Potencial de Semicelda ($E^\circ$).
\end{itemize}

\vspace{0.5cm}

\subsection*{Pregunta 10 - 2016-1}
\textbf{Enunciado:} Balancear $\mathrm{Fe}^{2+}+\mathrm{Cr}_2 \mathrm{O}_7^{2-} \rightarrow \mathrm{Fe}^{3+}+\mathrm{Cr}^{3+}$.

\textbf{Referencia FE Handbook:}
El balanceo de ecuaciones redox es un procedimiento sistemático (método del ión-electrón).
\begin{itemize}
    \item \textbf{Pág. 86}: Se presentan potenciales de semicelda que implican reacciones de reducción balanceadas (para referencia de especies).
\end{itemize}

\vspace{0.5cm}

\subsection*{Pregunta 11 - 2016-1}
\textbf{Enunciado:} Volumen de 6.69 moles de gas a 257°C y 10.10 atm.

\textbf{Referencia FE Handbook:}
\begin{itemize}
    \item \textbf{Pág. 119 (Thermodynamics)}: Ecuación de estado de Gas Ideal.
    $$ P v = R T \quad \text{o} \quad PV = mRT \quad (\text{Notar unidades de R}) $$
    \item También referenciado implícitamente en la sección de Química (Pág. 86, constante R).
\end{itemize}

\vspace{0.5cm}

\vspace{0.5cm}

\vspace{0.5cm}

\section{2023-2}

\subsection*{Pregunta 28 - 2023-2}
\textbf{Enunciado:} Radio atómico de Plata 172 pm. Convertir a cm. ($1 \text{ pm} = 10^{-10} \text{ cm}$).

\textbf{Referencia FE Handbook:}
Sección \textbf{Units and Conversion Factors} (Pág. 1-2).
\begin{itemize}
    \item Prefijos del SI (pico = $10^{-12}$).
    \item Factores de conversión de longitud.
\end{itemize}

\vspace{0.5cm}

\subsection*{Pregunta 29 - 2023-2}
\textbf{Enunciado:} Agua hirviendo a 100°C (1 atm). ¿Afirmación FALSA?

\textbf{Referencia FE Handbook:}
Sección \textbf{Thermodynamics / Vapor-Liquid Equilibrium} (Pág. 153).
\begin{itemize}
    \item \textbf{Pág. 153}: Propiedades de saturación ($P_{sat}$ a $T_{sat}$).
    \item Definición de punto de ebullición (Presión de vapor = Presión externa).
\end{itemize}

\vspace{0.5cm}

\subsection*{Pregunta 30 - 2023-2}
\textbf{Enunciado:} $\mathrm{N}_2 \mathrm{O}_4 \leftrightarrows 2 \mathrm{NO}_2$. $K_c = 4.7 \times 10^{-3}$. Disociación 24.0\%.

\textbf{Referencia FE Handbook:}
Sección \textbf{Chemical Reaction Equilibrium} (Pág. 156).
\begin{itemize}
    \item \textbf{Pág. 156}: Definición de $K_a$ (o $K_c$ en general).
    \item Relación estequiométrica en el equilibrio.
\end{itemize}

\vspace{0.5cm}

\subsection*{Pregunta 31 - 2023-2}
\textbf{Enunciado:} Buffer pH=7.4. $n_{HA} = 0.1 \text{ mol}$. $V=200 \text{ mL}$. $K_{HA} = 2.5 \times 10^{-8}$. Calcular $[KA]$.

\textbf{Referencia FE Handbook:}
Sección \textbf{Acids, Bases, and pH} (Pág. 86).
\begin{itemize}
    \item \textbf{Pág. 86}: Constante de disociación de ácidos débiles.
    \item (Henderson-Hasselbalch no está explícita por nombre pero se deriva de $K_a$).
\end{itemize}

\vspace{0.5cm}

\subsection*{Pregunta 32 - 2023-2}
\textbf{Enunciado:} Reacción $3 \mathrm{Cu}+2 \mathrm{HNO}_3+6 \mathrm{H}^{+} \rightarrow 3 \mathrm{Cu}^{2+}+2 \mathrm{NO}+4 \mathrm{H}_2 \mathrm{O}$.

\textbf{Referencia FE Handbook:}
Sección \textbf{Electrochemistry} (Pág. 86).
\begin{itemize}
    \item \textbf{Pág. 86}: Definiciones de Oxidación (Pérdida de electrones, Agente Reductor) y Reducción (Ganancia de electrones, Agente Oxidante).
\end{itemize}

\vspace{0.5cm}

\subsection*{Pregunta 33 - 2023-2}
\textbf{Enunciado:} Electrodo de Hidrógeno Estándar (SHE).

\textbf{Referencia FE Handbook:}
Sección \textbf{Electrochemistry} (Pág. 86).
\begin{itemize}
    \item \textbf{Pág. 86}: Los potenciales de semicelda ($E^\circ$) están referidos a la reducción de $2H^+ + 2e^- \to H_2$, cuyo potencial se define como $0.00$ V bajo condiciones estándar ($1 \text{ M}, 1 \text{ atm}, Pt$).
\end{itemize}

\vspace{0.5cm}

\section{2024-2}

\subsection*{Pregunta 28 - 2024-2}
\textbf{Enunciado:} Densidad de gas. $m=0.967 \text{ kg}$. $V=1 \text{ m}^3$ a 0°C, 1 atm. Calcular en kg/L.

\textbf{Referencia FE Handbook:}
\begin{itemize}
    \item \textbf{Pág. 119}: Ecuación de Gas Ideal ($Pv = RT$).
    \item \textbf{Pág. 1-2}: Conversión de unidades (m3 a L).
\end{itemize}

\vspace{0.5cm}

\subsection*{Pregunta 29 - 2024-2}
\textbf{Enunciado:} Diagrama de fases del agua. ¿Afirmación FALSA?

\textbf{Referencia FE Handbook:}
Sección \textbf{Thermodynamics} (Pág. 155).
\begin{itemize}
    \item \textbf{Pág. 155}: Regla de las Fases de Gibbs.
    \item Concepto de Punto Triple y Punto Crítico (General Chemistry).
\end{itemize}

\vspace{0.5cm}

\subsection*{Pregunta 30 - 2024-2}
\textbf{Enunciado:} $K_c=14.8$. $Q=10$. ¿Afirmación CORRECTA?

\textbf{Referencia FE Handbook:}
Sección \textbf{Chemical Reaction Equilibrium} (Pág. 156).
\begin{itemize}
    \item \textbf{Pág. 156}: Comparación de $Q$ vs $K$ (implícito en $\Delta G = RT \ln (Q/K)$).
    \item Si $Q < K$, la reacción avanza hacia productos.
\end{itemize}

\vspace{0.5cm}

\subsection*{Pregunta 31 - 2024-2}
\textbf{Enunciado:} Mayor cantidad de iones (concentración de partículas iónicas).

\textbf{Referencia FE Handbook:}
Sección \textbf{Chemistry} (Pág. 86).
\begin{itemize}
    \item \textbf{Pág. 86}: Ácidos, Bases y Sales. Electrolitos fuertes se disocian completamente.
\end{itemize}

\vspace{0.5cm}

\subsection*{Pregunta 32 - 2024-2}
\textbf{Enunciado:} Redox $3 \mathrm{Cu}+2 \mathrm{HNO}_3+6 \mathrm{H}^{+} \rightarrow \dots$. ¿Afirmación INCORRECTA?

\textbf{Referencia FE Handbook:}
\begin{itemize}
    \item \textbf{Pág. 86}: Balanceo de ecuaciones Redox / Electroquímica.
\end{itemize}

\vspace{0.5cm}

\subsection*{Pregunta 33 - 2024-2}
\textbf{Enunciado:} Calcular K para $2 \mathrm{Ag}^{+} + \mathrm{Fe} \rightleftharpoons 2 \mathrm{Ag} + \mathrm{Fe}^{2+}$.

\textbf{Referencia FE Handbook:}
Sección \textbf{Electrochemistry} (Pág. 86).
\begin{itemize}
    \item \textbf{Pág. 86}: Potenciales de semicelda ($E^\circ$).
    \item \textbf{Pág. 86}: Ecuación que relaciona $E^\circ_{cell}$ con $K$.
    $$ \Delta G^\circ = -nFE^\circ = -RT \ln K \implies \ln K = \frac{nFE^\circ}{RT} $$
\end{itemize}

\end{document}
