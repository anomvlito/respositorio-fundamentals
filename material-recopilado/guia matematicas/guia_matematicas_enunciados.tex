\documentclass{article}
\usepackage{fullpage}
\usepackage{graphicx}
\usepackage[utf8]{inputenc}
\usepackage[T1]{fontenc}
\usepackage[spanish]{babel}
\usepackage{amssymb}
\usepackage{amsmath}
\usepackage{cancel}
\usepackage{booktabs} 
\usepackage{tikz}
\usepackage{float}
\usepackage{url}
\usetikzlibrary{arrows.meta}

%%%%% Comandos Personalizados %%%%%
\newcommand{\N}{\mathbb{N}}
\newcommand{\R}{\mathbb{R}}
\newcommand{\Q}{\mathbb{Q}}
\newcommand{\E}{\mathbb{E}}
\newcommand{\PP}{\mathbb{P}}
\newcommand{\la}{\leftarrow}
\newcommand{\ra}{\rightarrow}
\newcommand{\lra}{\leftrightarrow}
\newcommand{\Ra}{\Rightarrow}
\newcommand{\La}{\Leftarrow}
\newcommand{\LRa}{\Leftrightarrow}
\newcommand{\sub}{\subseteq}
\newcommand{\matro}{\mathcal{M}}

\newcommand{\twopartdef}[4]
{
	\left\{
		\begin{array}{ll}
			#1 &  \text{#2} \\
			#3 &  \text{#4}
		\end{array}
	\right.
}

%%%%%  Fin Comandos Personalizados %%%%%

%%%%%%%%%% MODIFICAR %%%%%%%%%%
\newcommand{\alumnos}{Solucionario Generado}
\newcommand{\departamento}{Departamento de Ingeniería Mecánica y Metalúrgica}
\newcommand{\ramo}{Matemáticas}
\newcommand{\sigla}{DIM100}
\newcommand{\titulo}{Guía de Ejercicios Matemáticas (Cálculo, EDO, Álgebra Lineal, Probabilidades)}
\newcommand{\semestre}{Recopilación}
\newcommand{\anio}{2025}
\newcommand{\med}{\frac{1}{2}}
\newcommand{\indep}{\mathcal{I}}
%%%%%%%%%% FIN MODIFICAR %%%%%%%%%%

\renewcommand{\thesubsection}{\alph{subsection}}

\begin{document}

\title{Guía de Ejercicios Matemáticas (Cálculo, EDO, Álgebra Lineal, Probabilidades)}
\maketitle

\section{2016-1}

\subsection*{Pregunta 1 - 2016-1 (Cálculo I, II y III)}
\textbf{Enunciado:}

Considere la función $f(x)=-x e^{-\frac{x^2}{2}}$.
La función posee un máximo en:

\begin{enumerate}
    \item[a)] $\left(1,-e^{-\frac{1}{2}}\right)$
    \item[b)] $\left(-1,-e^{-\frac{1}{2}}\right)$
    \item[c)] $\left(-1, e^{-\frac{1}{2}}\right)$
    \item[d)] $\left(1, e^{-\frac{1}{2}}\right)$
\end{enumerate}
\vspace{0.5cm}

\subsection*{Pregunta 2 - 2016-1 (Cálculo I, II y III)}
\textbf{Enunciado:}

¿Cuál de las siguientes series converge?

\begin{enumerate}
    \item[a)] $\sum_{n=1}^{\infty} \frac{n^3+n^2+n}{n^4+n^3+n^2+n}$
    \item[b)] $\sum_{n=0}^{\infty} \frac{\mathrm{n}^2}{2 n^3+1}$
    \item[c)] $\sum_{n=0}^{\infty} \frac{3^{\mathrm{n}}}{n!}$
    \item[d)] $\sum_{n=1}^{\infty} \frac{\ln (\mathrm{n})}{n+2}$
\end{enumerate}
\vspace{0.5cm}

\subsection*{Pregunta 5 - 2016-1 (Ecuaciones Diferenciales)}
\textbf{Enunciado:}

Considere la función $g: \mathbb{R}^2 \rightarrow \mathbb{R}$ dada por:
$$
g(x, y)=\cos (x) \cos (y)+\tan (x y)+\frac{y^2}{2}
$$

Se calcula la derivada direccional en el punto $(0, \pi)$ según la dirección unitaria $\hat{u}=(1,0)$.
¿Cuánto vale la derivada direccional descrita?

\begin{enumerate}
    \item[a)] 0
    \item[b)] $\pi$
    \item[c)] $\pi+1 / \pi$
    \item[d)] $\pi-1 / \pi$
\end{enumerate}
\vspace{0.5cm}

\subsection*{Pregunta 6 - 2016-1 (Álgebra Lineal)}
\textbf{Enunciado:}

Se tiene $A=U U^T U$ con $U \in \mathbb{R}^{n \times n}$, y donde $U^{-1}$ existe.
¿Cuál de las siguientes alternativas corresponde a una condición correcta para el cálculo del determinante de $A$ ?

\begin{enumerate}
    \item[a)] $\operatorname{Det}(A) \neq 0$
    \item[b)] $\operatorname{Det}(A)=0$
    \item[c)] $\operatorname{Det}(A) \geq 0$
    \item[d)] $\operatorname{Det}(A) \leq 0$
\end{enumerate}
\vspace{0.5cm}

\subsection*{Pregunta 7 - 2016-1 (Ecuaciones Diferenciales)}
\textbf{Enunciado:}

¿Cuál de las siguientes ecuaciones diferenciales es lineal, no homogénea y de segundo orden?

\begin{enumerate}
    \item[a)] $y^{\prime \prime}+\cos (x) y^{\prime}+x=0$
    \item[b)] $y^{\prime \prime}+3 y^{\prime}=x y$
    \item[c)] $\left(y^{\prime}\right)^2=\mathrm{e}^x$
    \item[d)] $\left(y^{\prime}\right)^2-x^2 y=0$
\end{enumerate}
\vspace{0.5cm}

\subsection*{Pregunta 9 - 2016-1 (Álgebra Lineal)}
\textbf{Enunciado:}

Considere las siguientes afirmaciones con respecto a las matrices simétricas:

I. La diferencia de matrices simétricas es una matriz simétrica.

II. Si $A$ y $B$ son simétricas y $A B=B A$, entonces $A B$ es una matriz simétrica.

III. Todas las matrices simétricas de $n \times n$ tienen $n$ valores propios reales distintos.

De las afirmaciones anteriores, ¿cuáles son CORRECTAS?

\begin{enumerate}
    \item[a)] Sólo I y II
    \item[b)] Sólo II y III
    \item[c)] Sólo I y III
    \item[d)] Todas son correctas.
\end{enumerate}
\vspace{0.5cm}

\subsection*{Pregunta 17 - 2016-1 (Probabilidad y Estadística)}
\textbf{Enunciado:}

Suponga que se cuenta con un dado de seis caras mal construido, que tiene tres caras con el número 6 , dos caras con el número 4 y una cara con el número 5 .

Si se lanza dos veces este dado de manera independiente, ¿cuál es el valor más cercano a la probabilidad de que la suma de los dos números obtenidos sea 10 ?

\begin{enumerate}
    \item[a)] 0,1944
    \item[b)] 0,2777
    \item[c)] 0,3333
    \item[d)] 0,3611
\end{enumerate}
\vspace{0.5cm}

\subsection*{Pregunta 18 - 2016-1 (Probabilidad y Estadística)}
\textbf{Enunciado:}

La siguiente función representa la función de densidad de una variable aleatoria $X$, llamada ``exponencial trasladada'',
$$
f(x)=2 e^{-2(x-1)}, \quad x>1
$$

¿Cuál de los siguientes valores equivale a la varianza de $X$ ?

\begin{enumerate}
    \item[a)] $1 / 4$
    \item[b)] $5 / 4$
    \item[c)] $6 / 4$
    \item[d)] $9 / 4$
\end{enumerate}
\vspace{0.5cm}

\section{2016-2}

\subsection*{Pregunta 1 - 2016-2 (Cálculo I, II y III)}
\textbf{Enunciado:}

Considere la función $f(x)=\frac{\sqrt{1-x^2+x^4 / 2}}{x^2+1}$

La función posee un máximo en:

\begin{enumerate}
    \item[a)] $(0,1)$
    \item[b)] $\left(\sqrt{\frac{3}{2}}, \frac{1}{\sqrt{10}}\right)$
    \item[c)] $\left(-\sqrt{\frac{3}{2}}, \frac{1}{\sqrt{10}}\right)$
    \item[d)] $\left(1, \frac{1}{2 \sqrt{2}}\right)$
\end{enumerate}
\vspace{0.5cm}

\subsection*{Pregunta 2 - 2016-2 (Cálculo I, II y III)}
\textbf{Enunciado:}

Sea $0 < a < b < \infty$. ¿Cuál es el mayor intervalo al que puede pertenecer $p$ para que la siguiente integral converja?
$$ \int_a^b \frac{2 + \sin(x)}{(x - a)^p} dx $$

\begin{enumerate}
    \item[a)] $(-1,1)$
    \item[b)] $(-\infty, -1)$
    \item[c)] $(1, \infty)$
    \item[d)] $(-\infty, 1)$
\end{enumerate}
\vspace{0.5cm}

\subsection*{Pregunta 3 - 2016-2 (Cálculo I, II y III)}
\textbf{Enunciado:}

Sea $f(x, y)=x^y$.

La derivada direccional en el punto (1,2), en la dirección $\hat{u}=(1,1)$, es:

\begin{enumerate}
    \item[a)] 2
    \item[b)] 0
    \item[c)] $\sqrt{2}$
    \item[d)] 1
\end{enumerate}
\vspace{0.5cm}

\subsection*{Pregunta 4 - 2016-2 (Ecuaciones Diferenciales)}
\textbf{Enunciado:}

Sea el sistema de ecuaciones diferenciales
$$
\begin{aligned}
& \frac{d x}{d t}=3 x(t)-2 y(t) \\
& \frac{d y}{d t}=2 x(t)-2 y(t)
\end{aligned}
$$
¿Cuál es la solución a dicho sistema con $x(0)=1$ y $y(0)=5$?

\begin{enumerate}
    \item[a)] $\left\{\begin{array}{c}x(t)=-2 e^{2 t}+3 e^{-t} \\ y(t)=-e^{2 t}+6 e^{-t}\end{array}\right.$
    \item[b)] $\left\{\begin{array}{c}x(t)=-2 e^{-2 t}+3 e^t \\ y(t)=-e^{-2 t}+6 e^t\end{array}\right.$
    \item[c)] $\left\{\begin{array}{l}x(t)=3 e^{2 t}-2 e^{-t} \\ y(t)=6 e^{2 t}-e^{-t}\end{array}\right.$
    \item[d)] $\left\{\begin{array}{c}x(t)=e^{2 t} \\ y(t)=-e^{2 t}+6 e^{-t}\end{array}\right.$
\end{enumerate}
\vspace{0.5cm}

\subsection*{Pregunta 6 - 2016-2 (Álgebra Lineal)}
\textbf{Enunciado:}

Se tienen las matrices $C \in M_{n n}$ (matriz de $n$ filas y $n$ columnas). Se define la matriz $N=C-I_n$ (con $I_n$ la matriz identidad de $n$ filas y $n$ columnas).

Si se sabe que $N^n=0_{n n}$ (matriz de ceros), ¿cuál de las siguientes alternativas corresponde a la matriz $C^{-1}$ ?

\begin{enumerate}
    \item[a)] $C^{-1}=I_n-N$
    \item[b)] $C^{-1}=I_n-N+N^2-N^3+\cdots+(-1)^{n-1} N^{n-1}$
    \item[c)] $C^{-1}=I_n+N-N^2+N^3+\cdots+(1)^{n-1} N^{n-1}$
    \item[d)] $C^{-1}=I_n-N+N^2-N^3+\cdots+(-1)^{2 n-1} N^{2 n-1}$
\end{enumerate}
\vspace{0.5cm}

\subsection*{Pregunta 19 - 2016-2 (Probabilidad y Estadística)}
\textbf{Enunciado:}

Se registraron los siguientes datos pareados $\left(x_i, y_i\right)$ y se desea ajustar un modelo lineal de regresión simple. En particular, explicar la media de los datos $y_i$ en función de $x_i$. Los datos y sus operaciones básicas se resumen en la siguiente tabla.

\begin{center}
\begin{tabular}{|c|c|c|c|c|c|}
\hline
\textbf{Dato} & \boldmath$x_i$\unboldmath & \boldmath$y_i$\unboldmath & \boldmath$x_i^2$\unboldmath & \boldmath$y_i^2$\unboldmath & \boldmath$x_i y_i$\unboldmath \\ \hline
1 & 6,35 & 32,03 & 40,32 & 1025,92 & 203,39 \\
2 & 5,53 & 31,04 & 30,58 & 963,48 & 171,65 \\
3 & 2,21 & 21,1 & 4,88 & 445,21 & 46,63 \\
4 & 2,12 & 16,27 & 4,49 & 264,71 & 34,49 \\
5 & 4,9 & 27,29 & 24,01 & 744,74 & 133,72 \\
6 & 5,36 & 32,68 & 28,73 & 1067,98 & 175,16 \\ \hline
\end{tabular}
\end{center}

¿Cuál de las siguientes es la forma más cercana a la recta de regresión ajustada por los datos?

\begin{enumerate}
    \item[a)] $y=1,36+5,75 x$
    \item[b)] $y=11,15+3,53 x$
    \item[c)] $y=-2,45+0,26 x$
    \item[d)] $y=5,75+3,53 x$
\end{enumerate}
\vspace{0.5cm}

\subsection*{Pregunta 21 - 2016-2 (Probabilidad y Estadística)}
\textbf{Enunciado:}

En una línea de ensamblaje de automóviles se utilizan al menos ocho cajas de tornillos al día. La persona encargada de calidad abre la primera caja y selecciona dos tornillos al azar, y si al menos uno de ellos se encuentra dañado, entonces rechazará la caja entera. Luego repite este procedimiento de revisión en todas las cajas.

Según la empresa fabricante de tornillos, sólo un $4 \%$ de los tornillos de cada caja resultan dañados. Asuma que cada caja contiene varios miles de tornillos, y que cada extracción de tornillo es independiente.

De las siguientes alternativas, ¿cuál es el valor más cercano a la probabilidad de que la persona encargada de calidad rechace a lo más 1 caja de las 8 revisadas?

\begin{enumerate}
    \item[a)] 0,3541
    \item[b)] 0,4796
    \item[c)] 0,8746
    \item[d)] 0,9619
\end{enumerate}
\vspace{0.5cm}

\subsection*{Pregunta 22 - 2016-2 (Probabilidad y Estadística)}
\textbf{Enunciado:}

Suponga que una moneda se lanza 1000 veces. De ellas, 575 resultaron ser cara y 425 , sello. Se intenta dar evidencia estadística de que esta moneda no es equilibrada (es decir, rechazar la hipótesis $p=0,5)$.
¿Con qué nivel de significancia se puede concluir que la moneda no es equilibrada, dada esta muestra?

\begin{enumerate}
    \item[a)] Con $10 \%$, pero no con $5 \%$
    \item[b)] Con $5 \%$, pero no con $2 \%$
    \item[c)] Con $2 \%$, pero no con $1 \%$
    \item[d)] Con $1 \%$ sí
\end{enumerate}
\vspace{0.5cm}

\subsection*{Pregunta 23 - 2016-2 (Probabilidad y Estadística)}
\textbf{Enunciado:}

Considere una máquina electrónica de Transantiago ubicada en una calle, que carga la tarjeta ``Bip!'' en exactamente 30 segundos. Suponga que en cierta hora del día, los usuarios de la máquina llegan a ella para utilizarla (o hacer fila), siguiendo un proceso de Poisson, con una tasa media de llegada de 1 usuario cada dos minutos.

Si una persona A llega a la máquina sin fila y comienza a utilizarla, ¿cuál es la probabilidad de que llegue otra persona B a la máquina antes de que A termine de operarla?

\begin{enumerate}
    \item[a)] 0,2212
    \item[b)] 0,3935
    \item[c)] 0,6321
    \item[d)] 0,8647
\end{enumerate}
\vspace{0.5cm}

\section{2017-1}

\subsection*{Pregunta 1 - 2017-1 (Cálculo I, II y III)}
\textbf{Enunciado:}

Considere la función $f(x)=\frac{1}{a x^2+b x+c}$ y sean $x_1$ y $x_2$ las raíces del polinomio $a x^2+b x+c$ (con $x_1 \neq x_2$ ).

Una primitiva de la función es:

\begin{enumerate}
    \item[a)] $a\left(x_1-x_2\right) \ln \left|\frac{x-x_1}{x-x_2}\right|+C$
    \item[b)] $a\left(x_1-x_2\right) \mathrm{e}^{\left(\frac{x-x_1}{x-x_2}\right)}+C$
    \item[c)] $\frac{1}{a\left(x_1-x_2\right)} \tan ^{-1}\left(\frac{x-x_1}{x-x_2}\right)+C$
    \item[d)] $\frac{1}{a\left(x_1-x_2\right)} \ln \left|\frac{x-x_1}{x-x_2}\right|+C$
\end{enumerate}
\vspace{0.5cm}

\subsection*{Pregunta 2 - 2017-1 (Cálculo I, II y III)}
\textbf{Enunciado:}

¿Cuál de las siguientes series converge?

\begin{enumerate}
    \item[a)] $\sum_{n=0}^{\infty} \frac{(n!)^2}{(2 n)!}$
    \item[b)] $\sum_{n=0}^{\infty} \frac{1}{4 n+1}$
    \item[c)] $\sum_{n=1}^{\infty} \frac{\ln (\mathrm{n})}{n+2}$
    \item[d)] $\sum_{n=2}^{\infty} \frac{n^3+4 n}{n^4-8}$
\end{enumerate}
\vspace{0.5cm}

\subsection*{Pregunta 3 - 2017-1 (Cálculo I, II y III)}
\textbf{Enunciado:}

El sólido $\Omega \in \mathbb{R}^3$ se define por el volumen contenido sobre la superficie $z=\sqrt{3\left(x^2+y^2\right)}$ y bajo la superficie $x^2+y^2+z^2=4$.

El volumen de $\Omega$ es:

\begin{enumerate}
    \item[a)] $\frac{16}{3}\left(1-\frac{1}{\sqrt{2}}\right) \pi$
    \item[b)] $\frac{16}{3} \pi$
    \item[c)] $\frac{8}{3}\left(1-\frac{\sqrt{3}}{2}\right) \pi$
    \item[d)] $\frac{16}{3}\left(1-\frac{\sqrt{3}}{2}\right) \pi$
\end{enumerate}
\vspace{0.5cm}

\subsection*{Pregunta 4 - 2017-1 (Ecuaciones Diferenciales)}
\textbf{Enunciado:}

Sea la ecuación diferencial de segundo orden $y^{\prime \prime}-2 y^{\prime}+2 y=0$.

La solución a dicha ecuación con $x(0)=1$ y $x^{\prime}(0)=2$ es:

\begin{enumerate}
    \item[a)] $\cos (x)+\sin (x)$
    \item[b)] $e^x(\cos (x)-\sin (x))$
    \item[c)] $e^x(\cos (x)+\sin (x))$
    \item[d)] $\cos (x)-\sin (x)$
\end{enumerate}
\vspace{0.5cm}

\subsection*{Pregunta 5 - 2017-1 (Álgebra Lineal)}
\textbf{Enunciado:}

Se define el plano $\Pi$ como:
$$
x-2 y+3 z=12
$$
$Y$ se define la recta $L$ como:
$$
\left(\begin{array}{c}
1 \\
1 \\
-2
\end{array}\right)+t\left(\begin{array}{l}
2 \\
b \\
1
\end{array}\right)
$$
¿Cuál de las siguientes alternativas corresponde a la condición que debe cumplir el parámetro $b$ para que $\Pi \cap \mathrm{L}$ sea vacío?

\begin{enumerate}
    \item[a)] $b \geq 5 / 2$
    \item[b)] $b \leq 5 / 2$
    \item[c)] $b=5 / 2$
    \item[d)] no existe valor de $b$ que cumpla con lo solicitado.
\end{enumerate}
\vspace{0.5cm}

\subsection*{Pregunta 21 - 2017-1 (Probabilidad y Estadística)}
\textbf{Enunciado:}

Según un estudio, la probabilidad de que un neumático desgastado de automóvil sufra un pinchazo en un día cualquiera es de $5 \%$ si se utiliza sólo en caminos de asfalto, y de $20 \%$ si se utiliza en caminos de tierra y de asfalto. El $83 \%$ de los automóviles con un neumático desgastado circula únicamente en caminos de asfalto, mientras que el $17 \%$ restante utiliza también caminos de tierra.

Suponga que al final de un día en una autopista asfaltada se encontró un automóvil con un neumático desgastado, pero no estaba pinchado.
¿Cuál es el valor más cercano de la probabilidad de que ese automóvil haya circulado por caminos de tierra ese día?

\begin{enumerate}
    \item[a)] 0,1471
    \item[b)] 0,1700
    \item[c)] 0,4503
    \item[d)] 0,5497
\end{enumerate}
\vspace{0.5cm}

\subsection*{Pregunta 22 - 2017-1 (Probabilidad y Estadística)}
\textbf{Enunciado:}

Suponga que el porcentaje de sulfato en unas soluciones preparadas en un experimento se modelan como una variable con distribución beta, con la siguiente densidad, con $\alpha > 0$,
$$ f(x) = \alpha x^{\alpha-1} , \quad 0 < x < 1 $$
Se midió el porcentaje en $n$ soluciones preparadas con el mismo procedimiento, formando las mediciones $x_1, \ldots, x_n$.

¿Cuál de estas alternativas corresponde a la expresión del estimador de máxima verosimilitud del parámetro $\alpha$?

\begin{enumerate}
    \item[a)] $n / \left(\sum_{i=1}^n \log x_i\right)$
    \item[b)] $-n / \left(\sum_{i=1}^n \log x_i\right)$
    \item[c)] $n / \sum_{i=1}^n x_i$
    \item[d)] $\left(\sum_{i=1}^n \log x_i\right) / n$
\end{enumerate}
\vspace{0.5cm}

\subsection*{Pregunta 23 - 2017-1 (Probabilidad y Estadística)}
\textbf{Enunciado:}

Una familia de padre, madre y dos hijos decide un largo viaje en su automóvil, pero quieren revisar el peso de la maleta que llevará cada uno. Suponga que el peso de cada maleta es una variable aleatoria con distribución normal. Los pesos de las maletas del padre y de la madre tienen una media de 32 kg , y una desviación estándar de $4,2 \mathrm{~kg}$. Los pesos de las maletas de cada hijo tienen media 26 kg y una desviación estándar de $5,7 \mathrm{~kg}$. Asuma que el peso de cada maleta es independiente de las demás.

De las siguientes alternativas, ¿cuál es el valor más cercano de la probabilidad de que el peso total de las cuatro maletas juntas no supere los 126 kg ?

\begin{enumerate}
    \item[a)] 0,3085
    \item[b)] 0,6915
    \item[c)] 0,7580
    \item[d)] 0,8413
\end{enumerate}
\vspace{0.5cm}

\subsection*{Pregunta 24 - 2017-1 (Probabilidad y Estadística)}
\textbf{Enunciado:}

Se desea calcular un intervalo de confianza para la media poblacional de un fenómeno con distribución normal. Se asume que se tiene una muestra de tamaño $n$, y que la varianza poblacional es conocida e igual a $\sigma^2$. Si la muestra tiene media $\bar{x}$. La fórmula conocida para un intervalo de $(1-\alpha) 100 \%$ de confianza es,
$$
\left[\bar{x}-z_{1-\alpha / 2} \frac{\sigma}{\sqrt{n}} ; \bar{x}+z_{1-\alpha / 2} \frac{\sigma}{\sqrt{n}}\right]
$$
(los $z_b$ se denotan como el cuantil de una distribución normal estándar, de modo que el área a la izquierda de este valor sea $b$ )
Este intervalo se aplicó para una muestra con distribución normal y varianza conocida. El intervalo de $95 \%$ de confianza para la media $\mu$ resultó ser,
$$
[1,34 ; 2,81]
$$

¿Cuál de estas alternativas es correcta?

\begin{enumerate}
    \item[a)] La media poblacional $\mu$ se ubica entre 1,34 y 2,81, inclusive.
    \item[b)] Aproximadamente el $95 \%$ de los intervalos de $95 \%$ de confianza que se construyan van a contener al verdadero valor de $\mu$.
    \item[c)] Existe una probabilidad de $95 \%$ de que $\mu$ se ubique entre 1,34 y 2,81.
    \item[d)] Con la misma muestra, mientras más confianza, más corto será el intervalo.
\end{enumerate}
\vspace{0.5cm}

\section{2017-2}

\subsection*{Pregunta 1 - 2017-2 (Cálculo I, II y III)}
\textbf{Enunciado:}

Aún no hay solución propuesta
\vspace{0.5cm}

\subsection*{Pregunta 2 - 2017-2 (Cálculo I, II y III)}
\textbf{Enunciado:}

Una ecuación cartesiana del plano que pasa por el punto $A(7,-4,2)$ y la recta:
$$
\frac{x-2}{5}=\frac{y+5}{1}=\frac{z+1}{3}
$$
está dada por:

\begin{enumerate}
    \item[a)] $7 x-4 y+2 z=0$
    \item[b)] $5 x+y+3 z=0$
    \item[c)] $2 x-5 y-z=0$
    \item[d)] El plano no se encuentra determinado
\end{enumerate}
\vspace{0.5cm}

\subsection*{Pregunta 3 - 2017-2 (Cálculo I, II y III)}
\textbf{Enunciado:}

El sólido $\Omega \in \mathbb{R}^3$ se define por el volumen contenido entre las superficies $x^2+y^2+z^2=1, x^2+y^2=1, \mathrm{z}=1$ y los planos coordenados $x=0, y=0$ y $z=0$.

El volumen de $\Omega$ es:

\begin{enumerate}
    \item[a)] $\frac{1}{16} \pi$
    \item[b)] $\frac{1}{12} \pi$
    \item[c)] $\frac{3}{16} \pi$
    \item[d)] $\frac{1}{4} \pi$
\end{enumerate}
\vspace{0.5cm}

\subsection*{Pregunta 4 - 2017-2 (Ecuaciones Diferenciales)}
\textbf{Enunciado:}

Sea la ecuación diferencial del modelo poblacional de Verhulst dada por:

$$
d p-r p\left(1-\frac{p}{K}\right) d t=0 .
$$

La solución a dicha ecuación con $p(0)=p_0$ es:

\begin{enumerate}
    \item[a)] $p_0\left(1-K\left(1-e^{r t}\right)\right)$
    \item[b)] $\frac{K p_0}{\left(K-p_0\right) e^{-r t}+p_0}$
    \item[c)] $p_0 e^{r t}$
    \item[d)] $p_0$
\end{enumerate}
\vspace{0.5cm}

\subsection*{Pregunta 5 - 2017-2 (Álgebra Lineal)}
\textbf{Enunciado:}

Se define el plano $\Pi$ como:
$$
x-2 y+3 z=12
$$
$Y$ se define la recta $L$ como:
$$
\left(\begin{array}{c}
1 \\
1 \\
-2
\end{array}\right)+t\left(\begin{array}{l}
2 \\
b \\
1
\end{array}\right)
$$
¿Cuál de las siguientes alternativas corresponde a la condición que debe cumplir el parámetro $b$ para que $\Pi \cap \mathrm{L}$ sea vacío?

\begin{enumerate}
    \item[a)] $b \geq 5 / 2$
    \item[b)] $b \leq 5 / 2$
    \item[c)] $b=5 / 2$
    \item[d)] no existe valor de $b$ que cumpla con lo solicitado.
\end{enumerate}
\vspace{0.5cm}

\subsection*{Pregunta 21 - 2017-2 (Probabilidad y Estadística)}
\textbf{Enunciado:}

Un estudio meteorológico de una ciudad indicó que, de los días del año que presentan lluvia, un $13 \%$ de ellos va acompañado de fuertes vientos. Por otra parte, llueve un $26 \%$ de los días del año.

El estudio además registró fuertes vientos en $48 \%$ de los días del año.
¿Cuál de las alternativas es más cercana a la probabilidad de que en un día cualquiera haya fuertes vientos, pero no llueva?

\begin{enumerate}
    \item[a)] $35,00 \%$
    \item[b)] $44,62 \%$
    \item[c)] $48,00 \%$
    \item[d)] $60,30 \%$
\end{enumerate}
\vspace{0.5cm}

\subsection*{Pregunta 22 - 2017-2 (Probabilidad y Estadística)}
\textbf{Enunciado:}

Suponga que un camión de una marca de bebidas transporta diariamente $X$ miles de botellas de 5 litros cada una, e $Y$ miles de botellas de un litro cada una. Ambas cantidades $X$ e $Y$ se modelan como variables aleatorias independientes con distribución normal con media 2 y desviación estándar 0,8 (en miles de botellas).
¿Cuál es el valor más cercano a la probabilidad de que el camión transporte más de 10 mil litros en un día determinado?

\begin{enumerate}
    \item[a)] $15,87 \%$
    \item[b)] $30,85 \%$
    \item[c)] $69,15 \%$
    \item[d)] $84,13 \%$
\end{enumerate}
\vspace{0.5cm}

\subsection*{Pregunta 23 - 2017-2 (Probabilidad y Estadística)}
\textbf{Enunciado:}

En una universidad se desea hacer un estudio acerca de cuántos alumnos toman apuntes mediante su propio computador o tablet (u otro artefacto similar), respecto del total de alumnos. Preliminarmente se encuestó a 150 alumnos, de los cuales 62 afirman tomar apuntes en clase por medio de un dispositivo electrónico.

Utilizando esta muestra, ¿cuál de las siguientes alternativas representa aproximadamente un intervalo de $98 \%$ de confianza de dicha proporción? (intente utilizar precisión de 3 decimales)

\begin{enumerate}
    \item[a)] $[0,320 ; 0,506]$
    \item[b)] $[0,331 ; 0,495]$
    \item[c)] $[0,347 ; 0,479]$
    \item[d)] $[0,409 ; 0,417]$
\end{enumerate}
\vspace{0.5cm}

\subsection*{Pregunta 24 - 2017-2 (Probabilidad y Estadística)}
\textbf{Enunciado:}

En un conjunto de datos, se ajustó un modelo de regresión lineal que relaciona el ingreso familiar $Y$ (en miles de pesos) con respecto a la cantidad de integrantes de la familia que trabajan $X$. Los datos se muestran en la siguiente tabla

\begin{center}
\begin{tabular}{|c|c|c|c|c|c|c|c|}
\hline
\multicolumn{2}{|c|}{\textbf{Datos 1 a 7}} & \multicolumn{2}{c|}{\textbf{Datos 8 a 14}} & \multicolumn{2}{c|}{\textbf{Datos 15 a 21}} & \multicolumn{2}{c|}{\textbf{Datos 22 a 27}} \\ \hline
$x_i$ & $y_i$ & $x_i$ & $y_i$ & $x_i$ & $y_i$ & $x_i$ & $y_i$ \\ \hline
4 & 644 & 1 & 398 & 6 & 1.638 & 3 & 1.022 \\
2 & 477 & 3 & 953 & 1 & 314 & 5 & 1.194 \\
2 & 496 & 1 & 114 & 1 & 180 & 6 & 1.513 \\
3 & 902 & 6 & 1.721 & 4 & 1.107 & 2 & 761 \\
1 & 248 & 2 & 930 & 3 & 1.051 & 4 & 1.042 \\
1 & 426 & 2 & 447 & 3 & 1.184 & 6 & 1.642 \\
5 & 1.385 & 2 & 707 & 5 & 1.336 & \multicolumn{2}{c|}{} \\ \hline
\end{tabular}
\end{center}

\begin{center}
\begin{tabular}{|c|c|}
\hline
\multicolumn{2}{|c|}{\textbf{Resumen datos}} \\ \hline
$\Sigma x$ & 84 \\
$\Sigma y$ & 23.832 \\
$\Sigma x^2$ & 342 \\
$\Sigma y^2$ & 16.920.278 \\
$\Sigma x y$ & 94.483 \\
$n$ & 27 \\ \hline
\end{tabular}
\end{center}

Dado el modelo de regresión, ¿cuál de los siguientes valores se aproxima más a la predicción para el ingreso de una familia de la cual trabajan 4 personas?

\begin{enumerate}
    \item[a)] 712
    \item[b)] 931
    \item[c)] 1.008
    \item[d)] 1.107
\end{enumerate}
\vspace{0.5cm}

\section{2018-1}

\subsection*{Pregunta 1 - 2018-1 (Cálculo I, II y III)}
\textbf{Enunciado:}

Considere la función $f(x)=\frac{1}{x^{1 / 5}+2}$. Una primitiva de la función es:

\begin{enumerate}
    \item[a)] $\ln \left|x^{\frac{1}{5}}+2\right|+C$
    \item[b)] $\frac{5}{4} x^{\frac{4}{5}}-\frac{10}{3} x^{\frac{3}{5}}+10 x^{\frac{2}{5}}-40 x^{\frac{1}{5}}+80 \ln \left|x^{\frac{1}{5}}+2\right|+C$
    \item[c)] $\frac{1}{2} \sqrt{2} \arctan \left(\frac{1}{2} \sqrt{2} x^{\frac{1}{10}}\right)+C$
    \item[d)] $\frac{1}{2} \sqrt{2} \arctan \left(\frac{1}{2} \sqrt{2} x^{\frac{1}{5}}\right)+C$
\end{enumerate}
\vspace{0.5cm}

\subsection*{Pregunta 2 - 2018-1 (Cálculo I, II y III)}
\textbf{Enunciado:}

Considere las funciones $f(x)=\ln (x)$ y $g(x)=1-x$. El área de la región formada por las curvas $y=f(x)$ e $y=g(x)$, y el eje $y=2$ es:

\begin{enumerate}
    \item[a)] $2 \ln (2)-2$
    \item[b)] $\frac{1}{2} e^4$
    \item[c)] $2-2 \ln (2)$
    \item[d)] $e^2-1$
\end{enumerate}
\vspace{0.5cm}

\subsection*{Pregunta 3 - 2018-1 (Cálculo I, II y III)}
\textbf{Enunciado:}

Sea $f(x, y)=\sin \left(\sqrt{1+\ln ^2(x y)}\right)$
La derivada direccional en el punto $=\left(2, \frac{1}{2}\right)$, en la dirección unitaria $\theta=\frac{\pi}{2}$ (coordenadas polares), es:

\begin{enumerate}
    \item[a)] $2 \sin (1)$
    \item[b)] $\frac{1}{2} \sin (1)$
    \item[c)] 0
    \item[d)] $\frac{1}{2} \cos (1)$
\end{enumerate}
\vspace{0.5cm}

\subsection*{Pregunta 4 - 2018-1 (Ecuaciones Diferenciales)}
\textbf{Enunciado:}

Sea el sistema de ecuaciones diferenciales
$$
\begin{gathered}
\frac{d x}{d t}=3 x(t)-5 y(t) \\
\frac{d y}{d t}=x(t)-y(t)
\end{gathered}
$$

La solución a dicho sistema con $x(0)=3$ y $y(0)=1$ es:

\begin{enumerate}
    \item[a)] $\left\{\begin{array}{c}x(t)=e^{-t}(3 \cos (t)+\sin (t)) \\ y(t)=e^{-t}(\cos (t)+\sin (t))\end{array}\right.$
    \item[b)] $\left\{\begin{array}{c}x(t)=e^t(3 \cos (t)+\sin (t)) \\ y(t)=e^t(\cos (t)+\sin (t))\end{array}\right.$
    \item[c)] $\left\{\begin{array}{c}x(t)=e^{-t}(3 \cos (t)-\sin (t)) \\ y(t)=e^{-t}(\cos (t)-\sin (t))\end{array}\right.$
    \item[d)] $\left\{\begin{array}{c}x(t)=e^t(3 \cos (t)-\sin (t)) \\ y(t)=e^t(\cos (t)-\sin (t))\end{array}\right.$
\end{enumerate}
\vspace{0.5cm}

\subsection*{Pregunta 5 - 2018-1 (Álgebra Lineal)}
\textbf{Enunciado:}

Sea $X$ una matriz $3 \times 3$, y las siguientes tres matrices.
$$
A=\left[\begin{array}{lll}
0 & 1 & 0 \\
1 & 0 & 0 \\
0 & 0 & 1
\end{array}\right], \quad B=\left[\begin{array}{lll}
1 & 0 & 0 \\
0 & 2 & 0 \\
0 & 0 & 1
\end{array}\right], \quad C=\left[\begin{array}{lll}
0 & 1 & 0 \\
1 & 2 & 0 \\
0 & 0 & 1
\end{array}\right]
$$

Considere las matrices $A X, B X$ y $C X$, ¿cuál de las siguientes alternativas es generalmente FALSA?

\begin{enumerate}
    \item[a)] La matriz $A X$ es la matriz $X$ pero con las filas 1 y 2 intercambiadas
    \item[b)] La matriz $B X$ es la matriz $X$ con su segunda fila multiplicada por 2
    \item[c)] La matriz $C X$ es la matriz $X$ con su fila 1 intercambiada con 2 veces su fila 2
    \item[d)] Las matrices $A, B$ y $C$ son invertibles.
\end{enumerate}
\vspace{0.5cm}

\subsection*{Pregunta 21 - 2018-1 (Probabilidad y Estadística)}
\textbf{Enunciado:}

Un vino de marca UVA está destinado a comercializarse en ciertos puntos de comercio. Un $68 \%$ de ellos son botillerías, y el resto son supermercados. Un estudio de mercado determinó que el vino UVA se encuentra sólo en un $14 \%$ de los supermercados destinados, y en $38 \%$ de las botillerías asignadas.

Con esta información, si se escoge uno de los supermercados destinados, ¿cuál es la probabilidad de que no haya vino marca UVA?

\begin{enumerate}
    \item[a)] $6,43 \%$
    \item[b)] $27,52 \%$
    \item[c)] $69,68 \%$
    \item[d)] $86,00 \%$
\end{enumerate}
\vspace{0.5cm}

\subsection*{Pregunta 22 - 2018-1 (Probabilidad y Estadística)}
\textbf{Enunciado:}

Un computador debe ejecutar dos rutinas: Programa A y B. Durante el desarrollo de los programas, las dos rutinas demoran cada una un tiempo aleatorio con distribución exponencial con media 26 segundos. Ahora, la rutina B sólo comienza una vez terminado el programa A. Es de interés monitorear que el computador no demore más de un minuto en total (la suma de ambos tiempos de ejecución).

Suponga que, en una de las ejecuciones, el computador tomó 28,2 segundos en completar el programa A. ¿Cuál es el valor más cercano a la probabilidad de que el computador alcance a completar el programa B antes de que se cumpla el total de un minuto?

\begin{enumerate}
    \item[a)] $29,4 \%$
    \item[b)] $66,2 \%$
    \item[c)] $70,6 \%$
    \item[d)] $90,0 \%$
\end{enumerate}
\vspace{0.5cm}

\subsection*{Pregunta 23 - 2018-1 (Probabilidad y Estadística)}
\textbf{Enunciado:}

Durante una semana de entrenamiento, se ha medido 56 veces el tiempo que un nadador toma en la carrera de 100 metros nado libre. Se sabe que el tiempo medio que toma para esta carrera es de 63 segundos, pero la varianza $\sigma^2$ es desconocida. Suponga que los tiempos tienen distribución normal, y son independientes entre sí. La muestra obtenida $\left(t_1, t_2, \ldots, t_{56}\right)$ se resume en los siguientes estadísticos,
$$
\sum_{i=1}^{56} t_i=3530,3 \quad \sum_{i=1}^{56} t_i^2=222.779,1
$$

Utilizando la información, ¿cuál de las siguientes alternativas es más cercana a la estimación de momentos de $\sigma^2$ ?

\begin{enumerate}
    \item[a)] 3,55
    \item[b)] 3,95
    \item[c)] 4,09
    \item[d)] 9,20
\end{enumerate}
\vspace{0.5cm}

\subsection*{Pregunta 24 - 2018-1 (Probabilidad y Estadística)}
\textbf{Enunciado:}

Se desea ajustar una recta de regresión lineal simple, por medio del método de mínimos cuadrados, de la media de una variable respuesta $(Y)$, en función de una variable predictora $(X)$. Se cuenta con 8 datos, y se muestran en la tabla junto con otros cálculos.

\begin{center}
\begin{tabular}{|c|c|c|c|c|c|}
\hline
\boldmath$i$\unboldmath & \boldmath$x_i$\unboldmath & \boldmath$y_i$\unboldmath & \boldmath$x_i^2$\unboldmath & \boldmath$y_i^2$\unboldmath & \boldmath$x_i y_i$\unboldmath \\ \hline
1 & 4,7 & 0,9 & 22,09 & 0,81 & 4,23 \\
2 & 3,5 & 0,5 & 12,25 & 0,25 & 1,75 \\
3 & 2,7 & 0,7 & 7,29 & 0,49 & 1,89 \\
4 & 1,6 & -0,2 & 2,56 & 0,04 & -0,32 \\
5 & 1,5 & 0,1 & 2,25 & 0,01 & 0,15 \\
6 & 2,8 & 0,1 & 7,84 & 0,01 & 0,28 \\
7 & 2,6 & 0,5 & 6,76 & 0,25 & 1,30 \\
8 & 1,4 & 0,0 & 1,96 & 0,00 & 0,00 \\ \hline
\end{tabular}
\end{center}

¿Cuál de las siguientes alternativas es más cercana a la estimación de la pendiente $\hat{\beta}$ ?

\begin{enumerate}
    \item[a)] $\hat{\beta}=0,121$
    \item[b)] $\hat{\beta}=0,147$
    \item[c)] $\hat{\beta}=0,282$
    \item[d)] $\hat{\beta}=0,443$
\end{enumerate}
\vspace{0.5cm}

\section{2018-2}

\subsection*{Pregunta 1 - 2018-2 (Cálculo I, II y III)}
\textbf{Enunciado:}

Considere la función $f(x)=\frac{a x^2+b x+c}{x+d}\left(\operatorname{con} c \neq b d-a d^2\right)$. Las asíntotas de la función son:

\begin{enumerate}
    \item[a)] Asíntota vertical en $x=-d y$ asíntota oblicua con ecuación $y=a x-b$
    \item[b)] Asíntota vertical en $x=d$ y asíntota oblicua con ecuación $y=a x-b$
    \item[c)] Asíntota vertical en $x=-d$ y asíntota oblicua con ecuación $y=a x+b-a d$
    \item[d)] Asíntota vertical en $x=-d$ y asíntota oblicua con ecuación $y=a x-a d$
\end{enumerate}
\vspace{0.5cm}

\subsection*{Pregunta 2 - 2018-2 (Cálculo I, II y III)}
\textbf{Enunciado:}

¿Cuál de las siguientes integrales diverge?

\begin{enumerate}
    \item[a)] $\int_1^{\infty} \sin ^2(1 / x) d x$
    \item[b)] $\int_1^{\infty} \frac{\sin ^2(1 / x)}{x^2} d x$
    \item[c)] $\int_1^{\infty} \sin ^{1 / 2}(1 / x) d x$
    \item[d)] $\int_1^{\infty} \frac{\sin ^{1 / 2}(1 / x)}{x^2} d x$
\end{enumerate}
\vspace{0.5cm}

\subsection*{Pregunta 3 - 2018-2 (Cálculo I, II y III)}
\textbf{Enunciado:}

La región $\mathrm{D} \in \mathbb{R}^2$ se define por el área encerrada por la intersección de las parábolas $y=x^2$ y $x=y^2$. La densidad de esta región está dada por $\rho(x, y)=\sqrt{x}$ (en unidades de masa por unidad de área).

El centro de masa de D es:

\begin{enumerate}
    \item[a)] $\left(\frac{3}{14}, \frac{3}{14}\right)$
    \item[b)] $\left(\frac{6}{55}, \frac{1}{9}\right)$
    \item[c)] $\left(\frac{14}{27}, \frac{28}{55}\right)$
    \item[d)] $\left(\frac{27}{14}, \frac{9}{28}\right)$
\end{enumerate}
\vspace{0.5cm}

\subsection*{Pregunta 4 - 2018-2 (Ecuaciones Diferenciales)}
\textbf{Enunciado:}

Sean $m, n, p, q, t$ parámetros constantes
La ecuación diferencial $\frac{d^m y}{d x^m}\left(\frac{d y}{d x}\right)^p+x^t y^q=n x$ es:

\begin{enumerate}
    \item[a)] No-Lineal no-homogénea de tercer orden con coeficientes constantes si $m=2, n=1, p=$ $1, q=1, t=0$
    \item[b)] Lineal homogénea de tercer orden con coeficientes constantes si $m=1, n=1, p=0, q=$ $1, t=0$
    \item[c)] No-lineal no-homogénea de segundo orden con coeficientes constantes si $m=1, n=$ $2, p=1, q=1, t=1$
    \item[d)] No-lineal no-homogénea de segundo orden con coeficientes constantes si $m=2, n=$ $1, p=1, q=1, t=0$
\end{enumerate}
\vspace{0.5cm}

\subsection*{Pregunta 5 - 2018-2 (Álgebra Lineal)}
\textbf{Enunciado:}

Se tiene el siguiente sistema de ecuaciones
$$
\begin{array}{cc}
y-2 z & =1 \\
x+y+z & =1 \\
-x+z & =1
\end{array}
$$

¿Cuál de las siguientes alternativas indica la solución del problema por medio de la regla de Cramer?

\begin{enumerate}
    \item[a)] $x=\frac{\left|\begin{array}{ccc}1 & 1 & -2 \\ 1 & 1 & 1 \\ 1 & 0 & 1\end{array}\right|}{\left|\begin{array}{ccc}0 & 1 & -2 \\ 1 & 1 & 1 \\ -1 & 0 & 1\end{array}\right|}, \quad y=\frac{\left|\begin{array}{ccc}0 & 1 & -2 \\ 1 & 1 & 1 \\ -1 & 1 & 1\end{array}\right|}{\left|\begin{array}{ccc}0 & 1 & -2 \\ 1 & 1 & 1 \\ -1 & 0 & 1\end{array}\right|}, \quad z=\frac{\left|\begin{array}{ccc}0 & 1 & 1 \\ 1 & 1 & 1 \\ -1 & 0 & 1\end{array}\right|}{\left|\begin{array}{ccc}0 & 1 & -2 \\ 1 & 1 & 1 \\ -1 & 0 & 1\end{array}\right|}$
    \item[b)] $x=\frac{\left|\begin{array}{ccc}1 & 1 & 1 \\ 1 & 1 & 1 \\ -1 & 0 & 1\end{array}\right|}{\left|\begin{array}{ccc}0 & 1 & -2 \\ 1 & 1 & 1 \\ -1 & 0 & 1\end{array}\right|}, \quad y=\frac{\left|\begin{array}{ccc}0 & 1 & -2 \\ 1 & 1 & 1 \\ -1 & 0 & 1\end{array}\right|}{\left|\begin{array}{ccc}0 & 1 & -2 \\ 1 & 1 & 1 \\ -1 & 0 & 1\end{array}\right|}, \quad z=\frac{\left|\begin{array}{ccc}0 & 1 & -2 \\ 1 & 1 & 1 \\ 1 & 1 & 1\end{array}\right|}{\left|\begin{array}{ccc}0 & 1 & -2 \\ 1 & 1 & 1 \\ -1 & 0 & 1\end{array}\right|}$
    \item[c)] $\quad x=\frac{\left|\begin{array}{ccc}0 & 1 & -2 \\ 1 & 1 & 1 \\ -1 & 0 & 1\end{array}\right|}{\left|\begin{array}{ccc}1 & 1 & -2 \\ 1 & 1 & 1 \\ 1 & 0 & 1\end{array}\right|}, \quad y=\frac{\left|\begin{array}{ccc}0 & 1 & -2 \\ 1 & 1 & 1 \\ -1 & 0 & 1\end{array}\right|}{\left|\begin{array}{ccc}0 & 1 & -2 \\ 1 & 1 & 1 \\ -1 & 1 & 1\end{array}\right|}, \quad z=\frac{\left|\begin{array}{ccc}0 & 1 & -2 \\ 1 & 1 & 1 \\ -1 & 0 & 1\end{array}\right|}{\left|\begin{array}{ccc}0 & 1 & 1 \\ 1 & 1 & 1 \\ -1 & 0 & 1\end{array}\right|}$
    \item[d)] $\quad x=-\frac{\left|\begin{array}{ccc}1 & 1 & -2 \\ 1 & 1 & 1 \\ 1 & 0 & 1\end{array}\right|}{\left|\begin{array}{ccc}0 & 1 & -2 \\ 1 & 1 & 1 \\ -1 & 0 & 1\end{array}\right|}, \quad y=-\frac{\left|\begin{array}{ccc}0 & 1 & -2 \\ 1 & 1 & 1 \\ -1 & 1 & 1\end{array}\right|}{\left|\begin{array}{ccc}0 & 1 & -2 \\ 1 & 1 & 1 \\ -1 & 0 & 1\end{array}\right|}, \quad z=-\frac{\left|\begin{array}{ccc}0 & 1 & 1 \\ 1 & 1 & 1 \\ -1 & 0 & 1\end{array}\right|}{\left|\begin{array}{ccc}0 & 1 & -2 \\ 1 & 1 & 1 \\ -1 & 0 & 1\end{array}\right|}$
\end{enumerate}
\vspace{0.5cm}

\subsection*{Pregunta 19 - 2018-2 (Probabilidad y Estadística)}
\textbf{Enunciado:}

Un modelo meteorológico simple predice un día con o sin lluvia a partir del día anterior. En particular, estima que el día será lluvioso con $40 \%$ de probabilidad si es que el día anterior también es Iluvioso. Al mismo tiempo, el día será seco (no lluvioso) con un $66 \%$ de probabilidad si el día anterior también es seco.

Usando información externa, para hoy está pronosticado un día lluvioso con $24 \%$ de probabilidad. Según este modelo, ¿cuál es el valor más cercano a la probabilidad de que llueva mañana, si se sabe que hoy está lloviendo?

\begin{enumerate}
    \item[a)] $9,6 \%$
    \item[b)] $24,0 \%$
    \item[c)] $35,4 \%$
    \item[d)] $40,0 \%$
\end{enumerate}
\vspace{0.5cm}

\subsection*{Pregunta 20 - 2018-2 (Probabilidad y Estadística)}
\textbf{Enunciado:}

Los gastos mensuales de una cierta empresa se componen de ``materiales'', ``salarios'' y ``publicidad''. Los gastos por materiales y publicidad son variables aleatorias; también lo son los salarios, puesto que incluyen comisiones que dependen de las ventas.

Se pueden modelar los tres componentes de gasto como tres variables aleatorias con distribución normal, cuyas medias y desviaciones estándar se resumen en la tabla (en millones de pesos).

\begin{center}
\begin{tabular}{|l|c|c|}
\hline
\textbf{Item} & \textbf{Media $\mu$} & \textbf{Desviación estándar $\sigma$} \\ \hline
Materiales & 12 & 4 \\
Salarios & 22 & 3 \\
Publicidad & 8 & 3 \\ \hline
\end{tabular}
\end{center}

Además, la correlación entre ``materiales'' y ``publicidad'' es 0.8 , mientras que los gastos por salarios son independientes de los otros dos componentes.
¿Cuál de las siguientes alternativas corresponde al valor más cercano a la probabilidad de que en un cierto mes el total de gastos mensuales no exceda los 50 millones de pesos?

\begin{enumerate}
    \item[a)] $56 \%$
    \item[b)] $79 \%$
    \item[c)] $86 \%$
    \item[d)] $92 \%$
\end{enumerate}
\vspace{0.5cm}

\subsection*{Pregunta 21 - 2018-2 (Probabilidad y Estadística)}
\textbf{Enunciado:}

Suponga que usted cuenta con una muestra $x_1, \ldots, x_n$ de una misma población. Cada $x_i$ tiene distribución normal con media 1 y varianza desconocida $\sigma^2$.
¿Cuál de las siguientes alternativas representa la fórmula para el estimador de máxima verosimilitud (EMV) para la varianza desconocida $\sigma^2$ ?

\begin{enumerate}
    \item[a)] $\hat{\sigma}^2=\frac{1}{n} \sum_{i=1}^n\left(x_i-\bar{x}\right)^2$
    \item[b)] $\hat{\sigma}^2=\frac{1}{n-1} \sum_{i=1}^n\left(x_i-\bar{x}\right)^2$
    \item[c)] $\hat{\sigma}^2=\frac{1}{n} \sum_{i=1}^n\left(x_i-1\right)^2$
    \item[d)] $\hat{\sigma}^2=\frac{1}{n-1} \sum_{i=1}^n\left(x_i-1\right)^2$
\end{enumerate}
\vspace{0.5cm}

\subsection*{Pregunta 24 - 2018-2 (Probabilidad y Estadística)}
\textbf{Enunciado:}

Aún no hay solución propuesta
\vspace{0.5cm}

\section{2019-1}

\subsection*{Pregunta 1 - 2019-1 (Cálculo I, II y III)}
\textbf{Enunciado:}

Considere la función $f(x)=\ln (\ln (\ln (x)))$.
La derivada de esta función es:

\begin{enumerate}
    \item[a)] $\frac{1}{\ln (x) \ln (\ln (x))}$
    \item[b)] $\frac{1}{x \cdot \ln (x) \ln (\ln (x))}$
    \item[c)] $\frac{1}{\ln (\ln (x))}$
    \item[d)] $\frac{1}{\ln (x)}$
\end{enumerate}
\vspace{0.5cm}

\subsection*{Pregunta 2 - 2019-1 (Cálculo I, II y III)}
\textbf{Enunciado:}

¿Cuál de las siguientes series converge?

\begin{enumerate}
    \item[a)] $\sum_{n=1}^{\infty} \frac{n-1}{2 n+1}$
    \item[b)] $\sum_{n=0}^{\infty} \frac{\sqrt{n!}}{2^n}$
    \item[c)] $\sum_{n=0}^{\infty} \frac{e^n}{n!(\sqrt{n+1}-\sqrt{n})}$
    \item[d)] $\sum_{n=1}^{\infty} \frac{(-1)^n n}{4 n-1}$
\end{enumerate}
\vspace{0.5cm}

\subsection*{Pregunta 3 - 2019-1 (Cálculo I, II y III)}
\textbf{Enunciado:}

El sólido de revolución $\Omega \in \mathbb{R}^3$ se define al rotar la curva $z\left(a^2+x^2\right)^{3 / 2}=a^4$ (inserta en el plano $x-z$ ) respecto al eje de $z$, a su vez que esta superficie se intersecta con los planos $x=0, x=a$, $y=0$ e $y=a$ (con $a>0$ ). Se considera para dicho sólido solo el octante donde tanto $x, y$ como $z$ son positivos.

Encuentre el volumen de $\Omega$.

\begin{enumerate}
    \item[a)] $\frac{\pi}{5} a^3$
    \item[b)] $\frac{\pi}{6} a^3$
    \item[c)] $\frac{\pi}{7} a^3$
    \item[d)] $\frac{\pi}{8} a^3$
\end{enumerate}
\vspace{0.5cm}

\subsection*{Pregunta 4 - 2019-1 (Ecuaciones Diferenciales)}
\textbf{Enunciado:}

Una población posee una tasa de crecimiento en el tiempo que es proporcional a $r\left(1-\frac{p}{K}-\left(\frac{p}{K}\right)^2\right)$, donde $r$ y $K$ son parámetros positivos y $p$ es el nivel de la población.

¿A qué límite converge la población?

\begin{enumerate}
    \item[a)] $K e^{-r}$
    \item[b)] $\frac{\sqrt{5}-1}{2} K$
    \item[c)] $-\frac{\sqrt{5}-1}{2} K$
    \item[d)] $K e^r$
\end{enumerate}
\vspace{0.5cm}

\subsection*{Pregunta 5 - 2019-1 (Álgebra Lineal)}
\textbf{Enunciado:}

Sea $\mathbb{P}_2$ el espacio de los polinomios de segundo grado con coeficientes reales. Se define una base $B$ para $\mathbb{P}_2$ de la siguiente manera
$$
B=\left\{x^2, x, x+2\right\}
$$

Ahora, considere una transformación lineal $T: \mathbb{P}_2 \rightarrow \mathbb{P}_2$, tal que su matriz asociada respecto a la base $B$ es
$$
T_{B \rightarrow B}=\left[\begin{array}{ccc}
1 & -1 & 0 \\
0 & 1 & 1 \\
0 & 0 & 1
\end{array}\right]
$$

Sea $p \in \mathbb{P}_2$ un polinomio dado por $p(x)=x^2-4 x+4$. ¿Cuál de las siguientes alternativas corresponde a la transformación $T(p)$ ?

\begin{enumerate}
    \item[a)] $T(p)=5 x^2+4$
    \item[b)] $T(p)=5 x^2+4 x+8$
    \item[c)] $T(p)=7 x^2-4 x+2$
    \item[d)] $T(p)=7 x^2-2 x+4$
\end{enumerate}
\vspace{0.5cm}

\subsection*{Pregunta 6 - 2019-1 (Probabilidad y Estadística)}
\textbf{Enunciado:}

Una fábrica de automóviles está recibiendo una queja de una automotora extranjera pues aproximadamente un $16 \%$ de los vehículos que recibió vienen con una falla en su termostato. Un $65 \%$ de los vehículos son transportados por barco, y el $35 \%$ restante por avión. El jefe responsable del transporte aéreo aseguró que sólo un $4 \%$ de todos los vehículos que transportó a dicha automotora presentan la falla.

Según esta información, ¿cuál es el valor más cercano a la probabilidad de que un vehículo transportado por barco escogido al azar presente la falla mencionada?

\begin{enumerate}
    \item[a)] $14,6 \%$
    \item[b)] $22,5 \%$
    \item[c)] $28,0 \%$
    \item[d)] $38,3 \%$
\end{enumerate}
\vspace{0.5cm}

\subsection*{Pregunta 7 - 2019-1 (Probabilidad y Estadística)}
\textbf{Enunciado:}

En un cajero de estacionamiento, se ha instalado un aparato que mide el tiempo ( $T$, en horas) transcurrido cada 10 automóviles que pasan por la caja. En otras palabras, se mide la diferencia de tiempo entre la llegada de un automóvil y el décimo después de este. Suponga que el número de automóviles que pasan por caja tiene una distribución Poisson con tasa 20 llegadas por hora.

Considere las siguientes afirmaciones:
I. El tiempo transcurrido entre 10 llegadas de automóviles tiene una distribución Gamma $(10 ; 0,05)$.
II. El tiempo esperado entre 10 llegadas de automóviles es 10 veces el tiempo esperado entre llegadas consecutivas de automóviles.
III. El tiempo esperado entre llegadas consecutivas de automóviles es de 0,05 horas.

Son **CORRECTAS**:

\begin{enumerate}
    \item[a)] Sólo I y II
    \item[b)] Sólo I y III
    \item[c)] Sólo II y III
    \item[d)] I, II y III
\end{enumerate}
\vspace{0.5cm}

\subsection*{Pregunta 8 - 2019-1 (Probabilidad y Estadística)}
\textbf{Enunciado:}

En un hospital se está estudiando el peso promedio de los bebés que nacen de sus pacientes. En particular, quisieran probar que el peso promedio de los bebés recién nacidos en ese hospital es distinto a la media teórica $3,4 \mathrm{~kg}$.

Para ello se registró el peso de cada recién nacido durante un mes; en total fueron $n=86$. El peso promedio de esta muestra fue de $3,42 \mathrm{~kg}$ y la desviación estándar obtenida fue $0,32 \mathrm{~kg}$.

Asumiendo que el peso de un recién nacido tiene una distribución normal, ¿existe evidencia estadística para probar que el peso promedio de recién nacidos en ese hospital es diferente al promedio teórico?

\begin{enumerate}
    \item[a)] Con $1 \%$ de significancia sí.
    \item[b)] Con $1 \%$ de significancia no, pero con $5 \%$ de significancia sí.
    \item[c)] Con $5 \%$ de significancia no, pero con $10 \%$ de significancia sí.
    \item[d)] Con $10 \%$ de significancia no.
\end{enumerate}
\vspace{0.5cm}

\subsection*{Pregunta 22 - 2019-1 (Probabilidad y Estadística)}
\textbf{Enunciado:}

Suponga que se ajustó una recta de regresión simple a un conjunto de $n=34$ datos pareados $\left(x_i, y_i\right)$. La ecuación de la recta ajustada es la siguiente,
$$
y=25,97-4,68 \cdot x
$$

Para cada valor de $x_i$ se calculó el valor ajustado $\widehat{y}_l=25,97-4,68 \cdot x_i$, que corresponde al valor que toma la recta en $x=x_i$. De interés es la media cuadrática residual (o media cuadrática del error),
$$
M S E=\frac{1}{n-2} \sum_{i=1}^n\left(y_i-\widehat{y}_l\right)^2
$$
y se utiliza para estimar la varianza inherente al error del modelo, denotada $\sigma^2$. La varianza muestral de la variable $y$ es dada por
$$
s^2=\frac{1}{n-1} \sum_{i=1}^n\left(y_i-\bar{y}\right)^2=38,65
$$
y la $M S E$ tiene valor 23,94 .
Utilizando esta información, ¿cuál de la alternativas es el valor más cercano al coeficiente de determinación del ajuste $\left(R^2\right)$, o en otras palabras, la fracción de variabilidad de la variable $y$ explicada por el modelo?

\begin{enumerate}
    \item[a)] 0,05
    \item[b)] 0,40
    \item[c)] 0,60
    \item[d)] 0,95
\end{enumerate}
\vspace{0.5cm}

\section{2019-2}

\subsection*{Pregunta 1 - 2019-2 (Cálculo I, II y III)}
\textbf{Enunciado:}

¿Cuál es el gráfico que mejor representa la función $f(x) = e^{\sin(|x|)} + \ln(x)$?

\begin{center}
    \includegraphics[width=0.8\linewidth]{images/2019_2_mat_p_1.png}
\end{center}

\begin{enumerate}
    \item[a)] i)
    \item[b)] ii)
    \item[c)] iii)
    \item[d)] iv)
\end{enumerate}
\vspace{0.5cm}

\subsection*{Pregunta 2 - 2019-2 (Cálculo I, II y III)}
\textbf{Enunciado:}

Considere la función $f(x)=x^3$. El área de la región encerrada por la curva $y=f(x)$ y los ejes $x=0, x=1$ e $y=1$ es:

\begin{enumerate}
    \item[a)] $\frac{1}{4}$
    \item[b)] 1
    \item[c)] $\frac{1}{2}$
    \item[d)] $\frac{3}{4}$
\end{enumerate}
\vspace{0.5cm}

\subsection*{Pregunta 3 - 2019-2 (Cálculo I, II y III)}
\textbf{Enunciado:}

Aún no hay solución propuesta
\vspace{0.5cm}

\subsection*{Pregunta 4 - 2019-2 (Ecuaciones Diferenciales)}
\textbf{Enunciado:}

Durante una reacción química una sustancia $A$ es convertida en una $B$ a una tasa proporcional al cuadrado de la cantidad de $A$. Cuando $t=0$ hay 100 gramos de $A$ y después de 1 hora sólo quedan 50 gramos de $A$ por convertir.

¿Cuántos gramos de $A$ quedan luego de 4 horas de reacción?

\begin{enumerate}
    \item[a)] 6,25
    \item[b)] 12,5
    \item[c)] 20
    \item[d)] 25
\end{enumerate}
\vspace{0.5cm}

\subsection*{Pregunta 5 - 2019-2 (Álgebra Lineal)}
\textbf{Enunciado:}

Considere las siguientes 4 matrices,
$$
A=\left[\begin{array}{lll}
1 & 1 & 0 \\
0 & 1 & 1 \\
1 & 0 & 1
\end{array}\right], \quad B=\left[\begin{array}{lll}
4 & 4 & 4 \\
4 & 3 & 3 \\
4 & 3 & 2
\end{array}\right], \quad C=\left[\begin{array}{lll}
1 & 2 & 3 \\
2 & 3 & 4 \\
3 & 4 & 5
\end{array}\right], \quad D=\left[\begin{array}{ccc}
1 & 1 & -1 \\
1 & -1 & 1 \\
-1 & 1 & 1
\end{array}\right]
$$

¿Cuál de las matrices dadas $\underline{\text { NO }}$ puede ser transformada a la matriz identidad $I_3$ por medio de operaciones fila elementales?

\begin{enumerate}
    \item[a)] $A$
    \item[b)] $B$
    \item[c)] $C$
    \item[d)] $D$
\end{enumerate}
\vspace{0.5cm}

\subsection*{Pregunta 6 - 2019-2 (Probabilidad y Estadística)}
\textbf{Enunciado:}

Un pequeño ascensor en una construcción tiene capacidad máxima de 150 kilogramos, pero tiene espacio para que quepan 2 adultos. Considere que el peso de un obrero adulto tiene distribución normal con media 70 kilogramos y desviación estándar 10 kilogramos. El peso de un obrero es independiente a los demás.
¿Cuál de las siguientes alternativas es el valor más cercano a la probabilidad de que el ascensor exceda su capacidad máxima al ser utilizado por dos obreros adultos simultáneamente?

\begin{enumerate}
    \item[a)] $24 \%$
    \item[b)] $31 \%$
    \item[c)] $69 \%$
    \item[d)] $76 \%$
\end{enumerate}
\vspace{0.5cm}

\subsection*{Pregunta 7 - 2019-2 (Probabilidad y Estadística)}
\textbf{Enunciado:}

Según un estudio, se estima que durante una tormenta eléctrica, una antena pararrayos recibe en promedio 2 rayos por hora. Suponga que se modela la cantidad de rayos que impactan esta antena como una variable aleatoria con distribución Poisson, con tasa 2 rayos/hora.

¿Cuál de las siguientes alternativas es el valor más cercano a la probabilidad de que la antena pararrayos no reciba más de dos rayos durante una tormenta eléctrica que se extiende por exactamente tres horas?

\begin{enumerate}
    \item[a)] $1,7 \%$
    \item[b)] $6,2 \%$
    \item[c)] $40,6 \%$
    \item[d)] $67,7 \%$
\end{enumerate}
\vspace{0.5cm}

\subsection*{Pregunta 8 - 2019-2 (Probabilidad y Estadística)}
\textbf{Enunciado:}

Para un estudio acerca del área que alcanza una flor de girasol, se plantaron 20 girasoles en iguales condiciones, y se midió el área plana de su flor (incluyendo sus pétalos) luego de tres meses. El área promedio de las flores de esta muestra fue de $314,5 \mathrm{~cm}^2$, con una desviación estándar muestral de $111,1 \mathrm{~cm}^2$. Asuma que el área de la flor es una variable aleatoria con distribución normal.

Si se desea cuantificar la estimación por medio de un intervalo, ¿cuál de las siguientes alternativas se aproxima a un intervalo de $90 \%$ confianza para el área promedio?

\begin{enumerate}
    \item[a)] $[262,5 ; 366,5]$
    \item[b)] $[265,8 ; 363,2]$
    \item[c)] $[271,5 ; 357,5]$
    \item[d)] $[281,5$; 347,5]
\end{enumerate}
\vspace{0.5cm}

\subsection*{Pregunta 9 - 2019-2 (Probabilidad y Estadística)}
\textbf{Enunciado:}

En el contexto de un modelo de regresión lineal simple, existen algunos supuestos importantes que se sugiere sean verificados al momento de tomar conclusiones estadísticas. Sea $Y$ la variable respuesta (dependiente), y $X$ la variable explicativa (independiente) del modelo.
¿Cuál de las siguientes alternativas NO es un supuesto necesario para este modelo?

\begin{enumerate}
    \item[a)] Para cada valor de $X$, la distribución de $Y$ debe ser normal.
    \item[b)] Para cada valor de $X$, la desviación estándar de $Y$ debe ser la misma.
    \item[c)] La esperanza de $Y$ debe ser una función lineal de $X$.
    \item[d)] La variable $Y$ debe ser independiente de $X$.
\end{enumerate}
\vspace{0.5cm}

\section{2023-2}

\subsection*{Pregunta 1 - 2023-2 (Cálculo I, II y III)}
\textbf{Enunciado:}

Sea $f: \mathbb{R} \backslash\{0\} \rightarrow \mathbb{R}$ la función real definida por:
$$
f(x)=\frac{\operatorname{sen} x}{x}-\cos x, \quad x \neq 0
$$
¿Cuál de las siguientes alternativas corresponde a la derivada de $f(x)$ ?

\begin{enumerate}
    \item[a)] $f^{\prime}(x)=x^{-2}\left(x \cos x+\left(x^2-1\right) \operatorname{sen} x\right)$
    \item[b)] $f^{\prime}(x)=x^{-2}\left(-x \cos x+\left(x^2-1\right) \operatorname{sen} x\right)$
    \item[c)] $f^{\prime}(x)=x^{-2}\left(x \operatorname{sen} x+\left(1-x^2\right) \cos x\right)$
    \item[d)] $f^{\prime}(x)=x^{-2}\left(-x \operatorname{sen} x+\left(1-x^2\right) \cos x\right)$
\end{enumerate}
\vspace{0.5cm}

\subsection*{Pregunta 2 - 2023-2 (Cálculo I, II y III)}
\textbf{Enunciado:}

¿Cuál de las siguientes integrales diverge?

\begin{enumerate}
    \item[a)] $\int_1^{\infty} \frac{\cos x}{x^2} \mathrm{~d} x$
    \item[b)] $\int_1^{\infty} \frac{\sqrt{x^2+2}}{\sqrt{x^5+5}} \mathrm{~d} x$
    \item[c)] $\int_0^{\infty} \frac{\sin (1 / x)}{\exp (x)} \mathrm{d} x$
    \item[d)] $\int_e^{\infty} \frac{1}{x \ln x} \mathrm{~d} x$
\end{enumerate}
\vspace{0.5cm}

\subsection*{Pregunta 3 - 2023-2 (Cálculo I, II y III)}
\textbf{Enunciado:}

Sea $f(x, y)=\frac{x^2+y^2}{\sqrt{x^2+y^2}}$. La derivada direccional en el punto $(1,1)$, en la dirección unitaria $\theta=\frac{\pi}{4}$ (coordenadas polares), es:

\begin{enumerate}
    \item[a)] 0
    \item[b)] 2
    \item[c)] -1
    \item[d)] 1
\end{enumerate}
\vspace{0.5cm}

\subsection*{Pregunta 4 - 2023-2 (Cálculo I, II y III)}
\textbf{Enunciado:}

Sea $\Lambda \subset \mathbb{R}^3$ un cuerpo en el espacio definido por las siguientes desigualdades en coordenadas cilíndricas,
$$
\begin{aligned}
& 0 \leq r \leq 2+\operatorname{sen}(4 \theta) \\
& 0 \leq \theta \leq 2 \pi \\
& 0 \leq z \leq 1
\end{aligned}
$$
¿Cuál de las siguientes alternativas corresponde al volumen del cuerpo $\Lambda$ ?

\begin{enumerate}
    \item[a)] $2 \pi$
    \item[b)] $4 \pi$
    \item[c)] $9 \pi / 2$
    \item[d)] $9 \pi$
\end{enumerate}
\vspace{0.5cm}

\subsection*{Pregunta 4 - 2023-2 (Ecuaciones Diferenciales)}
\textbf{Enunciado:}

La temperatura de un objeto $T$ varía en el tiempo de acuerdo a la ecuación diferencial siguiente:

$$
\frac{d T}{d t}=k(A-T)
$$

donde $A$ es la temperatura del medio y $k$ es una constante de conductividad de calor del medio hacia el objeto.

Si la temperatura inicial del objeto es el doble que la temperatura del medio, ¿cuánto tiempo le tomará al objeto alcanzar una temperatura exactamente el $50 \%$ más alta que la del medio?

\begin{enumerate}
    \item[a)] $\frac{1}{k}$
    \item[b)] $\frac{1}{k \ln 2}$
    \item[c)] $\frac{k}{\ln 2}$
    \item[d)] $\frac{\ln 2}{k}$
\end{enumerate}
\vspace{0.5cm}

\subsection*{Pregunta 5 - 2023-2 (Álgebra Lineal)}
\textbf{Enunciado:}

Sean $A$ y $B$ dos matrices cuadradas de $n \times n$, ambas simétricas.

¿Cuál de las siguientes alternativas es FALSA?

\begin{enumerate}
    \item[a)] $A+B$ siempre es simétrica.
    \item[b)] $A A^T$ siempre es simétrica.
    \item[c)] $A-B^T$ siempre es simétrica.
    \item[d)] $A B(B A)^T$ siempre es simétrica.
\end{enumerate}
\vspace{0.5cm}

\subsection*{Pregunta 6 - 2023-2 (Ecuaciones Diferenciales)}
\textbf{Enunciado:}

Considere el siguiente sistema de ecuaciones diferenciales para $x(t)$ e $y(t)$ :
$$
\begin{aligned}
& \frac{d x}{d t}=2 x+3 y \\
& \frac{d y}{d t}=x-2 y
\end{aligned}
$$

¿Cuál de las siguientes alternativas corresponde a la solución $\{x(t), y(t)\}$ del sistema dado?

\begin{enumerate}
    \item[a)] $\binom{x(t)}{y(t)}=A\binom{2+\sqrt{7}}{1} e^{\sqrt{7} \cdot t}+B\binom{2-\sqrt{7}}{1} e^{-\sqrt{7} \cdot t}$
    \item[b)] $\binom{x(t)}{y(t)}=A\binom{2-\sqrt{7}}{1} e^{\sqrt{7} \cdot t}+B\binom{2+\sqrt{7}}{1} e^{-\sqrt{7} \cdot t}$
    \item[c)] $\binom{x(t)}{y(t)}=A\binom{-1}{1} e^t+B\binom{1}{1} e^{-t}$
    \item[d)] $\binom{x(t)}{y(t)}=A\binom{1}{1} e^t+B\binom{-1}{1} e^{-t}$
\end{enumerate}
\vspace{0.5cm}

\subsection*{Pregunta 8 - 2023-2 (Álgebra Lineal)}
\textbf{Enunciado:}

Considere la siguiente matriz ($A$ y $B$ son invertibles):
$$
M=(A B)^T\left(B A^T\right)^{-1}
$$
$M^T$ es igual a:

\begin{enumerate}
    \item[a)] $I$
    \item[b)] $\left(B^{-1}\right)^T B$
    \item[c)] $A B\left(B^T A\right)^{-1}$
    \item[d)] $A^T B^T B^{-1}\left(A^T\right)^{-1}$
\end{enumerate}
\vspace{0.5cm}

\subsection*{Pregunta 9 - 2023-2 (Probabilidad y Estadística)}
\textbf{Enunciado:}

Suponga que el valor de una acción $P$ tiene una distribución normal y, en circunstancias normales de mercado, el valor en cada día es aleatorio e independiente, con la misma distribución normal (media $\mu$ y varianza $\sigma^2$, desconocidas). Es de interés obtener una cuantificación de la varianza (o ``volatilidad'') del precio de la acción P por medio de un intervalo de confianza.

Para lograr el objetivo se registró el valor de la acción $\left(x_i\right)$ durante dos semanas hábiles (10 días) en que el mercado se encontraba en situación estable, y se obtuvo el siguiente resumen estadístico,
$$
n=10, \quad \bar{x}=268,6 \quad, \quad s^2=317,8
$$
donde los últimos dos valores están medidos en pesos.

En base a esta muestra, ¿cuál de las siguientes alternativas corresponde a un intervalo de $95 \%$ de confianza para la varianza $\sigma^2$ ?

\begin{enumerate}
    \item[a)] $[150,4 ; 1059,2]$
    \item[b)] $[169,1 ; 860,2]$
    \item[c)] $[139,6 ; 880,9]$
    \item[d)] $[156,2 ; 725,9]$
\end{enumerate}
\vspace{0.5cm}

\subsection*{Pregunta 10 - 2023-2 (Probabilidad y Estadística)}
\textbf{Enunciado:}

Suponga que en cierto terreno la probabilidad de encontrar gas natural subterráneo es de $30 \%$. Un experto petrolero quiere realizar una prueba sísmica en el terreno, la cual confirma correctamente la presencia de gas con una probabilidad de $90 \%$. La misma prueba confirma correctamente la ausencia de gas con probabilidad $70 \%$.

Aclaración: Confirmar correctamente la presencia (o ausencia) de gas significa que el resultado de la prueba sísmica es el correcto, dada la presencia (o ausencia) de gas en el terreno.

Suponga que la prueba sísmica indicó ausencia de gas, ¿cuál de las siguientes alternativas es más cercana a la probabilidad de que haya gas natural subterráneo en el terreno, a pesar del resultado de la prueba?

\begin{enumerate}
    \item[a)] $3 \%$
    \item[b)] $6 \%$
    \item[c)] $10 \%$
    \item[d)] $30 \%$
\end{enumerate}
\vspace{0.5cm}

\subsection*{Pregunta 11 - 2023-2 (Probabilidad y Estadística)}
\textbf{Enunciado:}

Un fabricante de automóviles tomó una muestra de 100 vehículos y midió su kilometraje al momento de ser necesario su cambio de transmisión. De la muestra se obtiene una media muestral de 122.240 km , y una desviación estándar de 8.400 km . Suponga que el rendimiento de cada vehículo es independiente de los demás y que el kilometraje recorrido antes de requerir un cambio de transmisión tiene distribución normal.

Según esta información, ¿cuál de las siguientes alternativas es la más cercana a un intervalo de $95 \%$ de confianza para el kilometraje esperado al momento de requerir un cambio de transmisión?

\begin{enumerate}
    \item[a)] $[120.286$; 124.194]
    \item[b)] $[120.594$; 123.886]
    \item[c)] $[120.858$; 123.621]
    \item[d)] $[121.163 ; 123.316]$
\end{enumerate}
\vspace{0.5cm}

\subsection*{Pregunta 12 - 2023-2 (Probabilidad y Estadística)}
\textbf{Enunciado:}

Un analista de una pequeña empresa busca relacionar los gastos mensuales $(y)$ como función del ingreso por ventas mensuales. Suponga que se registró una muestra de ventas y gastos por doce meses $\left(x_i, y_i\right)$. La información de los datos se resume en los siguientes estadísticos:
$$
\begin{aligned}
\sum_{i=1}^{12} x_i= & 2.618 \quad;\quad \sum_{i=1}^{12} y_i=325,8 \quad;\quad \sum_{i=1}^{12} x_i^2=587.099,08 \\
& \sum_{i=1}^{12} y_i^2=72.375,09 \quad;\quad \sum_{i=1}^{12} x_i y_i=9.041,74
\end{aligned}
$$

Asuma que se cumplen los supuestos de un modelo de regresión lineal simple.
¿Cuál de las siguientes alternativas corresponde a las estimaciones más cercanas de los parámetros $(a, b)$ de la recta de regresión $y=a+b x$, por el método de mínimos cuadrados?

\begin{enumerate}
    \item[a)] $\hat{a}=876,3 ; \hat{b}=-3,89$
    \item[b)] $\hat{a}=50,21 ; \hat{b}=-0,11$
    \item[c)] $\hat{a}=38,83 ; \hat{b}=-0,05$
    \item[d)] $\hat{a}=-1.069,5 ; \hat{b}=5,02$
\end{enumerate}
\vspace{0.5cm}

\subsection*{Pregunta 13 - 2023-2 (Probabilidad y Estadística)}
\textbf{Enunciado:}

Un proveedor de fibra óptica afirma que las velocidades de carga y descarga de su servicio son equivalentes. Para comprobarlo, Emilia ha realizado un test de velocidad en 50 ocasiones, obteniendo:
- Una media de 322 Mbps para velocidad de carga, con desviación estándar de 12 Mbps.
- Una media de 328 Mbps para velocidad de descarga, con desviación estándar de 9 Mbps.

Según los datos de Emilia, ¿existe suficiente evidencia para rechazar que las velocidades de carga y descarga sean equivalentes?

\begin{enumerate}
    \item[a)] Con un $1 \%$ de significancia sí.
    \item[b)] Con un $1 \%$ de significancia no, pero con un $5 \%$ de significancia sí.
    \item[c)] Con un $5 \%$ de significancia no, pero con un $10 \%$ de significancia sí.
    \item[d)] Con un $10 \%$ de significancia no.
\end{enumerate}
\vspace{0.5cm}

\subsection*{Pregunta 14 - 2023-2 (Probabilidad y Estadística)}
\textbf{Enunciado:}

Benjamín siempre ha vendido zapallo italiano por unidad, pero desea comenzar a venderlo por kg, así que está interesado en conocer, en promedio, cuánto masa uno de sus zapallos italianos. Para esto, ha masado 40 zapallos italianos, obteniendo un promedio de 240 g con una desviación estándar de 21 g .

Construya un intervalo de confianza al $90 \%$ para la masa de un zapallo italiano promedio, en gramos.

\begin{enumerate}
    \item[a)] $[234,5 ; 245,5]$
    \item[b)] $[233,5 ; 246,5]$
    \item[c)] $[232,5 ; 247,5]$
    \item[d)] $[231,5 ; 249,5]$
\end{enumerate}
\vspace{0.5cm}

\section{2024-2}

\subsection*{Pregunta 1 - 2024-2 (Cálculo I, II y III)}
\textbf{Enunciado:}

Se define la función $F: \mathbb{R} \rightarrow \mathbb{R}$ mediante:
$$
F(x)=\int_0^x \frac{3 t}{1+t^2} \mathrm{~d} t
$$
¿Cuánto vale $F(2)$ ?

\begin{enumerate}
    \item[a)] $\ln 3$
    \item[b)] $\frac{3}{2} \ln 3$
    \item[c)] $\ln 5$
    \item[d)] $\frac{3}{2} \ln 5$
\end{enumerate}
\vspace{0.5cm}

\subsection*{Pregunta 2 - 2024-2 (Cálculo I, II y III)}
\textbf{Enunciado:}

Sea $R$ la región delimitada por:
$$
0 \leq y \leq 2-|x|
$$
¿Cuál es el momento de $R$ con respecto al eje $X$ ?

\begin{enumerate}
    \item[a)] 1
    \item[b)] $4 / 3$
    \item[c)] 2
    \item[d)] $8 / 3$
\end{enumerate}
\vspace{0.5cm}

\subsection*{Pregunta 3 - 2024-2 (Cálculo I, II y III)}
\textbf{Enunciado:}

Los vectores $(x, y, z) \in \mathbb{R}^3$ que satisfacen la ecuación doble $x=-y+1=2 z$ corresponden a:

\begin{enumerate}
    \item[a)] Un plano cuyo vector normal es paralelo a ( $1,-1,2)$
    \item[b)] Un plano que pasa por el punto $(0,1,0)$
    \item[c)] Una recta cuyo vector director es paralelo a $(2,-2,1)$
    \item[d)] Una recta que pasa por el punto ( $-1,1,-1 / 2)$
\end{enumerate}
\vspace{0.5cm}

\subsection*{Pregunta 4 - 2024-2 (Cálculo I, II y III)}
\textbf{Enunciado:}

Considere el sólido de revolución conseguido al rotar la siguiente región del plano XY con respecto al eje X:
$$
\begin{aligned}
& 0 \leq x \leq 1 \\
& 0 \leq y \leq \mathrm{e}^x
\end{aligned}
$$
¿Cuál es el volumen del cuerpo descrito?

\begin{enumerate}
    \item[a)] $\pi \mathrm{e}^2 / 2$
    \item[b)] $\pi e^2$
    \item[c)] $\pi\left(\mathrm{e}^2-1\right) / 2$
    \item[d)] $\pi\left(\mathrm{e}^2-1\right)$
\end{enumerate}
\vspace{0.5cm}

\subsection*{Pregunta 5 - 2024-2 (Cálculo I, II y III)}
\textbf{Enunciado:}

Sea $g: \mathbb{R}^2 \rightarrow \mathbb{R}$ una función real definida como:
$$
g(x, y)=e^{\arctan (x+y)}
$$

Considere el punto $\boldsymbol{x}_{\mathbf{0}}=(1,0)$ y el vector unitario $\boldsymbol{u}=\left(\frac{1}{\sqrt{2}}, \frac{1}{\sqrt{2}}\right)$.
¿Cuál de las siguientes alternativas corresponde a la derivada direccional $\frac{\partial g}{\partial \boldsymbol{u}}$ en el punto $\boldsymbol{x}_{\mathbf{0}}$ ?

\begin{enumerate}
    \item[a)] $e^{\pi / 4} \sqrt{2}$
    \item[b)] $\frac{1}{2} e^{\pi / 4} \sqrt{2}$
    \item[c)] $e^{\pi / 2} \sqrt{2}$
    \item[d)] $\frac{1}{2} e^{\pi / 2} \sqrt{2}$
\end{enumerate}
\vspace{0.5cm}

\subsection*{Pregunta 6 - 2024-2 (Ecuaciones Diferenciales)}
\textbf{Enunciado:}

Se modela un sistema masa-resorte mediante la ecuación diferencial:
$$
m x^{\prime \prime}=-k x
$$

Donde $m$ es la masa del cuerpo, $k$ es la constante elástica, y $x$ es el estiramiento del resorte. Suponga que, en el instante inicial, la masa se está desplazando de modo que $x(0)=0 y$ $x^{\prime}(0)=v$.

¿Cuál es el menor valor de $t$ para el que $x^{\prime}(t)=0$ ?

\begin{enumerate}
    \item[a)] $\frac{\pi}{2} \sqrt{\frac{k}{m}}$
    \item[b)] $\frac{\pi}{2} \sqrt{\frac{m}{k}}$
    \item[c)] $\pi \sqrt{\frac{k}{m}}$
    \item[d)] $\pi \sqrt{\frac{m}{k}}$
\end{enumerate}
\vspace{0.5cm}

\subsection*{Pregunta 7 - 2024-2 (Ecuaciones Diferenciales)}
\textbf{Enunciado:}

Considere la siguiente ecuación diferencial para $y$ como función de $x$ :
$$
\left(x^2+y^2\right) \mathrm{d} x-x y \mathrm{~d} y=0
$$
¿Cuál de las siguientes alternativas describe mejor la ecuación diferencial?

\begin{enumerate}
    \item[a)] No lineal, homogénea y de primer orden.
    \item[b)] Lineal, no homogénea y de segundo orden.
    \item[c)] No lineal, no homogénea y de segundo orden.
    \item[d)] Lineal, homogénea y de primer orden.
\end{enumerate}
\vspace{0.5cm}

\subsection*{Pregunta 8 - 2024-2 (Álgebra Lineal)}
\textbf{Enunciado:}

Considere la matriz ampliada de un sistema de ecuaciones, $[A \mid \boldsymbol{b}]$, cuya forma escalonada reducida es:
$$
\left[\begin{array}{ccccc|c}
1 & -1 & 3 & 2 & 4 & 3 \\
0 & 0 & 1 & -3 & 2 & -1 \\
0 & 0 & 0 & 0 & 0 & 1 \\
0 & 0 & 0 & 0 & 0 & 0
\end{array}\right]
$$
¿Qué se puede afirmar de las soluciones del sistema?

\begin{enumerate}
    \item[a)] El sistema no tiene solución.
    \item[b)] El sistema tiene solución única.
    \item[c)] Las soluciones del sistema forman una recta o un plano.
    \item[d)] Las soluciones del sistema forman un espacio vectorial de 3 o más dimensiones.
\end{enumerate}
\vspace{0.5cm}

\subsection*{Pregunta 9 - 2024-2 (Álgebra Lineal)}
\textbf{Enunciado:}

Sean $A$ y $B$ dos matrices cuadradas del mismo tamaño. Suponga además que las matrices $A$ y $A+B$ son invertibles.

Considere las siguientes afirmaciones:

I. $B$ siempre es invertible.

II. $B A^{-1}$ siempre es invertible.

III. $\quad I+B A^{-1}$ siempre es invertible.

¿Cuál(es) de las afirmaciones anteriores es(son) FALSA(S)?

\begin{enumerate}
    \item[a)] Solo I
    \item[b)] Solo III
    \item[c)] Solo I y II
    \item[d)] Todas
\end{enumerate}
\vspace{0.5cm}

\subsection*{Pregunta 10 - 2024-2 (Probabilidad y Estadística)}
\textbf{Enunciado:}

Es bastante común asociar vientos fuertes y cálidos con la proximidad de una tormenta (Iluvia). Un estudio climatológico estimó un $30 \%$ de probabilidad de lluvia en un día cualquiera. Además, en días lluviosos, un $75 \%$ de las veces se registraron vientos fuertes y cálidos, mientras que, en días sin lluvia, se observaron vientos fuertes y cálidos en sólo un $20 \%$ de los casos.

Suponga que en un día cualquiera se sabe que existe presencia de vientos fuertes y cálidos. Según la información entregada, ¿cuál de las alternativas es el valor MÁS CERCANO a la probabilidad de que ese día sea lluvioso?

\begin{enumerate}
    \item[a)] $22,5 \%$
    \item[b)] $36,5 \%$
    \item[c)] $61,6 \%$
    \item[d)] $75 \%$
\end{enumerate}
\vspace{0.5cm}

\subsection*{Pregunta 11 - 2024-2 (Probabilidad y Estadística)}
\textbf{Enunciado:}

Valentina atiende pacientes en una clínica. Durante una jornada laboral, ella tiene agendados 20 pacientes, y recibirá un bono en dicho día si asisten 18 o más pacientes.
Suponga que cada paciente puede faltar con una probabilidad del $10 \%$.

¿Cuál es el valor más cercano de la probabilidad de que Valentina reciba un bono en un día determinado?

\begin{enumerate}
    \item[a)] $12,2 \%$
    \item[b)] $49,2 \%$
    \item[c)] $67,7 \%$
    \item[d)] $86,3 \%$
\end{enumerate}
\vspace{0.5cm}

\subsection*{Pregunta 12 - 2024-2 (Probabilidad y Estadística)}
\textbf{Enunciado:}

Considere 2 variables aleatorias $X$ e $Y$, cuya distribución de probabilidad conjunta está dada por:
$$
f(x, y)=k x \mathrm{e}^{-2 x y}
$$

En el dominio $x \in[1,5], y \in[0, \infty)$, y donde $k$ es una constante real desconocida, ¿cuál es el valor de $k$ ?
(hint: ¿cuánto debe valer la integral de $f$ en su dominio?)

\begin{enumerate}
    \item[a)] $1 / 4$
    \item[b)] $1 / 2$
    \item[c)] 2
    \item[d)] 4
\end{enumerate}
\vspace{0.5cm}

\subsection*{Pregunta 13 - 2024-2 (Probabilidad y Estadística)}
\textbf{Enunciado:}

Históricamente la temperatura promedio durante los meses de noviembre en Puerto Williams ha sido $8^{\circ} \mathrm{C}$. El último año se registró un promedio muestral de $8,9^{\circ} \mathrm{C}$ en sus $n=30$ días.

Asuma que la temperatura media de cada día en noviembre tiene distribución normal con media $\mu$ constante desconocida y desviación estándar $\sigma$ conocida igual a $1,2^{\circ} \mathrm{C}$, y que las temperaturas son independientes.

¿Se puede concluir que la temperatura diaria media en Puerto Williams es MAYOR que $8^{\circ} \mathrm{C}$ ?

\begin{enumerate}
    \item[a)] Con un nivel de significancia de $10 \%$ no.
    \item[b)] Con un nivel de significancia de $5 \%$ no, pero con un nivel de significancia de $10 \%$ sí.
    \item[c)] Con un nivel de significancia de $1 \%$ no, pero con un nivel de significancia de $5 \%$ sí.
    \item[d)] Con un nivel de significancia de $1 \%$ sí.
\end{enumerate}
\vspace{0.5cm}

\subsection*{Pregunta 14 - 2024-2 (Probabilidad y Estadística)}
\textbf{Enunciado:}

Usted está modelando la cantidad de vehículos que circulan por una autopista en una sección transversal determinada según una distribución de Poisson. Para esto, el procedimiento ha sido:
- Medir la cantidad de vehículos por minuto, durante 90 minutos.
- A partir de la muestra, estimar el parámetro de la distribución Poisson, que ha resultado ser $\lambda=5$ (vehículos por minuto).
- Construir la siguiente tabla:

\begin{center}
\begin{tabular}{|c|c|c|c|}
\hline
\textbf{Intervalo (vehículos en 1 minuto)} & \textbf{Frec. observada, $O_i$} & \textbf{Frec. esperada, $E_i$} & \boldmath$\frac{\left(O_i-E_i\right)^2}{E_i}$\unboldmath \\ \hline
$0-1$ & 4 & 7,27 & 1,48 \\
$2-3$ & 34 & 40,43 & 1,02 \\
$4-5$ & 59 & 63,17 & 0,28 \\
$6-7$ & 51 & 45,12 & 0,77 \\
$8-9$ & 23 & 18,28 & 1,22 \\
10 o más & 9 & 5,73 & 1,87 \\ \hline
\end{tabular}
\end{center}

Suponiendo que la medición fue perfecta, ¿existe evidencia suficiente para rechazar la hipótesis de que la distribución de vehículos que circula por la autopista distribuye Poisson?

\begin{enumerate}
    \item[a)] Con un $1 \%$ de significancia sí.
    \item[b)] Con un $1 \%$ de significancia no, pero con un $5 \%$ de significancia sí.
    \item[c)] Con un $5 \%$ de significancia no, pero con un $10 \%$ de significancia sí.
    \item[d)] Con un $10 \%$ de significancia no.
\end{enumerate}
\vspace{0.5cm}

\subsection*{Pregunta 15 - 2024-2 (Probabilidad y Estadística)}
\textbf{Enunciado:}

Un fabricante de ampolletas incandescentes está evaluando la calidad de su producto y está interesado en modelar la duración de las mismas (en horas de uso antes de quemarse).

Para esto, el procedimiento ha sido:
- Testear 100 ampolletas, registrando la cantidad de horas que duraron encendidas.
- A partir de la muestra anterior, conseguir el estimador de máxima verosimilitud para el parámetro de la distribución exponencial, que resultó ser $1 / \lambda=1.102$.
- Organizar la información en la siguiente tabla:

\begin{center}
\begin{tabular}{|l|c|c|c|}
\hline
\textbf{Intervalo (horas de duración)} & \textbf{Frec. observada, $O_i$} & \textbf{Frec. esperada, $E_i$} & \boldmath$\frac{\left(O_i-E_i\right)^2}{E_i}$\unboldmath \\ \hline
$[0,800)$ & 55 & 51,61 & 0,22 \\
$[800,1.600)$ & 21 & 24,97 & 0,63 \\
$[1.600,2.400)$ & 10 & 12,08 & 0,36 \\
$[2.400,3.200)$ & 10 & 5,85 & 2,94 \\
$[3.200,4.000)$ & 2 & 2,83 & 0,24 \\
$[4.000,+\infty)$ & 2 & 2,65 & 0,16 \\ \hline
\end{tabular}
\end{center}

Con esta información, ¿existe evidencia suficiente para rechazar la hipótesis de que la duración de las ampolletas distribuye exponencial?

\begin{enumerate}
    \item[a)] Con un $1 \%$ de significancia sí.
    \item[b)] Con un $1 \%$ de significancia no, pero con un $5 \%$ de significancia sí.
    \item[c)] Con un $5 \%$ de significancia no, pero con un $10 \%$ de significancia sí.
    \item[d)] Con un $10 \%$ de significancia no.
\end{enumerate}
\vspace{0.5cm}

\newpage
\begingroup
\let\clearpage\relax
\vspace*{1cm}
\section*{Tabla de Respuestas}
\begin{center}
\begin{tabular}{|c|c|c||c|c|c|}
\hline
\textbf{Año} & \textbf{Pre.} & \textbf{Res.} & \textbf{Año} & \textbf{Pre.} & \textbf{Res.} \\ \hline
2016-1 & 1 & a & 2018-2 & 20 & c \\
2016-1 & 2 & c & 2018-2 & 21 & c \\
2016-1 & 5 & b & 2018-2 & 24 & a \\ \cline{4-6}
2016-1 & 5 & a & 2019-1 & 1 & b \\
2016-1 & 7 & a & 2019-1 & 2 & c \\
2016-1 & 9 & a & 2019-1 & 3 & d \\
2016-1 & 17 & d & 2019-1 & 4 & b \\
2016-1 & 18 & a & 2019-1 & 5 & d \\ \cline{1-3}
2016-2 & 1 & a & 2019-1 & 6 & b \\
2016-2 & 2 & d & 2019-1 & 7 & d \\
2016-2 & 3 & c & 2019-1 & 8 & d \\
2016-2 & 4 & a & 2019-1 & 22 & b \\ \cline{4-6}
2016-2 & 6 & b & 2019-2 & 1 &  \\
2016-2 & 19 & b & 2019-2 & 2 & d \\
2016-2 & 21 & c & 2019-2 & 3 & d \\
2016-2 & 22 & d & 2019-2 & 4 & c \\
2016-2 & 23 & a & 2019-2 & 5 & c \\ \cline{1-3}
2017-1 & 1 & d & 2019-2 & 6 & a \\
2017-1 & 2 & a & 2019-2 & 7 & b \\
2017-1 & 3 & d & 2019-2 & 8 & c \\
2017-1 & 4 & c & 2019-2 & 9 & d \\ \cline{4-6}
2017-1 & 5 & c & 2023-2 & 1 & a \\
2017-1 & 21 & a & 2023-2 & 2 & d \\
2017-1 & 22 & c & 2023-2 & 3 & d \\
2017-1 & 23 & d & 2023-2 & 4 & c \\
2017-1 & 24 & b & 2023-2 & 4 & d \\ \cline{1-3}
2017-2 & 1 & c & 2023-2 & 5 & d \\
2017-2 & 2 & d & 2023-2 & 6 & a \\
2017-2 & 3 & b & 2023-2 & 8 & b \\
2017-2 & 4 & b & 2023-2 & 9 & a \\
2017-2 & 5 & c & 2023-2 & 10 & b \\
2017-2 & 21 & b & 2023-2 & 11 & b \\
2017-2 & 22 & c & 2023-2 & 12 & a \\
2017-2 & 23 & a & 2023-2 & 13 & a \\
2017-2 & 24 & d & 2023-2 & 14 & a \\ \hline
2018-1 & 1 & b & 2024-2 & 1 & d \\
2018-1 & 2 & d & 2024-2 & 2 & a \\
2018-1 & 3 & c & 2024-2 & 3 & c \\
2018-1 & 4 & b & 2024-2 & 4 & c \\
2018-1 & 5 & c & 2024-2 & 5 & b \\
2018-1 & 21 & d & 2024-2 & 6 & b \\
2018-1 & 22 & c & 2024-2 & 7 & a \\
2018-1 & 23 & d & 2024-2 & 8 & a \\
2018-1 & 24 & c & 2024-2 & 9 & c \\ \cline{1-3}
2018-2 & 1 & c & 2024-2 & 10 & c \\
2018-2 & 2 & c & 2024-2 & 11 & c \\
2018-2 & 3 & c & 2024-2 & 12 & b \\
2018-2 & 4 & d & 2024-2 & 13 & d \\
2018-2 & 5 & a & 2024-2 & 14 & d \\
2018-2 & 19 & d & 2024-2 & 15 & d \\
\hline
\end{tabular}
\end{center}
\endgroup

\vfill
\begin{center}
    \small Puedes ver este repositorio en \url{https://github.com/anomvlito/respositorio-fundamentals}
\end{center}

\end{document}
