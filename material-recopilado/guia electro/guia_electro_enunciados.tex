\documentclass{article}
\usepackage{fullpage}
\usepackage{graphicx}
\usepackage[utf8]{inputenc}
\usepackage[T1]{fontenc}
\usepackage[spanish]{babel}
\usepackage{amssymb}
\usepackage{amsmath}
\usepackage{cancel}
\usepackage{booktabs}
\usepackage{tikz}

%%%%% Comandos Personalizados %%%%%
\newcommand{\N}{\mathbb{N}}
\newcommand{\R}{\mathbb{R}}
\newcommand{\Q}{\mathbb{Q}}
\newcommand{\E}{\mathbb{E}}
\newcommand{\PP}{\mathbb{P}}
\newcommand{\la}{\leftarrow}
\newcommand{\ra}{\rightarrow}
\newcommand{\lra}{\leftrightarrow}
\newcommand{\Ra}{\Rightarrow}
\newcommand{\La}{\Leftarrow}
\newcommand{\LRa}{\Leftrightarrow}
\newcommand{\sub}{\subseteq}
\newcommand{\matro}{\mathcal{M}}

\title{Guía de Enunciados - Electromagnetismo}
\author{Generado por Breaking ECF Skill}
\date{\today}

\begin{document}

\maketitle

\section{2016-1}

\subsection*{Pregunta 15 - 2016-1}
\textbf{Enunciado:} Un capacitor de placas paralelas se carga hasta que la diferencia de potencial entre sus placas es $V$. De la placa cargada negativamente, se libera un electrón que es acelerado por el campo eléctrico entre las placas. Calcule la velocidad con la que llega el electrón hasta la otra placa del capacitor ($e$ es el módulo de la carga del electrón y $m$ su masa).

a) $v=\left(\frac{2 e V}{m}\right)^{1 / 2}$ \\
b) $v=\left(\frac{\mathrm{eV}}{\mathrm{m}}\right)^{1 / 2}$ \\
c) $\mathrm{v}=\left(\frac{2 \mathrm{~V}}{\mathrm{~m}}\right)^{1 / 2}$ \\
d) $\mathrm{v}=\frac{\mathrm{eV}}{\mathrm{m}}$

\vspace{0.5cm}

\subsection*{Pregunta 16 - 2016-1}
\textbf{Enunciado:} ¿Cuáles de los circuitos mostrados son filtros pasa altos?

\begin{center}
    \includegraphics[width=0.7\textwidth]{images/FIS1533-2016-1-P16.png}
\end{center}

a) I y II \\
b) I y III \\
c) I y IV \\
d) II y III

\vspace{0.5cm}

\section{2016-2}

\subsection*{Pregunta 17 - 2016-2}
\textbf{Enunciado:} Dos esferas conductoras de radios $R_1=2 \text{ cm}$ y $R_2=4 \text{ cm}$ están conectadas por un cable largo. El campo eléctrico en la superficie de la esfera de 4 cm es $100 \text{ kV/m}$. Calcular el potencial en la esfera de 2 cm.

\begin{center}
    \includegraphics[width=0.5\textwidth]{images/FIS1533-2016-2-P17.png}
\end{center}

a) 2 kV \\
b) 4 kV \\
c) 8 kV \\
d) Faltan datos

\vspace{0.5cm}

\subsection*{Pregunta 18 - 2016-2}
\textbf{Enunciado:} Un núcleo de un transformador está conectado a dos bobinas, A y B, de 20 y 100 vueltas, respectivamente. Si un voltaje alterno de 50 V de promedio (rms) se aplica a la bobina A, ¿cuál será el voltaje promedio (rms) estimado entre ambas bobinas?

\begin{center}
    \includegraphics[width=0.5\textwidth]{images/FIS1533-2016-2-P18.png}
\end{center}

a) 250 V \\
b) 200 V \\
c) 40 V \\
d) 10 V

\vspace{0.5cm}

\subsection*{Pregunta 19 - 2016-2}
\textbf{Enunciado:} Un modelo simple para representar la conducción de electrones en una resistencia consiste en dejar caer una pequeña bola por un plano inclinado que contiene clavos. ¿Qué representa la altura desde donde se deja caer la bola?

a) La energía transportada entre los dos extremos de la resistencia. \\
b) El voltaje aplicado a la resistencia. \\
c) Otros electrones de la resistencia. \\
d) La potencia disipada en la resistencia.

\vspace{0.5cm}

\subsection*{Pregunta 20 - 2016-2}
\textbf{Enunciado:} Un circuito RLC serie con $V_{rms}=35\text{V}$, $f=512\text{Hz}$, $R=148\Omega$, $C=1.5\mu\text{F}$, $L=35.7\text{mH}$. ¿Cuánto vale la potencia disipada en la resistencia?

\begin{center}
    \includegraphics[width=0.5\textwidth]{images/FIS1533-2016-2-P20.png}
\end{center}

a) $4,6 \text{ W}$ \\
b) $6,0 \text{ W}$ \\
c) $8,4 \text{ W}$ \\
d) $29,7 \text{ W}$

\vspace{0.5cm}

\section{2017-1}

\subsection*{Pregunta 17 - 2017-1}
\textbf{Enunciado:} La ley de Gauss aplicada a cuerpos puntuales sería inválida si:

a) la ley del inverso del cuadrado de la distancia no fuera válida. \\
b) la rapidez de la luz en el vació no fuera constante. \\
c) existieran los monopolos magnéticos. \\
d) solo existieran cargas negativas.

\vspace{0.5cm}

\subsection*{Pregunta 18 - 2017-1}
\textbf{Enunciado:} Una cierta cantidad de carga se distribuye en un casquete esférico de permitividad eléctrica $\varepsilon$, radio interno $R_1$ y radio externo $R_2$. La densidad de carga está dada por $\rho(r)= q r$. ¿Cuál es el campo eléctrico entre los radios?

a) $\frac{1}{\varepsilon r}\left[\frac{q}{4}\left(r^3-R_1^3\right)\right]$ \\
b) $\frac{1}{\varepsilon r^2}\left[\frac{q}{4}\left(r^4-R_1^4\right)\right]$ \\
c) $\frac{1}{\varepsilon r}\left[\frac{p}{3} \frac{q}{4}\left(R_2^3-R_1^3\right)\right]$ \\
d) $\frac{1}{\varepsilon r^2}\left[\frac{q}{4}\left(R_2^4-R_1^4\right)\right]$

\vspace{0.5cm}

\subsection*{Pregunta 19 - 2017-1}
\textbf{Enunciado:} Bobina cilíndrica $R=1$ m, $L=1$ m, $N_1=100$. Segunda bobina coaxial interna $r=10$ cm, $N_2=10$. ¿Inducción mutua?

a) $40 \mu \mathrm{H}$ \\
b) $400 \mu \mathrm{H}$ \\
c) $4000 \mu \mathrm{H}$ \\
d) No se puede calcular.

\vspace{0.5cm}

\subsection*{Pregunta 20 - 2017-1}
\textbf{Enunciado:} Circuito LRC serie, voltaje max 220 V, $R=100 \Omega$, $C=100 \text{ nF}$, $L=10 \text{ nH}$. ¿Cuál afirmación es correcta?

a) La suma de los voltajes en R, C y L es igual al voltaje fuente. \\
b) La frecuencia de resonancia es de 5 MHz. \\
c) La potencia disipada es de 484 W. \\
d) La corriente es de 2,2 A.

\vspace{0.5cm}

\section{2017-2}

\subsection*{Pregunta 17 - 2017-2}
\textbf{Enunciado:} ¿Qué es CORRECTO afirmar respecto de la corriente eléctrica?

a) Voltaje producido por cargas polarizadas. \\
b) Flujo de campo eléctrico por cargas en movimiento. \\
c) Tasa de movimiento de cargas eléctricas en el tiempo debido a un campo eléctrico. \\
d) Intensidad de electrones por unidad de área por unidad de tiempo.

\vspace{0.5cm}

\subsection*{Pregunta 18 - 2017-2}
\textbf{Enunciado:} ¿Por qué los pararrayos poseen forma lineal y vertical?

a) Distribución uniforme de energía. \\
b) Densidad de líneas de campo eléctrico se maximiza alrededor (efecto de puntas). \\
c) Simetría cilíndrica de líneas de potencial. \\
d) Cargas siguen caminos paralelos.

\vspace{0.5cm}

\subsection*{Pregunta 19 - 2017-2}
\textbf{Enunciado:} Televisor de rayos catódicos acelera electrones carga $Q$ con voltaje $V$. ¿Energía de impacto?

a) $Q V d$ \\
b) $Q V / d$ \\
c) $Q V$ \\
d) $V / Q$

\vspace{0.5cm}

\subsection*{Pregunta 20 - 2017-2}
\textbf{Enunciado:} Se introduce una inductancia (L) en serie a un circuito resistivo original (AC). ¿Qué sucede con la corriente original?

\begin{center}
    \includegraphics[width=0.5\textwidth]{images/FIS1533-2017-2-P20.png}
\end{center}

a) Disminuye su intensidad. \\
b) Aumenta su intensidad. \\
c) Cambia su frecuencia. \\
d) No cambia.

\vspace{0.5cm}

\section{2018-1}

\subsection*{Pregunta 17 - 2018-1}
\textbf{Enunciado:} Plano conductor cargado negativamente y esfera positiva. ¿Cuál afirmación es CORRECTA sobre el campo eléctrico en los puntos indicados?

\begin{center}
    \includegraphics[width=0.5\textwidth]{images/FIS1533-2018-1-P17.png}
\end{center}

a) En C es vertical y apunta hacia arriba. \\
b) En A es vertical y apunta hacia abajo. \\
c) En B es vertical y apunta hacia arriba. \\
d) En B es nulo.

\vspace{0.5cm}

\subsection*{Pregunta 18 - 2018-1}
\textbf{Enunciado:} Generador 100 A a 4 kV. Se eleva a 200 kV para transmisión por línea de $30 \Omega$. ¿Porcentaje de pérdida de potencia?

a) $0.002 \%$ \\
b) $0.015 \%$ \\
c) $0.030 \%$ \\
d) $0.060 \%$

\vspace{0.5cm}

\subsection*{Pregunta 19 - 2018-1}
\textbf{Enunciado:} Puente de Wheatstone. ¿Cuánto vale $R_x$ para que el galvanómetro no mida corriente (equilibrio)?

\begin{center}
    \includegraphics[width=0.5\textwidth]{images/FIS1533-2018-1-P19.png}
\end{center}

a) $R_1 \cdot R_2 / R_3$ \\
b) $R_3 \cdot R_1 / R_2$ \\
c) $R_1 \cdot R_3 / R_2$ \\
d) $R_3 \cdot R_2 / R_1$

\vspace{0.5cm}

\subsection*{Pregunta 20 - 2018-1}
\textbf{Enunciado:} El campo eléctrico corresponde a:

a) una propiedad de los cuerpos para interaccionar eléctricamente. \\
b) la fuerza ejercida por una carga positiva en un punto del espacio. \\
c) una propiedad del espacio y es causa de la interacción entre cargas. \\
d) la fuerza experimentada por una carga positiva en una región del espacio.

\vspace{0.5cm}

\section{2018-2}

\subsection*{Pregunta 15 - 2018-2}
\textbf{Enunciado:} Capacitor placas paralelas área $A$, distancia $d$. ¿Condición necesaria para funcionamiento correcto (modelo ideal/infinito)?

a) $A \rightarrow \infty$ \\
b) $\sqrt{ A } \gg d$ \\
c) $d \rightarrow 0$ \\
d) $A \ll d$

\vspace{0.5cm}

\subsection*{Pregunta 16 - 2018-2}
\textbf{Enunciado:} Conductor esférico radio $r$, carga $Q_r$ y otro radio $R$, carga $Q_R$, separados por medio $\varepsilon$. ¿Potencial en el punto medio?

\begin{center}
    \includegraphics[width=0.5\textwidth]{images/FIS1533-2018-2-P16.png}
\end{center}

a) $\displaystyle \frac{1}{4 \pi \varepsilon}\left[\frac{Q_R}{R}+\frac{2 Q_r}{R+r}\right]$ \\
b) $\displaystyle \frac{1}{4 \pi \varepsilon}\left[\frac{Q_r}{R}+\frac{2 Q_R}{R+r}\right]$ \\
c) $\displaystyle \frac{1}{4 \pi \varepsilon}\left[\frac{Q_R}{R}-\frac{2 Q_r}{R+r}\right]$ \\
d) $\displaystyle \frac{1}{4 \pi \varepsilon}\left[\frac{Q_R}{R}+\frac{Q_r}{R+r}\right]$

\vspace{0.5cm}

\subsection*{Pregunta 17 - 2018-2}
\textbf{Enunciado:} Barra conductora 10 cm desliza a 10 m/s en campo 0.1 T. Resistencia $10 \Omega$. ¿Corriente inducida?

\begin{center}
    \includegraphics[width=0.5\textwidth]{images/FIS1533-2018-2-P17.png}
\end{center}

a) $0,01 \mathrm{~mA}$ \\
b) $0,1 \mathrm{~mA}$ \\
c) $1 \mathrm{~mA}$ \\
d) $10 \mathrm{~mA}$

\vspace{0.5cm}

\subsection*{Pregunta 18 - 2018-2}
\textbf{Enunciado:} Circuito tipo Wheatstone. Potencia disipada en $R_x$ si el galvanómetro G mide corriente nula.

\begin{center}
    \includegraphics[width=0.5\textwidth]{images/FIS1533-2018-2-P18.png}
\end{center}

a) $\displaystyle \frac{V^2 R_1 R_3}{R_2\left(R_1+R_3\right)^2}$ \\
b) $\displaystyle \frac{V^2 R_2 R_3}{R_1\left(R_1+R_2\right)^2}$ \\
c) $\displaystyle \frac{V^2 R_1 R_2}{R_3\left(R_1+R_2\right)^2}$ \\
d) $\displaystyle \frac{V^2 R_1 R_3}{R_2\left(R_1+R_2\right)^2}$

\vspace{0.5cm}

\section{2019-1}

\subsection*{Pregunta 14 - 2019-1}
\textbf{Enunciado:} Densidad lineal relativa de líneas de campo eléctrico $\mu_2 / \mu_1$ en radios $R_2=3R_1$.

a) $1 / 9$ \\
b) $1 / 3$ \\
c) 3 \\
d) 9

\vspace{0.5cm}

\subsection*{Pregunta 15 - 2019-1}
\textbf{Enunciado:} Capacitor de placas paralelas. Afirmación SIEMPRE correcta:

a) La energía eléctrica en cada placa es aprox. la misma. \\
b) El campo eléctrico disminuye si se introduce un dieléctrico. \\
c) El potencial eléctrico no depende de la distancia. \\
d) Entre las placas existe un traspaso de cargas.

\vspace{0.5cm}

\subsection*{Pregunta 16 - 2019-1}
\textbf{Enunciado:} Experimento con ampolletas en serie y paralelo. Observación plausible:

a) En serie, la más cerca de la fuente brilla más. \\
b) En paralelo brillan igual y tienen la misma corriente. \\
c) Al aumentar voltaje, aumenta brillo y disminuye corriente. \\
d) Al disminuir voltaje dejan de brillar, pero aún se mide corriente.

\vspace{0.5cm}

\subsection*{Pregunta 17 - 2019-1}
\textbf{Enunciado:} Circuito RLC ($L=0,6 H ; R=250 \Omega ; C=3,5 \mu F$), $f=60$ Hz. Ángulo de fase.

a) $\tan^{-1}(-0,47)$ \\
b) $\tan^{-1}(0,90)$ \\
c) $\tan^{-1}(-2,13)$ \\
d) $\tan^{-1}(3,03)$

\vspace{0.5cm}

\section{2019-2}

\subsection*{Pregunta 14 - 2019-2}
\textbf{Enunciado:} La ley de Gauss establece que la carga encerrada es proporcional a:

a) intensidad líneas campo en superficie. \\
b) intensidad campo en superficie. \\
c) flujo de líneas de campo (conceptualmente). \\
d) el flujo de campo eléctrico que atraviesa la superficie.

\vspace{0.5cm}

\subsection*{Pregunta 15 - 2019-2}
\textbf{Enunciado:} Líneas de campo y equipotenciales. ¿Relación CORRECTA?

a) Tangentes. \\
b) Perpendiculares. \\
c) Se intersectan si cargas no puntuales. \\
d) Se intersectan en un punto si cargas puntuales.

\vspace{0.5cm}

\subsection*{Pregunta 16 - 2019-2}
\textbf{Enunciado:} Se tienen 4 cargas puntuales unidas por líneas (ver figura). ¿Qué representan las líneas?

a) Líneas de fuerza eléctrica. \\
b) Trayectorias equipotenciales. \\
c) Líneas de voltaje. \\
d) Líneas de campo eléctrico.

\vspace{0.5cm}

\subsection*{Pregunta 17 - 2019-2}
\textbf{Enunciado:} Dos cargas puntuales $Q$ separadas $d$. ¿Potencial en punto medio?

a) $\frac{Q}{\pi \varepsilon d}$ \\
b) $\frac{Q}{2 \pi \varepsilon d}$ \\
c) $\frac{Q}{4 \pi \varepsilon d}$ \\
d) $0$

\vspace{0.5cm}

\section{2023-2}

\subsection*{Pregunta 22 - 2023-2}
\textbf{Enunciado:} Dipolo eléctrico (cargas $+q, -q$) rodeado por superficie gaussiana. ¿Cuál afirmación es correcta respecto al flujo eléctrico?

a) Solo una forma de superficie da flujo no nulo. \\
b) Solo una forma de superficie da flujo nulo. \\
c) Para cualquier superficie el flujo es distinto de cero. \\
d) Para cualquier superficie que encierre a ambas cargas, el flujo es cero.

\vspace{0.5cm}

\subsection*{Pregunta 23 - 2023-2}
\textbf{Enunciado:} Cargas fijas $+Q, -Q$ separadas $2L$. Carga prueba $q$ a distancia horizontal $L$ del eje central. ¿Fuerza?

a) $k Q q / 2 L$ \\
b) $k Q q / L$ \\
c) $k Q q / 2 \sqrt{2} L^2$ \\
d) $k Q q / \sqrt{2} L^2$

\vspace{0.5cm}

\subsection*{Pregunta 24 - 2023-2}
\textbf{Enunciado:} Carga $q, m$ entra a $45^\circ$ con velocidad $v$. Placas $\pm V$ separadas $d$. ¿Largo $L$ para que salga horizontal?

a) $\frac{v^2 d m}{q V}$ \\
b) $\frac{v^2 d m}{2 q V}$ \\
c) $\frac{v^2 d m}{4 q V}$ \\
d) $\frac{v^2 d m}{8 q V}$

\vspace{0.5cm}

\subsection*{Pregunta 25 - 2023-2}
\textbf{Enunciado:} Solenoide con núcleo de sección $A$, flujo $\phi=\phi_0 \sin(t)$. ¿Cuál es el voltaje inducido?

a) $V=-\phi_0 A \cos (t)$ \\
b) $V=\phi_0 \cos (t)$ \\
c) $V=-5 \phi_0 A \cos (t)$ \\
d) $V=5 \phi_0 \cos (t)$

\vspace{0.5cm}

\subsection*{Pregunta 26 - 2023-2}
\textbf{Enunciado:} Cable enrollado en cilindro radio $R$, rotando a $N$ rev/s en campo $B$. ¿Diferencia de potencial?

a) $2 \pi N B R$ \\
b) $\pi N B R^2$ \\
c) $\pi B R^2 / N$ \\
d) $2 \pi N R / B$

\vspace{0.5cm}

\subsection*{Pregunta 27 - 2023-2}
\textbf{Enunciado:} Dada la topología del circuito (ver figura), aplicar LVK. ¿Cuál ecuación es correcta?

a) $v_7+v_1+v_4-v_9-v_8=0$ \\
b) $-v_6+v_2+v_3-v_9+v_8=0$ \\
c) $-v_4+v_9+v_8-v_6=0$ \\
d) $v_7-v_1+v_5+v_8=0$

\vspace{0.5cm}

\section{2024-2}

\subsection*{Pregunta 22 - 2024-2}
\textbf{Enunciado:} Se tienen cargas puntuales distribuidas alrededor de un punto O a distintas distancias. Se traza una superficie gaussiana esférica de radio $1.7r$. ¿El campo eléctrico es proporcional a qué?

\begin{center}
    \includegraphics[width=0.45\textwidth]{images/FIS1533-2023-2-P1-5.png}
\end{center}

a) Proporcional a $q$. \\
b) Proporcional a $-q$. \\
c) Proporcional a $2q$. \\
d) Nula.

\vspace{0.5cm}

\subsection*{Pregunta 23 - 2024-2}
\textbf{Enunciado:} Una carga negativa con masa se mueve en presencia de un campo eléctrico (ver figura). ¿Cuál trayectoria describe?

\begin{center}
    \includegraphics[width=0.45\textwidth]{images/FIS1533-2024-2-P2-5.png}
\end{center}

a) A \\
b) B \\
c) C \\
d) D

\vspace{0.5cm}

\subsection*{Pregunta 24 - 2024-2}
\textbf{Enunciado:} Capacitor formado por una grilla de $2 \times 3$ bloques dieléctricos: cinco bloques de permitividad $\varepsilon_1$ y uno de $\varepsilon_2$ (posición superior central). ¿Capacitancia equivalente?

\begin{center}
    \includegraphics[width=0.45\textwidth]{images/FIS1533-2023-2-P4-1.png}
\end{center}

a) $(1 + \dots) C_1$ \\
b) $(\frac{1}{2} + \dots) C_1$ \\
c) $(\frac{6C_1+3C_2}{5C_1+C_2}) C_1$ (aprox) \\
d) Forma compleja.

\vspace{0.5cm}

\subsection*{Pregunta 25 - 2024-2}
\textbf{Enunciado:} Un inductor es recorrido por una corriente periódica. ¿Cuál es el gráfico de voltaje correspondiente?

a) A \\
b) B \\
c) C \\
d) D

\vspace{0.5cm}

\subsection*{Pregunta 26 - 2024-2}
\textbf{Enunciado:} Circuito resistivo con fuente DC $V_0$. ¿Cuánto vale la corriente $i$?

\begin{center}
    \includegraphics[width=0.55\textwidth]{images/FIS1533-2023-2-P6-3.png}
\end{center}

a) $\frac{1}{5} \frac{V_0}{R}$ \\
b) $\frac{1}{4} \frac{V_0}{R}$ \\
c) $\frac{1}{2} \frac{V_0}{R}$ \\
d) $\frac{2}{7} \frac{V_0}{R}$

\vspace{0.5cm}

\subsection*{Pregunta 27 - 2024-2}
\textbf{Enunciado:} Red de 6 resistencias iguales $R$ (ver figura). ¿Resistencia equivalente entre los terminales $a$ y $b$?

a) $9/4 \, R$ \\
b) $13/5 \, R$ \\
c) $10/3 \, R$ \\
d) $9/2 \, R$

\vspace{0.5cm}

\end{document}
