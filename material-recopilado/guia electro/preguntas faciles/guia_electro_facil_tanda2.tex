\documentclass{article}
\usepackage{fullpage}
\usepackage{graphicx}
\usepackage[utf8]{inputenc}
\usepackage[T1]{fontenc}
\usepackage[spanish]{babel}
\usepackage{amssymb}
\usepackage{amsmath}
\usepackage{cancel}
\usepackage{booktabs}
\usepackage{url}
\usepackage{tikz}
\usetikzlibrary{arrows.meta}
\usepackage{tcolorbox}

%%%%% Comandos Personalizados %%%%%
\newcommand{\N}{\mathbb{N}}
\newcommand{\R}{\mathbb{R}}
\newcommand{\Q}{\mathbb{Q}}
\newcommand{\E}{\mathbb{E}}
\newcommand{\PP}{\mathbb{P}}
\newcommand{\la}{\leftarrow}
\newcommand{\ra}{\rightarrow}
\newcommand{\lra}{\leftrightarrow}
\newcommand{\Ra}{\Rightarrow}
\newcommand{\La}{\Leftarrow}
\newcommand{\LRa}{\Leftrightarrow}
\newcommand{\sub}{\subseteq}
\newcommand{\matro}{\mathcal{M}}
%%%%% Fin Comandos Personalizados %%%%%

\title{Guía de Victorias Rápidas -- Electromagnetismo\\[0.3cm]
\large Tanda 2: Consolidación y Primer Salto -- Soluciones}
\author{}
\date{\today}

\begin{document}

\maketitle

\begin{tcolorbox}[colback=blue!5!white, colframe=blue!50!black, title=Plan de Estudio -- Tanda 2]
\textbf{Objetivo:} Consolidar los fundamentos de la Tanda 1 y dar el primer paso hacia problemas con un poco más de cálculo.
\end{tcolorbox}

\newpage

%% ============================================================
%% EJERCICIO 1
%% ============================================================

\section*{Ejercicio 1 -- Circuitos RC y RL (Filtros pasa altos) \normalsize{(Análisis en frecuencia)}}
\textit{Fuente: Pregunta 16 -- 2016-1}

\subsection*{Enunciado}
¿Cuáles de los circuitos mostrados son filtros pasa altos?

\begin{center}
    \includegraphics[width=0.65\textwidth]{../images/FIS1533-2016-1-P16.png}
\end{center}

\begin{enumerate}
    \item[a)] I y II.
    \item[b)] I y III.
    \item[c)] I y IV.
    \item[d)] II y III.
\end{enumerate}

\subsection*{Solución paso a paso}

\textbf{Paso 1: Concepto de filtro pasa altos}

Un filtro pasa alto permite el paso de señales de alta frecuencia y atenúa drásticamente las señales de baja frecuencia. Para lograrlo, debemos analizar la impedancia.
\begin{itemize}
    \item Inductor: $Z_L = j\omega L$ (A baja frecuencia es un cable; a alta frecuencia se opone al paso de corriente fuertemente).
    \item Capacitor: $Z_C = 1/j\omega C$ (A baja frecuencia bloquea; a alta frecuencia es un cable corto).
\end{itemize}

\textbf{Paso 2: Análisis por circuito}

\begin{itemize}
    \item \textbf{I.} La salida está en paralelo a $L$ no, el dibujo muestra un R y $L$ en serie con salida paralela. A alta frecuencia la bobina bloquea. \textbf{Pasa bajos}.
    \item \textbf{II.} La salida está en paralelo a $L$ y $R$ está en serie. A baja frecuencia, $L$ cortocircuita la señal. A alta frecuencia, $L$ actúa como cable abierto favoreciendo voltaje en salida. \textbf{Pasa altos}.
    \item \textbf{III.} La salida está en paralelo a $R$ con $C$ en serie. A baja frecuencia, $C$ bloquea; a alta frecuencia $C$ permite paso directo. \textbf{Pasa altos}.
    \item \textbf{IV.} La salida está en paralelo a $C$. A alta frecuencia, $C$ se convierte en un cortocircuito que anula el voltaje. \textbf{Pasa bajos}.
\end{itemize}

\textbf{Paso 3: Conclusión}

Los circuitos que son pasa altos son II y III.

\[
\boxed{\text{Respuesta: d)}}
\]

\noindent\fbox{%
    \parbox{\textwidth}{%
        \textbf{¡Lo que dice el Handbook FE!} Recuerda que en el examen no necesitas memorizar esto:
        \begin{itemize}
            \item \textbf{Analog Filter Circuits (Pág. 379):} Detalla el comportamiento y funciones de transferencia para filtros de primer orden.
            \item \textbf{Impedance Table (Pág. 361):} Muestra comportamientos frecuenciales de los componentes.
        \end{itemize}
    }%
}

\vspace{1cm}

%% ============================================================
%% EJERCICIO 2
%% ============================================================

\section*{Ejercicio 2 -- ¿Por qué los pararrayos tienen punta?}
\textit{Fuente: Pregunta 18 -- 2017-2}

\subsection*{Enunciado}
¿Por qué los pararrayos poseen forma lineal y vertical?

\begin{enumerate}
    \item[a)] Distribución uniforme de energía.
    \item[b)] La densidad de líneas de campo eléctrico se maximiza alrededor (efecto de puntas).
    \item[c)] Simetría cilíndrica de líneas de potencial.
    \item[d)] Las cargas siguen caminos paralelos.
\end{enumerate}

\subsection*{Solución paso a paso}

\textbf{Paso 1: El Efecto de Puntas}

En un conductor en \textbf{equilibrio electrostático}, las cargas se distribuyen en la superficie. Pero no uniformemente: la densidad superficial de carga $\sigma$ es \textbf{mayor} donde el radio de curvatura es \textbf{menor} (las puntas).

\textbf{¿Por qué?} Imagina que comprimes las cargas de una esfera grande en una esfera muy pequeña: la misma carga en menos área $\Rightarrow$ más densidad.

\textbf{Paso 2: De $\sigma$ al campo eléctrico}

En la superficie de un conductor:
\[
E = \frac{\sigma}{\varepsilon_0}
\]

Si $\sigma$ es máxima en la punta, entonces $E$ también es \textbf{máximo} en la punta. Este campo intenso:
\begin{enumerate}
    \item \textbf{Ioniza} el aire circundante (arranca electrones de las moléculas)
    \item Crea un \textbf{canal conductor} (plasma)
    \item El rayo sigue preferentemente este camino de mínima resistencia
\end{enumerate}

\textbf{Paso 3: Las otras opciones}
\begin{itemize}
    \item \textbf{a)} ``Distribución uniforme de energía'' es lo \emph{opuesto} a lo que ocurre -- la energía se concentra en la punta.
    \item \textbf{c)} No hay ``simetría cilíndrica de líneas de potencial'' relevante aquí.
    \item \textbf{d)} Las cargas no siguen caminos paralelos; siguen el gradiente de potencial.
\end{itemize}

\[
\boxed{\text{Respuesta: b)}}
\]

\noindent\fbox{%
    \parbox{\textwidth}{%
        \textbf{¡Lo que dice el Handbook FE!} Recuerda que en el examen no necesitas memorizar esto:
        \begin{itemize}
            \item \textbf{Electrostatic Fields (Pág. 355):} $E_s = \rho_s / (2\varepsilon)$. En superficies curvas, la carga se concentra en zonas de alta curvatura (puntas).
        \end{itemize}
    }%
}

\vspace{1cm}

%% ============================================================
%% EJERCICIO 3
%% ============================================================

\section*{Ejercicio 3 -- Dipolo dentro de superficie gaussiana}
\textit{Fuente: Pregunta 22 -- 2023-2}

\subsection*{Enunciado}
Un dipolo eléctrico (cargas $+q$ y $-q$) está rodeado por una superficie gaussiana. ¿Cuál afirmación es correcta respecto al flujo eléctrico?

\begin{enumerate}
    \item[a)] Solo una forma de superficie da flujo no nulo.
    \item[b)] Solo una forma de superficie da flujo nulo.
    \item[c)] Para cualquier superficie el flujo es distinto de cero.
    \item[d)] Para cualquier superficie que encierre a ambas cargas, el flujo es cero.
\end{enumerate}

\subsection*{Solución paso a paso}

\textbf{Paso 1: Carga neta del dipolo}
\[
Q_{neta} = (+q) + (-q) = 0
\]

\textbf{Paso 2: Aplicamos Gauss directamente}
\[
\Phi_E = \frac{Q_{enc}}{\varepsilon_0} = \frac{0}{\varepsilon_0} = 0
\]

\textbf{Paso 3: ¿Importa la forma de la superficie?}

¡NO! La Ley de Gauss dice que el flujo depende \textbf{únicamente} de la carga encerrada, sin importar la forma de la superficie. Puede ser esférica, cúbica, irregular -- siempre que encierre a ambas cargas, el flujo es cero.

\textbf{Paso 4: Cuidado con un error conceptual}

Que el flujo sea cero NO significa que el campo sea cero en la superficie. El campo del dipolo es complicado y no nulo. Lo que pasa es que la misma cantidad de ``líneas de campo'' que \emph{salen} de la superficie, \emph{entran} por otro lado. El flujo neto (salida $-$ entrada) es cero.

\[
\boxed{\text{Respuesta: d)}}
\]

\noindent\fbox{%
    \parbox{\textwidth}{%
        \textbf{¡Lo que dice el Handbook FE!} Recuerda que en el examen no necesitas memorizar esto:
        \begin{itemize}
            \item \textbf{Gauss' Law (Pág. 355):} Si $Q_{encl} = 0$, el flujo neto es nulo para \emph{cualquier} superficie cerrada.
        \end{itemize}
    }%
}

\vspace{1cm}

%% ============================================================
%% EJERCICIO 4
%% ============================================================

\section*{Ejercicio 4 -- Gauss con distribución de cargas}
\textit{Fuente: Pregunta 22 -- 2024-2}

\subsection*{Enunciado}
Cargas puntuales distribuidas alrededor de un punto O. Superficie gaussiana esférica de radio $1{,}7r$. ¿El flujo es proporcional a qué?

\begin{center}
    \includegraphics[width=0.45\textwidth]{../images/FIS1533-2023-2-P1-5.png}
\end{center}

\begin{enumerate}
    \item[a)] Proporcional a $q$.
    \item[b)] Proporcional a $-q$.
    \item[c)] Proporcional a $2q$.
    \item[d)] Nula.
\end{enumerate}

\subsection*{Solución paso a paso}

\textbf{Paso 1: Identificar qué cargas están DENTRO}

De la figura, las cargas y sus distancias al centro O son:
\begin{itemize}
    \item A distancia $1{,}5r$: una carga \textbf{negativa} ($-q$) $\leftarrow$ \textbf{DENTRO} (pues $1{,}5r < 1{,}7r$)
    \item A distancia $2r$: cargas positivas $\leftarrow$ FUERA ($2r > 1{,}7r$)
    \item A distancia $2{,}5r$, $3r$, $4r$: más cargas $\leftarrow$ todas FUERA
\end{itemize}

\textbf{Paso 2: Aplicamos Gauss}

Solo la carga a $1{,}5r$ está dentro de la superficie de radio $1{,}7r$:
\[
Q_{enc} = -q
\]
\[
\Phi_E = \frac{Q_{enc}}{\varepsilon_0} = \frac{-q}{\varepsilon_0} \propto -q
\]

\textbf{Paso 3: Lección clave}

Las cargas \textbf{fuera} de la superficie gaussiana NO contribuyen al flujo neto. Sí afectan el campo $\vec{E}$ en puntos individuales de la superficie, pero cuando integras sobre toda la superficie cerrada, su contribución se cancela (lo que entra = lo que sale).

\[
\boxed{\text{Respuesta: b)}}
\]

\noindent\fbox{%
    \parbox{\textwidth}{%
        \textbf{¡Lo que dice el Handbook FE!} Recuerda que en el examen no necesitas memorizar esto:
        \begin{itemize}
            \item \textbf{Gauss' Law (Pág. 355):} Solo las cargas \emph{dentro} de la superficie contribuyen a $Q_{encl}$.
        \end{itemize}
    }%
}

\vspace{1cm}

%% ============================================================
%% EJERCICIO 5
%% ============================================================

\section*{Ejercicio 5 -- Modelo bola-clavos (Drude simplificado)}
\textit{Fuente: Pregunta 19 -- 2016-2}

\subsection*{Enunciado}
Modelo de bola cayendo por plano inclinado con clavos (conducción electrónica). ¿Qué representa la \emph{altura}?

\begin{enumerate}
    \item[a)] La energía transportada entre los dos extremos de la resistencia.
    \item[b)] El voltaje aplicado a la resistencia.
    \item[c)] Otros electrones de la resistencia.
    \item[d)] La potencia disipada en la resistencia.
\end{enumerate}

\subsection*{Solución paso a paso}

\textbf{Paso 1: La analogía mecánica-eléctrica}

\begin{center}
\begin{tabular}{l|l}
\textbf{Modelo mecánico} & \textbf{Circuito eléctrico} \\
\hline
Bola & Electrón \\
Clavos & Iones de la red cristalina \\
Altura $h$ & ¿? \\
Gravedad $g$ & Campo eléctrico $E$ \\
Energía potencial $mgh$ & Energía potencial $qV$
\end{tabular}
\end{center}

\textbf{Paso 2: ¿Qué es la altura en mecánica?}

La altura define la \textbf{energía potencial por unidad de masa}: $gh$. Es lo que ``impulsa'' a la bola hacia abajo.

\textbf{Paso 3: ¿Cuál es el equivalente eléctrico?}

El concepto de ``energía potencial por unidad de carga'' en electricidad es exactamente el \textbf{voltaje} (diferencia de potencial):
\[
V = \frac{W}{Q} = \frac{\text{energía potencial eléctrica}}{\text{carga}}
\]

Así como la altura ``empuja'' la bola, el voltaje ``empuja'' a los electrones.

\textbf{Paso 4: ¿Por qué no energía (opción a)?}

La \emph{energía} sería $mgh$ (no solo $h$) o $qV$ (no solo $V$). La altura por sí sola es energía \emph{por unidad de masa}, análoga al voltaje (energía por unidad de carga), no a la energía total.

\[
\boxed{\text{Respuesta: b)}}
\]

\noindent\fbox{%
    \parbox{\textwidth}{%
        \textbf{¡Lo que dice el Handbook FE!} Recuerda que en el examen no necesitas memorizar esto:
        \begin{itemize}
            \item \textbf{Voltage (Pág. 356):} $V = W/Q$. El voltaje es la diferencia de potencial = trabajo por unidad de carga.
        \end{itemize}
    }%
}

\vspace{1cm}

%% ============================================================
%% EJERCICIO 6
%% ============================================================

\section*{Ejercicio 6 -- Condición de idealidad del capacitor}
\textit{Fuente: Pregunta 15 -- 2018-2}

\subsection*{Enunciado}
Capacitor de placas paralelas, área $A$, separación $d$. ¿Condición para modelo ideal?

\begin{enumerate}
    \item[a)] $A \rightarrow \infty$
    \item[b)] $\sqrt{A} \gg d$
    \item[c)] $d \rightarrow 0$
    \item[d)] $A \ll d$
\end{enumerate}

\subsection*{Solución paso a paso}

\textbf{Paso 1: ¿Qué asume el modelo ideal?}

El modelo ideal de capacitor de placas paralelas asume:
\begin{itemize}
    \item Campo \textbf{uniforme} entre las placas
    \item Campo \textbf{nulo} fuera de las placas
    \item Sin ``efectos de borde'' (el campo no se ``escapa'' por los lados)
\end{itemize}

\textbf{Paso 2: ¿Cuándo se cumple?}

Para que los efectos de borde sean despreciables, las dimensiones de las placas deben ser \textbf{mucho mayores} que la separación. Si las placas son ``infinitas'' comparadas con $d$, los bordes están tan lejos que no importan.

\textbf{Paso 3: ¿Cómo comparar?}

El área $A$ tiene unidades de m², la separación $d$ tiene unidades de m. Para compararlos, necesitamos una longitud: $\sqrt{A}$ (la ``dimensión característica'' de la placa). La condición es:
\[
\boxed{\sqrt{A} \gg d}
\]

\textbf{Paso 4: ¿Por qué no a) ni c)?}
\begin{itemize}
    \item \textbf{a)} $A \to \infty$ es suficiente pero innecesaria. No necesitas placas infinitas, solo que $\sqrt{A}$ sea ``mucho mayor'' que $d$.
    \item \textbf{c)} $d \to 0$ no garantiza nada si $A$ también es pequeña.
\end{itemize}

La clave es la \textbf{razón} $\sqrt{A}/d$, no el valor absoluto de $A$ ni $d$.

\[
\boxed{\text{Respuesta: b)}}
\]

\noindent\fbox{%
    \parbox{\textwidth}{%
        \textbf{¡Lo que dice el Handbook FE!} Recuerda que en el examen no necesitas memorizar esto:
        \begin{itemize}
            \item \textbf{Capacitors (Pág. 358):} $C = \varepsilon A/d$. Fórmula válida cuando $\sqrt{A} \gg d$ (campo uniforme, sin efectos de borde).
        \end{itemize}
    }%
}

\vspace{1cm}

%% ============================================================
%% EJERCICIO 7
%% ============================================================

\section*{Ejercicio 7 -- Dieléctrico en capacitor}
\textit{Fuente: Pregunta 15 -- 2019-1}

\subsection*{Enunciado}
Sobre un capacitor de placas paralelas, ¿cuál afirmación es SIEMPRE correcta?

\begin{enumerate}
    \item[a)] La energía eléctrica en cada placa es aproximadamente la misma.
    \item[b)] El campo eléctrico disminuye si se introduce un dieléctrico.
    \item[c)] El potencial eléctrico no depende de la distancia.
    \item[d)] Entre las placas existe un traspaso de cargas.
\end{enumerate}

\subsection*{Solución paso a paso}

\textbf{Paso 1: ¿Qué hace un dieléctrico?}

Un dieléctrico (material aislante como vidrio, plástico, etc.) insertado entre las placas se \textbf{polariza}: sus moléculas se alinean con el campo externo, creando un campo interno opuesto. El resultado neto:
\[
E_{con\ diel} = \frac{E_0}{\kappa}
\]
donde $\kappa > 1$ es la constante dieléctrica. El campo siempre \textbf{disminuye}.

\textbf{Paso 2: ¿Es SIEMPRE cierto?}

Sí, para carga fija en las placas (caso más general). Si las placas están conectadas a una batería (voltaje fijo), el campo se mantiene igual pero la carga aumenta. Sin embargo, la afirmación ``el campo disminuye'' es correcta en el caso estándar (carga fija), que es el que se asume por defecto.

\textbf{Paso 3: Las otras opciones}
\begin{itemize}
    \item \textbf{a) FALSO:} La energía se almacena en el \textbf{campo entre} las placas ($u = \frac{1}{2}\varepsilon E^2$), no ``en cada placa''.
    \item \textbf{c) FALSO:} $V = E \cdot d$. Claramente depende de la distancia $d$.
    \item \textbf{d) FALSO:} El dieléctrico (o vacío) entre las placas es \textbf{aislante}. No fluye carga entre ellas.
\end{itemize}

\[
\boxed{\text{Respuesta: b)}}
\]

\noindent\fbox{%
    \parbox{\textwidth}{%
        \textbf{¡Lo que dice el Handbook FE!} Recuerda que en el examen no necesitas memorizar esto:
        \begin{itemize}
            \item \textbf{Capacitors (Pág. 358):} $C = \varepsilon_r \varepsilon_0 A/d$. Con dieléctrico ($\varepsilon_r > 1$), $C$ aumenta y, para carga constante, $E$ disminuye.
        \end{itemize}
    }%
}

\vspace{1cm}

%% ============================================================
%% EJERCICIO 8
%% ============================================================

\section*{Ejercicio 8 -- Densidad de líneas de campo $\propto 1/r^2$}
\textit{Fuente: Pregunta 14 -- 2019-1}

\subsection*{Enunciado}
Densidad relativa de líneas de campo $\mu_2/\mu_1$ a distancias $R_2 = 3R_1$ de una carga puntual:

\begin{enumerate}
    \item[a)] $1/9$
    \item[b)] $1/3$
    \item[c)] $3$
    \item[d)] $9$
\end{enumerate}

\subsection*{Solución paso a paso}

\textbf{Paso 1: Relación densidad de líneas $\leftrightarrow$ campo}

La densidad de líneas de campo es una representación visual de la \textbf{magnitud} del campo eléctrico. Donde las líneas están más juntas, $|\vec{E}|$ es mayor.

\textbf{Paso 2: Campo de carga puntual}
\[
E \propto \frac{1}{r^2}
\]

\textbf{Paso 3: Calculamos la razón}
\[
\frac{\mu_2}{\mu_1} = \frac{E(R_2)}{E(R_1)} = \frac{1/R_2^2}{1/R_1^2} = \left(\frac{R_1}{R_2}\right)^2 = \left(\frac{R_1}{3R_1}\right)^2 = \left(\frac{1}{3}\right)^2 = \boxed{\frac{1}{9}}
\]

\textbf{Paso 4: Interpretación}

A distancia triple, la densidad de líneas cae a $1/9$. Las líneas se ``reparten'' sobre una esfera de área $4\pi r^2$: si el radio se triplica, el área se multiplica por 9, y la densidad por línea cae a $1/9$.

\[
\boxed{\text{Respuesta: a)}}
\]

\noindent\fbox{%
    \parbox{\textwidth}{%
        \textbf{¡Lo que dice el Handbook FE!} Recuerda que en el examen no necesitas memorizar esto:
        \begin{itemize}
            \item \textbf{Electrostatic Fields (Pág. 355):} $E = Q/(4\pi\varepsilon r^2)$. La densidad de líneas decae como $1/r^2$.
        \end{itemize}
    }%
}

\vspace{1cm}

%% ============================================================
%% EJERCICIO 9
%% ============================================================

\section*{Ejercicio 9 -- Esferas conductoras conectadas: $V = ER$}
\textit{Fuente: Pregunta 17 -- 2016-2}

\subsection*{Enunciado}
Dos esferas conductoras ($R_1 = 2$ cm, $R_2 = 4$ cm) conectadas por cable. Campo en la superficie de la esfera grande: $E_2 = 100$ kV/m. Potencial en la esfera de 2 cm.

\begin{center}
    \includegraphics[width=0.5\textwidth]{../images/FIS1533-2016-2-P17.png}
\end{center}

\begin{enumerate}
    \item[a)] 2 kV
    \item[b)] 4 kV
    \item[c)] 8 kV
    \item[d)] Faltan datos
\end{enumerate}

\subsection*{Solución paso a paso}

\textbf{Paso 1: Conductores conectados $=$ mismo potencial}

Si dos conductores están conectados por un cable, las cargas se redistribuyen hasta que ambos están al \textbf{mismo potencial}:
\[
V_1 = V_2
\]

Esto es porque un conductor en equilibrio es una equipotencial. Si hubiera diferencia de potencial, habría corriente hasta que se igualaran.

\textbf{Paso 2: Relación $V$ y $E$ para una esfera}

Para una esfera conductora de radio $R$:
\[
E = \frac{kQ}{R^2}, \qquad V = \frac{kQ}{R}
\]

Dividiendo: $V = E \cdot R$. Esta relación es muy útil para esferas.

\textbf{Paso 3: Calculamos}
\[
V_2 = E_2 \cdot R_2 = (100 \times 10^3 \text{ V/m}) \times (0{,}04 \text{ m}) = 4{.}000 \text{ V} = 4 \text{ kV}
\]

Como $V_1 = V_2$:
\[
\boxed{V_1 = 4 \text{ kV}}
\]

\textbf{Paso 4: Dato curioso -- ¿dónde es más fuerte el campo?}

Aunque $V_1 = V_2$, los campos son diferentes:
\[
E_1 = \frac{V_1}{R_1} = \frac{4000}{0{,}02} = 200 \text{ kV/m}
\]

¡El campo en la esfera \textbf{pequeña} es el doble! Esto se conecta con el efecto de puntas (Tanda 2, Ej. 2): menor radio $\Rightarrow$ mayor campo.

\[
\boxed{\text{Respuesta: b)}}
\]

\noindent\fbox{%
    \parbox{\textwidth}{%
        \textbf{¡Lo que dice el Handbook FE!} Recuerda que en el examen no necesitas memorizar esto:
        \begin{itemize}
            \item \textbf{Electrostatic Fields (Pág. 355):} Esfera conductora: $V = kQ/R$, $E = kQ/R^2 = V/R$.
            \item \textbf{Voltage (Pág. 356):} Conductores conectados: $V_1 = V_2$.
        \end{itemize}
    }%
}

\vspace{1cm}

%% ============================================================
%% EJERCICIO 10
%% ============================================================

\section*{Ejercicio 10 -- FEM en barra conductora: $\varepsilon = vBl$}
\textit{Fuente: Pregunta 17 -- 2018-2}

\subsection*{Enunciado}
Barra conductora de 10 cm, velocidad 10 m/s, campo $B = 0{,}1$ T, resistencia $R = 10\ \Omega$. ¿Corriente inducida?

\begin{center}
    \includegraphics[width=0.55\textwidth]{../images/FIS1533-2018-2-P17.png}
\end{center}

\begin{enumerate}
    \item[a)] 0,01 mA
    \item[b)] 0,1 mA
    \item[c)] 1 mA
    \item[d)] 10 mA
\end{enumerate}

\subsection*{Solución paso a paso}

\textbf{Paso 1: ¿Por qué hay FEM?}

Cuando una barra conductora se mueve en un campo magnético, los electrones dentro de la barra sienten una fuerza magnética ($\vec{F} = q\vec{v}\times\vec{B}$) que los empuja hacia un extremo. Esto crea una separación de cargas $\Rightarrow$ diferencia de potencial $\Rightarrow$ FEM.

\textbf{Paso 2: Fórmula de la FEM motional}
\[
\varepsilon = v \cdot B \cdot l
\]

Condición: $\vec{v}$, $\vec{B}$ y $\vec{l}$ deben ser mutuamente perpendiculares.

\textbf{Paso 3: Calculamos la FEM}

¡Cuidado con las unidades! $l = 10$ cm $= 0{,}1$ m:
\[
\varepsilon = 10 \text{ m/s} \times 0{,}1 \text{ T} \times 0{,}1 \text{ m} = 0{,}1 \text{ V}
\]

\textbf{Paso 4: Corriente por Ley de Ohm}
\[
I = \frac{\varepsilon}{R} = \frac{0{,}1 \text{ V}}{10\ \Omega} = 0{,}01 \text{ A} = \boxed{10 \text{ mA}}
\]

\textbf{Paso 5: Verificación de unidades}

$[\text{m/s}] \times [\text{T}] \times [\text{m}] = [\text{m/s}] \times [\text{kg/(A·s²)}] \times [\text{m}] = [\text{kg·m²/(A·s³)}] = [\text{V}]$ \checkmark

\[
\boxed{\text{Respuesta: d)}}
\]

\noindent\fbox{%
    \parbox{\textwidth}{%
        \textbf{¡Lo que dice el Handbook FE!} Recuerda que en el examen no necesitas memorizar esto:
        \begin{itemize}
            \item \textbf{Faraday's Law (Pág. 356):} FEM motional: $\varepsilon = vBl$ (con $\vec{v} \perp \vec{B} \perp \vec{l}$).
            \item \textbf{Ohm's Law (Pág. 357):} $V = IR \Rightarrow I = V/R$.
        \end{itemize}
    }%
}

\vspace{1cm}

%% ============================================================
%% RESUMEN FINAL
%% ============================================================

\section*{Resumen de Conceptos Clave -- Tanda 2}

\begin{tcolorbox}[colback=green!5!white, colframe=green!50!black, title=Lo que deberías dominar después de Tanda 1 + 1B + 2]

\textbf{Conceptuales profundos:}
\begin{enumerate}
    \item Gauss funciona \textbf{porque} $E \propto 1/r^2$ (geometría del espacio 3D).
    \item Efecto de puntas: $\sigma$ mayor $\Rightarrow$ $E$ mayor $\Rightarrow$ ionización $\Rightarrow$ descarga.
    \item Dipolo: $Q_{neta} = 0 \Rightarrow \Phi = 0$ para cualquier superficie.
    \item Solo las cargas \textbf{dentro} de la superficie gaussiana contribuyen al flujo.
    \item Voltaje $\equiv$ energía potencial por unidad de carga (análogo a altura/$g$).
    \item Capacitor ideal: $\sqrt{A} \gg d$ (sin efectos de borde).
    \item Dieléctrico: siempre reduce $E$ (para carga fija): $E = E_0/\kappa$.
\end{enumerate}

\textbf{Cálculos de 1--2 pasos:}
\begin{enumerate}
    \setcounter{enumi}{7}
    \item $\mu \propto E \propto 1/r^2$: a distancia triple, densidad $\times 1/9$.
    \item Esferas conectadas: $V = ER$, mismo potencial, campo mayor en la pequeña.
    \item FEM motional: $\varepsilon = vBl$, luego $I = \varepsilon/R$.
\end{enumerate}
\end{tcolorbox}

\begin{tcolorbox}[colback=red!5!white, colframe=red!50!black, title=Mapa de progreso]
\textbf{Tanda 1} (7 ejercicios): Definiciones + 1 fórmula $\checkmark$\\
\textbf{Tanda 1B} (10 ejercicios): Conceptuales profundos + circuitos básicos $\checkmark$\\
\textbf{Tanda 2} (10 ejercicios): Gauss aplicado + cálculos de 1--2 pasos $\checkmark$\\[0.3cm]
\textbf{Total: 27 ejercicios resueltos de los 42 de la guía} (64\% de cobertura).\\
Los 15 restantes son de mayor complejidad (RLC multi-paso, integrales, redes de resistencias).
\end{tcolorbox}

\vfill
\begin{center}
    \small Puedes ver este repositorio en \url{https://github.com/anomvlito/respositorio-fundamentals}
\end{center}

\end{document}
