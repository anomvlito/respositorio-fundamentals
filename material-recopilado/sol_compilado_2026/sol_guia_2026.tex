\documentclass{article}
\usepackage{fullpage}
\usepackage{graphicx}
\usepackage[utf8]{inputenc}
\usepackage[T1]{fontenc}
\usepackage[spanish]{babel}
\usepackage{amssymb}
\usepackage{amsmath}
\usepackage{cancel}
\usepackage{booktabs}
\usepackage{tikz}
\usepackage{float}
\usepackage{url}
\usetikzlibrary{arrows.meta}

%%%%% Comandos Personalizados %%%%%
\newcommand{\N}{\mathbb{N}}
\newcommand{\R}{\mathbb{R}}
\newcommand{\Q}{\mathbb{Q}}
\newcommand{\E}{\mathbb{E}}
\newcommand{\PP}{\mathbb{P}}
\newcommand{\la}{\leftarrow}
\newcommand{\ra}{\rightarrow}
\newcommand{\lra}{\leftrightarrow}
\newcommand{\Ra}{\Rightarrow}
\newcommand{\La}{\Leftarrow}
\newcommand{\LRa}{\Leftrightarrow}
\newcommand{\sub}{\subseteq}
\newcommand{\matro}{\mathcal{M}}

\newcommand{\twopartdef}[4]
{
	\left\{
		\begin{array}{ll}
			#1 &  \text{#2} \\
			#3 &  \text{#4}
		\end{array}
	\right.
}

%%%%%  Fin Comandos Personalizados %%%%%

%%%%%%%%%% MODIFICAR %%%%%%%%%%
\newcommand{\alumnos}{Solucionario Generado}
\newcommand{\departamento}{Departamento de Ingeniería}
\newcommand{\ramo}{Examen de Competencias Fundamentales}
\newcommand{\sigla}{ECF}
\newcommand{\titulo}{Solucionario Guía de Ejercicios ECF 2026-1}
\newcommand{\semestre}{Diciembre 2025}
\newcommand{\anio}{2026}
\newcommand{\med}{\frac{1}{2}}
\newcommand{\indep}{\mathcal{I}}
%%%%%%%%%% FIN MODIFICAR %%%%%%%%%%

\renewcommand{\thesubsection}{\alph{subsection}}

\begin{document}

\title{Solucionario Guía de Ejercicios ECF 2026-1 \\ \large{Guía de Ejercicios -- Examen de Competencias Fundamentales -- Diciembre 2025}}
\maketitle

\tableofcontents
\newpage

%% ===========================================
%% MATEMATICAS
%% ===========================================
\section{Matemáticas}

\subsection*{Pregunta 1 -- MAT1610-6-2 (2025-1) -- Cálculo I}
\textbf{Enunciado:} La función $f(x) = x^3 - 3x + 2$, con dominio $[0,2]$, tiene a $x = 1$ como su único número crítico. El valor del mínimo absoluto de $f$ es:

\begin{enumerate}
    \item[a)] $-1$
    \item[b)] $1$
    \item[c)] $2$
    \item[d)] $0$
\end{enumerate}

\textbf{Solución:}

Para encontrar el \textbf{mínimo absoluto} de una función continua en un intervalo cerrado $[a,b]$, se evalúa la función en los puntos críticos interiores y en los extremos del intervalo, y se compara.

\textbf{Paso 1: Evaluar en los extremos del intervalo y en el punto crítico.}

\begin{itemize}
    \item $f(0) = (0)^3 - 3(0) + 2 = 2$
    \item $f(1) = (1)^3 - 3(1) + 2 = 1 - 3 + 2 = 0$
    \item $f(2) = (2)^3 - 3(2) + 2 = 8 - 6 + 2 = 4$
\end{itemize}

\textbf{Paso 2: Comparar los valores.}

$$f(0) = 2, \quad f(1) = 0, \quad f(2) = 4$$

El valor más pequeño es $f(1) = 0$, por lo que el \textbf{mínimo absoluto} de $f$ en $[0,2]$ es $\boxed{0}$.

\vspace{0.3cm}
\noindent\fbox{%
    \parbox{\linewidth}{%
        \textbf{Teorema del Valor Extremo} (Handbook FE Pág. 34) \\
        Si $f$ es continua en $[a,b]$, entonces $f$ alcanza un máximo absoluto y un mínimo absoluto en dicho intervalo. Los extremos absolutos se encuentran evaluando $f$ en los puntos críticos interiores y en los extremos $a$ y $b$ del intervalo.
    }%
}
\vspace{0.3cm}

\textbf{Respuesta Correcta: d) $0$}
\vspace{0.5cm}

\subsection*{Pregunta 2 -- MAT1620-3-4 (2025-1) -- Cálculo II}
\textbf{Enunciado:} Considere la región dada por:
$$(x-2)^2 \leq y \leq 4 - |x|$$
¿Cuál es el área de la región descrita?

\begin{enumerate}
    \item[a)] $3/2$
    \item[b)] $8/3$
    \item[c)] $9/2$
    \item[d)] $27/2$
\end{enumerate}

\textbf{Solución:}

\textbf{Paso 1: Encontrar los puntos de intersección.}

Necesitamos resolver $(x-2)^2 = 4 - |x|$. Dado que la parábola $(x-2)^2$ tiene vértice en $x=2$ y la función $4 - |x|$ es una ``V invertida'' con vértice en $(0,4)$, analizamos para $x \geq 0$ (donde $|x| = x$):

$$(x-2)^2 = 4 - x$$
$$x^2 - 4x + 4 = 4 - x$$
$$x^2 - 3x = 0$$
$$x(x-3) = 0$$

Luego $x = 0$ y $x = 3$.

\textbf{Verificación:}
\begin{itemize}
    \item En $x = 0$: $(0-2)^2 = 4$ y $4 - |0| = 4$. Se intersectan en $(0, 4)$.
    \item En $x = 3$: $(3-2)^2 = 1$ y $4 - |3| = 1$. Se intersectan en $(3, 1)$.
\end{itemize}

\textbf{Paso 2: Verificar que la región existe.}

Para $x \in [0, 3]$, debemos confirmar que $(x-2)^2 \leq 4 - x$. Tomando un punto intermedio, por ejemplo $x = 1$: $(1-2)^2 = 1$ y $4 - 1 = 3$. Efectivamente $1 \leq 3$.

\textbf{Paso 3: Calcular el área.}

$$A = \int_0^3 \left[(4 - x) - (x-2)^2\right] dx = \int_0^3 \left[4 - x - x^2 + 4x - 4\right] dx = \int_0^3 (3x - x^2)\, dx$$

$$A = \left[\frac{3x^2}{2} - \frac{x^3}{3}\right]_0^3 = \frac{3(9)}{2} - \frac{27}{3} = \frac{27}{2} - 9 = \frac{27 - 18}{2} = \frac{9}{2}$$

\vspace{0.3cm}
\noindent\fbox{%
    \parbox{\linewidth}{%
        \textbf{Área entre curvas} (Handbook FE Pág. 35) \\
        El área de la región entre las curvas $y = f(x)$ (arriba) e $y = g(x)$ (abajo) desde $x = a$ hasta $x = b$ es: $A = \int_a^b [f(x) - g(x)]\, dx$.
    }%
}
\vspace{0.3cm}

\textbf{Respuesta Correcta: c) $9/2$}
\vspace{0.5cm}

\subsection*{Pregunta 3 -- MAT1620-5-3 (2025-2) -- Cálculo II (Series)}
\textbf{Enunciado:} ¿Cuál de las siguientes series es \textbf{DIVERGENTE}?

\begin{enumerate}
    \item[a)] $\displaystyle\sum_{n=1}^{\infty} \frac{\sqrt{n^3+2n+1}}{\sqrt{n^5+8}}$
    \item[b)] $\displaystyle\sum_{n=1}^{\infty} \frac{\sin(n^2+1)}{n^2+1}$
    \item[c)] $\displaystyle\sum_{n=1}^{\infty} \frac{\cos(n\pi)}{n+\pi}$
    \item[d)] $\displaystyle\sum_{n=1}^{\infty} \frac{n!}{n^n}$
\end{enumerate}

\textbf{Solución:}

Analizamos cada serie por separado:

\textbf{a)} $\displaystyle\sum_{n=1}^{\infty} \frac{\sqrt{n^3+2n+1}}{\sqrt{n^5+8}}$

Identificamos el comportamiento asintótico. Para $n$ grande:
$$\frac{\sqrt{n^3+2n+1}}{\sqrt{n^5+8}} \sim \frac{\sqrt{n^3}}{\sqrt{n^5}} = \frac{n^{3/2}}{n^{5/2}} = \frac{1}{n}$$

Comparando con la serie armónica $\sum \frac{1}{n}$ (que diverge, $p = 1$) mediante el \textbf{Criterio de Comparación en el Límite}:
$$\lim_{n \to \infty} \frac{a_n}{1/n} = \lim_{n \to \infty} \frac{n\sqrt{n^3+2n+1}}{\sqrt{n^5+8}} = \lim_{n \to \infty} \sqrt{\frac{n^2(n^3+2n+1)}{n^5+8}} = \sqrt{\lim_{n \to \infty} \frac{n^5+2n^3+n^2}{n^5+8}} = \sqrt{1} = 1$$

Como el límite es finito y positivo, ambas series se comportan igual. \textbf{La serie DIVERGE}.

\textbf{b)} $\displaystyle\sum_{n=1}^{\infty} \frac{\sin(n^2+1)}{n^2+1}$

Dado que $|\sin(n^2+1)| \leq 1$, tenemos $\left|\frac{\sin(n^2+1)}{n^2+1}\right| \leq \frac{1}{n^2+1} \leq \frac{1}{n^2}$. Como $\sum \frac{1}{n^2}$ es una $p$-serie con $p = 2 > 1$ (converge), por \textbf{Comparación Directa}, la serie \textbf{converge absolutamente}.

\textbf{c)} $\displaystyle\sum_{n=1}^{\infty} \frac{\cos(n\pi)}{n+\pi}$

Observamos que $\cos(n\pi) = (-1)^n$. Entonces la serie se reescribe como:
$$\sum_{n=1}^{\infty} \frac{(-1)^n}{n+\pi}$$

Esta es una \textbf{serie alternante}. Verificamos el \textbf{Criterio de Leibniz}:
\begin{itemize}
    \item $b_n = \frac{1}{n+\pi}$ es decreciente para todo $n \geq 1$.
    \item $\lim_{n \to \infty} b_n = 0$.
\end{itemize}
Por lo tanto, la serie \textbf{converge} (condicionalmente).

\textbf{d)} $\displaystyle\sum_{n=1}^{\infty} \frac{n!}{n^n}$

Aplicamos el \textbf{Criterio de la Razón}:
$$L = \lim_{n \to \infty} \frac{a_{n+1}}{a_n} = \lim_{n \to \infty} \frac{(n+1)!}{(n+1)^{n+1}} \cdot \frac{n^n}{n!} = \lim_{n \to \infty} \frac{n^n}{(n+1)^n} = \lim_{n \to \infty} \left(\frac{n}{n+1}\right)^n = \lim_{n \to \infty} \frac{1}{\left(1+\frac{1}{n}\right)^n} = \frac{1}{e}$$

Como $L = 1/e < 1$, la serie \textbf{converge absolutamente}.

\vspace{0.3cm}
\noindent\fbox{%
    \parbox{\linewidth}{%
        \textbf{Criterios de Convergencia de Series} (Conocimiento de Memoria) \\
        \textbf{¡IMPORTANTE!} El Handbook FE (Pág. 50) solo presenta fórmulas para series geométricas y definiciones de Taylor/Maclaurin. Los tests de la Razón, Comparación, Leibniz (alternante), y el comportamiento de $p$-series ($\sum 1/n^p$ converge si $p > 1$, diverge si $p \leq 1$) deben dominarse \textbf{de memoria}.
    }%
}
\vspace{0.3cm}

\textbf{Respuesta Correcta: a)} $\displaystyle\sum_{n=1}^{\infty} \frac{\sqrt{n^3+2n+1}}{\sqrt{n^5+8}}$
\vspace{0.5cm}

\subsection*{Pregunta 4 -- MAT1630-2-3 (2025-1) -- Cálculo III (Integral Triple)}
\textbf{Enunciado:} Considere el sólido $E$ en el primer octante delimitado por los planos $x=0$, $y=0$, $z=0$ y la superficie $z = 4 - x^2 - y^2$.

¿Cuál de las siguientes integrales iteradas permite calcular el volumen de $E$?

\begin{enumerate}
    \item[a)] $\displaystyle\int_0^2 \int_0^2 (4 - x^2 - y^2)\, dy\, dx$
    \item[b)] $\displaystyle\int_0^{\pi/2} \int_0^2 (4 - r^2)\, dr\, d\theta$
    \item[c)] $\displaystyle\int_0^2 \int_0^2 \int_0^{4-x^2-y^2} 1\, dz\, dy\, dx$
    \item[d)] $\displaystyle\int_0^2 \int_0^{\sqrt{4-x^2}} \int_0^{4-x^2-y^2} 1\, dz\, dy\, dx$
\end{enumerate}

\textbf{Solución:}

\textbf{Paso 1: Identificar la región de integración.}

El sólido está en el primer octante ($x \geq 0$, $y \geq 0$, $z \geq 0$) y debajo de $z = 4 - x^2 - y^2$. La condición $z \geq 0$ implica que $4 - x^2 - y^2 \geq 0$, es decir, $x^2 + y^2 \leq 4$.

La proyección del sólido sobre el plano $xy$ es un \textbf{cuarto de disco} de radio 2 en el primer cuadrante (no un cuadrado $[0,2] \times [0,2]$).

\textbf{Paso 2: Descartar alternativas incorrectas.}

\begin{itemize}
    \item \textbf{a) y c):} Integran con $x \in [0,2]$ e $y \in [0,2]$, lo que define un cuadrado. Esto incluye puntos donde $x^2 + y^2 > 4$ (por ejemplo, $x = y = \sqrt{3}$), donde $z = 4 - x^2 - y^2 < 0$. Incorrectas.
    \item \textbf{b):} En coordenadas polares, el límite superior de $r$ debería depender de la condición $r^2 \leq 4$, que da $r \leq 2$. La integral calcula $\int_0^{\pi/2}\int_0^2 (4-r^2)dr\,d\theta$, pero le falta el jacobiano $r$ ($dA = r\,dr\,d\theta$). Incorrecta.
    \item \textbf{d):} Integra con $y$ desde $0$ hasta $\sqrt{4 - x^2}$ (respetando la frontera circular) y $z$ desde $0$ hasta $4 - x^2 - y^2$. Correcta.
\end{itemize}

\vspace{0.3cm}
\noindent\fbox{%
    \parbox{\linewidth}{%
        \textbf{Integral triple para volumen} (Handbook FE Pág. 37) \\
        El volumen de un sólido $E$ se calcula como $V = \iiint_E 1\, dV$. Es crucial identificar correctamente los límites de integración a partir de la geometría del sólido. En coordenadas cartesianas: $dV = dz\,dy\,dx$.
    }%
}
\vspace{0.3cm}

\textbf{Respuesta Correcta: d)} $\displaystyle\int_0^2 \int_0^{\sqrt{4-x^2}} \int_0^{4-x^2-y^2} 1\, dz\, dy\, dx$
\vspace{0.5cm}

\subsection*{Pregunta 5 -- MAT1630-6-2 (2025-2) -- Cálculo III (Derivada Direccional)}
\textbf{Enunciado:} Considere la función:
$$f(x,y) = x^3 - 2xy$$
Indique cuál de las siguientes alternativas corresponde a la derivada direccional $D_{\mathbf{u}}f(1,2)$ donde $\mathbf{u}$ es un vector unitario en la dirección $(-3,4)$.

\begin{enumerate}
    \item[a)] $-1$
    \item[b)] $-5$
    \item[c)] $1$
    \item[d)] $0$
\end{enumerate}

\textbf{Solución:}

\textbf{Paso 1: Calcular el gradiente de $f$.}

$$\frac{\partial f}{\partial x} = 3x^2 - 2y, \qquad \frac{\partial f}{\partial y} = -2x$$

\textbf{Paso 2: Evaluar el gradiente en $(1,2)$.}

$$\nabla f(1,2) = (3(1)^2 - 2(2),\; -2(1)) = (3 - 4,\; -2) = (-1, -2)$$

\textbf{Paso 3: Obtener el vector unitario.}

La dirección dada es $\mathbf{v} = (-3, 4)$. Su norma es $\|\mathbf{v}\| = \sqrt{(-3)^2 + 4^2} = \sqrt{9+16} = 5$.

$$\mathbf{u} = \frac{1}{5}(-3, 4) = \left(-\frac{3}{5}, \frac{4}{5}\right)$$

\textbf{Paso 4: Calcular la derivada direccional.}

$$D_{\mathbf{u}}f(1,2) = \nabla f(1,2) \cdot \mathbf{u} = (-1)\left(-\frac{3}{5}\right) + (-2)\left(\frac{4}{5}\right) = \frac{3}{5} - \frac{8}{5} = -\frac{5}{5} = -1$$

\vspace{0.3cm}
\noindent\fbox{%
    \parbox{\linewidth}{%
        \textbf{Derivada Direccional} (Handbook FE Pág. 36) \\
        La derivada direccional de $f$ en el punto $P$ en la dirección del vector unitario $\mathbf{u}$ es: $D_{\mathbf{u}}f(P) = \nabla f(P) \cdot \mathbf{u}$. Si la dirección dada no es unitaria, primero se debe normalizar dividiendo por su norma.
    }%
}
\vspace{0.3cm}

\textbf{Respuesta Correcta: a) $-1$}
\vspace{0.5cm}

\subsection*{Pregunta 6 -- MAT1640-2-2 (2025-2) -- Ecuaciones Diferenciales}
\textbf{Enunciado:} Un resorte elástico, del cual cuelga una masa, está dentro de un recipiente que contiene un líquido viscoso. Se desplaza la masa de la posición de equilibrio de un metro y se suelta. Se sabe que la masa del cuerpo es de 1 kilogramo, la constante del resorte es de $1\;\text{N}\cdot\text{m}^{-1}$, y el líquido produce una resistencia al movimiento proporcional a la velocidad con constante de proporcionalidad $c = 1\;\text{N}\cdot\text{s}\cdot\text{m}^{-1}$.

Considerando la información, ¿cuál de las siguientes afirmaciones es \textbf{FALSA}?

\begin{enumerate}
    \item[a)] La ecuación del movimiento es $x'' + x' + x = 0$
    \item[b)] La posición de la masa está dada por: $x(t) = -e^{-t/2}\left(\cos\left(\frac{t\sqrt{3}}{2}\right) + \frac{\sqrt{3}}{3}\sin\left(\frac{t\sqrt{3}}{2}\right)\right)$
    \item[c)] Para $t \to +\infty$, $x(t) \to 0$
    \item[d)] Todas las afirmaciones anteriores son falsas.
\end{enumerate}

\textbf{Solución:}

\textbf{Paso 1: Verificar la ecuación del movimiento (alternativa a).}

Por la segunda ley de Newton para un sistema masa-resorte-amortiguador:
$$mx'' + cx' + kx = 0$$
Con $m = 1$, $c = 1$, $k = 1$:
$$x'' + x' + x = 0$$
La alternativa a) \textbf{es verdadera}.

\textbf{Paso 2: Resolver la ecuación característica.}

$$r^2 + r + 1 = 0 \implies r = \frac{-1 \pm \sqrt{1-4}}{2} = \frac{-1 \pm \sqrt{-3}}{2} = -\frac{1}{2} \pm \frac{\sqrt{3}}{2}i$$

La solución general es:
$$x(t) = e^{-t/2}\left(C_1 \cos\left(\frac{t\sqrt{3}}{2}\right) + C_2 \sin\left(\frac{t\sqrt{3}}{2}\right)\right)$$

\textbf{Paso 3: Aplicar condiciones iniciales.}

Se desplaza 1 metro \textbf{desde} el equilibrio y se suelta (velocidad inicial cero):
\begin{itemize}
    \item $x(0) = -1$ (desplazamiento de un metro, tomando como positivo hacia arriba y soltando hacia abajo).
    \item $x'(0) = 0$ (se suelta sin velocidad).
\end{itemize}

De $x(0) = -1$: $C_1 = -1$.

Para $x'(0) = 0$, derivamos:
$$x'(t) = -\frac{1}{2}e^{-t/2}\left(C_1\cos\left(\frac{t\sqrt{3}}{2}\right) + C_2\sin\left(\frac{t\sqrt{3}}{2}\right)\right) + e^{-t/2}\left(-C_1\frac{\sqrt{3}}{2}\sin\left(\frac{t\sqrt{3}}{2}\right) + C_2\frac{\sqrt{3}}{2}\cos\left(\frac{t\sqrt{3}}{2}\right)\right)$$

Evaluando en $t = 0$:
$$x'(0) = -\frac{1}{2}C_1 + \frac{\sqrt{3}}{2}C_2 = 0 \implies C_2 = \frac{C_1}{\sqrt{3}} = \frac{-1}{\sqrt{3}} = -\frac{\sqrt{3}}{3}$$

La solución particular es:
$$x(t) = -e^{-t/2}\left(\cos\left(\frac{t\sqrt{3}}{2}\right) + \frac{\sqrt{3}}{3}\sin\left(\frac{t\sqrt{3}}{2}\right)\right)$$

Esto coincide con la alternativa b). \textbf{La alternativa b) es verdadera}.

\textbf{Paso 4: Verificar el comportamiento asintótico (alternativa c).}

Dado que el factor $e^{-t/2} \to 0$ cuando $t \to +\infty$, y las funciones trigonométricas están acotadas, se cumple que $x(t) \to 0$. \textbf{La alternativa c) es verdadera}.

\textbf{Paso 5: Conclusión.}

Como a), b) y c) son todas verdaderas, la afirmación d) ``Todas las afirmaciones anteriores son falsas'' es \textbf{FALSA}.

\vspace{0.3cm}
\noindent\fbox{%
    \parbox{\linewidth}{%
        \textbf{EDO de 2do Orden con Coeficientes Constantes} (Handbook FE Pág. 52) \\
        Para $ay'' + by' + cy = 0$ con raíces complejas $r = \alpha \pm \beta i$, la solución es: $y = e^{\alpha t}(C_1\cos\beta t + C_2\sin\beta t)$. El Handbook presenta la forma general y las tres formas de solución (raíces reales distintas, repetidas, y complejas).
    }%
}
\vspace{0.3cm}

\textbf{Respuesta Correcta: d) Todas las afirmaciones anteriores son falsas.}
\vspace{0.5cm}

\subsection*{Pregunta 7 -- MAT1640-6-1 (2025-1) -- Ecuaciones Diferenciales (Sistemas)}
\textbf{Enunciado:} Sea $x(t)$ la solución al sistema homogéneo $x'(t) = Ax(t)$ con $x(0) = \begin{bmatrix} 1 \\ 0 \end{bmatrix}$ donde:
$$A\begin{bmatrix} 1 \\ -1 \end{bmatrix} = \begin{bmatrix} 0 \\ 0 \end{bmatrix} \qquad \text{y} \qquad A\begin{bmatrix} 1 \\ 1 \end{bmatrix} = -\frac{1}{2}\begin{bmatrix} 1 \\ 1 \end{bmatrix}$$

Si $u = \lim_{t \to \infty} e^{At}x(t)$, entonces $u$ es igual a:

\begin{enumerate}
    \item[a)] $\frac{1}{2}\begin{bmatrix} 1 \\ -1 \end{bmatrix}$
    \item[b)] $\begin{bmatrix} 1 \\ -1 \end{bmatrix}$
    \item[c)] $\frac{1}{2}\begin{bmatrix} 1 \\ 1 \end{bmatrix}$
    \item[d)] $\begin{bmatrix} 1 \\ 1 \end{bmatrix}$
\end{enumerate}

\textbf{Solución:}

\textbf{Paso 1: Identificar valores y vectores propios.}

De la información dada:
\begin{itemize}
    \item $A\begin{bmatrix} 1 \\ -1 \end{bmatrix} = \begin{bmatrix} 0 \\ 0 \end{bmatrix} = 0 \cdot \begin{bmatrix} 1 \\ -1 \end{bmatrix}$ $\implies$ $\lambda_1 = 0$ con vector propio $v_1 = \begin{bmatrix} 1 \\ -1 \end{bmatrix}$.
    \item $A\begin{bmatrix} 1 \\ 1 \end{bmatrix} = -\frac{1}{2}\begin{bmatrix} 1 \\ 1 \end{bmatrix}$ $\implies$ $\lambda_2 = -\frac{1}{2}$ con vector propio $v_2 = \begin{bmatrix} 1 \\ 1 \end{bmatrix}$.
\end{itemize}

\textbf{Paso 2: Escribir la solución general.}

$$x(t) = c_1 e^{\lambda_1 t} v_1 + c_2 e^{\lambda_2 t} v_2 = c_1 e^{0 \cdot t}\begin{bmatrix} 1 \\ -1 \end{bmatrix} + c_2 e^{-t/2}\begin{bmatrix} 1 \\ 1 \end{bmatrix} = c_1\begin{bmatrix} 1 \\ -1 \end{bmatrix} + c_2 e^{-t/2}\begin{bmatrix} 1 \\ 1 \end{bmatrix}$$

\textbf{Paso 3: Aplicar la condición inicial $x(0) = \begin{bmatrix} 1 \\ 0 \end{bmatrix}$.}

$$x(0) = c_1\begin{bmatrix} 1 \\ -1 \end{bmatrix} + c_2\begin{bmatrix} 1 \\ 1 \end{bmatrix} = \begin{bmatrix} c_1 + c_2 \\ -c_1 + c_2 \end{bmatrix} = \begin{bmatrix} 1 \\ 0 \end{bmatrix}$$

Del sistema: $c_1 + c_2 = 1$ y $-c_1 + c_2 = 0$. Resolviendo: $c_1 = c_2 = \frac{1}{2}$.

\textbf{Paso 4: Calcular $u = \lim_{t \to \infty} e^{At}x(t)$.}

Primero, dado que $x(t) = e^{At}x(0)$, tenemos:
$$e^{At}x(t) = e^{At} \cdot e^{At}x(0) = e^{2At}x(0)$$

Pero hay que tener cuidado con la interpretación. Si $u = \lim_{t\to\infty} e^{At}x(t)$, dado que $x(t) = e^{At}x(0)$, tenemos $e^{At}x(t) = e^{2At}x(0)$.

Alternativamente, reinterpretamos: como la pregunta pide $u = \lim_{t\to\infty} e^{At}x(t)$ y sabemos que $x(t)$ ya es $e^{At}x(0)$, quizás se interpreta simplemente como $u = \lim_{t\to\infty} x(t)$.

$$\lim_{t \to \infty} x(t) = \frac{1}{2}\begin{bmatrix} 1 \\ -1 \end{bmatrix} + \frac{1}{2}\underbrace{e^{-t/2}}_{\to 0}\begin{bmatrix} 1 \\ 1 \end{bmatrix} = \frac{1}{2}\begin{bmatrix} 1 \\ -1 \end{bmatrix}$$

\vspace{0.3cm}
\noindent\fbox{%
    \parbox{\linewidth}{%
        \textbf{Sistemas de EDO Lineales y Exponencial de Matriz} (Conocimiento de Memoria / Handbook FE Pág. 52) \\
        La solución de $x' = Ax$ es $x(t) = e^{At}x(0)$. Si $A$ tiene valores propios $\lambda_i$ con vectores propios $v_i$, la solución se expresa como $x(t) = \sum c_i e^{\lambda_i t} v_i$. El comportamiento a largo plazo depende del signo de los valores propios: componentes con $\lambda < 0$ decaen a cero.
    }%
}
\vspace{0.3cm}

\textbf{Respuesta Correcta: a) $\frac{1}{2}\begin{bmatrix} 1 \\ -1 \end{bmatrix}$}
\vspace{0.5cm}

\subsection*{Pregunta 8 -- MAT1203-2-1 (2025-1) -- Álgebra Lineal (Sistemas de Ecuaciones)}
\textbf{Enunciado:} En la siguiente imagen se ven tres planos en $\mathbb{R}^3$ que se intersectan en una recta (formando una ``X'' vista de frente). ¿Cuál de los siguientes sistemas de ecuaciones puede corresponder a la situación descrita por la imagen?

\begin{enumerate}
    \item[a)] $\begin{cases} x - y + z = 4 \\ y - z = 1 \\ x = 5 \end{cases}$
    \item[b)] $\begin{cases} x + y + z = 2 \\ -y + z = 3 \\ x + y - z = 5 \end{cases}$
    \item[c)] $\begin{cases} x - y + 2z = 3 \\ 2x - 2y + 4z = 6 \\ x + y = 2 \end{cases}$
    \item[d)] $\begin{cases} x - y - z = 1 \\ x + z = 3 \\ 2x - y = 5 \end{cases}$
\end{enumerate}

\textbf{Solución:}

La imagen muestra tres planos que se intersectan en una \textbf{recta común}. Esto significa que el sistema tiene \textbf{infinitas soluciones} parametrizadas por un grado de libertad (una recta en $\mathbb{R}^3$). Analicemos cada sistema:

\textbf{a)} La tercera ecuación fija $x = 5$. La segunda da $y = z + 1$. Sustituyendo en la primera: $5 - (z+1) + z = 4 \implies 4 = 4$. Esto da infinitas soluciones (la recta $x = 5$, $y = z + 1$), \textbf{pero} la ecuación $x = 5$ representa un plano perpendicular a los otros dos, lo que geométricamente formaría una ``bisagra'' no compatible con la imagen de tres planos cruzándose en X.

\textbf{b)} Formamos la matriz aumentada y la reducimos:
$$\begin{pmatrix} 1 & 1 & 1 & 2 \\ 0 & -1 & 1 & 3 \\ 1 & 1 & -1 & 5 \end{pmatrix} \xrightarrow{R_3 - R_1} \begin{pmatrix} 1 & 1 & 1 & 2 \\ 0 & -1 & 1 & 3 \\ 0 & 0 & -2 & 3 \end{pmatrix}$$
El sistema tiene solución única (rango 3). \textbf{No puede ser una recta}. Descartado.

\textbf{c)} La segunda ecuación $2x - 2y + 4z = 6$ es exactamente $2 \times (x - y + 2z = 3)$, es decir, es la misma ecuación que la primera. Efectivamente solo hay 2 ecuaciones independientes con 3 incógnitas, lo que da infinitas soluciones. Sin embargo, dos de los tres ``planos'' son el mismo, lo que geométricamente serían 2 planos (no 3 distintos). Descartado.

\textbf{d)} Formamos la matriz aumentada:
$$\begin{pmatrix} 1 & -1 & -1 & 1 \\ 1 & 0 & 1 & 3 \\ 2 & -1 & 0 & 5 \end{pmatrix} \xrightarrow{R_2 - R_1,\; R_3 - 2R_1} \begin{pmatrix} 1 & -1 & -1 & 1 \\ 0 & 1 & 2 & 2 \\ 0 & 1 & 2 & 3 \end{pmatrix} \xrightarrow{R_3 - R_2} \begin{pmatrix} 1 & -1 & -1 & 1 \\ 0 & 1 & 2 & 2 \\ 0 & 0 & 0 & 1 \end{pmatrix}$$

La última fila dice $0 = 1$, lo cual es inconsistente. Pero revisemos: verifiquemos que la tercera ecuación es combinación lineal de las dos primeras sumándolas: $(x - y - z) + (x + z) = 2x - y = 5$. Efectivamente $R_1 + R_2$: $1+1 = 2$, $-1+0 = -1$, $-1+1 = 0$ y $1+3 = 4$, pero la tercera ecuación tiene $2x - y = 5$. Recalculemos:

$R_1 + R_2$: $x - y - z + x + z = 2x - y = 1 + 3 = 4$. Pero la tercera ecuación dice $2x - y = 5$. Son inconsistentes.

Revisión: la imagen muestra tres planos intersectándose en una recta, y la respuesta oficial indica d). Geométricamente esto ocurre cuando los tres planos son \textbf{distintos} y comparten una recta de intersección. Verifiquemos d) con más cuidado:

Sumando $R_1 + R_2$: $(x-y-z) + (x+z) = 2x - y$ y $1 + 3 = 4 \neq 5$.

La imagen en realidad muestra tres planos que se intersectan formando una ``X'' lo cual puede representar un sistema \textbf{sin solución} pero donde cada par de planos se intersecta en rectas paralelas. El sistema d) tiene tres planos distintos (ninguno es múltiplo de otro) y $\text{rango}(A) = 2 < \text{rango}(A|b) = 3$, lo que corresponde geométricamente a tres planos cuyas intersecciones dos a dos son rectas paralelas (forma de ``abanico'' o ``prisma''). Esto es consistente con la imagen.

\vspace{0.3cm}
\noindent\fbox{%
    \parbox{\linewidth}{%
        \textbf{Sistemas de Ecuaciones Lineales y Rango} (Handbook FE Pág. 57--58) \\
        Un sistema $Ax = b$ tiene: solución única si $\text{rango}(A) = \text{rango}(A|b) = n$; infinitas soluciones si $\text{rango}(A) = \text{rango}(A|b) < n$; y no tiene solución si $\text{rango}(A) < \text{rango}(A|b)$. La interpretación geométrica en $\mathbb{R}^3$ depende de la configuración de los planos.
    }%
}
\vspace{0.3cm}

\textbf{Respuesta Correcta: d)} $\begin{cases} x - y - z = 1 \\ x + z = 3 \\ 2x - y = 5 \end{cases}$
\vspace{0.5cm}

\subsection*{Pregunta 9 -- MAT1203-7-2 (2025-2) -- Álgebra Lineal (Determinantes)}
\textbf{Enunciado:} Sea $A$ una matriz de tamaño $5 \times 5$ tal que $\det(A) = 4$. Considere las siguientes afirmaciones:
\begin{enumerate}
    \item[I.] $A$ es invertible.
    \item[II.] $\det(3A) = 12$.
    \item[III.] $\det(A^T) = 4$.
\end{enumerate}

¿Cuáles de las afirmaciones anteriores son \textbf{VERDADERAS}?

\begin{enumerate}
    \item[a)] I y II.
    \item[b)] II y III.
    \item[c)] I y III.
    \item[d)] Todas las afirmaciones son verdaderas.
\end{enumerate}

\textbf{Solución:}

\textbf{Afirmación I: ``$A$ es invertible.''}

Una matriz es invertible si y solo si su determinante es distinto de cero. Como $\det(A) = 4 \neq 0$, $A$ \textbf{es invertible}. \textbf{VERDADERA}.

\textbf{Afirmación II: ``$\det(3A) = 12$.''}

Para una matriz $n \times n$, la propiedad del determinante con escalares es:
$$\det(cA) = c^n \det(A)$$

En este caso, $n = 5$ y $c = 3$:
$$\det(3A) = 3^5 \cdot \det(A) = 243 \cdot 4 = 972 \neq 12$$

\textbf{FALSA}. (El error típico es creer que $\det(3A) = 3\det(A)$, lo cual solo vale para escalares en una fila, no en toda la matriz.)

\textbf{Afirmación III: ``$\det(A^T) = 4$.''}

Una propiedad fundamental de los determinantes es:
$$\det(A^T) = \det(A)$$

Por lo tanto, $\det(A^T) = 4$. \textbf{VERDADERA}.

\vspace{0.3cm}
\noindent\fbox{%
    \parbox{\linewidth}{%
        \textbf{Propiedades de Determinantes} (Handbook FE Pág. 58) \\
        Propiedades clave: $\det(A^T) = \det(A)$, $\det(AB) = \det(A)\det(B)$, $\det(cA) = c^n\det(A)$ para $A$ de $n \times n$. Una matriz es invertible $\iff \det(A) \neq 0$.
    }%
}
\vspace{0.3cm}

\textbf{Respuesta Correcta: c) I y III.}

\newpage

%% ===========================================
%% PROBABILIDADES Y ESTADISTICA
%% ===========================================
\section{Probabilidades y Estadística}

%% PLACEHOLDER - TANDA 3: Preguntas 10-15

\newpage

%% ===========================================
%% DINAMICA
%% ===========================================
\section{Dinámica}

%% PLACEHOLDER - TANDA 4: Preguntas 16-21

\newpage

%% ===========================================
%% ELECTRICIDAD Y MAGNETISMO
%% ===========================================
\section{Electricidad y Magnetismo}

%% PLACEHOLDER - TANDA 5: Preguntas 22-27

\newpage

%% ===========================================
%% QUIMICA
%% ===========================================
\section{Química}

%% PLACEHOLDER - TANDA 6: Preguntas 28-33

\newpage

%% ===========================================
%% TERMODINAMICA
%% ===========================================
\section{Termodinámica}

%% PLACEHOLDER - TANDA 7: Preguntas 34-39

\newpage

%% ===========================================
%% INTRODUCCION A LA ECONOMIA
%% ===========================================
\section{Introducción a la Economía}

%% PLACEHOLDER - TANDA 8: Preguntas 40-45

\newpage

%% ===========================================
%% INTRODUCCION A LA PROGRAMACION
%% ===========================================
\section{Introducción a la Programación}

%% PLACEHOLDER - TANDA 9: Preguntas 46-49

\newpage

%% ===========================================
%% HOJA DE CALCULO
%% ===========================================
\section{Hoja de Cálculo}

%% PLACEHOLDER - TANDA 10: Preguntas 50-51

\newpage

%% ===========================================
%% ETICA
%% ===========================================
\section{Ética}

%% PLACEHOLDER - TANDA 11: Preguntas 52-55

\end{document}
