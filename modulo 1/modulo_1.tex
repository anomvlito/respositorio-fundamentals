% PREÁMBULO
\documentclass[12pt]{article}

% Configurar ruta de búsqueda para evitar conflictos de path
\makeatletter
\def\input@path{{./}{unidades/}{modulo 1/}}
\makeatother

% --- CONFIGURACIÓN DE PÁGINA Y FUENTES ---
\PassOptionsToPackage{dvipsnames,svgnames,table,xcdraw}{xcolor}
\usepackage[utf8]{inputenc}
\usepackage[T1]{fontenc}
\usepackage[spanish,es-tabla]{babel}
\usepackage{geometry}
\geometry{a4paper, top=2.5cm, bottom=2.5cm, left=2.5cm, right=2.5cm, headheight=15pt}
\usepackage{lmodern}
\usepackage{helvet}

% --- PAQUETES MATEMÁTICOS Y DE UTILIDAD ---
\usepackage{amsmath, amssymb, amsthm, amsfonts}
\usepackage{mathtools}
\usepackage{graphicx}
\graphicspath{{imagenes/}}
\usepackage{circuitikz}
\usepackage{siunitx}
\DeclareSIUnit\million{M}
\usepackage{chemformula}
\usepackage{listings}
\usepackage{float}
\usepackage{enumitem}
\setlist{nosep}
\usepackage{multicol}
\usepackage{longtable}
\usepackage{booktabs}
\usepackage{array}

% --- DISEÑO Y COLORES ---
\usepackage{xcolor}
\usepackage[many]{tcolorbox}

\definecolor{DeepBlue}{HTML}{003B5C}
\definecolor{BrightBlue}{HTML}{007ACC}
\definecolor{Emerald}{HTML}{00A388}
\definecolor{Sunset}{HTML}{FF6B6B}
\definecolor{WarningOrange}{HTML}{FF9F43}
\definecolor{LightGray}{HTML}{F8F9FA}

% --- CAJAS PERSONALIZADAS ---
\newtcolorbox{definicion}[1][]{enhanced, breakable, colback=BrightBlue!5!white, colframe=DeepBlue, coltitle=white, fonttitle=\bfseries\sffamily, title=Definición, borderline west={4pt}{0pt}{DeepBlue}, sharp corners, boxrule=0.5pt, #1}

\newtcolorbox{teorema}[1][]{enhanced, breakable, colback=Emerald!5!white, colframe=Emerald, coltitle=white, fonttitle=\bfseries\sffamily, title=Teorema, borderline west={4pt}{0pt}{Emerald}, sharp corners, boxrule=0.5pt, #1}

\newtcolorbox{ejercicio}[1][]{enhanced, breakable, colback=white, colframe=gray!50!black, coltitle=white, fonttitle=\bfseries, title=Ejercicio Resuelto, boxrule=1pt, arc=4mm, shadow={2mm}{-2mm}{0mm}{black!20}, #1}

\newtcolorbox{tip}[1][]{enhanced, colback=WarningOrange!10!white, colframe=WarningOrange, title={Tip / Cuidado}, fonttitle=\bfseries, coltitle=black, attach boxed title to top left={yshift=-2mm, xshift=2mm}, boxed title style={colback=WarningOrange, sharp corners}, boxrule=1pt, arc=2mm, #1}

\newtcolorbox{notabox}[1][]{enhanced, colback=yellow!10!white, colframe=gray, title={Nota}, fonttitle=\bfseries, coltitle=black, boxrule=0.5pt, #1}

\newtcolorbox{solbox}[1][]{enhanced, colback=Emerald!5!white, colframe=Emerald, title={Solución}, fonttitle=\bfseries, coltitle=white, boxrule=0.5pt, #1}

\newcommand{\nota}[1]{\begin{notabox}\textbf{Nota Estratégica:} #1\end{notabox}}

% --- REFERENCIA AL HANDBOOK ---
\newtcolorbox{febox}[1][]{enhanced, colback=gray!5!white, colframe=gray!50!black, title={Manual FE v10.1}, fonttitle=\bfseries\sffamily\small, attach boxed title to top right={yshift=-2mm, xshift=-2mm}, boxed title style={colback=gray!60!black, sharp corners}, boxrule=0.5pt, arc=2mm, #1}

\newcommand{\fehandbook}[1]{%
    \begin{febox}
    \centering \small \sffamily \textbf{Página #1} \\ Consúltalo en el examen.
    \end{febox}
}

\newcommand{\fehandbookinline}[1]{%
    \fcolorbox{gray!50!black}{gray!5!white}{\small \sffamily \textbf{Manual FE p.#1}}%
}

\newcommand{\feimply}[1]{%
    \begin{tcolorbox}[colback=red!10!white, colframe=red!50!black, boxrule=0.5pt, arc=1mm, left=2pt, right=2pt, top=1pt, bottom=1pt]
    \centering \small \sffamily \textbf{¡NO ESTÁ EN EL MANUAL!} \\ #1
    \end{tcolorbox}
}

% --- NAVEGACIÓN Y ENCABEZADOS ---
\usepackage{fancyhdr}
\pagestyle{fancy}
\fancyhf{}
\fancyhead[L]{\small \sffamily \textbf{Módulo 1: Matemáticas}}
\fancyhead[R]{\small \sffamily \leftmark}
\fancyfoot[C]{\thepage}
\renewcommand{\headrulewidth}{1pt}
\renewcommand{\footrulewidth}{0pt}

% Hyperref (Siempre al final del preámbulo)
\usepackage{url}
\usepackage[colorlinks=true, linkcolor=DeepBlue, citecolor=Emerald, urlcolor=BrightBlue]{hyperref}

% --- MACROS MATEMÁTICOS ---
\newcommand{\R}{\mathbb{R}}
\newcommand{\N}{\mathbb{N}}
\newcommand{\Z}{\mathbb{Z}}
\newcommand{\C}{\mathbb{C}}
\newcommand{\dd}{\mathrm{d}}
\newcommand{\norm}[1]{\left\lVert#1\right\rVert}
\newcommand{\abs}[1]{\left\lvert#1\right\rvert}
\newcommand{\vect}[1]{\mathbf{#1}}

\newenvironment{pasos}{\begin{enumerate}[label=\textbf{Paso \arabic*}:, leftmargin=*]}{\end{enumerate}}

% Título del Documento (Opcional, para portada)
\title{\textbf{\Huge Fundamentals Resumen} \\ \Large Resumen para el Examen de Conocimientos Fundamentales}
\author{Ingeniería UC}
\date{\today}

\begin{document}

% Portada simple
\maketitle
\tableofcontents
\newpage

% Introducción / Temario
\section*{Programa del Módulo}
\addcontentsline{toc}{section}{Programa del Módulo}
% \section*{MÓDULO 1: Matemáticas y Probabilidades}

\subsection{Cálculo I (MAT1610)}
\begin{definicion}[title=Contenidos]
\begin{enumerate}
    \item[1.] Geometría Analítica
\end{enumerate}
\end{definicion}

\begin{teorema}[title=Indicadores a evaluar (Números corresponden al correlativo del programa de cada curso)]
\begin{enumerate}
    \item[1.] Identificar gráficos de funciones básicas, exponenciales, logarítmicas. \feimply{Aprender de memoria las formas gráficas elementales. No están en el manual.}
    \item[4.] Calcular derivadas de funciones obtenidas por álgebra de funciones elementales. \fehandbook{48 (Tabla de Derivadas)}
    \item[6.] Reconocer gráfica y analíticamente propiedades de los gráficos de funciones. \fehandbook{45 (Máximos, Mínimos, Inflexión)}
    \item[9.] Conocer el cálculo de primitivas de funciones básicas. \fehandbook{49 (Tabla de Integrales)}
\end{enumerate}
\end{teorema}

\subsection{Cálculo II (MAT1620)}
\begin{definicion}[title=Contenidos]
\begin{enumerate}
    \item[1.] Cálculo Integral
\end{enumerate}
\end{definicion}

\begin{teorema}[title=Indicadores a evaluar (Números corresponden al correlativo del programa de cada curso)]
\begin{enumerate}
    \item[3.] Aplicar el concepto de integral definida para calcular áreas y momentos de regiones del plano. \fehandbook{108-112 (Centroides en sección de Estática)}
    \item[5.] Aplicar los criterios básicos de convergencia de series e integrales impropias. \feimply{¡CUIDADO! El manual solo tiene Geométrica (p.50) y Taylor (p.51). Faltan Razón, Raíz e Integral.}
    \item[8.] Conocer las ecuaciones paramétricas, vectoriales y cartesianas de rectas y planos. \fehandbook{35 (Rectas 2D) y 59 (Vectores)}
\end{enumerate}
\end{teorema}

\subsection{Cálculo III (MAT1630)}
\begin{definicion}[title=Contenidos]
\begin{enumerate}
    \item[1.] Cálculo Diferencial
\end{enumerate}
\end{definicion}

\begin{teorema}[title=Indicadores a evaluar (Números corresponden al correlativo del programa de cada curso)]
\begin{enumerate}
    \item[2.] Aplicar el concepto de integral múltiple para evaluar volúmenes y centros de masa. \fehandbook{42 (Volúmenes de Sólidos Básicos)}
    \item[5.] Reconocer y explicar el concepto de “curvas de nivel” y calcularlas. \feimply{Concepto visual. No hay fórmula.}
    \item[6.] Calcular derivadas direccionales. \fehandbook{59 (Gradiente, Divergencia, Rotor)}
\end{enumerate}
\end{teorema}

\subsection{Ecuaciones Diferenciales (MAT1640)}
\begin{definicion}[title=Contenidos]
\begin{enumerate}
    \item[1.] Ecuaciones Diferenciales
\end{enumerate}
\end{definicion}

\begin{teorema}[title=Indicadores a evaluar (Números corresponden al correlativo del programa de cada curso)]
\begin{enumerate}
    \item[2.] Modelar situaciones sencillas de la realidad y fenómenos mediante ecuaciones diferenciales. \feimply{Habilidad de modelado.}
    \item[3.] Reconocer tipo de EDO, identificar y utilizar métodos de solución según el caso. \fehandbook{51-52 (Primer y Segundo Orden)}
    \item[6.] Calcular soluciones de sistemas lineales de $2\times2$ y $3\times3$ (coef. constantes). \feimply{Teoría de valores propios para sistemas no está explícita.}
\end{enumerate}
\end{teorema}

\subsection{Álgebra Lineal (MAT1203)}
\begin{definicion}[title=Contenidos]
\begin{enumerate}
    \item[1.] Matrices
    \item[2.] Raíces de Ecuaciones
    \item[3.] Análisis Vectorial
\end{enumerate}
\end{definicion}

\begin{teorema}[title=Indicadores a evaluar (Números corresponden al correlativo del programa de cada curso)]
\begin{enumerate}
    \item[1.] Determinar escalonada reducida, resolver $Ax=b$, calcular inversas y bases. \fehandbook{57 (Matrices e Inversa)}
    \item[2.] Interpretar geométricamente dependencia lineal, complemento ortogonal. \feimply{Concepto teórico.}
    \item[4.] Explicar y utilizar propiedades de operaciones matriciales. \fehandbook{57}
    \item[6.] Explicar y utilizar matrices elementales, simétricas, ortogonales, etc. \fehandbook{57 (Definiciones)}
    \item[7.] Calcular determinantes, resolver sistemas y evaluar inversas. \fehandbook{58 (Determinantes)}
    \item[9.] Determinar matriz de Transformación Lineal y relación con cambio de base. \feimply{No está explícito.}
    \item[12.] Explicar valores/vectores propios, diagonalización y aplicaciones (simétricas). \feimply{Definición básica solamente. Algoritmo no está.}
\end{enumerate}
\end{teorema}

\subsection{Probabilidades y Estadística (EYP1113)}
\begin{definicion}[title=Contenidos]
\begin{enumerate}
    \item[1.] Álgebra de eventos, axiomas, prob. condicional, Bayes.
    \item[2.] Medidas descriptivas teóricas (media, varianza, percentil, etc.).
    \item[3.] Modelos Discretos/Continuos (Binomial, Poisson, Normal, Exp, etc.) y uso de R.
    \item[4.] Distribuciones conjuntas, covarianza, correlación.
    \item[5.] Estimación y propiedades.
    \item[6.] Test de hipótesis e intervalos de confianza.
    \item[7.] Bondad de ajuste (Chi-cuadrado).
    \item[8.] Regresión lineal (test-t, test-F, $R^2$).
\end{enumerate}
\end{definicion}

\begin{teorema}[title=Indicadores a evaluar (Números corresponden al correlativo del programa de cada curso)]
\begin{enumerate}
    \item[1.] Ajustar distribuciones de probabilidad a datos reales. \fehandbook{66-68 (Distribuciones)}
    \item[2.] Describir fenómenos de incertidumbre usando variables aleatorias. \fehandbook{63-65 (Medidas)}
    \item[3.] Realizar estimaciones de parámetros e intervalos de confianza. \fehandbook{73-74 (Tests y C.I.)}
    \item[4.] Ajustar e interpretar modelos de regresión lineal. \fehandbook{69-70 (Regresión)}
\end{enumerate}
\end{teorema}

\newpage

% --- UNIDADES ---

% 1. Cálculo I
\section{Cálculo I (MAT1610): Análisis de Funciones}
\subsection{Funciones Elementales y Límites}

\begin{definicion}[title=Funciones Básicas]
\begin{itemize}
    \item \textbf{Exponencial ($e^x$):} Dominio $\R$, Recorrido $(0, \infty)$. $\exp(x+y) = e^x e^y$.
    \item \textbf{Logaritmo Natural ($\ln x$):} Dominio $(0, \infty)$. $y = \ln x \iff e^y = x$.
    \item \textbf{Propiedades ($a,b > 0$):} $\ln(ab) = \ln a + \ln b$, $\ln(a^b) = b \ln a$.
\end{itemize}
\end{definicion}

\begin{teorema}[title=Límites y Continuidad]
\textbf{Límites Notables:}
$$\lim_{x\to 0} \frac{\sin x}{x} = 1, \quad \lim_{x\to \infty} \left(1 + \frac{1}{x}\right)^x = e$$
\textbf{Continuidad en $c$:} Se requiere que $\lim_{x\to c} f(x) = f(c)$.
\end{teorema}

\begin{teorema}[title=Asíntotas]
\begin{itemize}
    \item \textbf{Vertical ($x=c$):} Si $\lim_{x\to c^\pm} f(x) = \pm \infty$.
    \item \textbf{Horizontal ($y=L$):} Si $\lim_{x\to \pm\infty} f(x) = L$.
    \item \textbf{Oblicua ($y=mx+n$):} $m = \lim_{x\to\infty} \frac{f(x)}{x}, \quad n = \lim_{x\to\infty} (f(x)-mx)$.
\end{itemize}
\end{teorema}

\subsection{Cálculo Diferencial}

\begin{tip}[title=Reglas de Derivación (Cheat Sheet)]
\begin{itemize}
    \item \textbf{Producto:} $(f \cdot g)' = f'g + fg'$.
    \item \textbf{Cuociente:} $\left(\frac{f}{g}\right)' = \frac{f'g - fg'}{g^2}$ \textbf{(con $g \neq 0$)}.
    \item \textbf{Cadena:} $[f(g(x))]' = f'(g(x)) \cdot g'(x)$.
\end{itemize}
\end{tip}

\begin{teorema}[title=Teoremas de Existencia (Rigor MAT1610)]
\begin{itemize}
    \item \textbf{Bolzano:} Si $f$ es cont. en $[a,b]$ y $f(a)f(b)<0 \implies \exists c \in (a,b) / f(c)=0$.
    \item \textbf{Valor Medio (MVT):} Si $f$ es cont. en $[a,b]$ y derivable en $(a,b)$, $\exists c \in (a,b)$ tal que $f'(c) = \frac{f(b)-f(a)}{b-a}$.
    \item \textbf{Rolle:} Si $MVT$ y $f(a)=f(b) \implies \exists c \in (a,b) / f'(c)=0$.
\end{itemize}
\end{teorema}

\begin{definicion}[title=Interpretación y Gráfica]
\textbf{1ra Derivada ($f'$):}
$f' > 0 \uparrow$ (Creciente), $f' < 0 \downarrow$ (Decreciente), $f'=0 \to$ P. Crítico.

\textbf{2da Derivada ($f''$):}
\begin{itemize}
    \item $f'' > 0 \cup$ (Punto mínimo local si $f'=0$).
    \item $f'' < 0 \cap$ (Punto máximo local si $f'=0$).
    \item \textbf{Punto de Inflexión:} Si $f''(c)=0$ y hay cambio de signo en $f''$.
\end{itemize}
\end{definicion}

\subsection{Primitivas (Integrales Indefinidas)}

\begin{teorema}[title=Tabla de Primitivas Básicas]
\begin{minipage}{\linewidth} % Estabilidad para multicols
\begin{multicols}{2}
\begin{itemize}
    \item $\int x^n \, \dd x = \frac{x^{n+1}}{n+1} + C \quad (n \ne -1)$
    \item $\int \frac{1}{x} \, \dd x = \ln|x| + C$
    \item $\int e^x \, \dd x = e^x + C$
    \item $\int a^x \, \dd x = \frac{a^x}{\ln a} + C$
    \item $\int \sin x \, \dd x = -\cos x + C$
    \item $\int \cos x \, \dd x = \sin x + C$
    \item $\int \sec^2 x \, \dd x = \tan x + C$
    \item $\int \frac{1}{1+x^2} \, \dd x = \arctan x + C$
\end{itemize}
\end{multicols}
\end{minipage}
\end{teorema}



\newpage

% 2. Cálculo II
\section{Cálculo II (MAT1620): Integrales y Series}
\subsection{Técnicas de Integración}

\begin{tip}[title=Integración por Partes]
Recuerda: ``\textbf{U}n \textbf{D}ía \textbf{V}i \textbf{U}na \textbf{V}aca \textbf{V}estida \textbf{D}e \textbf{U}nforme''.
$$
\int u \, \dd v = u v - \int v \, \dd u
$$
\textbf{Estrategia LIATE} para elegir $u$: \textbf{L}ogarítmicas, \textbf{I}nversas, \textbf{A}lgebraicas, \textbf{T}rigonométricas, \textbf{E}xponenciales.
\end{tip}

\subsection{Aplicaciones de la Integral Definida}

\begin{definicion}[title=Área entre Curvas]
Si $f(x) \ge g(x)$ en $[a,b]$:
$$
A = \int_{a}^{b} [f(x) - g(x)] \, \dd x
$$
\end{definicion}

\begin{ejercicio}[title=Área entre Curvas con Valor Absoluto]
\textbf{Problema (MAT1620-3-4):} Considere la región dada por:
$$
(x-2)^2 \le y \le 4-|x|
$$
¿Cuál es el área de la región descrita?

\textbf{Solución:}

\textbf{Paso 1: Identificar las funciones}
\begin{itemize}
    \item Curva inferior: $g(x) = (x-2)^2$ (parábola con vértice en $(2,0)$)
    \item Curva superior: $f(x) = 4-|x|$ (función valor absoluto invertida)
\end{itemize}

\textbf{Paso 2: Encontrar puntos de intersección}

Resolver $(x-2)^2 = 4-|x|$. Dividimos en dos casos:

\textit{Caso 1: $x \ge 0$} → $|x| = x$
\begin{align*}
(x-2)^2 &= 4-x \\
x^2 - 4x + 4 &= 4-x \\
x^2 - 3x &= 0 \\
x(x-3) &= 0 \implies x=0 \text{ o } x=3
\end{align*}

\textit{Caso 2: $x < 0$} → $|x| = -x$
\begin{align*}
(x-2)^2 &= 4+x \\
x^2 - 4x + 4 &= 4+x \\
x^2 - 5x &= 0 \\
x(x-5) &= 0 \implies x=0 \text{ (no válido en } x<0 \text{)}
\end{align*}

Para $x < 0$, si probamos un punto como $x=-1$: $( -1-2)^2 = 9$ y $4-|-1|=3$. Como $9 \not\le 3$, \textbf{no existe región} para $x < 0$.
La región está definida únicamente en el intervalo $x \in [0, 3]$.

\textbf{Paso 3: Calcular el área}

Dado que $x \ge 0$, tenemos $|x|=x$, por lo tanto $f(x) = 4-x$.
\begin{align*}
A &= \int_{0}^{3} [(4-x) - (x-2)^2] \, \dd x \\
&= \int_{0}^{3} [4-x - (x^2 - 4x + 4)] \, \dd x \\
&= \int_{0}^{3} [-x^2 + 3x] \, \dd x \\
&= \left[-\frac{x^3}{3} + \frac{3x^2}{2}\right]_{0}^{3} \\
&= \left(-\frac{27}{3} + \frac{3 \cdot 9}{2}\right) - 0 \\
&= -9 + \frac{27}{2} \\
&= -\frac{18}{2} + \frac{27}{2} = \frac{9}{2}
\end{align*}

\textbf{Alternativa correcta: c) $9/2$}
\end{ejercicio}

\begin{teorema}[title=Centro de Masa (Centroide)]
Para una región plana de densidad constante $\rho$, el centroide $(\bar{x}, \bar{y})$ es:
\begin{align*}
    \bar{x} &= \frac{1}{A} \int_{a}^{b} x [f(x)-g(x)] \, \dd x \\
    \bar{y} &= \frac{1}{A} \int_{a}^{b} \frac{1}{2} ([f(x)]^2 - [g(x)]^2) \, \dd x
\end{align*}
\end{teorema}

\subsection{Series e Integrales Impropias}

\begin{tip}[title=Resumen de Criterios de Convergencia]
\begin{enumerate}
    \item \textbf{Criterio del Término $n$-ésimo (Divergencia):} Si $\lim_{n \to \infty} a_n \neq 0$, la serie diverge.
    
    \item \textbf{Serie Geométrica:} $\sum ar^n$ converge si $|r| < 1$, diverge si $|r| \ge 1$.
    
    \item \textbf{Serie-p:} $\sum \frac{1}{n^p}$ converge si $p > 1$, diverge si $p \le 1$.
    
    \item \textbf{Criterio de la Integral:} Si $f(x)$ es continua, positiva y decreciente, entonces $\sum a_n$ y $\int_{1}^{\infty} f(x) \dd x$ convergen o divergen juntas.
    
    \item \textbf{Criterio de Computación Directa/Límite:} Compara con series conocidas (generalmente p-series o geométricas).
    
    \item \textbf{Criterio de Series Alternantes (Leibniz):} $\sum (-1)^n b_n$ converge si $b_n$ decrece a 0.
    
    \item \textbf{Criterio de la Razón:} $L = \lim |\frac{a_{n+1}}{a_n}|$. $L<1$ (Conv), $L>1$ (Div), $L=1$ (No decide).
    
    \item \textbf{Criterio de la Raíz:} $L = \lim \sqrt[n]{|a_n|}$. Mismas condiciones que la Razón.
\end{enumerate}
\end{tip}

\begin{teorema}[title=Criterio de la Razón (D'Alembert)]
Sea $\sum a_n$. Calculamos $L = \lim_{n \to \infty} \left| \frac{a_{n+1}}{a_n} \right|$:
\begin{itemize}
    \item Si $L < 1 \implies$ Converge Absolutamente.
    \item Si $L > 1 \implies$ Diverge.
    \item Si $L = 1 \implies$ El criterio no decide.
\end{itemize}
\end{teorema}

\begin{teorema}[title=Criterio de Comparación en el Límite]
Si $a_n, b_n > 0$ y $\lim_{n \to \infty} \frac{a_n}{b_n} = c$ ($0 < c < \infty$), entonces:
$$
\sum a_n \text{ y } \sum b_n \text{ se comportan igual (ambas Convergen o ambas Divergen).}
$$
Útil para comparar con p-series: $\sum \frac{1}{n^p}$ conv. si $p>1$.
\end{teorema}

\begin{ejercicio}[title=Divergencia de Series]
\textbf{Problema:} ¿Cuál de las siguientes series es \textbf{DIVERGENTE}?
\begin{enumerate}[label=\alph*)]
    \item $\displaystyle \sum_{n=1}^{\infty} \frac{\sqrt{n^3+2n+1}}{\sqrt{n^5+8}}$
    \item $\displaystyle \sum_{n=1}^{\infty} \frac{\sin(n^2+1)}{n^2+1}$
    \item $\displaystyle \sum_{n=1}^{\infty} \frac{\cos(n\pi)}{n+\pi}$
    \item $\displaystyle \sum_{n=1}^{\infty} \frac{n!}{n^n}$
\end{enumerate}

\textbf{Solución Detallada:}

\textbf{a)} \textbf{Análisis Riguroso (Limit Comparison Test):}

1. \textbf{Definimos las series:}
Comparamos nuestra serie original ($a_n$) con la serie armónica ($b_n$), que sabemos que DIVERGE.
$$ a_n = \frac{\sqrt{n^3+2n+1}}{\sqrt{n^5+8}} \quad , \quad b_n = \frac{1}{n} $$

2. \textbf{Planteamos el Límite:}
Calculamos $L = \lim_{n \to \infty} \frac{a_n}{b_n}$. Si $0 < L < \infty$, ambas se comportan igual.
$$ L = \lim_{n \to \infty} \frac{\frac{\sqrt{n^3+2n+1}}{\sqrt{n^5+8}}}{\frac{1}{n}} = \lim_{n \to \infty} \left( n \cdot \frac{\sqrt{n^3+2n+1}}{\sqrt{n^5+8}} \right) $$

3. \textbf{Factorización de términos dominantes:}
No basta con decir "se parece a". Factorizamos la potencia mayor \textit{dentro} de cada raíz para demostrar formalmente que los términos menores desaparecen.
$$ L = \lim_{n \to \infty} n \cdot \frac{\sqrt{n^3(1 + \frac{2n}{n^3} + \frac{1}{n^3})}}{\sqrt{n^5(1 + \frac{8}{n^5})}} = \lim_{n \to \infty} n \cdot \frac{\sqrt{n^3} \cdot \sqrt{1 + \frac{2}{n^2} + \frac{1}{n^3}}}{\sqrt{n^5} \cdot \sqrt{1 + \frac{8}{n^5}}} $$

4. \textbf{Cancelación y Evaluación:}
Sabemos que $\sqrt{n^3} = n^{3/2}$ y $\sqrt{n^5} = n^{5/2}$. Agrupamos las potencias de $n$:
$$ L = \lim_{n \to \infty} \left( \frac{n \cdot n^{3/2}}{n^{5/2}} \right) \cdot \frac{\sqrt{1 + \frac{2}{n^2} + \frac{1}{n^3}}}{\sqrt{1 + \frac{8}{n^5}}} $$
Observamos que $n \cdot n^{3/2} = n^{1+1.5} = n^{2.5} = n^{5/2}$. Los términos se cancelan exactamente ($\frac{n^{5/2}}{n^{5/2}} = 1$).
$$ L = 1 \cdot \frac{\sqrt{1+0+0}}{\sqrt{1+0}} = 1 $$

Como $L=1$ (finito y positivo), y $\sum b_n$ diverge, \textbf{la serie a) DIVERGE}.

\textbf{Análisis de las otras opciones:}

\textbf{b)} Convergencia Absoluta:
$$
\left| \frac{\sin(n^2+1)}{n^2+1} \right| \le \frac{1}{n^2+1} < \frac{1}{n^2}
$$
Como la función seno está acotada entre $[-1, 1]$, el numerador no crece. Comparamos con la p-serie convergente $\sum \frac{1}{n^2}$ ($p=2 > 1$). Converge.

\textbf{c)} Serie Alternante:
La serie se puede reescribir considerando que $\cos(n\pi) = (-1)^n$:
$$
\sum_{n=1}^{\infty} \frac{(-1)^n}{n+\pi}
$$
Esta es una \textbf{Serie Alternante}. Verificamos el Criterio de Leibniz:
1. ¿Son los términos decrecientes? Sí, $\frac{1}{n+1+\pi} < \frac{1}{n+\pi}$.
2. ¿El límite es 0? Sí, $\lim_{n \to \infty} \frac{1}{n+\pi} = 0$.
Por lo tanto, la serie \textbf{converge} (condicionalmente).

\textbf{d)} Criterio de la Razón (Converge):
Usamos D'Alembert para términos con factoriales y potencias $n$-ésimas:
\begin{align*}
L &= \lim_{n \to \infty} \left| \frac{a_{n+1}}{a_n} \right| = \lim_{n \to \infty} \frac{(n+1)!}{(n+1)^{n+1}} \cdot \frac{n^n}{n!} \\
&= \lim_{n \to \infty} \frac{(n+1)n! \cdot n^n}{(n+1) \cdot (n+1)^n \cdot n!} \\
&= \lim_{n \to \infty} \left(\frac{n}{n+1}\right)^n = \lim_{n \to \infty} \frac{1}{(1+1/n)^n} = \frac{1}{e}
\end{align*}
Como $L = \frac{1}{e} \approx \frac{1}{2.718} < 1$, la serie \textbf{converge} absolutamente.

\textbf{Alternativa correcta: a)}
\end{ejercicio}

\subsection{Geometría en el Espacio}

\begin{definicion}[title=Rectas y Planos]
\begin{itemize}
    \item \textbf{Recta} por $P_0$ con dirección $\vec{v}$: 
    $$ \vec{r}(t) = P_0 + t\vec{v} $$
    \item \textbf{Plano} por $P_0(x_0, y_0, z_0)$ con normal $\vec{n}=\langle a,b,c \rangle$:
    $$ a(x-x_0) + b(y-y_0) + c(z-z_0) = 0 $$
\end{itemize}
\end{definicion}


\newpage

% 3. Cálculo III
\section{Cálculo III (MAT1630): Cálculo Multivariable}
\subsection{Diferenciación Multivariable}

\begin{definicion}[title=Gradiente y su Significado] \fehandbook{59 (Vectors: Gradient, Divergence, Curl)}
Sea $f: \R^n \to \R$. El gradiente es el vector:
$$ \nabla f = \left\langle \frac{\partial f}{\partial x_1}, \dots, \frac{\partial f}{\partial x_n} \right\rangle $$
\textbf{Propiedades Clave:}
\begin{itemize}
    \item $\nabla f$ apunta a la dirección de \textbf{máximo crecimiento}.
    \item La tasa máxima de cambio es $\norm{\nabla f}$.
    \item $\nabla f$ es \textbf{perpendicular} (ortogonal) a las curvas/superficies de nivel.
\end{itemize}
\end{definicion}

\begin{teorema}[title=Derivada Direccional] \feimply{Concepto deducible de producto punto (p.59), pero no explícito.}
La derivada de $f$ en la dirección del vector unitario $\vec{u}$:
$$ D_{\vec{u}}f(P) = \nabla f(P) \cdot \vec{u} $$
{\small \textbf{Nota:} Si $\vec{u}$ no es unitario, normalizar $\vec{u} \leftarrow \frac{\vec{u}}{\norm{\vec{u}}}$ antes de usar la fórmula.}
\end{teorema}

\begin{tip}[title=Optimización con Multiplicadores de Lagrange] \feimply{NO ESTÁ en el manual.}
Para maximizar/minimizar $f(x,y,z)$ sujeto a la restricción $g(x,y,z)=k$:
\begin{enumerate}
    \item Resolver el sistema: $\nabla f = \lambda \nabla g$.
    \item Considerar también la restricción: $g(x,y,z) = k$.
\end{enumerate}
\end{tip}

\subsection{Integrales Múltiples y Cambios de Coordenadas}

\begin{definicion}[title=Coordenadas Polares (En el plano $xy$)] \fehandbook{36 (Coordinate Systems)}
$$ x = r\cos\theta, \quad y = r\sin\theta, \quad x^2+y^2=r^2 $$
\textbf{Jacobiano (Factor de corrección):} $\dd A = r \, \dd r \, \dd\theta$ \feimply{Jacobiano $r$ no explícito en sección integración.}
\end{definicion}

\begin{teorema}[title=Coordenadas Esféricas (Espacio 3D)] \fehandbook{36}
Usar para esferas o conos.
\begin{itemize}
    \item $x = \rho \sin\phi \cos\theta$
    \item $y = \rho \sin\phi \sin\theta$
    \item $z = \rho \cos\phi$
\end{itemize}
\textbf{Jacobiano de volumen:} $\dd V = \rho^2 \sin\phi \, \dd\rho \, \dd\phi \, \dd\theta$
\end{teorema}

\begin{ejercicio}[title=MAT1630-2-3 (2025-1)]
Considere el sólido $E$ en el primer octante delimitado por los planos $x = 0$, $y = 0$, $z = 0$ y la superficie $z=4-x^2-y^2$.

¿Cuál de las siguientes integrales iteradas permite calcular el volumen de $E$?

\begin{enumerate}[label=\alph*)]
    \item $\displaystyle \int_0^2 \int_0^2 (4-x^2-y^2) \, dy \, dx$
    
    \item $\displaystyle \int_0^{\pi/2} \int_0^2 (4-r^2) \, dr \, d\theta$
    
    \item $\displaystyle \int_0^2 \int_0^2 \int_0^{4-x^2-y^2} 1 \, dz \, dy \, dx$
    
    \item $\displaystyle \int_0^2 \int_0^{\sqrt{4-x^2}} \int_0^{4-x^2-y^2} 1 \, dz \, dy \, dx$
\end{enumerate}
\end{ejercicio}

\begin{solbox}
\textbf{Respuesta correcta: d)}

\textbf{Análisis de cada opción:}

\begin{itemize}
    \item \textbf{Opción a):} Los límites de integración son incorrectos. Para $x=2$ y $y=2$, tendríamos $z=4-4-4=-4<0$, lo cual está fuera del primer octante.
    
    \item \textbf{Opción b):} Falta el factor $r$ del Jacobiano en coordenadas polares. Debería ser $\int_0^{\pi/2} \int_0^2 r(4-r^2) \, dr \, d\theta$.
    
    \item \textbf{Opción c):} Similar a la opción a), los límites de $y$ son incorrectos. No considera que la región de integración en el plano $xy$ es circular.
    
    \item \textbf{Opción d):} \textbf{CORRECTA.} 
    \begin{itemize}
        \item El límite superior de $z$ es la superficie $z=4-x^2-y^2$.
        \item Para cada $x$ fijo, $y$ varía desde $0$ hasta $\sqrt{4-x^2}$ (semicírculo en el primer cuadrante).
        \item $x$ varía de $0$ a $2$ (donde la superficie intersecta el plano $xy$ cuando $z=0$: $4-x^2-y^2=0 \Rightarrow x^2+y^2=4$).
    \end{itemize}
\end{itemize}

\textbf{Verificación:} La región de integración en el plano $xy$ es un cuarto de círculo de radio 2 (primer cuadrante de $x^2+y^2 \leq 4$), y la altura va desde $z=0$ hasta $z=4-x^2-y^2$.
\end{solbox}



\newpage

% 4. Ecuaciones Diferenciales
\section{Ecuaciones Diferenciales (MAT1640)}
\subsection{Clasificación y Primer Orden}

\begin{definicion}[title=Conceptos Básicos]
\begin{itemize}
    \item \textbf{Orden:} Derivada más alta presente (ej: $y''$ es orden 2).
    \item \textbf{Linealidad:} La variable $y$ y sus derivadas tienen potencia 1 y no se multiplican entre sí.
\end{itemize}
\end{definicion}

\begin{teorema}[title=Método: Variables Separables]
Si la EDO se puede escribir como $f(y) \, \dd y = g(x) \, \dd x$:
\begin{enumerate}
    \item Separar variables a cada lado del igual.
    \item Integrar ambos lados: $\int f(y) \dd y = \int g(x) \dd x$.
    \item Despejar $y(x)$ si es posible.
\end{enumerate}
\end{teorema}

\begin{teorema}[title=Método: Factor Integrante (Ec. Lineales 1er Orden)]
Para ecuaciones de la forma $y' + P(x)y = Q(x)$:
\begin{enumerate}
    \item Calcular el factor integrante $\mu(x) = e^{\int P(x) \dd x}$.
    \item Multiplicar toda la ecuación por $\mu(x)$. El lado izquierdo colapsa a $(\mu \cdot y)'$.
    \item Integrar: $\mu(x) \cdot y = \int \mu(x) Q(x) \dd x$.
    \item Despejar $y$.
\end{enumerate}
\end{teorema}

\subsection{Ecuaciones de Segundo Orden (Coef. Constantes)}

Para resolver $ay'' + by' + cy = 0$:
\begin{pasos}
\item Escribir la \textbf{ecuación característica}: $ar^2 + br + c = 0$.
\item Hallar las raíces $r_1, r_2$:
\begin{itemize}
    \item \textbf{Reales distintas} ($r_1 \ne r_2$): $y = C_1 e^{r_1 x} + C_2 e^{r_2 x}$
    \item \textbf{Reales iguales} ($r_1 = r_2 = r$): $y = C_1 e^{rx} + C_2 x e^{rx}$
    \item \textbf{Complejas} ($\alpha \pm \beta i$): $y = e^{\alpha x}(C_1 \cos(\beta x) + C_2 \sin(\beta x))$
\end{itemize}
\end{pasos}


\newpage

% 5. Álgebra Lineal
\section{Álgebra Lineal (MAT1203)}
\subsection{Matrices y Determinantes}

\begin{definicion}[title=Propiedades Clave]
\begin{itemize}
    \item \textbf{Invertible:} $A$ es invertible $\iff \det(A) \ne 0$.
    \item \textbf{Simétrica:} $A^T = A$. (Sus valores propios son siempre reales).
    \item \textbf{Ortogonal:} $A^T = A^{-1}$ (o $A^T A = I$). Preserva distancias y ángulos.
\end{itemize}
\end{definicion}

\begin{teorema}[title=Propiedades del Determinante]
Si $A, B \in M_{n\times n}(\R)$:
\begin{itemize}
    \item $\det(AB) = \det(A)\det(B)$
    \item $\det(A^T) = \det(A)$
    \item $\det(kA) = k^n \det(A)$ (¡Ojo con el $n$!)
    \item $\det(A^{-1}) = \frac{1}{\det(A)}$
\end{itemize}
\end{teorema}

\subsection{Diagonalización}

\begin{definicion}[title=Valores y Vectores Propios]
Un vector $v \ne 0$ es vector propio de $A$ con valor propio $\lambda$ si:
$$ A v = \lambda v \iff (A - \lambda I)v = 0 $$
\end{definicion}

\textbf{Algoritmo para Diagonalizar Matriz $A$}:
\begin{pasos}
\item \textbf{Polinomio Característico:} Calcular $p(\lambda) = \det(A - \lambda I)$.
\item \textbf{Valores Propios:} Hallar las raíces de $p(\lambda) = 0$.
\item \textbf{Vectores Propios:} Para cada $\lambda_i$, resolver $(A - \lambda_i I)v = 0$ para hallar una base del espacio propio $E_{\lambda_i}$.
\item \textbf{Matriz de Paso $P$:} Formar $P$ con los vectores propios como columnas.
\item \textbf{Diagonalización:} $D = P^{-1}AP$, donde $D$ tiene los $\lambda_i$ en la diagonal.
\end{pasos}


\newpage

% 6. Probabilidades y Estadística
\section{Probabilidades y Estadística (EYP1113)}

\subsection{Álgebra de Eventos y Probabilidad Básica}
\begin{definicion}[title=Conceptos Básicos] \fehandbook{63-65 (Probability)}
\begin{itemize}
    \item \textbf{Espacio Muestral ($\Omega$)}: Conjunto de todos los resultados posibles.
    \item \textbf{Evento ($A$)}: Subconjunto de $\Omega$.
    \item \textbf{Axiomas}: $P(\Omega)=1$, $P(A) \ge 0$, etc. \fehandbook{64 (Laws of Probability)}
\end{itemize}
\end{definicion}

% Contenido placeholder basado en intro.tex
\subsection{Variables Aleatorias y Medidas Descriptivas}

\begin{definicion}[title=Medidas Teóricas] \fehandbook{63 (Dispersion, Mean, Mode)}
Sea $X$ una variable aleatoria:
\begin{itemize}
    \item \textbf{Esperanza (Media):} $E[X] = \mu$.
    \item \textbf{Varianza:} $Var(X) = E[(X-\mu)^2] = E[X^2] - (E[X])^2$.
    \item \textbf{Coef. de Variación:} $CV = \frac{\sigma}{|\mu|}$. \fehandbook{63}
\end{itemize}
\end{definicion}

\subsection{Modelos de Distribución}

\begin{table}[H]
\centering
\caption{Distribuciones Comunes \fehandbook{66-68}}
\rowcolors{2}{gray!10}{white}
\begin{tabular}{|l|l|l|}
\hline
\textbf{Modelo} & \textbf{Parámetros} & \textbf{Aplicación Típica} \\ \hline
Binomial & $n, p$ & Conteo de éxitos en $n$ intentos ($p$). \fehandbookinline{66} \\
Geométrica & $p$ & Intentos hasta el primer éxito. \fehandbookinline{66} \\
Poisson & $\lambda$ & Tasa ocurrencia eventos raros. \fehandbookinline{66} \\
Normal & $\mu, \sigma^2$ & Fenómenos naturales. \fehandbookinline{67} \\
Exponencial & $\lambda$ & Tiempo entre eventos Poisson. \fehandbookinline{67} \\ \hline
\end{tabular}
\end{table}

\begin{tip}[title=Uso de R (Cheat Sheet)]
\begin{itemize}
    \item \texttt{dnorm(x, mean, sd)}: Densidad (altura de la curva).
    \item \texttt{pnorm(q, mean, sd)}: Probabilidad Acumulada $P(X \le q)$.
    \item \texttt{qnorm(p, mean, sd)}: Cuantil (valor $x$ tal que acumula prob $p$).
    \item \texttt{rnorm(n, mean, sd)}: Generar $n$ datos aleatorios.
\end{itemize}
Prefixos: \textbf{d} (denstiy), \textbf{p} (probability), \textbf{q} (quantile), \textbf{r} (random).
\end{tip}

\subsection{Inferencia y Regresión}

\begin{teorema}[title=Teorema del Límite Central] \fehandbook{74 (Confidence Intervals for Mean)}
Para $n$ grande ($n > 30$), la media muestral $\bar{X}$ se distribuye aproximadamente Normal:
$$ \bar{X} \sim N\left(\mu, \frac{\sigma^2}{n}\right) $$
Esto permite construir intervalos de seguridad para $\mu$ sin conocer la distribucion original.
\end{teorema}

\begin{definicion}[title=Regresión Lineal Simple] \fehandbook{69-70}
Modelo: $Y = \beta_0 + \beta_1 X + \epsilon$
\begin{itemize}
    \item Coeficiente de Determinación ($R^2$): \% de variabilidad explicada.
    \item Comando R: \texttt{lm(y $\sim$ x, data=datos)}
\end{itemize}
\end{definicion}

\newpage

% --- EJERCICIOS Y SOLUCIONES ---
\section*{Preguntas de Práctica Seleccionadas}
\begin{ejercicio}[title=Pregunta 1 (MAT1610 - Cálculo I)]
Considere la función $f(x)=-xe^{-\frac{x^{2}}{2}}.$ La función posee un máximo en:
\begin{enumerate}
    \item[a)] $(1,-e^{-\frac{1}{2}})$ \quad \item[b)] $(-1,e^{-\frac{1}{2}})$ \quad \item[c)] $(-1,-e^{-\frac{1}{2}})$ \quad \item[d)] $(1,e^{-\frac{1}{2}})$
\end{enumerate}
\end{ejercicio}

\begin{ejercicio}[title=Pregunta 2 (MAT1620 - Cálculo II)]
Sea R la región delimitada por $0\le y\le2-|x|$. ¿Cuál es el momento de R con respecto al eje X?
\begin{enumerate}
    \item[a)] $1$ \quad \item[b)] $4/3$ \quad \item[c)] $2$ \quad \item[d)] $8/3$
\end{enumerate}
\end{ejercicio}

\begin{ejercicio}[title=Pregunta 3 (MAT1630 - Cálculo III)]
Sea $f(x,y)=x^{y}$. La derivada direccional en el punto (1,2), en la dirección de $\mathbf{v}=(1,1),$ es:
\begin{enumerate}
    \item[a)] $2$ \quad \item[b)] $0$ \quad \item[c)] $\sqrt{2}$ \quad \item[d)] $1$
\end{enumerate}
\end{ejercicio}

\begin{ejercicio}[title=Pregunta 4 (MAT1630 - Cálculo III)]
Sea un cuerpo en el espacio definido por las siguientes desigualdades en coordenadas cilíndricas: $0\le r\le2+\sin(4\theta)$, $0\le\theta\le2\pi$, $0\le z\le1$. ¿Cuál de las siguientes alternativas corresponde al volumen del cuerpo?
\begin{enumerate}
    \item[a)] $2\pi$ \quad \item[b)] $4\pi$ \quad \item[c)] $9\pi/2$ \quad \item[d)] $9\pi$
\end{enumerate}
\end{ejercicio}

\begin{ejercicio}[title=Pregunta 5 (MAT1640 - Ecuaciones)]
Una población posee una tasa de crecimiento instantánea anual de 2\%. ¿Cuántos años le tomará aproximadamente a dicha población triplicar su tamaño?
\begin{enumerate}
    \item[a)] $25 \ln(3)$ \quad \item[b)] $50 \ln(3)$ \quad \item[c)] $2 \ln(3)$ \quad \item[d)] $3 \ln(2)$
\end{enumerate}
\end{ejercicio}

\begin{ejercicio}[title=Pregunta 6 (MAT1640 - Ecuaciones)]
Considere la ecuación: $(x^{2}+y^{2})\dd x-xy\,\dd y=0$. ¿Cuál alternativa la describe mejor?
\begin{enumerate}
    \item[a)] No lineal, homogénea, orden 1. \quad \item[b)] Lineal, no homogénea, orden 2.
    \item[c)] No lineal, no homogénea, orden 2. \quad \item[d)] Lineal, homogénea, orden 1.
\end{enumerate}
\end{ejercicio}

\begin{ejercicio}[title=Pregunta 7 (MAT1203 - Álgebra Lineal)]
Plano $\Pi$: $x-2y+3z=12$. Recta $L$: $\vec{r}(t) = (1,1,-2) + t(2,b,1)$. ¿Cuál es la condición sobre $b$ para que $\Pi\cap L = \emptyset$ (paralelos)?
\begin{enumerate}
    \item[a)] $b\ge5/2$ \quad \item[b)] $b\le5/2$ \quad \item[c)] $b=5/2$ \quad \item[d)] No existe valor
\end{enumerate}
\end{ejercicio}

\begin{ejercicio}[title=Pregunta 8 (MAT1203 - Álgebra Lineal)]
Sobre matrices simétricas, ¿cuáles son verdaderas?
\begin{itemize}
    \item[I.] Resta de simétricas es simétrica.
    \item[II.] Si $AB=BA$, entonces $AB$ es simétrica.
    \item[III.] Matriz $n \times n$ tiene $n$ valores propios reales.
\end{itemize}
\begin{enumerate}
    \item[a)] I y II \quad \item[b)] II y III \quad \item[c)] I y III \quad \item[d)] Todas correctas
\end{enumerate}
\end{ejercicio}

\begin{ejercicio}[title=Respuesta 9 (EYP1113 - Probabilidades)]
Si $X \sim N(\mu=10, \sigma^2=4)$, ¿cuál es el valor estandarizado $Z$ correspondiente a $X=13$?
\begin{enumerate}
    \item[a)] $1.5$ \quad \item[b)] $0.75$ \quad \item[c)] $3$ \quad \item[d)] $0.3$
\end{enumerate}
\end{ejercicio}


\newpage
\section*{Solucionario}
% Tabla Resumen de Respuestas
\begin{center}
\begin{tcolorbox}[colback=Emerald!10, colframe=Emerald, title=Tabla de Respuestas Correctas, width=0.8\textwidth]
\begin{minipage}{\linewidth}
\begin{multicols}{2}
\begin{enumerate}
    \item[1.] \textbf{b)} $(-1,e^{-\frac{1}{2}})$
    \item[2.] \textbf{d)} $8/3$
    \item[3.] \textbf{c)} $\sqrt{2}$
    \item[4.] \textbf{c)} $9\pi/2$
    \item[5.] \textbf{b)} $50 \ln(3)$
    \item[6.] \textbf{a)} No lineal, homogénea, 1er orden.
    \item[7.] \textbf{c)} $b=5/2$
    \item[8.] \textbf{a)} Sólo I y II
\end{enumerate}
\end{multicols}
\end{minipage}
\end{tcolorbox}
\end{center}
\vspace{1cm}

\newpage
%-------------------------------------------------------------------------------
\section*{Solución Pregunta 1 (MAT1610)}
%-------------------------------------------------------------------------------
\begin{ejercicio}[title=Enunciado]
\textbf{Problema:} Encontrar el máximo de $f(x)=-xe^{-x^2/2}$.
\end{ejercicio}

\begin{pasos}
    \item \textbf{Calcular la primera derivada.} Usamos la regla del producto.
    \begin{align*}
    f'(x) &= (-1) \cdot e^{-x^2/2} + (-x) \cdot e^{-x^2/2} \cdot (-x) \\
    &= -e^{-x^2/2} + x^2 e^{-x^2/2} = e^{-x^2/2} (x^2 - 1)
    \end{align*}
    \item \textbf{Encontrar puntos críticos.} Igualamos $f'(x)$ a cero. Como $e^{-x^2/2} > 0$, solo $x^2-1=0 \implies x = \pm 1$.
    \item \textbf{Clasificar los puntos críticos.} Usamos la segunda derivada.
    $$ f''(x) = xe^{-x^2/2}(3-x^2) $$
    Evaluamos:
    \begin{itemize}
        \item $f''(1) = 2e^{-1/2} > 0 \implies$ Mínimo local.
        \item $f''(-1) = -2e^{-1/2} < 0 \implies$ Máximo local.
    \end{itemize}
    El valor máximo es $f(-1) = e^{-1/2}$.
\end{pasos}

\begin{teorema}[title=Respuesta Correcta]
\textbf{b)} $(-1,e^{-\frac{1}{2}})$
\end{teorema}

%-------------------------------------------------------------------------------
\newpage
\section*{Solución Pregunta 2 (MAT1620)}
%-------------------------------------------------------------------------------
\begin{ejercicio}[title=Enunciado]
\textbf{Problema:} Calcular el momento $M_x$ de la región delimitada por $0\le y\le2-|x|$.
\end{ejercicio}

\begin{pasos}
    \item \textbf{Configurar la integral.} $M_x = \frac{1}{2} \int_a^b [f(x)]^2 dx$. Por simetría (de -2 a 2), calculamos de 0 a 2 y multiplicamos por 2. Para $x \ge 0$, $y=2-x$.
    $$ M_x = 2 \int_{0}^{2} \frac{1}{2}(2-x)^2 \, dx = \int_{0}^{2} (4 - 4x + x^2) \, dx $$
    \item \textbf{Evaluar la integral.}
    $$ M_x = \left[ 4x - 2x^2 + \frac{x^3}{3} \right]_0^2 = \left( 8 - 8 + \frac{8}{3} \right) = \frac{8}{3} $$
\end{pasos}

\begin{teorema}[title=Respuesta Correcta]
\textbf{d)} $8/3$
\end{teorema}

%-------------------------------------------------------------------------------
\newpage
\section*{Solución Pregunta 3 (MAT1630)}
%-------------------------------------------------------------------------------
\begin{ejercicio}[title=Enunciado]
\textbf{Problema:} Derivada direccional de $f(x,y)=x^y$ en $(1,2)$ dirección $\mathbf{v}=(1,1)$.
\end{ejercicio}

\begin{pasos}
    \item \textbf{Gradiente.} $\nabla f = \langle yx^{y-1}, x^y \ln(x) \rangle$. En $(1,2)$: $\nabla f(1,2) = \langle 2, 0 \rangle$.
    \item \textbf{Vector unitario.} $||\mathbf{v}|| = \sqrt{1^2+1^2}=\sqrt{2} \implies \mathbf{u} = \langle 1/\sqrt{2}, 1/\sqrt{2} \rangle$.
    \item \textbf{Producto punto.} $D_{\mathbf{u}}f = \langle 2, 0 \rangle \cdot \langle 1/\sqrt{2}, 1/\sqrt{2} \rangle = 2/\sqrt{2} = \sqrt{2}$.
\end{pasos}

\begin{teorema}[title=Respuesta Correcta]
\textbf{c)} $\sqrt{2}$
\end{teorema}

%-------------------------------------------------------------------------------
\newpage
\section*{Solución Pregunta 4 (MAT1630)}
%-------------------------------------------------------------------------------
\begin{ejercicio}[title=Enunciado]
\textbf{Problema:} Volumen en cilíndricas: $\int_{0}^{2\pi} \int_{0}^{2+\sin(4\theta)} \int_{0}^{1} r \, dz \, dr \, d\theta$.
\end{ejercicio}

\begin{pasos}
    \item \textbf{Integral en z.} $\int_{0}^{1} r \, dz = r$.
    \item \textbf{Integral en r.} $\int_{0}^{2+\sin(4\theta)} r \, dr = \frac{1}{2}(2+\sin(4\theta))^2$.
    \item \textbf{Integral en $\theta$.}
    Expandir: $\frac{1}{2}(4 + 4\sin(4\theta) + \sin^2(4\theta))$.
    La integral de $\sin(4\theta)$ en periodo completo es 0.
    La integral de $\sin^2(4\theta)$ es $\pi$. Integral de 4 es $8\pi$.
    Total: $\frac{1}{2}(8\pi + \pi) = \frac{9\pi}{2}$.
\end{pasos}

\begin{teorema}[title=Respuesta Correcta]
\textbf{c)} $9\pi/2$
\end{teorema}

%-------------------------------------------------------------------------------
\newpage
\section*{Solución Pregunta 5 (MAT1640)}
%-------------------------------------------------------------------------------
\begin{ejercicio}[title=Enunciado]
\textbf{Problema:} Tiempo para triplicar población con tasa 2\% ($k=0.02$).
\end{ejercicio}

\begin{pasos}
    \item \textbf{Modelo.} $P(t) = P_0 e^{kt} = P_0 e^{0.02t}$.
    \item \textbf{Resolver.} $3P_0 = P_0 e^{0.02t} \implies 3 = e^{0.02t}$.
    \item \textbf{Despejar t.} $\ln(3) = 0.02t \implies t = \frac{\ln(3)}{0.02} = 50\ln(3)$.
\end{pasos}

\begin{teorema}[title=Respuesta Correcta]
\textbf{b)} $50 \ln(3)$
\end{teorema}

%-------------------------------------------------------------------------------
\newpage
\section*{Solución Pregunta 6 (MAT1640)}
%-------------------------------------------------------------------------------
\begin{ejercicio}[title=Enunciado]
\textbf{Problema:} Clasificar $(x^{2}+y^{2})dx-xy\,dy=0$.
\end{ejercicio}

\begin{pasos}
    \item \textbf{Orden.} Primera derivada ($dy/dx$), primer orden.
    \item \textbf{Linealidad.} Término $y^2$ o $1/y$ implica NO lineal.
    \item \textbf{Homogeneidad.} Grado 2 en todos los términos ($x^2, y^2, xy$). Es Homogénea.
\end{pasos}

\begin{teorema}[title=Respuesta Correcta]
\textbf{a)} No lineal, homogénea, 1er orden.
\end{teorema}

%-------------------------------------------------------------------------------
\newpage
\section*{Solución Pregunta 7 (MAT1203)}
%-------------------------------------------------------------------------------
\begin{ejercicio}[title=Enunciado]
\textbf{Problema:} Intersección vacía entre recta $L$ y plano $\Pi: x-2y+3z=12$.
\end{ejercicio}

\begin{pasos}
    \item \textbf{Vectores.} $\mathbf{n}_{\Pi} = \langle 1, -2, 3 \rangle$, $\mathbf{v}_L = \langle 2, b, 1 \rangle$.
    \item \textbf{Condición.} $\mathbf{n} \cdot \mathbf{v} = 0 \implies 2 - 2b + 3 = 0 \implies 2b = 5 \implies b=5/2$.
    \item \textbf{Verificación.} Punto de recta no debe estar en plano. Confirmado en desarrollo previo.
\end{pasos}

\begin{teorema}[title=Respuesta Correcta]
\textbf{c)} $b=5/2$
\end{teorema}

%-------------------------------------------------------------------------------
\newpage
\section*{Solución Pregunta 8 (MAT1203)}
%-------------------------------------------------------------------------------
\begin{ejercicio}[title=Enunciado]
\textbf{Problema:} Afirmaciones sobre matrices simétricas.
\end{ejercicio}

\begin{pasos}
    \item \textbf{I. Diferencia es simétrica.} VERDADERO. $(A-B)^T = A^T-B^T = A-B$.
    \item \textbf{II. Si $AB=BA$, producto es simétrico.} VERDADERO. $(AB)^T = B^TA^T = BA$. Si $BA=AB$, entonces $(AB)^T=AB$.
    \item \textbf{III. Valores propios reales distintos.} FALSO. Son reales, pero pueden repetirse (ej: Identidad).
\end{pasos}

\begin{teorema}[title=Respuesta Correcta]
\textbf{a)} Sólo I y II
\end{teorema}



\vfill
\begin{center}
    \small Puedes ver este repositorio en \url{https://github.com/anomvlito/respositorio-fundamentals}
\end{center}

\end{document}
