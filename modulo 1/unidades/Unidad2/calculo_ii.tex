\subsection{Técnicas de Integración}

\begin{tip}[title=Integración por Partes]
Recuerda: ``\textbf{U}n \textbf{D}ía \textbf{V}i \textbf{U}na \textbf{V}aca \textbf{V}estida \textbf{D}e \textbf{U}nforme''.
$$
\int u \, \dd v = u v - \int v \, \dd u
$$
\textbf{Estrategia LIATE} para elegir $u$: \textbf{L}ogarítmicas, \textbf{I}nversas, \textbf{A}lgebraicas, \textbf{T}rigonométricas, \textbf{E}xponenciales.
\end{tip}

\subsection{Aplicaciones de la Integral Definida}

\begin{definicion}[title=Área entre Curvas]
Si $f(x) \ge g(x)$ en $[a,b]$:
$$
A = \int_{a}^{b} [f(x) - g(x)] \, \dd x
$$
\end{definicion}

\begin{ejercicio}[title=Área entre Curvas con Valor Absoluto]
\textbf{Problema (MAT1620-3-4):} Considere la región dada por:
$$
(x-2)^2 \le y \le 4-|x|
$$
¿Cuál es el área de la región descrita?

\textbf{Solución:}

\textbf{Paso 1: Identificar las funciones}
\begin{itemize}
    \item Curva inferior: $g(x) = (x-2)^2$ (parábola con vértice en $(2,0)$)
    \item Curva superior: $f(x) = 4-|x|$ (función valor absoluto invertida)
\end{itemize}

\textbf{Paso 2: Encontrar puntos de intersección}

Resolver $(x-2)^2 = 4-|x|$. Dividimos en dos casos:

\textit{Caso 1: $x \ge 0$} → $|x| = x$
\begin{align*}
(x-2)^2 &= 4-x \\
x^2 - 4x + 4 &= 4-x \\
x^2 - 3x &= 0 \\
x(x-3) &= 0 \implies x=0 \text{ o } x=3
\end{align*}

\textit{Caso 2: $x < 0$} → $|x| = -x$
\begin{align*}
(x-2)^2 &= 4+x \\
x^2 - 4x + 4 &= 4+x \\
x^2 - 5x &= 0 \\
x(x-5) &= 0 \implies x=0 \text{ (no válido en } x<0 \text{)}
\end{align*}

Para $x < 0$, si probamos un punto como $x=-1$: $( -1-2)^2 = 9$ y $4-|-1|=3$. Como $9 \not\le 3$, \textbf{no existe región} para $x < 0$.
La región está definida únicamente en el intervalo $x \in [0, 3]$.

\textbf{Paso 3: Calcular el área}

Dado que $x \ge 0$, tenemos $|x|=x$, por lo tanto $f(x) = 4-x$.
\begin{align*}
A &= \int_{0}^{3} [(4-x) - (x-2)^2] \, \dd x \\
&= \int_{0}^{3} [4-x - (x^2 - 4x + 4)] \, \dd x \\
&= \int_{0}^{3} [-x^2 + 3x] \, \dd x \\
&= \left[-\frac{x^3}{3} + \frac{3x^2}{2}\right]_{0}^{3} \\
&= \left(-\frac{27}{3} + \frac{3 \cdot 9}{2}\right) - 0 \\
&= -9 + \frac{27}{2} \\
&= -\frac{18}{2} + \frac{27}{2} = \frac{9}{2}
\end{align*}

\textbf{Alternativa correcta: c) $9/2$}
\end{ejercicio}

\begin{teorema}[title=Centro de Masa (Centroide)]
Para una región plana de densidad constante $\rho$, el centroide $(\bar{x}, \bar{y})$ es:
\begin{align*}
    \bar{x} &= \frac{1}{A} \int_{a}^{b} x [f(x)-g(x)] \, \dd x \\
    \bar{y} &= \frac{1}{A} \int_{a}^{b} \frac{1}{2} ([f(x)]^2 - [g(x)]^2) \, \dd x
\end{align*}
\end{teorema}

\subsection{Series e Integrales Impropias}

\begin{tip}[title=Resumen de Criterios de Convergencia]
\begin{enumerate}
    \item \textbf{Criterio del Término $n$-ésimo (Divergencia):} Si $\lim_{n \to \infty} a_n \neq 0$, la serie diverge.
    
    \item \textbf{Serie Geométrica:} $\sum ar^n$ converge si $|r| < 1$, diverge si $|r| \ge 1$.
    
    \item \textbf{Serie-p:} $\sum \frac{1}{n^p}$ converge si $p > 1$, diverge si $p \le 1$.
    
    \item \textbf{Criterio de la Integral:} Si $f(x)$ es continua, positiva y decreciente, entonces $\sum a_n$ y $\int_{1}^{\infty} f(x) \dd x$ convergen o divergen juntas.
    
    \item \textbf{Criterio de Computación Directa/Límite:} Compara con series conocidas (generalmente p-series o geométricas).
    
    \item \textbf{Criterio de Series Alternantes (Leibniz):} $\sum (-1)^n b_n$ converge si $b_n$ decrece a 0.
    
    \item \textbf{Criterio de la Razón:} $L = \lim |\frac{a_{n+1}}{a_n}|$. $L<1$ (Conv), $L>1$ (Div), $L=1$ (No decide).
    
    \item \textbf{Criterio de la Raíz:} $L = \lim \sqrt[n]{|a_n|}$. Mismas condiciones que la Razón.
\end{enumerate}
\end{tip}

\begin{teorema}[title=Criterio de la Razón (D'Alembert)]
Sea $\sum a_n$. Calculamos $L = \lim_{n \to \infty} \left| \frac{a_{n+1}}{a_n} \right|$:
\begin{itemize}
    \item Si $L < 1 \implies$ Converge Absolutamente.
    \item Si $L > 1 \implies$ Diverge.
    \item Si $L = 1 \implies$ El criterio no decide.
\end{itemize}
\end{teorema}

\begin{teorema}[title=Criterio de Comparación en el Límite]
Si $a_n, b_n > 0$ y $\lim_{n \to \infty} \frac{a_n}{b_n} = c$ ($0 < c < \infty$), entonces:
$$
\sum a_n \text{ y } \sum b_n \text{ se comportan igual (ambas Convergen o ambas Divergen).}
$$
Útil para comparar con p-series: $\sum \frac{1}{n^p}$ conv. si $p>1$.
\end{teorema}

\begin{ejercicio}[title=Divergencia de Series]
\textbf{Problema:} ¿Cuál de las siguientes series es \textbf{DIVERGENTE}?
\begin{enumerate}[label=\alph*)]
    \item $\displaystyle \sum_{n=1}^{\infty} \frac{\sqrt{n^3+2n+1}}{\sqrt{n^5+8}}$
    \item $\displaystyle \sum_{n=1}^{\infty} \frac{\sin(n^2+1)}{n^2+1}$
    \item $\displaystyle \sum_{n=1}^{\infty} \frac{\cos(n\pi)}{n+\pi}$
    \item $\displaystyle \sum_{n=1}^{\infty} \frac{n!}{n^n}$
\end{enumerate}

\textbf{Solución Detallada:}

\textbf{a)} \textbf{Análisis Riguroso (Limit Comparison Test):}

1. \textbf{Definimos las series:}
Comparamos nuestra serie original ($a_n$) con la serie armónica ($b_n$), que sabemos que DIVERGE.
$$ a_n = \frac{\sqrt{n^3+2n+1}}{\sqrt{n^5+8}} \quad , \quad b_n = \frac{1}{n} $$

2. \textbf{Planteamos el Límite:}
Calculamos $L = \lim_{n \to \infty} \frac{a_n}{b_n}$. Si $0 < L < \infty$, ambas se comportan igual.
$$ L = \lim_{n \to \infty} \frac{\frac{\sqrt{n^3+2n+1}}{\sqrt{n^5+8}}}{\frac{1}{n}} = \lim_{n \to \infty} \left( n \cdot \frac{\sqrt{n^3+2n+1}}{\sqrt{n^5+8}} \right) $$

3. \textbf{Factorización de términos dominantes:}
No basta con decir "se parece a". Factorizamos la potencia mayor \textit{dentro} de cada raíz para demostrar formalmente que los términos menores desaparecen.
$$ L = \lim_{n \to \infty} n \cdot \frac{\sqrt{n^3(1 + \frac{2n}{n^3} + \frac{1}{n^3})}}{\sqrt{n^5(1 + \frac{8}{n^5})}} = \lim_{n \to \infty} n \cdot \frac{\sqrt{n^3} \cdot \sqrt{1 + \frac{2}{n^2} + \frac{1}{n^3}}}{\sqrt{n^5} \cdot \sqrt{1 + \frac{8}{n^5}}} $$

4. \textbf{Cancelación y Evaluación:}
Sabemos que $\sqrt{n^3} = n^{3/2}$ y $\sqrt{n^5} = n^{5/2}$. Agrupamos las potencias de $n$:
$$ L = \lim_{n \to \infty} \left( \frac{n \cdot n^{3/2}}{n^{5/2}} \right) \cdot \frac{\sqrt{1 + \frac{2}{n^2} + \frac{1}{n^3}}}{\sqrt{1 + \frac{8}{n^5}}} $$
Observamos que $n \cdot n^{3/2} = n^{1+1.5} = n^{2.5} = n^{5/2}$. Los términos se cancelan exactamente ($\frac{n^{5/2}}{n^{5/2}} = 1$).
$$ L = 1 \cdot \frac{\sqrt{1+0+0}}{\sqrt{1+0}} = 1 $$

Como $L=1$ (finito y positivo), y $\sum b_n$ diverge, \textbf{la serie a) DIVERGE}.

\textbf{Análisis de las otras opciones:}

\textbf{b)} Convergencia Absoluta:
$$
\left| \frac{\sin(n^2+1)}{n^2+1} \right| \le \frac{1}{n^2+1} < \frac{1}{n^2}
$$
Como la función seno está acotada entre $[-1, 1]$, el numerador no crece. Comparamos con la p-serie convergente $\sum \frac{1}{n^2}$ ($p=2 > 1$). Converge.

\textbf{c)} Serie Alternante:
La serie se puede reescribir considerando que $\cos(n\pi) = (-1)^n$:
$$
\sum_{n=1}^{\infty} \frac{(-1)^n}{n+\pi}
$$
Esta es una \textbf{Serie Alternante}. Verificamos el Criterio de Leibniz:
1. ¿Son los términos decrecientes? Sí, $\frac{1}{n+1+\pi} < \frac{1}{n+\pi}$.
2. ¿El límite es 0? Sí, $\lim_{n \to \infty} \frac{1}{n+\pi} = 0$.
Por lo tanto, la serie \textbf{converge} (condicionalmente).

\textbf{d)} Criterio de la Razón (Converge):
Usamos D'Alembert para términos con factoriales y potencias $n$-ésimas:
\begin{align*}
L &= \lim_{n \to \infty} \left| \frac{a_{n+1}}{a_n} \right| = \lim_{n \to \infty} \frac{(n+1)!}{(n+1)^{n+1}} \cdot \frac{n^n}{n!} \\
&= \lim_{n \to \infty} \frac{(n+1)n! \cdot n^n}{(n+1) \cdot (n+1)^n \cdot n!} \\
&= \lim_{n \to \infty} \left(\frac{n}{n+1}\right)^n = \lim_{n \to \infty} \frac{1}{(1+1/n)^n} = \frac{1}{e}
\end{align*}
Como $L = \frac{1}{e} \approx \frac{1}{2.718} < 1$, la serie \textbf{converge} absolutamente.

\textbf{Alternativa correcta: a)}
\end{ejercicio}

\subsection{Geometría en el Espacio}

\begin{definicion}[title=Rectas y Planos]
\begin{itemize}
    \item \textbf{Recta} por $P_0$ con dirección $\vec{v}$: 
    $$ \vec{r}(t) = P_0 + t\vec{v} $$
    \item \textbf{Plano} por $P_0(x_0, y_0, z_0)$ con normal $\vec{n}=\langle a,b,c \rangle$:
    $$ a(x-x_0) + b(y-y_0) + c(z-z_0) = 0 $$
\end{itemize}
\end{definicion}
