\subsection{Clasificación y Primer Orden}

\begin{definicion}[title=Conceptos Básicos]
\begin{itemize}
    \item \textbf{Orden:} Derivada más alta presente (ej: $y''$ es orden 2).
    \item \textbf{Linealidad:} La variable $y$ y sus derivadas tienen potencia 1 y no se multiplican entre sí.
\end{itemize}
\end{definicion}

\begin{teorema}[title=Método: Variables Separables]
Si la EDO se puede escribir como $f(y) \, \dd y = g(x) \, \dd x$:
\begin{enumerate}
    \item Separar variables a cada lado del igual.
    \item Integrar ambos lados: $\int f(y) \dd y = \int g(x) \dd x$.
    \item Despejar $y(x)$ si es posible.
\end{enumerate}
\end{teorema}

\begin{teorema}[title=Método: Factor Integrante (Ec. Lineales 1er Orden)]
Para ecuaciones de la forma $y' + P(x)y = Q(x)$:
\begin{enumerate}
    \item Calcular el factor integrante $\mu(x) = e^{\int P(x) \dd x}$.
    \item Multiplicar toda la ecuación por $\mu(x)$. El lado izquierdo colapsa a $(\mu \cdot y)'$.
    \item Integrar: $\mu(x) \cdot y = \int \mu(x) Q(x) \dd x$.
    \item Despejar $y$.
\end{enumerate}
\end{teorema}

\subsection{Ecuaciones de Segundo Orden (Coef. Constantes)}

Para resolver $ay'' + by' + cy = 0$:
\begin{pasos}
\item Escribir la \textbf{ecuación característica}: $ar^2 + br + c = 0$.
\item Hallar las raíces $r_1, r_2$:
\begin{itemize}
    \item \textbf{Reales distintas} ($r_1 \ne r_2$): $y = C_1 e^{r_1 x} + C_2 e^{r_2 x}$
    \item \textbf{Reales iguales} ($r_1 = r_2 = r$): $y = C_1 e^{rx} + C_2 x e^{rx}$
    \item \textbf{Complejas} ($\alpha \pm \beta i$): $y = e^{\alpha x}(C_1 \cos(\beta x) + C_2 \sin(\beta x))$
\end{itemize}
\end{pasos}

\subsection{Sistemas de Ecuaciones Diferenciales Lineales}

\begin{definicion}[title=Sistema Lineal Homogéneo]
Un sistema de la forma $\mathbf{x}'(t) = A\mathbf{x}(t)$ donde $A$ es una matriz constante $n \times n$.

\textbf{Solución general:} $\mathbf{x}(t) = e^{At}\mathbf{x}(0)$

donde $e^{At}$ es la matriz exponencial definida por:
$$e^{At} = I + At + \frac{(At)^2}{2!} + \frac{(At)^3}{3!} + \cdots$$
\end{definicion}

\begin{teorema}[title=Comportamiento Asintótico]
El comportamiento de $\mathbf{x}(t)$ cuando $t \to \infty$ depende de los valores propios $\lambda$ de $A$:
\begin{itemize}
    \item Si $\text{Re}(\lambda) < 0$ para todos los $\lambda$: el sistema es \textbf{estable} y $\mathbf{x}(t) \to \mathbf{0}$
    \item Si existe $\lambda$ con $\text{Re}(\lambda) > 0$: el sistema es \textbf{inestable} y diverge
    \item Si $\text{Re}(\lambda) = 0$: el comportamiento depende de la multiplicidad algebraica/geométrica
\end{itemize}
\end{teorema}

\begin{ejercicio}[title=Comportamiento Asintótico de Sistemas Lineales]
Sea $\mathbf{x}(t)$ la solución al sistema homogéneo $\mathbf{x}'(t) = A\mathbf{x}(t)$ con $\mathbf{x}(0) = \begin{bmatrix}1\\0\end{bmatrix}$ donde:

$$A\begin{bmatrix}1\\-1\end{bmatrix} = \begin{bmatrix}0\\0\end{bmatrix} \quad \text{y} \quad A\begin{bmatrix}1\\1\end{bmatrix} = -\frac{1}{2}\begin{bmatrix}1\\1\end{bmatrix}$$

Si $u = \lim_{t\to\infty} e^{At}\mathbf{x}(t)$, entonces $u$ es igual a:

\begin{enumerate}[label=\alph*)]
    \item $\displaystyle \frac{1}{2}\begin{bmatrix}1\\-1\end{bmatrix}$
    
    \item $\displaystyle \begin{bmatrix}1\\-1\end{bmatrix}$
    
    \item $\displaystyle \frac{1}{2}\begin{bmatrix}1\\1\end{bmatrix}$
    
    \item $\displaystyle \begin{bmatrix}1\\1\end{bmatrix}$
\end{enumerate}
\end{ejercicio}

\begin{solbox}
\textbf{Respuesta correcta: a)} $\displaystyle \frac{1}{2}\begin{bmatrix}1\\-1\end{bmatrix}$

\textbf{Análisis paso a paso:}

\textbf{1. Identificar valores y vectores propios:}

De la información dada:
\begin{itemize}
    \item $\mathbf{v}_1 = \begin{bmatrix}1\\-1\end{bmatrix}$ es vector propio con $\lambda_1 = 0$ (ya que $A\mathbf{v}_1 = \mathbf{0}$)
    \item $\mathbf{v}_2 = \begin{bmatrix}1\\1\end{bmatrix}$ es vector propio con $\lambda_2 = -\frac{1}{2}$
\end{itemize}

\textbf{2. Expresar la condición inicial en la base de vectores propios:}

Escribimos $\mathbf{x}(0) = \begin{bmatrix}1\\0\end{bmatrix}$ como combinación lineal:

$$\begin{bmatrix}1\\0\end{bmatrix} = c_1\begin{bmatrix}1\\-1\end{bmatrix} + c_2\begin{bmatrix}1\\1\end{bmatrix}$$

Resolviendo: $c_1 + c_2 = 1$ y $-c_1 + c_2 = 0$ $\Rightarrow$ $c_1 = c_2 = \frac{1}{2}$

\textbf{3. Solución del sistema:}

$$\mathbf{x}(t) = e^{At}\mathbf{x}(0) = \frac{1}{2}e^{0 \cdot t}\begin{bmatrix}1\\-1\end{bmatrix} + \frac{1}{2}e^{-\frac{t}{2}}\begin{bmatrix}1\\1\end{bmatrix}$$

Cuando $t \to \infty$: el término $e^{-\frac{t}{2}} \to 0$, por lo tanto:

$$\lim_{t\to\infty} \mathbf{x}(t) = \frac{1}{2}\begin{bmatrix}1\\-1\end{bmatrix}$$

\textbf{4. Calcular el límite pedido:}

$$u = \lim_{t\to\infty} e^{At}\mathbf{x}(t) = \lim_{t\to\infty} e^{At} \cdot e^{At}\mathbf{x}(0) = \lim_{t\to\infty} e^{2At}\mathbf{x}(0)$$

Usando la descomposición en vectores propios:
$$e^{2At}\mathbf{x}(0) = \frac{1}{2}e^{2\lambda_1 t}\mathbf{v}_1 + \frac{1}{2}e^{2\lambda_2 t}\mathbf{v}_2 = \frac{1}{2}e^{0}\begin{bmatrix}1\\-1\end{bmatrix} + \frac{1}{2}e^{-t}\begin{bmatrix}1\\1\end{bmatrix}$$

Cuando $t \to \infty$: $e^{-t} \to 0$, entonces:

$$u = \frac{1}{2}\begin{bmatrix}1\\-1\end{bmatrix}$$

\textbf{Interpretación:} El vector propio asociado al valor propio $\lambda = 0$ domina el comportamiento a largo plazo, mientras que el componente con $\lambda = -\frac{1}{2}$ decae exponencialmente.
\end{solbox}

