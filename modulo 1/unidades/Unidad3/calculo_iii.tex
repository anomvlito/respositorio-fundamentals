\subsection{Diferenciación Multivariable}

\begin{definicion}[title=Gradiente y su Significado]
Sea $f: \R^n \to \R$. El gradiente es el vector:
$$ \nabla f = \left\langle \frac{\partial f}{\partial x_1}, \dots, \frac{\partial f}{\partial x_n} \right\rangle $$
\textbf{Propiedades Clave:}
\begin{itemize}
    \item $\nabla f$ apunta a la dirección de \textbf{máximo crecimiento}.
    \item La tasa máxima de cambio es $\norm{\nabla f}$.
    \item $\nabla f$ es \textbf{perpendicular} (ortogonal) a las curvas/superficies de nivel.
\end{itemize}
\end{definicion}

\begin{teorema}[title=Derivada Direccional]
La derivada de $f$ en la dirección del vector unitario $\vec{u}$:
$$ D_{\vec{u}}f(P) = \nabla f(P) \cdot \vec{u} $$
{\small \textbf{Nota:} Si $\vec{u}$ no es unitario, normalizar $\vec{u} \leftarrow \frac{\vec{u}}{\norm{\vec{u}}}$ antes de usar la fórmula.}
\end{teorema}

\begin{tip}[title=Optimización con Multiplicadores de Lagrange]
Para maximizar/minimizar $f(x,y,z)$ sujeto a la restricción $g(x,y,z)=k$:
\begin{enumerate}
    \item Resolver el sistema: $\nabla f = \lambda \nabla g$.
    \item Considerar también la restricción: $g(x,y,z) = k$.
\end{enumerate}
\end{tip}

\subsection{Integrales Múltiples y Cambios de Coordenadas}

\begin{definicion}[title=Coordenadas Polares (En el plano $xy$)]
$$ x = r\cos\theta, \quad y = r\sin\theta, \quad x^2+y^2=r^2 $$
\textbf{Jacobiano (Factor de corrección):} $\dd A = r \, \dd r \, \dd\theta$
\end{definicion}

\begin{teorema}[title=Coordenadas Esféricas (Espacio 3D)]
Usar para esferas o conos.
\begin{itemize}
    \item $x = \rho \sin\phi \cos\theta$
    \item $y = \rho \sin\phi \sin\theta$
    \item $z = \rho \cos\phi$
\end{itemize}
\textbf{Jacobiano de volumen:} $\dd V = \rho^2 \sin\phi \, \dd\rho \, \dd\phi \, \dd\theta$
\end{teorema}
