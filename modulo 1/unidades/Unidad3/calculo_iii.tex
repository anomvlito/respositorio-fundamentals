\subsection{Diferenciación Multivariable}

\begin{definicion}[title=Gradiente y su Significado] \fehandbook{59 (Vectors: Gradient, Divergence, Curl)}
Sea $f: \R^n \to \R$. El gradiente es el vector:
$$ \nabla f = \left\langle \frac{\partial f}{\partial x_1}, \dots, \frac{\partial f}{\partial x_n} \right\rangle $$
\textbf{Propiedades Clave:}
\begin{itemize}
    \item $\nabla f$ apunta a la dirección de \textbf{máximo crecimiento}.
    \item La tasa máxima de cambio es $\norm{\nabla f}$.
    \item $\nabla f$ es \textbf{perpendicular} (ortogonal) a las curvas/superficies de nivel.
\end{itemize}
\end{definicion}

\begin{teorema}[title=Derivada Direccional] \feimply{Concepto deducible de producto punto (p.59), pero no explícito.}
La derivada de $f$ en la dirección del vector unitario $\vec{u}$:
$$ D_{\vec{u}}f(P) = \nabla f(P) \cdot \vec{u} $$
{\small \textbf{Nota:} Si $\vec{u}$ no es unitario, normalizar $\vec{u} \leftarrow \frac{\vec{u}}{\norm{\vec{u}}}$ antes de usar la fórmula.}
\end{teorema}

\begin{tip}[title=Optimización con Multiplicadores de Lagrange] \feimply{NO ESTÁ en el manual.}
Para maximizar/minimizar $f(x,y,z)$ sujeto a la restricción $g(x,y,z)=k$:
\begin{enumerate}
    \item Resolver el sistema: $\nabla f = \lambda \nabla g$.
    \item Considerar también la restricción: $g(x,y,z) = k$.
\end{enumerate}
\end{tip}

\subsection{Integrales Múltiples y Cambios de Coordenadas}

\begin{definicion}[title=Coordenadas Polares (En el plano $xy$)] \fehandbook{36 (Coordinate Systems)}
$$ x = r\cos\theta, \quad y = r\sin\theta, \quad x^2+y^2=r^2 $$
\textbf{Jacobiano (Factor de corrección):} $\dd A = r \, \dd r \, \dd\theta$ \feimply{Jacobiano $r$ no explícito en sección integración.}
\end{definicion}

\begin{teorema}[title=Coordenadas Esféricas (Espacio 3D)] \fehandbook{36}
Usar para esferas o conos.
\begin{itemize}
    \item $x = \rho \sin\phi \cos\theta$
    \item $y = \rho \sin\phi \sin\theta$
    \item $z = \rho \cos\phi$
\end{itemize}
\textbf{Jacobiano de volumen:} $\dd V = \rho^2 \sin\phi \, \dd\rho \, \dd\phi \, \dd\theta$
\end{teorema}

\begin{ejercicio}[title=MAT1630-2-3 (2025-1)]
Considere el sólido $E$ en el primer octante delimitado por los planos $x = 0$, $y = 0$, $z = 0$ y la superficie $z=4-x^2-y^2$.

¿Cuál de las siguientes integrales iteradas permite calcular el volumen de $E$?

\begin{enumerate}[label=\alph*)]
    \item $\displaystyle \int_0^2 \int_0^2 (4-x^2-y^2) \, dy \, dx$
    
    \item $\displaystyle \int_0^{\pi/2} \int_0^2 (4-r^2) \, dr \, d\theta$
    
    \item $\displaystyle \int_0^2 \int_0^2 \int_0^{4-x^2-y^2} 1 \, dz \, dy \, dx$
    
    \item $\displaystyle \int_0^2 \int_0^{\sqrt{4-x^2}} \int_0^{4-x^2-y^2} 1 \, dz \, dy \, dx$
\end{enumerate}
\end{ejercicio}

\begin{solbox}
\textbf{Respuesta correcta: d)}

\textbf{Análisis de cada opción:}

\begin{itemize}
    \item \textbf{Opción a):} Los límites de integración son incorrectos. Para $x=2$ y $y=2$, tendríamos $z=4-4-4=-4<0$, lo cual está fuera del primer octante.
    
    \item \textbf{Opción b):} Falta el factor $r$ del Jacobiano en coordenadas polares. Debería ser $\int_0^{\pi/2} \int_0^2 r(4-r^2) \, dr \, d\theta$.
    
    \item \textbf{Opción c):} Similar a la opción a), los límites de $y$ son incorrectos. No considera que la región de integración en el plano $xy$ es circular.
    
    \item \textbf{Opción d):} \textbf{CORRECTA.} 
    \begin{itemize}
        \item El límite superior de $z$ es la superficie $z=4-x^2-y^2$.
        \item Para cada $x$ fijo, $y$ varía desde $0$ hasta $\sqrt{4-x^2}$ (semicírculo en el primer cuadrante).
        \item $x$ varía de $0$ a $2$ (donde la superficie intersecta el plano $xy$ cuando $z=0$: $4-x^2-y^2=0 \Rightarrow x^2+y^2=4$).
    \end{itemize}
\end{itemize}

\textbf{Verificación:} La región de integración en el plano $xy$ es un cuarto de círculo de radio 2 (primer cuadrante de $x^2+y^2 \leq 4$), y la altura va desde $z=0$ hasta $z=4-x^2-y^2$.
\end{solbox}

