\section{Probabilidades y Estadística (EYP1113)}

\subsection{Álgebra de Eventos y Probabilidad Básica}
\begin{definicion}[title=Conceptos Básicos] \fehandbook{63-65 (Probability)}
\begin{itemize}
    \item \textbf{Espacio Muestral ($\Omega$)}: Conjunto de todos los resultados posibles.
    \item \textbf{Evento ($A$)}: Subconjunto de $\Omega$.
    \item \textbf{Axiomas}: $P(\Omega)=1$, $P(A) \ge 0$, etc. \fehandbook{64 (Laws of Probability)}
\end{itemize}
\end{definicion}

% Contenido placeholder basado en intro.tex
\subsection{Variables Aleatorias y Medidas Descriptivas}

\begin{definicion}[title=Medidas Teóricas] \fehandbook{63 (Dispersion, Mean, Mode)}
Sea $X$ una variable aleatoria:
\begin{itemize}
    \item \textbf{Esperanza (Media):} $E[X] = \mu$.
    \item \textbf{Varianza:} $Var(X) = E[(X-\mu)^2] = E[X^2] - (E[X])^2$.
    \item \textbf{Coef. de Variación:} $CV = \frac{\sigma}{|\mu|}$. \fehandbook{63}
\end{itemize}
\end{definicion}

\subsection{Modelos de Distribución}

\begin{table}[H]
\centering
\caption{Distribuciones Comunes \fehandbook{66-68}}
\rowcolors{2}{gray!10}{white}
\begin{tabular}{|l|l|l|}
\hline
\textbf{Modelo} & \textbf{Parámetros} & \textbf{Aplicación Típica} \\ \hline
Binomial & $n, p$ & Conteo de éxitos en $n$ intentos ($p$). \fehandbookinline{66} \\
Geométrica & $p$ & Intentos hasta el primer éxito. \fehandbookinline{66} \\
Poisson & $\lambda$ & Tasa ocurrencia eventos raros. \fehandbookinline{66} \\
Normal & $\mu, \sigma^2$ & Fenómenos naturales. \fehandbookinline{67} \\
Exponencial & $\lambda$ & Tiempo entre eventos Poisson. \fehandbookinline{67} \\ \hline
\end{tabular}
\end{table}

\begin{tip}[title=Uso de R (Cheat Sheet)]
\begin{itemize}
    \item \texttt{dnorm(x, mean, sd)}: Densidad (altura de la curva).
    \item \texttt{pnorm(q, mean, sd)}: Probabilidad Acumulada $P(X \le q)$.
    \item \texttt{qnorm(p, mean, sd)}: Cuantil (valor $x$ tal que acumula prob $p$).
    \item \texttt{rnorm(n, mean, sd)}: Generar $n$ datos aleatorios.
\end{itemize}
Prefixos: \textbf{d} (denstiy), \textbf{p} (probability), \textbf{q} (quantile), \textbf{r} (random).
\end{tip}

\subsection{Inferencia y Regresión}

\begin{teorema}[title=Teorema del Límite Central] \fehandbook{74 (Confidence Intervals for Mean)}
Para $n$ grande ($n > 30$), la media muestral $\bar{X}$ se distribuye aproximadamente Normal:
$$ \bar{X} \sim N\left(\mu, \frac{\sigma^2}{n}\right) $$
Esto permite construir intervalos de seguridad para $\mu$ sin conocer la distribucion original.
\end{teorema}

\begin{definicion}[title=Regresión Lineal Simple] \fehandbook{69-70}
Modelo: $Y = \beta_0 + \beta_1 X + \epsilon$
\begin{itemize}
    \item Coeficiente de Determinación ($R^2$): \% de variabilidad explicada.
    \item Comando R: \texttt{lm(y $\sim$ x, data=datos)}
\end{itemize}
\end{definicion}
