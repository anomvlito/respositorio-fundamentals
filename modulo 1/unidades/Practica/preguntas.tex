\begin{ejercicio}[title=Pregunta 1 (MAT1610 - Cálculo I)]
Considere la función $f(x)=-xe^{-\frac{x^{2}}{2}}.$ La función posee un máximo en:
\begin{enumerate}
    \item[a)] $(1,-e^{-\frac{1}{2}})$ \quad \item[b)] $(-1,e^{-\frac{1}{2}})$ \quad \item[c)] $(-1,-e^{-\frac{1}{2}})$ \quad \item[d)] $(1,e^{-\frac{1}{2}})$
\end{enumerate}
\end{ejercicio}

\begin{ejercicio}[title=Pregunta 2 (MAT1620 - Cálculo II)]
Sea R la región delimitada por $0\le y\le2-|x|$. ¿Cuál es el momento de R con respecto al eje X?
\begin{enumerate}
    \item[a)] $1$ \quad \item[b)] $4/3$ \quad \item[c)] $2$ \quad \item[d)] $8/3$
\end{enumerate}
\end{ejercicio}

\begin{ejercicio}[title=Pregunta 3 (MAT1630 - Cálculo III)]
Sea $f(x,y)=x^{y}$. La derivada direccional en el punto (1,2), en la dirección de $\mathbf{v}=(1,1),$ es:
\begin{enumerate}
    \item[a)] $2$ \quad \item[b)] $0$ \quad \item[c)] $\sqrt{2}$ \quad \item[d)] $1$
\end{enumerate}
\end{ejercicio}

\begin{ejercicio}[title=Pregunta 4 (MAT1630 - Cálculo III)]
Sea un cuerpo en el espacio definido por las siguientes desigualdades en coordenadas cilíndricas: $0\le r\le2+\sin(4\theta)$, $0\le\theta\le2\pi$, $0\le z\le1$. ¿Cuál de las siguientes alternativas corresponde al volumen del cuerpo?
\begin{enumerate}
    \item[a)] $2\pi$ \quad \item[b)] $4\pi$ \quad \item[c)] $9\pi/2$ \quad \item[d)] $9\pi$
\end{enumerate}
\end{ejercicio}

\begin{ejercicio}[title=Pregunta 5 (MAT1640 - Ecuaciones)]
Una población posee una tasa de crecimiento instantánea anual de 2\%. ¿Cuántos años le tomará aproximadamente a dicha población triplicar su tamaño?
\begin{enumerate}
    \item[a)] $25 \ln(3)$ \quad \item[b)] $50 \ln(3)$ \quad \item[c)] $2 \ln(3)$ \quad \item[d)] $3 \ln(2)$
\end{enumerate}
\end{ejercicio}

\begin{ejercicio}[title=Pregunta 6 (MAT1640 - Ecuaciones)]
Considere la ecuación: $(x^{2}+y^{2})\dd x-xy\,\dd y=0$. ¿Cuál alternativa la describe mejor?
\begin{enumerate}
    \item[a)] No lineal, homogénea, orden 1. \quad \item[b)] Lineal, no homogénea, orden 2.
    \item[c)] No lineal, no homogénea, orden 2. \quad \item[d)] Lineal, homogénea, orden 1.
\end{enumerate}
\end{ejercicio}

\begin{ejercicio}[title=Pregunta 7 (MAT1203 - Álgebra Lineal)]
Plano $\Pi$: $x-2y+3z=12$. Recta $L$: $\vec{r}(t) = (1,1,-2) + t(2,b,1)$. ¿Cuál es la condición sobre $b$ para que $\Pi\cap L = \emptyset$ (paralelos)?
\begin{enumerate}
    \item[a)] $b\ge5/2$ \quad \item[b)] $b\le5/2$ \quad \item[c)] $b=5/2$ \quad \item[d)] No existe valor
\end{enumerate}
\end{ejercicio}

\begin{ejercicio}[title=Pregunta 8 (MAT1203 - Álgebra Lineal)]
Sobre matrices simétricas, ¿cuáles son verdaderas?
\begin{itemize}
    \item[I.] Resta de simétricas es simétrica.
    \item[II.] Si $AB=BA$, entonces $AB$ es simétrica.
    \item[III.] Matriz $n \times n$ tiene $n$ valores propios reales.
\end{itemize}
\begin{enumerate}
    \item[a)] I y II \quad \item[b)] II y III \quad \item[c)] I y III \quad \item[d)] Todas correctas
\end{enumerate}
\end{ejercicio}

\begin{ejercicio}[title=Respuesta 9 (EYP1113 - Probabilidades)]
Si $X \sim N(\mu=10, \sigma^2=4)$, ¿cuál es el valor estandarizado $Z$ correspondiente a $X=13$?
\begin{enumerate}
    \item[a)] $1.5$ \quad \item[b)] $0.75$ \quad \item[c)] $3$ \quad \item[d)] $0.3$
\end{enumerate}
\end{ejercicio}
