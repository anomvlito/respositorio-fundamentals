% \section*{MÓDULO 1: Matemáticas y Probabilidades}

\subsection{Cálculo I (MAT1610)}
\begin{definicion}[title=Contenidos]
\begin{enumerate}
    \item[1.] Geometría Analítica
\end{enumerate}
\end{definicion}

\begin{teorema}[title=Indicadores a evaluar (Números corresponden al correlativo del programa de cada curso)]
\begin{enumerate}
    \item[1.] Identificar gráficos de funciones básicas, exponenciales, logarítmicas.
    \item[4.] Calcular derivadas de funciones obtenidas por álgebra de funciones elementales.
    \item[6.] Reconocer gráfica y analíticamente propiedades de los gráficos de funciones (crecimiento, concavidad, máx/mín, asíntotas).
    \item[9.] Conocer el cálculo de primitivas de funciones básicas.
\end{enumerate}
\end{teorema}

\subsection{Cálculo II (MAT1620)}
\begin{definicion}[title=Contenidos]
\begin{enumerate}
    \item[1.] Cálculo Integral
\end{enumerate}
\end{definicion}

\begin{teorema}[title=Indicadores a evaluar (Números corresponden al correlativo del programa de cada curso)]
\begin{enumerate}
    \item[3.] Aplicar el concepto de integral definida para calcular áreas y momentos de regiones del plano.
    \item[5.] Aplicar los criterios básicos de convergencia de series e integrales impropias.
    \item[8.] Conocer las ecuaciones paramétricas, vectoriales y cartesianas de rectas y planos.
\end{enumerate}
\end{teorema}

\subsection{Cálculo III (MAT1630)}
\begin{definicion}[title=Contenidos]
\begin{enumerate}
    \item[1.] Cálculo Diferencial
\end{enumerate}
\end{definicion}

\begin{teorema}[title=Indicadores a evaluar (Números corresponden al correlativo del programa de cada curso)]
\begin{enumerate}
    \item[2.] Aplicar el concepto de integral múltiple para evaluar volúmenes y centros de masa.
    \item[5.] Reconocer y explicar el concepto de “curvas de nivel” y calcularlas.
    \item[6.] Calcular derivadas direccionales.
\end{enumerate}
\end{teorema}

\subsection{Ecuaciones Diferenciales (MAT1640)}
\begin{definicion}[title=Contenidos]
\begin{enumerate}
    \item[1.] Ecuaciones Diferenciales
\end{enumerate}
\end{definicion}

\begin{teorema}[title=Indicadores a evaluar (Números corresponden al correlativo del programa de cada curso)]
\begin{enumerate}
    \item[2.] Modelar situaciones sencillas de la realidad y fenómenos mediante ecuaciones diferenciales.
    \item[3.] Reconocer tipo de EDO, identificar y utilizar métodos de solución según el caso.
    \item[6.] Calcular soluciones de sistemas lineales de $2\times2$ y $3\times3$ (coef. constantes).
\end{enumerate}
\end{teorema}

\subsection{Álgebra Lineal (MAT1203)}
\begin{definicion}[title=Contenidos]
\begin{enumerate}
    \item[1.] Matrices
    \item[2.] Raíces de Ecuaciones
    \item[3.] Análisis Vectorial
\end{enumerate}
\end{definicion}

\begin{teorema}[title=Indicadores a evaluar (Números corresponden al correlativo del programa de cada curso)]
\begin{enumerate}
    \item[1.] Determinar escalonada reducida, resolver $Ax=b$, calcular inversas y bases.
    \item[2.] Interpretar geométricamente dependencia lineal, complemento ortogonal.
    \item[4.] Explicar y utilizar propiedades de operaciones matriciales.
    \item[6.] Explicar y utilizar matrices elementales, simétricas, ortogonales, etc.
    \item[7.] Calcular determinantes, resolver sistemas y evaluar inversas.
    \item[9.] Determinar matriz de Transformación Lineal y relación con cambio de base.
    \item[12.] Explicar valores/vectores propios, diagonalización y aplicaciones (simétricas).
\end{enumerate}
\end{teorema}

\subsection{Probabilidades y Estadística (EYP1113)}
\begin{definicion}[title=Contenidos]
\begin{enumerate}
    \item[1.] Álgebra de eventos, axiomas, prob. condicional, Bayes.
    \item[2.] Medidas descriptivas teóricas (media, varianza, percentil, etc.).
    \item[3.] Modelos Discretos/Continuos (Binomial, Poisson, Normal, Exp, etc.) y uso de R.
    \item[4.] Distribuciones conjuntas, covarianza, correlación.
    \item[5.] Estimación y propiedades.
    \item[6.] Test de hipótesis e intervalos de confianza.
    \item[7.] Bondad de ajuste (Chi-cuadrado).
    \item[8.] Regresión lineal (test-t, test-F, $R^2$).
\end{enumerate}
\end{definicion}

\begin{teorema}[title=Indicadores a evaluar (Números corresponden al correlativo del programa de cada curso)]
\begin{enumerate}
    \item[1.] Ajustar distribuciones de probabilidad a datos reales.
    \item[2.] Describir fenómenos de incertidumbre usando variables aleatorias.
    \item[3.] Realizar estimaciones de parámetros e intervalos de confianza.
    \item[4.] Ajustar e interpretar modelos de regresión lineal.
\end{enumerate}
\end{teorema}
