% \section*{MÓDULO 1: Matemáticas y Probabilidades}

\subsection{Cálculo I (MAT1610)}
\begin{definicion}[title=Contenidos]
\begin{enumerate}
    \item[1.] Geometría Analítica
\end{enumerate}
\end{definicion}

\begin{teorema}[title=Indicadores a evaluar (Números corresponden al correlativo del programa de cada curso)]
\begin{enumerate}
    \item[1.] Identificar gráficos de funciones básicas, exponenciales, logarítmicas. \feimply{Aprender de memoria las formas gráficas elementales. No están en el manual.}
    \item[4.] Calcular derivadas de funciones obtenidas por álgebra de funciones elementales. \fehandbook{48 (Tabla de Derivadas)}
    \item[6.] Reconocer gráfica y analíticamente propiedades de los gráficos de funciones. \fehandbook{45 (Máximos, Mínimos, Inflexión)}
    \item[9.] Conocer el cálculo de primitivas de funciones básicas. \fehandbook{49 (Tabla de Integrales)}
\end{enumerate}
\end{teorema}

\subsection{Cálculo II (MAT1620)}
\begin{definicion}[title=Contenidos]
\begin{enumerate}
    \item[1.] Cálculo Integral
\end{enumerate}
\end{definicion}

\begin{teorema}[title=Indicadores a evaluar (Números corresponden al correlativo del programa de cada curso)]
\begin{enumerate}
    \item[3.] Aplicar el concepto de integral definida para calcular áreas y momentos de regiones del plano. \fehandbook{108-112 (Centroides en sección de Estática)}
    \item[5.] Aplicar los criterios básicos de convergencia de series e integrales impropias. \feimply{¡CUIDADO! El manual solo tiene Geométrica (p.50) y Taylor (p.51). Faltan Razón, Raíz e Integral.}
    \item[8.] Conocer las ecuaciones paramétricas, vectoriales y cartesianas de rectas y planos. \fehandbook{35 (Rectas 2D) y 59 (Vectores)}
\end{enumerate}
\end{teorema}

\subsection{Cálculo III (MAT1630)}
\begin{definicion}[title=Contenidos]
\begin{enumerate}
    \item[1.] Cálculo Diferencial
\end{enumerate}
\end{definicion}

\begin{teorema}[title=Indicadores a evaluar (Números corresponden al correlativo del programa de cada curso)]
\begin{enumerate}
    \item[2.] Aplicar el concepto de integral múltiple para evaluar volúmenes y centros de masa. \fehandbook{42 (Volúmenes de Sólidos Básicos)}
    \item[5.] Reconocer y explicar el concepto de “curvas de nivel” y calcularlas. \feimply{Concepto visual. No hay fórmula.}
    \item[6.] Calcular derivadas direccionales. \fehandbook{59 (Gradiente, Divergencia, Rotor)}
\end{enumerate}
\end{teorema}

\subsection{Ecuaciones Diferenciales (MAT1640)}
\begin{definicion}[title=Contenidos]
\begin{enumerate}
    \item[1.] Ecuaciones Diferenciales
\end{enumerate}
\end{definicion}

\begin{teorema}[title=Indicadores a evaluar (Números corresponden al correlativo del programa de cada curso)]
\begin{enumerate}
    \item[2.] Modelar situaciones sencillas de la realidad y fenómenos mediante ecuaciones diferenciales. \feimply{Habilidad de modelado.}
    \item[3.] Reconocer tipo de EDO, identificar y utilizar métodos de solución según el caso. \fehandbook{51-52 (Primer y Segundo Orden)}
    \item[6.] Calcular soluciones de sistemas lineales de $2\times2$ y $3\times3$ (coef. constantes). \feimply{Teoría de valores propios para sistemas no está explícita.}
\end{enumerate}
\end{teorema}

\subsection{Álgebra Lineal (MAT1203)}
\begin{definicion}[title=Contenidos]
\begin{enumerate}
    \item[1.] Matrices
    \item[2.] Raíces de Ecuaciones
    \item[3.] Análisis Vectorial
\end{enumerate}
\end{definicion}

\begin{teorema}[title=Indicadores a evaluar (Números corresponden al correlativo del programa de cada curso)]
\begin{enumerate}
    \item[1.] Determinar escalonada reducida, resolver $Ax=b$, calcular inversas y bases. \fehandbook{57 (Matrices e Inversa)}
    \item[2.] Interpretar geométricamente dependencia lineal, complemento ortogonal. \feimply{Concepto teórico.}
    \item[4.] Explicar y utilizar propiedades de operaciones matriciales. \fehandbook{57}
    \item[6.] Explicar y utilizar matrices elementales, simétricas, ortogonales, etc. \fehandbook{57 (Definiciones)}
    \item[7.] Calcular determinantes, resolver sistemas y evaluar inversas. \fehandbook{58 (Determinantes)}
    \item[9.] Determinar matriz de Transformación Lineal y relación con cambio de base. \feimply{No está explícito.}
    \item[12.] Explicar valores/vectores propios, diagonalización y aplicaciones (simétricas). \feimply{Definición básica solamente. Algoritmo no está.}
\end{enumerate}
\end{teorema}

\subsection{Probabilidades y Estadística (EYP1113)}
\begin{definicion}[title=Contenidos]
\begin{enumerate}
    \item[1.] Álgebra de eventos, axiomas, prob. condicional, Bayes.
    \item[2.] Medidas descriptivas teóricas (media, varianza, percentil, etc.).
    \item[3.] Modelos Discretos/Continuos (Binomial, Poisson, Normal, Exp, etc.) y uso de R.
    \item[4.] Distribuciones conjuntas, covarianza, correlación.
    \item[5.] Estimación y propiedades.
    \item[6.] Test de hipótesis e intervalos de confianza.
    \item[7.] Bondad de ajuste (Chi-cuadrado).
    \item[8.] Regresión lineal (test-t, test-F, $R^2$).
\end{enumerate}
\end{definicion}

\begin{teorema}[title=Indicadores a evaluar (Números corresponden al correlativo del programa de cada curso)]
\begin{enumerate}
    \item[1.] Ajustar distribuciones de probabilidad a datos reales. \fehandbook{66-68 (Distribuciones)}
    \item[2.] Describir fenómenos de incertidumbre usando variables aleatorias. \fehandbook{63-65 (Medidas)}
    \item[3.] Realizar estimaciones de parámetros e intervalos de confianza. \fehandbook{73-74 (Tests y C.I.)}
    \item[4.] Ajustar e interpretar modelos de regresión lineal. \fehandbook{69-70 (Regresión)}
\end{enumerate}
\end{teorema}
