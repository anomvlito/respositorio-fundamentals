% \section*{MÓDULO 2: Ciencias Naturales Física y Química}

\subsection{Dinámica (FIS1514)}
\begin{definicion}[title=Contenidos]
\begin{enumerate}
    \item[1.] Estática:
    \begin{itemize}
        \item Resultantes de sistemas de fuerzas
        \item Sistemas de fuerzas concurrentes
        \item Equilibrio de cuerpos rígidos
        \item Marcos
        \item Centroide del área
        \item Momentos del area de inercia
        \item Fricción
    \end{itemize}
    \item[2.] Dinámica:
    \begin{itemize}
        \item Movimiento lineal (fuerza, masa, aceleración, momento)
        \item Movimiento angular (par, inercia, aceleración, momento)
        \item Momentos de inercia
        \item Principio de Impulso y cantidad de movimiento aplicados a partículas y cuerpos rígidos
        \item Trabajo, energía, potencia y como se aplica a partículas y cuerpos rígidos
    \end{itemize}
\end{enumerate}
\end{definicion}

\begin{teorema}[title=Indicadores a evaluar]
\begin{enumerate}
    \item[2.] Establecer las ecuaciones del movimiento y equilibrio de sistemas utilizando la cinemática, las leyes constitutivas, y las condiciones de equilibrio.
    \item[3.] Resolver problemas de equilibrio estático y dinámico de sistemas.
    \item[4.] Plantear el equilibrio de sistemas utilizando los principios de energía y trabajo virtual.
    \item[5.] Manejar el concepto de restricciones cinemáticas y fuerzas de vínculo.
    \item[6.] Transformar fuerzas y desplazamientos entre distintos sistemas coordenados.
    \item[7.] Conocer planteamientos algorítmicos y numéricos para resolver eficientemente problemas de la mecánica clásica.
\end{enumerate}
\end{teorema}

\subsection{Electricidad y Magnetismo (FIS1533)}
\begin{definicion}[title=Contenidos]
\begin{enumerate}
    \item[1.] Carga
    \item[2.] Corriente
    \item[3.] Energía
    \item[4.] Voltaje y poder
    \item[5.] Voltaje y trabajo
    \item[6.] Fuerza entre cargas
    \item[7.] Leyes de voltaje y corriente (Kirchhoff, Ohm)
    \item[8.] Circuitos Equivalentes (series y paralelo)
    \item[9.] Capacitancia e inductancia
    \item[10.] Circuitos de corriente alterna
    \item[11.] Reactancia e impedancia
    \item[12.] Algebra compleja básica
\end{enumerate}
\end{definicion}

\begin{teorema}[title=Indicadores a evaluar]
\begin{enumerate}
    \item[1.] Describir el fenómeno del campo eléctrico, la conceptualización de carga eléctrica así como la corriente eléctrica (Ley de Gauss).
    \item[2.] Identificar los campos vectoriales creados a través de arreglos discretos y continuos de cargas eléctricas.
    \item[4.] Calcular el potencial electroestático de un sistema y explicar su relación con dispositivos reales como el capacitor.
    \item[5.] Explicar el principio de inducción magnética y su relación con dispositivos reales como la inductancia.
    \item[6.] Describir un circuito de corriente continua mediante las ecuaciones que lo gobiernan y de calcular la corriente y el voltaje en cada uno de sus nodos.
    \item[7.] Describir un circuito de corriente alterna mediante las ecuaciones que lo gobiernan y de predecir su comportamiento inicial y estacionario.
\end{enumerate}
\end{teorema}

\subsection{Química para Ingeniería (QIM100E)}
\begin{definicion}[title=Contenidos]
\begin{enumerate}
    \item[1.] Nomenclatura
    \item[2.] Oxidación-Reducción
    \item[3.] Tabla periódica
    \item[4.] Estados de la materia
    \item[5.] Ácidos y Bases
    \item[6.] Ecuaciones (estequiometría)
    \item[7.] Metales y No Metales
    \item[8.] Equilibrio
\end{enumerate}
\end{definicion}

\begin{teorema}[title=Indicadores a evaluar]
\begin{enumerate}
    \item[1.2] Manejar y aplicar la conversión de unidades (Sistema Internacional y sus prefijos).
    \item[3.2] Discutir cómo los cambios de temperatura afectan el estado de una sustancia (sólido, líquido y gas).
    \item[3.3] Describir los diferentes tipos de enlaces presentes en materiales sólidos y sus estructuras cristalinas.
    \item[6.1] Describir el significado de equilibrio dinámico y diferenciar equilibrios homogéneos y heterogéneos.
    \item[6.2] Entender y relacionar cuociente de reacción (Q), concentración de especies, y constante de equilibrio (K).
    \item[6.3] Explicar el significado de la constante de equilibrio, y su cálculo a partir de las concentraciones en el equilibrio.
    \item[7.1] Diferenciar conceptos de electrolitos fuertes y débiles.
    \item[7.3] Identificar las relaciones entre la concentración de iones y el pH. Discutir las relaciones entre Ka y el grado de ionización de ácido. Describir el comportamiento de ácidos y bases fuertes y débiles en disolución.
    \item[7.4] Calcular el pH de soluciones de ácidos, bases y sistemas buffer.
    \item[9.1] Formular semi-reacciones balanceadas en masa y carga.
    \item[9.2] Describir los componentes de una celda electroquímica.
    \item[9.3] Describir el electrodo estándar de hidrógeno.
    \item[9.4] Identificar las relaciones entre energía de Gibbs, potencial estándar y la constante de equilibrio K.
\end{enumerate}
\end{teorema}

\subsection{Termodinámica (FIS1523)}
\begin{definicion}[title=Contenidos]
\begin{enumerate}
    \item[1.] Leyes termodinámicas (primera ley, segunda ley)
    \item[2.] Energía, calor y trabajo
    \item[3.] Disponibilidad y reversibilidad
    \item[4.] Ciclos
    \item[5.] Gases ideales
    \item[6.] Mezcla de gases
    \item[7.] Fase cambios
    \item[8.] Transferencia de calor
    \item[9.] Propiedades de: entalpía y entropía
\end{enumerate}
\end{definicion}

\begin{teorema}[title=Indicadores a evaluar]
\begin{enumerate}
    \item[1.] Definir el concepto de temperatura y temperatura absoluta.
    \item[2.] Explicar el equilibrio térmico y el principio de expansión térmica.
    \item[4.] Explicar la primera ley de la termodinámica y aplicar la ley a ejemplos con gases ideales.
    \item[5.] Describir el concepto de entropía y de la dirección de los procesos.
    \item[6.] Calcular la entropía, potencial termodinámico y eficiencia en distintos ciclos ideales y reales.
    \item[7.] Calcular varias cantidades termodinámicas como promedios de propiedades mecánicas de sistemas de gran número de partículas.
\end{enumerate}
\end{teorema}
