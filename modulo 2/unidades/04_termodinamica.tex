\section{Termodinámica}

%-------------------------------------------------------------------------------
\subsection{Conceptos Fundamentales}
%-------------------------------------------------------------------------------
La termodinámica estudia la energía y sus transformaciones. Es vital para entender motores, refrigeradores y procesos industriales.

\subsubsection{Sistemas y Propiedades}
\begin{itemize}
    \item \textbf{Sistema:} Lo que estudiamos.
    \item \textbf{Entorno:} Todo lo demás.
    \item \textbf{Sistema Cerrado:} Masa fija, energía puede cruzar.
    \item \textbf{Sistema Abierto (Volumen de Control):} Masa y energía cruzan la frontera.
    \item \textbf{Propiedad Intensiva:} Independiente de la masa (T, P, v).
    \item \textbf{Propiedad Extensiva:} Depende de la masa (V, U, H).
\end{itemize}

\subsubsection{Primera Ley de la Termodinámica (Conservación de Energía)}
\begin{teorema}[title=Primera Ley (Sistema Cerrado)] \fehandbook{147}
$$ Q - W = \Delta U $$
Convención: $Q_{entra} (+)$, $W_{sale} (+)$.
\end{teorema}
Para sistemas cerrados: $Q - W = \Delta U + \Delta EC + \Delta EP$
\begin{itemize}
    \item $Q > 0$: Calor entra al sistema.
    \item $W > 0$: Trabajo hecho POR el sistema.
\end{itemize}

Para volúmenes de control (flujo estacionario):
$$ \dot{Q} - \dot{W} = \dot{m} \left( h_2 - h_1 + \frac{v_2^2 - v_1^2}{2} + g(z_2 - z_1) \right) $$

\subsubsection{Segunda Ley de la Termodinámica}
La energía tiene calidad, no solo cantidad. Los procesos ocurren en una dirección específica.
\begin{itemize}
    \item \textbf{Entropía (S):} Medida del desorden o irreversibilidad.
    \item \textbf{Enunciado de Kelvin-Planck:} Es imposible construir una máquina térmica que opere en un ciclo y produzca trabajo neto intercambiando calor con un solo depósito térmico.
    \item \textbf{Eficiencia Térmica ($\eta$):}
    \begin{definicion}[title=Eficiencia Térmica] \fehandbook{149 (Cycles)}
$$ \eta = \frac{W_{neto}}{Q_{entra}} = 1 - \frac{Q_{sale}}{Q_{entra}} $$
Para ciclo de Carnot (máxima teórica): $\eta = 1 - \frac{T_L}{T_H}$. \fehandbook{149}
\end{definicion}
\end{itemize}

%-------------------------------------------------------------------------------
\subsection{Propiedades de Sustancias Puras}
%-------------------------------------------------------------------------------
El uso de tablas termodinámicas es esencial.
\begin{itemize}
    \item \textbf{Mezcla Saturada:} Conviven líquido y vapor. La propiedad se calcula como:
    $$ y = y_f + x (y_g - y_f) $$
    Donde $x$ es la calidad (fracción de vapor).
    $$ x = \frac{m_{vapor}}{m_{total}} $$
    \begin{tip}[title=Uso de Tablas de Vapor] \fehandbook{157+}
\begin{enumerate}
    \item Determinar estado: Comparar $T$ vs $T_{sat}$ (o $P$ vs $P_{sat}$).
    \item \textbf{Líquido Comprimido:} Aproximar como líquido saturado a la temperatura $T$. \feimply{Aproximación usual no explícita en fórmulas.}
    \item \textbf{Mezcla Saturada:} Usar calidad $x$: $h = h_f + x h_{fg}$. \fehandbook{144 (Properties of Two-Phase Systems)}
    \item \textbf{Vapor Sobrecalentado:} Interpolar en tablas de sobrecalentado.
\end{enumerate}
\end{tip}
    \item \textbf{Gas Ideal:} $Pv = RT$. Válido a bajas presiones y altas temperaturas respecto al punto crítico.
\end{itemize}

%-------------------------------------------------------------------------------
\subsection{Ejercicios Seleccionados}
%-------------------------------------------------------------------------------

\subsubsection{Ejercicio 1: Primera Ley en Sistema Cerrado}
\textbf{Problema:} Un gas ideal se expande isotérmicamente (temperatura constante) absorbiendo 100 J de calor. ¿Cuál es el cambio en su energía interna ($\Delta U$)?
\begin{enumerate}
    \item[a)] 100 J
    \item[b)] -100 J
    \item[c)] 0 J
    \item[d)] Depende de la presión.
\end{enumerate}

\subsubsection{Ejercicio 2: Calidad de Vapor}
\textbf{Problema:} Un recipiente rígido contiene 10 kg de agua a 90°C. Si 8 kg están en forma líquida y 2 kg en forma de vapor, calcule la calidad ($x$) de la mezcla.
\begin{enumerate}
    \item[a)] 0.2
    \item[b)] 0.8
    \item[c)] 0.25
    \item[d)] 80\%
\end{enumerate}

\subsubsection{Ejercicio 3: Eficiencia de Carnot}
\textbf{Problema:} Una máquina térmica opera entre dos fuentes a temperaturas de $T_H = 1000 \, \text{K}$ y $T_L = 300 \, \text{K}$. ¿Cuál es la máxima eficiencia teórica posible?
\begin{enumerate}
    \item[a)] 30\%
    \item[b)] 70\%
    \item[c)] 100\%
    \item[d)] 43\%
\end{enumerate}

%-------------------------------------------------------------------------------
\newpage
\subsection{Soluciones}
%-------------------------------------------------------------------------------

\subsubsection*{Solución Ejercicio 1}
\begin{ejercicio}
\begin{enumerate}
    \item Para un \textbf{gas ideal}, la energía interna $U$ depende \textbf{exclusivamente de la temperatura}.
    \item Dado que el proceso es \textbf{isotérmico} ($T$ constante), el cambio de temperatura $\Delta T = 0$.
    \item Por lo tanto, el cambio de energía interna $\Delta U = C_v \Delta T = 0$.
    \item Todo el calor absorbido se convierte en trabajo ($Q=W$).
\end{enumerate}
\textbf{Respuesta Correcta: c) 0 J}
\end{ejercicio}

\subsubsection*{Solución Ejercicio 2}
\begin{ejercicio}
\begin{enumerate}
    \item La calidad se define como la fracción másica de vapor.
    $$ x = \frac{m_{vapor}}{m_{total}} $$
    \item $m_{vapor} = 2 \, \text{kg}$.
    \item $m_{total} = m_{liq} + m_{vapor} = 8 + 2 = 10 \, \text{kg}$.
    \item $x = \frac{2}{10} = 0.2$.
\end{enumerate}
\textbf{Respuesta Correcta: a) 0.2}
\end{ejercicio}

\subsubsection*{Solución Ejercicio 3}
\begin{ejercicio}
\begin{enumerate}
    \item La máxima eficiencia posible es la del Ciclo de Carnot.
    $$ \eta_{Carnot} = 1 - \frac{T_L}{T_H} $$
    \item Las temperaturas deben estar en Kelvin (absolutas).
    \item $\eta = 1 - \frac{300}{1000} = 1 - 0.3 = 0.7$.
    \item Expresado en porcentaje: 70\%.
\end{enumerate}
\textbf{Respuesta Correcta: b) 70\%}
\end{ejercicio}
