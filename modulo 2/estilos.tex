%%%%%%%%%%%%%%%%%%%%%%%%%%%%%%%%%%%%%%%%%%%%%%%%%%%%%%%%%%%%%%%%%%%%%%%%%%%%%%%%
% ESTILOS.TEX - Configuración visual y macros del Proyecto Fundamentals Premium
%%%%%%%%%%%%%%%%%%%%%%%%%%%%%%%%%%%%%%%%%%%%%%%%%%%%%%%%%%%%%%%%%%%%%%%%%%%%%%%%

% --- 1. CONFIGURACIÓN DE PÁGINA Y FUENTES ---
\PassOptionsToPackage{dvipsnames,svgnames,table,xcdraw}{xcolor}
\usepackage[utf8]{inputenc}
\usepackage[T1]{fontenc}
\usepackage[spanish,es-tabla]{babel} % 'es-tabla' usa "Tabla" en vez de "Cuadro"
\usepackage{geometry}
\geometry{
    a4paper,
    top=2.5cm,
    bottom=2.5cm,
    left=2.5cm,
    right=2.5cm,
    headheight=15pt
}

% Fuentes más modernas (opcional, requiere compilador compatible o fuentes instaladas, 
% pero usaremos paquetes estándar que se ven bien)
\usepackage{lmodern} % Mejora la fuente Computer Modern estandar
\usepackage{helvet}  % Fuente Helvetica para sans-serif
% \renewcommand{\familydefault}{\sfdefault} % Descomentar si se prefiere todo el documento en Sans Serif

% --- 2. PAQUETES MATEMÁTICOS Y DE UTILIDAD ---
\usepackage{amsmath, amssymb, amsthm, amsfonts}
\usepackage{mathtools} % Mejoras a amsmath
\usepackage{graphicx}
\usepackage{circuitikz} % Para circuitos eléctricos
\usepackage{siunitx} % Unidades del SI y notación científica
\DeclareSIUnit\million{M} % Definición para millones (ej. 50M)
\usepackage{chemformula} % Para fórmulas químicas (\ch)
\usepackage{listings} % Para bloques de código
\usepackage{float}
\usepackage{enumitem} % Para personalizar listas
\setlist{nosep} % Listas compactas por defecto
\usepackage{multicol} % Para layouts de múltiples columnas

% Paquetes para Tablas (Requeridos por intro.tex)
\usepackage{longtable}
\usepackage{booktabs}
\usepackage{array}

% --- 3. DISEÑO Y COLORES (PREMIUM) ---
\usepackage{xcolor} % Carga xcolor con opciones definidas arriba
\usepackage[many]{tcolorbox} % El motor de nuestras cajas bonitas

% Definición de Colores Corporativos (Modern Palette)
\definecolor{DeepBlue}{HTML}{003B5C}    % Azul corporativo serio
\definecolor{BrightBlue}{HTML}{007ACC}  % Azul brillante para destacados
\definecolor{Emerald}{HTML}{00A388}     % Verde para teoremas/éxito
\definecolor{Sunset}{HTML}{FF6B6B}      % Rojo suave para alertas
\definecolor{WarningOrange}{HTML}{FF9F43} % Naranja para tips
\definecolor{LightGray}{HTML}{F8F9FA}   % Fondo muy suave

% --- 4. CAJAS PERSONALIZADAS (TCOLORBOX) ---

% Caja de DEFINICIÓN (Azul)
\newtcolorbox{definicion}[1][]{
    enhanced,
    breakable,
    colback=BrightBlue!5!white, % Fondo azul muy suave
    colframe=DeepBlue,          % Borde azul oscuro
    coltitle=white,             % Texto del título blanco
    fonttitle=\bfseries\sffamily,
    title=Definición,           % Título por defecto
    borderline west={4pt}{0pt}{DeepBlue}, % Barra lateral gruesa
    sharp corners,
    boxrule=0.5pt,
    #1 % Permite cambiar opciones al usarla (ej: title={Nueva Def})
}

% Caja de TEOREMA / PROPIEDAD (Verde)
% Caja de TEOREMA / PROPIEDAD (Verde)
\newtcolorbox{teorema}[1][]{
    enhanced,
    breakable,
    colback=Emerald!5!white,
    colframe=Emerald,
    coltitle=white,
    fonttitle=\bfseries\sffamily,
    title=Teorema,
    borderline west={4pt}{0pt}{Emerald},
    sharp corners,
    boxrule=0.5pt,
    #1
}

% Caja de EJERCICIO RESUELTO (Gris profesional)
\newtcolorbox{ejercicio}[1][]{
    enhanced,
    breakable,
    colback=white,
    colframe=gray!50!black,
    coltitle=white,
    fonttitle=\bfseries,
    title=Ejercicio Resuelto,
    boxrule=1pt,
    arc=4mm, % Bordes redondeados
    shadow={2mm}{-2mm}{0mm}{black!20}, % Sombra sutil
    #1
}

% Caja de TIP / ALERTA (Naranja)
\newtcolorbox{tip}[1][]{
    enhanced,
    colback=WarningOrange!10!white,
    colframe=WarningOrange,
    title={Tip / Cuidado},
    fonttitle=\bfseries,
    coltitle=black,
    attach boxed title to top left={yshift=-2mm, xshift=2mm},
    boxed title style={colback=WarningOrange, sharp corners},
    boxrule=1pt,
    arc=2mm,
    #1
}

% Caja de NOTA (Gris/Amarillo suave)
\newtcolorbox{notabox}[1][]{
    enhanced,
    colback=yellow!10!white,
    colframe=gray,
    title={Nota},
    fonttitle=\bfseries,
    coltitle=black,
    boxrule=0.5pt,
    #1
}

% Caja de SOLUCIÓN (Verde suave)
\newtcolorbox{solbox}[1][]{
    enhanced,
    colback=Emerald!5!white,
    colframe=Emerald,
    title={Solución},
    fonttitle=\bfseries,
    coltitle=white,
    boxrule=0.5pt,
    #1
}

\newcommand{\nota}[1]{
\begin{notabox}
    \textbf{Nota Estratégica:} #1
\end{notabox}
}

% --- 5. NAVEGACIÓN Y ENCABEZADOS ---
\usepackage{fancyhdr}
\pagestyle{fancy}
\fancyhf{} % Limpiar cabeceras y pies por defecto
\fancyhead[L]{\small \sffamily \textbf{Resumen de Matemáticas}}
\fancyhead[R]{\small \sffamily \leftmark} % Muestra la sección actual
\fancyfoot[C]{\thepage}
\renewcommand{\headrulewidth}{1pt}
\renewcommand{\footrulewidth}{0pt}

% Hyperref (Siempre al final del preámbulo)
\usepackage[colorlinks=true, linkcolor=DeepBlue, citecolor=Emerald, urlcolor=BrightBlue]{hyperref}

% --- 6. MACROS MATEMÁTICOS (Cheat Sheet Style) ---
\newcommand{\R}{\mathbb{R}} % Reales
\newcommand{\N}{\mathbb{N}} % Naturales
\newcommand{\Z}{\mathbb{Z}} % Enteros
\newcommand{\C}{\mathbb{C}} % Complejos
\newcommand{\dd}{\mathrm{d}} % d de derivada
\newcommand{\norm}[1]{\left\lVert#1\right\rVert} % Norma
\newcommand{\abs}[1]{\left\lvert#1\right\rvert} % Valor absoluto
\newcommand{\vect}[1]{\mathbf{#1}} % Vectores en negrita

% Macro para "Receta de Cocina" (Pasos)
\newenvironment{pasos}
    {\begin{enumerate}[label=\textbf{Paso \arabic*}:, leftmargin=*]}
    {\end{enumerate}}
