%%%%%%%%%%%%%%%%%%%%%%%%%%%%%%%%%%%%%%%%%%%%%%%%%%%%%%%%%%%%%%%%%%%%%%%%%%%%%%%%
% PREÁMBULO: Configuración del documento
%%%%%%%%%%%%%%%%%%%%%%%%%%%%%%%%%%%%%%%%%%%%%%%%%%%%%%%%%%%%%%%%%%%%%%%%%%%%%%%%
\documentclass[12pt, a4paper]{article}

% --- Paquetes Esenciales ---
\usepackage[utf8]{inputenc}
\usepackage[spanish]{babel} % Para soporte de español
\usepackage{amsmath, amssymb, amsfonts} % Paquetes matemáticos avanzados
\usepackage{graphicx} % Para incluir imágenes
\usepackage{siunitx} % Para unidades del SI
\usepackage{xcolor} % Para definir colores
\usepackage[svgnames]{xcolor} % Nombres de colores adicionales
\usepackage{mdframed} % Para crear cajas y marcos
\usepackage{hyperref} % Para links internos y externos (clickable ToC)
\usepackage{enumitem} % Para personalizar listas
\usepackage{listings} % Para formatear código

% --- Configuración de Página y Estilo ---
\usepackage{geometry}
\geometry{a4paper, total={170mm,257mm}, left=20mm, top=25mm} % Márgenes
\usepackage{parskip} % Espacio entre párrafos en lugar de indentación
\linespread{1.1} % Interlineado para mayor legibilidad

% --- Colores y Estilos Personalizados ---
\definecolor{sectioncolor}{RGB}{200, 80, 0} % Naranja para programación
\definecolor{noteback}{RGB}{255, 245, 230} % Naranja pálido para notas
\definecolor{solback}{RGB}{235, 235, 235} % Gris claro para soluciones
\definecolor{codegray}{rgb}{0.95,0.95,0.95}
\definecolor{commentgreen}{rgb}{0,0.5,0}
\definecolor{keywordblue}{rgb}{0,0,0.6}

% --- Configuración para el código ---
\lstdefinestyle{mystyle}{
    backgroundcolor=\color{codegray},   
    commentstyle=\color{commentgreen},
    keywordstyle=\color{keywordblue}\bfseries,
    numberstyle=\tiny\color{gray},
    stringstyle=\color{purple},
    basicstyle=\footnotesize\ttfamily,
    breakatwhitespace=false,         
    breaklines=true,                 
    captionpos=b,                    
    keepspaces=true,                 
    numbers=left,                    
    numbersep=5pt,                   
    showspaces=false,                
    showstringspaces=false,
    showtabs=false,                  
    tabsize=2
}
\lstset{style=mystyle}

% --- Comandos Personalizados para Notas Estratégicas ---
\newmdenv[
    linecolor=OrangeRed,
    linewidth=1.5pt,
    roundcorner=5pt,
    backgroundcolor=noteback,
    topline=false,
    bottomline=false,
    rightline=false,
    leftline=true,
    skipabove=\baselineskip,
    skipbelow=\baselineskip
]{notabox}

\newcommand{\nota}[1]{
\begin{notabox}
    \textbf{Nota Estratégica:} #1
\end{notabox}
}

\newmdenv[
    linecolor=DarkSlateGray,
    linewidth=1.5pt,
    roundcorner=5pt,
    backgroundcolor=solback,
    topline=true,
    bottomline=true,
    rightline=false,
    leftline=false,
    skipabove=\baselineskip,
    skipbelow=\baselineskip
]{solbox}


%%%%%%%%%%%%%%%%%%%%%%%%%%%%%%%%%%%%%%%%%%%%%%%%%%%%%%%%%%%%%%%%%%%%%%%%%%%%%%%%
% DOCUMENTO PRINCIPAL
%%%%%%%%%%%%%%%%%%%%%%%%%%%%%%%%%%%%%%%%%%%%%%%%%%%%%%%%%%%%%%%%%%%%%%%%%%%%%%%%
\begin{document}

% --- Portada ---
\title{
    \Huge\bfseries
    Guía de Razonamiento Lógico y Hojas de Cálculo \\[0.5cm]
    \Large IIC1103: Introducción a la Programación
}
\author{Una guía para el Examen de Competencias Fundamentales}
\date{Julio de 2025}
\maketitle
\thispagestyle{empty}
\newpage

% --- Tabla de Contenidos ---
\tableofcontents
\newpage

%%%%%%%%%%%%%%%%%%%%%%%%%%%%%%%%%%%%%%%%%%%%%%%%%%%%%%%%%%%%%%%%%%%%%%%%%%%%%%%%
% SECCIÓN 1: INTRODUCCIÓN ESTRATÉGICA
%%%%%%%%%%%%%%%%%%%%%%%%%%%%%%%%%%%%%%%%%%%%%%%%%%%%%%%%%%%%%%%%%%%%%%%%%%%%%%%%
\section{Cómo Abordar Problemas de Lógica en el ECF}

Esta sección del examen no mide tu habilidad para escribir código o fórmulas de memoria, sino tu capacidad para \textbf{leer, interpretar y predecir el comportamiento} de un algoritmo o una hoja de cálculo. La clave es el razonamiento lógico, no la sintaxis perfecta.

\nota{¡Alerta Crítica! El manual de referencia oficial (FE Handbook) \textbf{no tiene una sección relevante} para programación introductoria o uso de hojas de cálculo a este nivel. El contenido es demasiado avanzado. Debes tratar esta parte del examen como si no tuvieras apuntes. Todo el conocimiento debe venir de tu preparación.}

\subsection{Habilidades Clave}
\begin{itemize}
    \item \textbf{Seguimiento de Variables (Trazas):} La habilidad más importante para el código. Debes ser capaz de seguir el valor de cada variable, línea por línea.
    \item \textbf{Comprensión de Control de Flujo:} Entender exactamente cómo y cuándo se ejecuta cada parte del código (`if`, `while`, `for`).
    \item \textbf{Manejo de Índices y Referencias:} Los errores con los índices de listas y strings son muy comunes. En hojas de cálculo, entender las \textbf{referencias relativas, absolutas y mixtas} es fundamental.
    \item \textbf{Abstracción con Funciones:} Entender qué hace una función basándose en su descripción y cómo su valor de retorno afecta al resto del programa o la hoja de cálculo.
\end{itemize}


%%%%%%%%%%%%%%%%%%%%%%%%%%%%%%%%%%%%%%%%%%%%%%%%%%%%%%%%%%%%%%%%%%%%%%%%%%%%%%%%
% SECCIÓN 2: CONCEPTOS DE PROGRAMACIÓN
%%%%%%%%%%%%%%%%%%%%%%%%%%%%%%%%%%%%%%%%%%%%%%%%%%%%%%%%%%%%%%%%%%%%%%%%%%%%%%%%
\section{Conceptos Fundamentales de Programación}

\subsection{Algoritmos, Variables y Expresiones}
\begin{itemize}
    \item \textbf{Algoritmo:} Una secuencia de pasos finitos y bien definidos para resolver un problema.
    \item \textbf{Variables:} Espacios en memoria que guardan un valor y tienen un nombre y un tipo de dato.
    \item \textbf{Expresiones:} Combinaciones de valores, variables y operadores.
        \begin{itemize}
            \item \textbf{Aritméticas:} `+`, `-`, `*`, `/`, `//` (división entera), `\%` (módulo).
            \item \textbf{Lógicas:} `AND`, `OR`, `NOT`.
            \item \textbf{Relacionales:} `==`, `!=`, `>`, `<`, `GTE`, `<=`.
        \end{itemize}
\end{itemize}

\subsection{Control de Flujo y Estructuras de Datos}
\begin{itemize}
    \item \textbf{Condicionales (`if`/`elif`/`else`):} Ejecutan bloques de código si se cumple una condición.
    \item \textbf{Bucles `while` y `for`:} Repiten un bloque de código. `while` lo hace mientras una condición sea verdadera; `for` itera sobre una secuencia.
    \item \textbf{Strings y Listas:} Secuencias ordenadas de elementos a los que se accede por un índice que comienza en 0.
\end{itemize}

%%%%%%%%%%%%%%%%%%%%%%%%%%%%%%%%%%%%%%%%%%%%%%%%%%%%%%%%%%%%%%%%%%%%%%%%%%%%%%%%
% SECCIÓN 3: CONCEPTOS DE HOJAS DE CÁLCULO
%%%%%%%%%%%%%%%%%%%%%%%%%%%%%%%%%%%%%%%%%%%%%%%%%%%%%%%%%%%%%%%%%%%%%%%%%%%%%%%%
\section{Conceptos Fundamentales de Hojas de Cálculo}

La clave para resolver estos problemas es entender cómo las fórmulas cambian cuando se copian y pegan.

\subsection{Referencias de Celda: El Concepto Clave}
El símbolo \texttt{\$} congela o ancla una fila o una columna, evitando que cambie al copiar la fórmula.

\begin{center}
\begin{tabular}{|p{4cm}|p{2.5cm}|p{8cm}|}
\hline
\textbf{Tipo de Referencia} & \textbf{Ejemplo} & \textbf{Comportamiento al Copiar y Pegar} \\
\hline
\textbf{Relativa} & \texttt{A1} & Tanto la columna (A) como la fila (1) \textbf{cambian} según la dirección en que se mueva la fórmula. \\
\hline
\textbf{Absoluta} & \texttt{\$A\$1} & Ni la columna (A) ni la fila (1) \textbf{cambian}. La referencia está totalmente anclada. \\
\hline
\textbf{Mixta (Columna Absoluta)} & \texttt{\$A1} & La columna (A) está anclada y \textbf{no cambia}, pero la fila (1) \textbf{sí cambia}. \\
\hline
\textbf{Mixta (Fila Absoluta)} & \texttt{A\$1} & La columna (A) \textbf{sí cambia}, pero la fila (1) está anclada y \textbf{no cambia}. \\
\hline
\end{tabular}
\end{center}

\subsection{Funciones Comunes Utilizadas}
\begin{itemize}
    \item \texttt{PROMEDIO(rango)}: Calcula la media aritmética.
    \item \texttt{MAX(rango)}: Devuelve el valor más alto del rango.
    \item \texttt{SI(prueba\_logica; valor\_si\_verdadero; valor\_si\_falso)}: Evalúa una condición y devuelve un valor según el resultado.
    \item \texttt{CONTAR.SI(rango; criterio)}: Cuenta celdas que cumplen un criterio.
    \item \texttt{SUMA(rango)}: Suma todos los números en un rango.
    \item \texttt{MEDIANA(rango)}: Devuelve el número central de un conjunto de datos ordenado.
\end{itemize}

%%%%%%%%%%%%%%%%%%%%%%%%%%%%%%%%%%%%%%%%%%%%%%%%%%%%%%%%%%%%%%%%%%%%%%%%%%%%%%%%
% SECCIÓN 4: EJERCICIOS DE PRÁCTICA
%%%%%%%%%%%%%%%%%%%%%%%%%%%%%%%%%%%%%%%%%%%%%%%%%%%%%%%%%%%%%%%%%%%%%%%%%%%%%%%%
\newpage
\section{Ejercicios de Práctica Tipo Prueba}

\subsection{Ejercicio 1: Tipos de Datos y Expresiones (Código)}
\textbf{Problema (IIC1103-1-5):} Dado el siguiente pseudocódigo, ¿qué valores quedan almacenados en las variables \texttt{e}, \texttt{f} y \texttt{g} al final?
\begin{lstlisting}[language=Python]
a = 3
b = 15.0
c = VERDADERO
d = a - b
e = d > 0
f = b / 2 == 7
g = e AND f
\end{lstlisting}
\begin{enumerate}[label=\alph*)]
    \item \texttt{e} = -12.0, \texttt{f} = FALSO, \texttt{g} = FALSO
    \item \texttt{e} = FALSO, \texttt{f} = FALSO, \texttt{g} = FALSO
    \item \texttt{e} = FALSO, \texttt{f} = VERDADERO, \texttt{g} = FALSO
    \item \texttt{e} = 15.0, \texttt{f} = FALSO, \texttt{g} = VERDADERO
\end{enumerate}

\subsection{Ejercicio 2: Bucles y Índices (Código)}
\textbf{Problema (IIC1103-1-6):} El siguiente código intenta sumar los valores de dos listas de enteros (\texttt{v1} y \texttt{v2}). Asume que \texttt{largo\_v1} es el largo de \texttt{v1}.
\begin{lstlisting}[language=Python]
total = 0
i = 0
while i < largo_v1:
    total = total + v1[i] + v2[i]
    i = i + 1
\end{lstlisting}
¿Cuál de las siguientes afirmaciones es cierta para este código?
\begin{enumerate}[label=\alph*)]
    \item El ciclo siempre termina sin considerar el último elemento.
    \item El código no considera listas de largos distintos, pero se arregla cambiando la condición del `while`.
    \item Este código tendrá un error si \texttt{v1} tiene más elementos que \texttt{v2}.
    \item El código funciona como se espera en cualquier situación.
\end{enumerate}

\subsection{Ejercicio 3: Referencias de Celda (Hoja de Cálculo)}
\textbf{Problema (TRANS-3):} La celda A4, que contiene la fórmula \texttt{=PROMEDIO(\$A1:B\$2)}, se copia en la celda C5. ¿Qué fórmula queda en C5?
\begin{center}
\begin{tabular}{|c|c|c|c|c|}
\hline
\textbf{} & \textbf{A} & \textbf{B} & \textbf{C} & \textbf{D} \\
\hline
\textbf{1} & 2 & 2 & 0 & 1 \\
\hline
\textbf{2} & 1 & 1 & 1 & 1 \\
\hline
\textbf{3} & 1 & 1 & 1 & 1 \\
\hline
\textbf{4} & \texttt{=PROMEDIO(\$A1:B\$2)} & & & \\
\hline
\textbf{5} & & & & \\
\hline
\end{tabular}
\end{center}
\begin{enumerate}[label=\alph*)]
    \item \texttt{=PROMEDIO(\$A1:B\$2)}
    \item \texttt{=PROMEDIO(C2:D3)}
    \item \texttt{=PROMEDIO(\$C2:D\$3)}
    \item \texttt{=PROMEDIO(\$A2:D\$2)}
\end{enumerate}

\subsection{Ejercicio 4: Funciones y Referencias (Hoja de Cálculo)}
\textbf{Problema (Trans-9):} La celda E1, que contiene la fórmula \texttt{=MAX(A\$1:C\$2)}, se copia en la celda E2. ¿Qué valor queda almacenado en E2?
\begin{center}
\begin{tabular}{|c|c|c|c|c|c|}
\hline
\textbf{} & \textbf{A} & \textbf{B} & \textbf{C} & \textbf{D} & \textbf{E} \\
\hline
\textbf{1} & 1 & 2 & 1 & & \texttt{=MAX(A\$1:C\$2)} \\
\hline
\textbf{2} & 1 & 0 & 2 & & \\
\hline
\textbf{3} & 1 & 1 & 1 & & \\
\hline
\textbf{4} & 3 & 2 & 1 & & \\
\hline
\textbf{5} & & & & & \\
\hline
\end{tabular}
\end{center}
\begin{enumerate}[label=\alph*)]
    \item 2
    \item 3
    \item 1
\end{enumerate}

\subsection{Ejercicio 5: Lógica Condicional (Hoja de Cálculo)}
\textbf{Problema (TRANS-8):} ¿Qué fórmula debe ponerse en la celda D3 para que ésta quede con valor "NO"?
\begin{center}
\begin{tabular}{|c|c|c|c|c|}
\hline
\textbf{} & \textbf{A} & \textbf{B} & \textbf{C} & \textbf{D} \\
\hline
\textbf{1} & 1 & 1 & 3 & \\
\hline
\textbf{2} & 2 & 1 & 0 & \\
\hline
\textbf{3} & & & & \\
\hline
\end{tabular}
\end{center}
\begin{enumerate}[label=\alph*)]
    \item \texttt{=SI(MEDIANA(A1:C1)>MEDIANA(A2:C2);"SI";"NO")}
    \item \texttt{=SI(SUMA(A1:C1)>SUMA(A2:C2);"SI";"NO")}
    \item \texttt{=SI(PROMEDIO(A1:C1)>PROMEDIO(A2:C2);"SI";"NO")}
    \item \texttt{=SI(MAX(A1:C1)>MAX(A2:C2);"SI";"NO")}
\end{enumerate}

%%%%%%%%%%%%%%%%%%%%%%%%%%%%%%%%%%%%%%%%%%%%%%%%%%%%%%%%%%%%%%%%%%%%%%%%%%%%%%%%
% SECCIÓN 5: SOLUCIONES DETALLADAS
%%%%%%%%%%%%%%%%%%%%%%%%%%%%%%%%%%%%%%%%%%%%%%%%%%%%%%%%%%%%%%%%%%%%%%%%%%%%%%%%
\newpage
\section{Soluciones Detalladas}

\subsection*{Solución Ejercicio 1 (Código)}
\begin{solbox}
\textbf{Estrategia:} Realizar una traza, evaluando cada línea.
\begin{itemize}
    \item \texttt{d = 3 - 15.0 = -12.0}
    \item \texttt{e = -12.0 > 0} es \texttt{FALSO}.
    \item \texttt{f = 15.0 / 2 == 7} es \texttt{7.5 == 7}, que es \texttt{FALSO}.
    \item \texttt{g = FALSO AND FALSO} es \texttt{FALSO}.
\end{itemize}
\textbf{Alternativa correcta: b)}
\end{solbox}

\subsection*{Solución Ejercicio 2 (Código)}
\begin{solbox}
\textbf{Estrategia:} Analizar el acceso a los elementos de las listas en el caso límite de largos distintos.
\begin{itemize}
    \item El bucle se controla solo con el largo de \texttt{v1}.
    \item Si \texttt{v1} es más larga que \texttt{v2}, el índice \texttt{i} eventualmente superará el tamaño de \texttt{v2}, causando un error de "índice fuera de rango" al intentar acceder a \texttt{v2[i]}.
\end{itemize}
\textbf{Alternativa correcta: c)}
\end{solbox}

\subsection*{Solución Ejercicio 3 (Hoja de Cálculo)}
\begin{solbox}
\textbf{Estrategia:} Analizar el movimiento y aplicar las reglas de referencias.
\begin{itemize}
    \item \textbf{Movimiento:} De A4 a C5 (+2 columnas, +1 fila).
    \item \textbf{Fórmula:} \texttt{=PROMEDIO(\$A1:B\$2)}
    \item \texttt{\$A1}: Columna `A` anclada, no cambia. Fila `1` relativa, `1+1=2` $\rightarrow$ \texttt{\$A2}.
    \item \texttt{B\$2}: Columna `B` relativa, `B+2=D`. Fila `2` anclada, no cambia $\rightarrow$ \texttt{D\$2}.
    \item \textbf{Fórmula resultante:} \texttt{=PROMEDIO(\$A2:D\$2)}.
\end{itemize}
\textbf{Alternativa correcta: d)}
\end{solbox}

\subsection*{Solución Ejercicio 4 (Hoja de Cálculo)}
\begin{solbox}
\textbf{Estrategia:} Copiar la fórmula y luego calcular su valor.
\begin{itemize}
    \item \textbf{Movimiento:} De E1 a E2 (0 columnas, +1 fila).
    \item \textbf{Fórmula:} \texttt{=MAX(A\$1:C\$2)}
    \item `A\$1`: Columna `A` relativa, `A+0=A`. Fila `1` anclada, no cambia $\rightarrow$ \texttt{A\$1}.
    \item `C\$2`: Columna `C` relativa, `C+0=C`. Fila `2` anclada, no cambia $\rightarrow$ \texttt{C\$2}.
    \item \textbf{Fórmula resultante en E2:} \texttt{=MAX(A\$1:C\$2)}. Es la misma.
    \item \textbf{Cálculo:} El rango `A1:C2` contiene los valores `{1, 2, 1, 1, 0, 2}`. El valor máximo es 2.
\end{itemize}
\textbf{Alternativa correcta: a)}
\end{solbox}

\subsection*{Solución Ejercicio 5 (Hoja de Cálculo)}
\begin{solbox}
\textbf{Estrategia:} Evaluar la condición lógica de cada alternativa para ver cuál resulta `FALSA`.
\begin{itemize}
    \item \textbf{Datos:} Rango A1:C1 = `{1, 1, 3}`, Rango A2:C2 = `{2, 1, 0}`.
    \item a) \texttt{MEDIANA(A1:C1)>MEDIANA(A2:C2)} $\rightarrow$ \texttt{1 > 1} es \textbf{FALSO}. Esta devuelve "NO".
    \item b) \texttt{SUMA(A1:C1)>SUMA(A2:C2)} $\rightarrow$ \texttt{5 > 3} es VERDADERO.
    \item c) \texttt{PROMEDIO(A1:C1)>PROMEDIO(A2:C2)} $\rightarrow$ \texttt{1.66 > 1} es VERDADERO.
    \item d) \texttt{MAX(A1:C1)>MAX(A2:C2)} $\rightarrow$ \texttt{3 > 2} es VERDADERO.
\end{itemize}
\textbf{Alternativa correcta: a)}
\end{solbox}

\end{document}
