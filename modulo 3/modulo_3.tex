% PREÁMBULO
\documentclass[12pt]{article}

% Configurar ruta de búsqueda para archivos
\makeatletter
\def\input@path{{./}{unidades/}{modulo 3/}}
\makeatother

% --- CONFIGURACIÓN DE PÁGINA Y FUENTES ---
\PassOptionsToPackage{dvipsnames,svgnames,table,xcdraw}{xcolor}
\usepackage[utf8]{inputenc}
\usepackage[T1]{fontenc}
\usepackage[spanish,es-tabla]{babel}
\usepackage{geometry}
\geometry{a4paper, top=2.5cm, bottom=2.5cm, left=2.5cm, right=2.5cm, headheight=15pt}
\usepackage{lmodern}
\usepackage{helvet}

% --- PAQUETES MATEMÁTICOS Y DE UTILIDAD ---
\usepackage{amsmath, amssymb, amsthm, amsfonts}
\usepackage{mathtools}
\usepackage{graphicx}
\graphicspath{{imagenes/}}
\usepackage{circuitikz}
\usepackage{siunitx}
\DeclareSIUnit\million{M}
\usepackage{chemformula}
\usepackage{listings}
\usepackage{float}
\usepackage{enumitem}
\setlist{nosep}
\usepackage{multicol}
\usepackage{longtable}
\usepackage{booktabs}
\usepackage{array}

% --- DISEÑO Y COLORES ---
\usepackage{xcolor}
\usepackage[many]{tcolorbox}

\definecolor{DeepBlue}{HTML}{003B5C}
\definecolor{BrightBlue}{HTML}{007ACC}
\definecolor{Emerald}{HTML}{00A388}
\definecolor{Sunset}{HTML}{FF6B6B}
\definecolor{WarningOrange}{HTML}{FF9F43}
\definecolor{LightGray}{HTML}{F8F9FA}

% --- CAJAS PERSONALIZADAS ---
\newtcolorbox{definicion}[1][]{enhanced, breakable, colback=BrightBlue!5!white, colframe=DeepBlue, coltitle=white, fonttitle=\bfseries\sffamily, title=Definición, borderline west={4pt}{0pt}{DeepBlue}, sharp corners, boxrule=0.5pt, #1}

\newtcolorbox{teorema}[1][]{enhanced, breakable, colback=Emerald!5!white, colframe=Emerald, coltitle=white, fonttitle=\bfseries\sffamily, title=Teorema, borderline west={4pt}{0pt}{Emerald}, sharp corners, boxrule=0.5pt, #1}

\newtcolorbox{ejercicio}[1][]{enhanced, breakable, colback=white, colframe=gray!50!black, coltitle=white, fonttitle=\bfseries, title=Ejercicio Resuelto, boxrule=1pt, arc=4mm, shadow={2mm}{-2mm}{0mm}{black!20}, #1}

\newtcolorbox{tip}[1][]{enhanced, colback=WarningOrange!10!white, colframe=WarningOrange, title={Tip / Cuidado}, fonttitle=\bfseries, coltitle=black, attach boxed title to top left={yshift=-2mm, xshift=2mm}, boxed title style={colback=WarningOrange, sharp corners}, boxrule=1pt, arc=2mm, #1}

\newtcolorbox{notabox}[1][]{enhanced, colback=yellow!10!white, colframe=gray, title={Nota}, fonttitle=\bfseries, coltitle=black, boxrule=0.5pt, #1}

\newtcolorbox{solbox}[1][]{enhanced, colback=Emerald!5!white, colframe=Emerald, title={Solución}, fonttitle=\bfseries, coltitle=white, boxrule=0.5pt, #1}

\newcommand{\nota}[1]{\begin{notabox}\textbf{Nota Estratégica:} #1\end{notabox}}

% --- NAVEGACIÓN Y ENCABEZADOS ---
\usepackage{fancyhdr}
\pagestyle{fancy}
\fancyhf{}
\fancyhead[L]{\small \sffamily \textbf{Módulo 3: Ingeniería}}
\fancyhead[R]{\small \sffamily \leftmark}
\fancyfoot[C]{\thepage}
\renewcommand{\headrulewidth}{1pt}
\renewcommand{\footrulewidth}{0pt}

% Hyperref (Siempre al final del preámbulo)
\usepackage[colorlinks=true, linkcolor=DeepBlue, citecolor=Emerald, urlcolor=BrightBlue]{hyperref}

% --- MACROS MATEMÁTICOS ---
\newcommand{\R}{\mathbb{R}}
\newcommand{\N}{\mathbb{N}}
\newcommand{\Z}{\mathbb{Z}}
\newcommand{\C}{\mathbb{C}}
\newcommand{\dd}{\mathrm{d}}
\newcommand{\norm}[1]{\left\lVert#1\right\rVert}
\newcommand{\abs}[1]{\left\lvert#1\right\rvert}
\newcommand{\vect}[1]{\mathbf{#1}}

\newenvironment{pasos}{\begin{enumerate}[label=\textbf{Paso \arabic*}:, leftmargin=*]}{\end{enumerate}}

% Título del Documento
\title{\textbf{\Huge Fundamentals Resumen} \\ \Large MÓDULO 3: Ingeniería}
\author{Ingeniería UC}
\date{\today}

\begin{document}

% Portada simple
\maketitle
\tableofcontents
\newpage

% Introducción / Temario
\section*{Programa del Módulo}
\addcontentsline{toc}{section}{Programa del Módulo}
% \section*{MÓDULO 1: Matemáticas y Probabilidades}

\subsection{Cálculo I (MAT1610)}
\begin{definicion}[title=Contenidos]
\begin{enumerate}
    \item[1.] Geometría Analítica
\end{enumerate}
\end{definicion}

\begin{teorema}[title=Indicadores a evaluar (Números corresponden al correlativo del programa de cada curso)]
\begin{enumerate}
    \item[1.] Identificar gráficos de funciones básicas, exponenciales, logarítmicas. \feimply{Aprender de memoria las formas gráficas elementales. No están en el manual.}
    \item[4.] Calcular derivadas de funciones obtenidas por álgebra de funciones elementales. \fehandbook{48 (Tabla de Derivadas)}
    \item[6.] Reconocer gráfica y analíticamente propiedades de los gráficos de funciones. \fehandbook{45 (Máximos, Mínimos, Inflexión)}
    \item[9.] Conocer el cálculo de primitivas de funciones básicas. \fehandbook{49 (Tabla de Integrales)}
\end{enumerate}
\end{teorema}

\subsection{Cálculo II (MAT1620)}
\begin{definicion}[title=Contenidos]
\begin{enumerate}
    \item[1.] Cálculo Integral
\end{enumerate}
\end{definicion}

\begin{teorema}[title=Indicadores a evaluar (Números corresponden al correlativo del programa de cada curso)]
\begin{enumerate}
    \item[3.] Aplicar el concepto de integral definida para calcular áreas y momentos de regiones del plano. \fehandbook{108-112 (Centroides en sección de Estática)}
    \item[5.] Aplicar los criterios básicos de convergencia de series e integrales impropias. \feimply{¡CUIDADO! El manual solo tiene Geométrica (p.50) y Taylor (p.51). Faltan Razón, Raíz e Integral.}
    \item[8.] Conocer las ecuaciones paramétricas, vectoriales y cartesianas de rectas y planos. \fehandbook{35 (Rectas 2D) y 59 (Vectores)}
\end{enumerate}
\end{teorema}

\subsection{Cálculo III (MAT1630)}
\begin{definicion}[title=Contenidos]
\begin{enumerate}
    \item[1.] Cálculo Diferencial
\end{enumerate}
\end{definicion}

\begin{teorema}[title=Indicadores a evaluar (Números corresponden al correlativo del programa de cada curso)]
\begin{enumerate}
    \item[2.] Aplicar el concepto de integral múltiple para evaluar volúmenes y centros de masa. \fehandbook{42 (Volúmenes de Sólidos Básicos)}
    \item[5.] Reconocer y explicar el concepto de “curvas de nivel” y calcularlas. \feimply{Concepto visual. No hay fórmula.}
    \item[6.] Calcular derivadas direccionales. \fehandbook{59 (Gradiente, Divergencia, Rotor)}
\end{enumerate}
\end{teorema}

\subsection{Ecuaciones Diferenciales (MAT1640)}
\begin{definicion}[title=Contenidos]
\begin{enumerate}
    \item[1.] Ecuaciones Diferenciales
\end{enumerate}
\end{definicion}

\begin{teorema}[title=Indicadores a evaluar (Números corresponden al correlativo del programa de cada curso)]
\begin{enumerate}
    \item[2.] Modelar situaciones sencillas de la realidad y fenómenos mediante ecuaciones diferenciales. \feimply{Habilidad de modelado.}
    \item[3.] Reconocer tipo de EDO, identificar y utilizar métodos de solución según el caso. \fehandbook{51-52 (Primer y Segundo Orden)}
    \item[6.] Calcular soluciones de sistemas lineales de $2\times2$ y $3\times3$ (coef. constantes). \feimply{Teoría de valores propios para sistemas no está explícita.}
\end{enumerate}
\end{teorema}

\subsection{Álgebra Lineal (MAT1203)}
\begin{definicion}[title=Contenidos]
\begin{enumerate}
    \item[1.] Matrices
    \item[2.] Raíces de Ecuaciones
    \item[3.] Análisis Vectorial
\end{enumerate}
\end{definicion}

\begin{teorema}[title=Indicadores a evaluar (Números corresponden al correlativo del programa de cada curso)]
\begin{enumerate}
    \item[1.] Determinar escalonada reducida, resolver $Ax=b$, calcular inversas y bases. \fehandbook{57 (Matrices e Inversa)}
    \item[2.] Interpretar geométricamente dependencia lineal, complemento ortogonal. \feimply{Concepto teórico.}
    \item[4.] Explicar y utilizar propiedades de operaciones matriciales. \fehandbook{57}
    \item[6.] Explicar y utilizar matrices elementales, simétricas, ortogonales, etc. \fehandbook{57 (Definiciones)}
    \item[7.] Calcular determinantes, resolver sistemas y evaluar inversas. \fehandbook{58 (Determinantes)}
    \item[9.] Determinar matriz de Transformación Lineal y relación con cambio de base. \feimply{No está explícito.}
    \item[12.] Explicar valores/vectores propios, diagonalización y aplicaciones (simétricas). \feimply{Definición básica solamente. Algoritmo no está.}
\end{enumerate}
\end{teorema}

\subsection{Probabilidades y Estadística (EYP1113)}
\begin{definicion}[title=Contenidos]
\begin{enumerate}
    \item[1.] Álgebra de eventos, axiomas, prob. condicional, Bayes.
    \item[2.] Medidas descriptivas teóricas (media, varianza, percentil, etc.).
    \item[3.] Modelos Discretos/Continuos (Binomial, Poisson, Normal, Exp, etc.) y uso de R.
    \item[4.] Distribuciones conjuntas, covarianza, correlación.
    \item[5.] Estimación y propiedades.
    \item[6.] Test de hipótesis e intervalos de confianza.
    \item[7.] Bondad de ajuste (Chi-cuadrado).
    \item[8.] Regresión lineal (test-t, test-F, $R^2$).
\end{enumerate}
\end{definicion}

\begin{teorema}[title=Indicadores a evaluar (Números corresponden al correlativo del programa de cada curso)]
\begin{enumerate}
    \item[1.] Ajustar distribuciones de probabilidad a datos reales. \fehandbook{66-68 (Distribuciones)}
    \item[2.] Describir fenómenos de incertidumbre usando variables aleatorias. \fehandbook{63-65 (Medidas)}
    \item[3.] Realizar estimaciones de parámetros e intervalos de confianza. \fehandbook{73-74 (Tests y C.I.)}
    \item[4.] Ajustar e interpretar modelos de regresión lineal. \fehandbook{69-70 (Regresión)}
\end{enumerate}
\end{teorema}

\newpage

% --- UNIDADES ---

% 1. Economía
\input{unidades/01_economia.tex}
\newpage

% 2. Introducción a la Programación
%%%%%%%%%%%%%%%%%%%%%%%%%%%%%%%%%%%%%%%%%%%%%%%%%%%%%%%%%%%%%%%%%%%%%%%%%%%%%%%%
% CONTENIDO DE PROGRAMACIÓN
%%%%%%%%%%%%%%%%%%%%%%%%%%%%%%%%%%%%%%%%%%%%%%%%%%%%%%%%%%%%%%%%%%%%%%%%%%%%%%%%

\section{Introducción a la Programación}

%%%%%%%%%%%%%%%%%%%%%%%%%%%%%%%%%%%%%%%%%%%%%%%%%%%%%%%%%%%%%%%%%%%%%%%%%%%%%%%%
% SECCIÓN 1: INTRODUCCIÓN ESTRATÉGICA
%%%%%%%%%%%%%%%%%%%%%%%%%%%%%%%%%%%%%%%%%%%%%%%%%%%%%%%%%%%%%%%%%%%%%%%%%%%%%%%%
\subsection{Cómo Abordar Problemas de Lógica en el ECF}

Esta sección del examen no mide tu habilidad para escribir código de memoria, sino tu capacidad para \textbf{leer, interpretar y predecir el comportamiento} de un algoritmo. La clave es el razonamiento lógico, no la sintaxis perfecta.

\nota{¡Alerta Crítica! El manual de referencia oficial (FE Handbook) \textbf{no tiene una sección relevante} para programación introductoria a este nivel. Debes tratar esta parte del examen como si no tuvieras apuntes.}

\subsubsection{Habilidades Clave}
\begin{itemize}
    \item \textbf{Seguimiento de Variables (Trazas):} La habilidad más importante para el código. Debes ser capaz de seguir el valor de cada variable, línea por línea.
    \item \textbf{Comprensión de Control de Flujo:} Entender exactamente cómo y cuándo se ejecuta cada parte del código (`if`, `while`, `for`).
    \item \textbf{Manejo de Índices:} Los errores con los índices de listas y strings son muy comunes. Recuerda que comienzan en 0.
    \item \textbf{Abstracción con Funciones:} Entender qué hace una función basándose en su descripción y cómo su valor de retorno afecta al resto del programa.
\end{itemize}

%%%%%%%%%%%%%%%%%%%%%%%%%%%%%%%%%%%%%%%%%%%%%%%%%%%%%%%%%%%%%%%%%%%%%%%%%%%%%%%%
% SECCIÓN 2: CONCEPTOS DE PROGRAMACIÓN
%%%%%%%%%%%%%%%%%%%%%%%%%%%%%%%%%%%%%%%%%%%%%%%%%%%%%%%%%%%%%%%%%%%%%%%%%%%%%%%%
\section{Conceptos Fundamentales}

\subsection{Algoritmos, Variables y Expresiones}
\begin{itemize}
    \item \textbf{Algoritmo:} Una secuencia de pasos finitos y bien definidos para resolver un problema.
    \item \textbf{Variables:} Espacios en memoria que guardan un valor y tienen un nombre y un tipo de dato.
    \item \textbf{Expresiones:} Combinaciones de valores, variables y operadores.
        \begin{itemize}
            \item \textbf{Aritméticas:} `+`, `-`, `*`, `/`, `//` (división entera), `\%` (módulo).
            \item \textbf{Lógicas:} `AND`, `OR`, `NOT`.
            \item \textbf{Relacionales:} `==`, `!=`, `>`, `<`, `GTE`, `<=`.
        \end{itemize}
\end{itemize}

\subsection{Control de Flujo y Estructuras de Datos}
\begin{itemize}
    \item \textbf{Condicionales (`if`/`elif`/`else`):} Ejecutan bloques de código si se cumple una condición.
    \item \textbf{Bucles `while` y `for`:} Repiten un bloque de código. `while` lo hace mientras una condición sea verdadera; `for` itera sobre una secuencia.
    \item \textbf{Strings y Listas:} Secuencias ordenadas de elementos a los que se accede por un índice que comienza en 0.
\end{itemize}

%%%%%%%%%%%%%%%%%%%%%%%%%%%%%%%%%%%%%%%%%%%%%%%%%%%%%%%%%%%%%%%%%%%%%%%%%%%%%%%%
% SECCIÓN 4: EJERCICIOS DE PRÁCTICA (Programación)
%%%%%%%%%%%%%%%%%%%%%%%%%%%%%%%%%%%%%%%%%%%%%%%%%%%%%%%%%%%%%%%%%%%%%%%%%%%%%%%%
\newpage
\section{Ejercicios de Práctica Tipo Prueba}

\subsection{Ejercicio 1: Tipos de Datos y Expresiones}
\textbf{Problema (IIC1103-1-5):} Dado el siguiente pseudocódigo, ¿qué valores quedan almacenados en las variables \texttt{e}, \texttt{f} y \texttt{g} al final?
\begin{lstlisting}[language=Python]
a = 3
b = 15.0
c = VERDADERO
d = a - b
e = d > 0
f = b / 2 == 7
g = e AND f
\end{lstlisting}
\begin{enumerate}[label=\alph*)]
    \item \texttt{e} = -12.0, \texttt{f} = FALSO, \texttt{g} = FALSO
    \item \texttt{e} = FALSO, \texttt{f} = FALSO, \texttt{g} = FALSO
    \item \texttt{e} = FALSO, \texttt{f} = VERDADERO, \texttt{g} = FALSO
    \item \texttt{e} = 15.0, \texttt{f} = FALSO, \texttt{g} = VERDADERO
\end{enumerate}

\subsection{Ejercicio 2: Bucles y Índices}
\textbf{Problema (IIC1103-1-6):} El siguiente código intenta sumar los valores de dos listas de enteros (\texttt{v1} y \texttt{v2}). Asume que \texttt{largo\_v1} es el largo de \texttt{v1}.
\begin{lstlisting}[language=Python]
total = 0
i = 0
while i < largo_v1:
    total = total + v1[i] + v2[i]
    i = i + 1
\end{lstlisting}
¿Cuál de las siguientes afirmaciones es cierta para este código?
\begin{enumerate}[label=\alph*)]
    \item El ciclo siempre termina sin considerar el último elemento.
    \item El código no considera listas de largos distintos, pero se arregla cambiando la condición del `while`.
    \item Este código tendrá un error si \texttt{v1} tiene más elementos que \texttt{v2}.
    \item El código funciona como se espera en cualquier situación.
\end{enumerate}

%%%%%%%%%%%%%%%%%%%%%%%%%%%%%%%%%%%%%%%%%%%%%%%%%%%%%%%%%%%%%%%%%%%%%%%%%%%%%%%%
% SECCIÓN 5: SOLUCIONES DETALLADAS (Programación)
%%%%%%%%%%%%%%%%%%%%%%%%%%%%%%%%%%%%%%%%%%%%%%%%%%%%%%%%%%%%%%%%%%%%%%%%%%%%%%%%
\newpage
\section{Soluciones Detalladas}

\subsection*{Solución Ejercicio 1}
\begin{solbox}
\textbf{Estrategia:} Realizar una traza, evaluando cada línea.
\begin{itemize}
    \item \texttt{d = 3 - 15.0 = -12.0}
    \item \texttt{e = -12.0 > 0} es \texttt{FALSO}.
    \item \texttt{f = 15.0 / 2 == 7} es \texttt{7.5 == 7}, que es \texttt{FALSO}.
    \item \texttt{g = FALSO AND FALSO} es \texttt{FALSO}.
\end{itemize}
\textbf{Alternativa correcta: b)}
\end{solbox}

\subsection*{Solución Ejercicio 2}
\begin{solbox}
\textbf{Estrategia:} Analizar el acceso a los elementos de las listas en el caso límite de largos distintos.
\begin{itemize}
    \item El bucle se controla solo con el largo de \texttt{v1}.
    \item Si \texttt{v1} es más larga que \texttt{v2}, el índice \texttt{i} eventualmente superará el tamaño de \texttt{v2}, causando un error de "índice fuera de rango" al intentar acceder a \texttt{v2[i]}.
\end{itemize}
\textbf{Alternativa correcta: c)}
\end{solbox}

\newpage

% 3. Hojas de Cálculo
% Configuración local movida a estilos.tex

%%%%%%%%%%%%%%%%%%%%%%%%%%%%%%%%%%%%%%%%%%%%%%%%%%%%%%%%%%%%%%%%%%%%%%%%%%%%%%%%
% CONTENIDO DE HOJAS DE CÁLCULO
%%%%%%%%%%%%%%%%%%%%%%%%%%%%%%%%%%%%%%%%%%%%%%%%%%%%%%%%%%%%%%%%%%%%%%%%%%%%%%%%

\section{Hojas de Cálculo (Excel)}

\subsection{Conceptos Fundamentales de Hojas de Cálculo}

La clave para resolver estos problemas es entender cómo las fórmulas cambian cuando se copian y pegan.

\subsubsection{Referencias de Celda: El Concepto Clave}
El símbolo \texttt{\$} congela o ancla una fila o una columna, evitando que cambie al copiar la fórmula.

\begin{center}
\begin{tabular}{|p{4cm}|p{2.5cm}|p{8cm}|}
\hline
\textbf{Tipo de Referencia} & \textbf{Ejemplo} & \textbf{Comportamiento al Copiar y Pegar} \\
\hline
\textbf{Relativa} & \texttt{A1} & Tanto la columna (A) como la fila (1) \textbf{cambian} según la dirección en que se mueva la fórmula. \\
\hline
\textbf{Absoluta} & \texttt{\$A\$1} & Ni la columna (A) ni la fila (1) \textbf{cambian}. La referencia está totalmente anclada. \\
\hline
\textbf{Mixta (Columna Absoluta)} & \texttt{\$A1} & La columna (A) está anclada y \textbf{no cambia}, pero la fila (1) \textbf{sí cambia}. \\
\hline
\textbf{Mixta (Fila Absoluta)} & \texttt{A\$1} & La columna (A) \textbf{sí cambia}, pero la fila (1) está anclada y \textbf{no cambia}. \\
\hline
\end{tabular}
\end{center}

\subsubsection{Funciones Comunes Utilizadas}
\begin{itemize}
    \item \texttt{PROMEDIO(rango)}: Calcula la media aritmética.
    \item \texttt{MAX(rango)}: Devuelve el valor más alto del rango.
    \item \texttt{SI(prueba\_logica; valor\_si\_verdadero; valor\_si\_falso)}: Evalúa una condición y devuelve un valor según el resultado.
    \item \texttt{CONTAR.SI(rango; criterio)}: Cuenta celdas que cumplen un criterio.
    \item \texttt{SUMA(rango)}: Suma todos los números en un rango.
    \item \texttt{MEDIANA(rango)}: Devuelve el número central de un conjunto de datos ordenado.
\end{itemize}

%%%%%%%%%%%%%%%%%%%%%%%%%%%%%%%%%%%%%%%%%%%%%%%%%%%%%%%%%%%%%%%%%%%%%%%%%%%%%%%%
% SECCIÓN 4: EJERCICIOS DE PRÁCTICA (Excel)
%%%%%%%%%%%%%%%%%%%%%%%%%%%%%%%%%%%%%%%%%%%%%%%%%%%%%%%%%%%%%%%%%%%%%%%%%%%%%%%%
\newpage
\section{Ejercicios de Práctica Tipo Prueba}

\subsection{Ejercicio 1: Referencias de Celda}
\textbf{Problema (TRANS-3):} La celda A4, que contiene la fórmula \texttt{=PROMEDIO(\$A1:B\$2)}, se copia en la celda C5. ¿Qué fórmula queda en C5?
\begin{center}
\begin{tabular}{|c|c|c|c|c|}
\hline
\textbf{} & \textbf{A} & \textbf{B} & \textbf{C} & \textbf{D} \\
\hline
\textbf{1} & 2 & 2 & 0 & 1 \\
\hline
\textbf{2} & 1 & 1 & 1 & 1 \\
\hline
\textbf{3} & 1 & 1 & 1 & 1 \\
\hline
\textbf{4} & \texttt{=PROMEDIO(\$A1:B\$2)} & & & \\
\hline
\textbf{5} & & & & \\
\hline
\end{tabular}
\end{center}
\begin{enumerate}[label=\alph*)]
    \item \texttt{=PROMEDIO(\$A1:B\$2)}
    \item \texttt{=PROMEDIO(C2:D3)}
    \item \texttt{=PROMEDIO(\$C2:D\$3)}
    \item \texttt{=PROMEDIO(\$A2:D\$2)}
\end{enumerate}

\subsection{Ejercicio 2: Funciones y Referencias}
\textbf{Problema (Trans-9):} La celda E1, que contiene la fórmula \texttt{=MAX(A\$1:C\$2)}, se copia en la celda E2. ¿Qué valor queda almacenado en E2?
\begin{center}
\begin{tabular}{|c|c|c|c|c|c|}
\hline
\textbf{} & \textbf{A} & \textbf{B} & \textbf{C} & \textbf{D} & \textbf{E} \\
\hline
\textbf{1} & 1 & 2 & 1 & & \texttt{=MAX(A\$1:C\$2)} \\
\hline
\textbf{2} & 1 & 0 & 2 & & \\
\hline
\textbf{3} & 1 & 1 & 1 & & \\
\hline
\textbf{4} & 3 & 2 & 1 & & \\
\hline
\textbf{5} & & & & & \\
\hline
\end{tabular}
\end{center}
\begin{enumerate}[label=\alph*)]
    \item 2
    \item 3
    \item 1
\end{enumerate}

\subsection{Ejercicio 3: Lógica Condicional}
\textbf{Problema (TRANS-8):} ¿Qué fórmula debe ponerse en la celda D3 para que ésta quede con valor "NO"?
\begin{center}
\begin{tabular}{|c|c|c|c|c|}
\hline
\textbf{} & \textbf{A} & \textbf{B} & \textbf{C} & \textbf{D} \\
\hline
\textbf{1} & 1 & 1 & 3 & \\
\hline
\textbf{2} & 2 & 1 & 0 & \\
\hline
\textbf{3} & & & & \\
\hline
\end{tabular}
\end{center}
\begin{enumerate}[label=\alph*)]
    \item \texttt{=SI(MEDIANA(A1:C1)>MEDIANA(A2:C2);"SI";"NO")}
    \item \texttt{=SI(SUMA(A1:C1)>SUMA(A2:C2);"SI";"NO")}
    \item \texttt{=SI(PROMEDIO(A1:C1)>PROMEDIO(A2:C2);"SI";"NO")}
    \item \texttt{=SI(MAX(A1:C1)>MAX(A2:C2);"SI";"NO")}
\end{enumerate}

%%%%%%%%%%%%%%%%%%%%%%%%%%%%%%%%%%%%%%%%%%%%%%%%%%%%%%%%%%%%%%%%%%%%%%%%%%%%%%%%
% SECCIÓN 5: SOLUCIONES DETALLADAS (Excel)
%%%%%%%%%%%%%%%%%%%%%%%%%%%%%%%%%%%%%%%%%%%%%%%%%%%%%%%%%%%%%%%%%%%%%%%%%%%%%%%%
\newpage
\section{Soluciones Detalladas}

\subsection*{Solución Ejercicio 1}
\begin{solbox}
\textbf{Estrategia:} Analizar el movimiento y aplicar las reglas de referencias.
\begin{itemize}
    \item \textbf{Movimiento:} De A4 a C5 (+2 columnas, +1 fila).
    \item \textbf{Fórmula:} \texttt{=PROMEDIO(\$A1:B\$2)}
    \item \texttt{\$A1}: Columna `A` anclada, no cambia. Fila `1` relativa, `1+1=2` $\rightarrow$ \texttt{\$A2}.
    \item \texttt{B\$2}: Columna `B` relativa, `B+2=D`. Fila `2` anclada, no cambia $\rightarrow$ \texttt{D\$2}.
    \item \textbf{Fórmula resultante:} \texttt{=PROMEDIO(\$A2:D\$2)}.
\end{itemize}
\textbf{Alternativa correcta: d)}
\end{solbox}

\subsection*{Solución Ejercicio 2}
\begin{solbox}
\textbf{Estrategia:} Copiar la fórmula y luego calcular su valor.
\begin{itemize}
    \item \textbf{Movimiento:} De E1 a E2 (0 columnas, +1 fila).
    \item \textbf{Fórmula:} \texttt{=MAX(A\$1:C\$2)}
    \item `A\$1`: Columna `A` relativa, `A+0=A`. Fila `1` anclada, no cambia $\rightarrow$ \texttt{A\$1}.
    \item `C\$2`: Columna `C` relativa, `C+0=C`. Fila `2` anclada, no cambia $\rightarrow$ \texttt{C\$2}.
    \item \textbf{Fórmula resultante en E2:} \texttt{=MAX(A\$1:C\$2)}. Es la misma.
    \item \textbf{Cálculo:} El rango `A1:C2` contiene los valores `{1, 2, 1, 1, 0, 2}`. El valor máximo es 2.
\end{itemize}
\textbf{Alternativa correcta: a)}
\end{solbox}

\subsection*{Solución Ejercicio 3}
\begin{solbox}
\textbf{Estrategia:} Evaluar la condición lógica de cada alternativa para ver cuál resulta `FALSA`.
\begin{itemize}
    \item \textbf{Datos:} Rango A1:C1 = `{1, 1, 3}`, Rango A2:C2 = `{2, 1, 0}`.
    \item a) \texttt{MEDIANA(A1:C1)>MEDIANA(A2:C2)} $\rightarrow$ \texttt{1 > 1} es \textbf{FALSO}. Esta devuelve "NO".
    \item b) \texttt{SUMA(A1:C1)>SUMA(A2:C2)} $\rightarrow$ \texttt{5 > 3} es VERDADERO.
    \item c) \texttt{PROMEDIO(A1:C1)>PROMEDIO(A2:C2)} $\rightarrow$ \texttt{1.66 > 1} es VERDADERO.
    \item d) \texttt{MAX(A1:C1)>MAX(A2:C2)} $\rightarrow$ \texttt{3 > 2} es VERDADERO.
\end{itemize}
\textbf{Alternativa correcta: a)}
\end{solbox}

\newpage

% 4. Ética
% Configuración local movida a estilos.tex

%%%%%%%%%%%%%%%%%%%%%%%%%%%%%%%%%%%%%%%%%%%%%%%%%%%%%%%%%%%%%%%%%%%%%%%%%%%%%%%%
% CONTENIDO DE ÉTICA
%%%%%%%%%%%%%%%%%%%%%%%%%%%%%%%%%%%%%%%%%%%%%%%%%%%%%%%%%%%%%%%%%%%%%%%%%%%%%%%%

\section{Ética para Ingenieros (FIL188)}

%%%%%%%%%%%%%%%%%%%%%%%%%%%%%%%%%%%%%%%%%%%%%%%%%%%%%%%%%%%%%%%%%%%%%%%%%%%%%%%%
% SECCIÓN 1: INTRODUCCIÓN ESTRATÉGICA
%%%%%%%%%%%%%%%%%%%%%%%%%%%%%%%%%%%%%%%%%%%%%%%%%%%%%%%%%%%%%%%%%%%%%%%%%%%%%%%%
\subsection{Cómo Abordar la Ética Profesional (y no morir en el intento)}

Esta no es una guía de fórmulas, sino un mapa de pensamiento. En FIL188, el éxito no depende de memorizar reglas, sino de entender y aplicar \textbf{marcos éticos} para analizar situaciones complejas.

\nota{¡Alerta Crítica! El manual de referencia estándar (como el FE Handbook) es \textbf{completamente inútil} para esta materia. Se enfoca en códigos de EE.UU. y no cubre las teorías filosóficas (Aristóteles, Kant, Rawls) ni el marco legal de Chile, que son el núcleo de este curso. Tu única fuente confiable son los materiales de FIL188.}

\subsubsection{Habilidades Clave}
\begin{itemize}
    \item \textbf{Identificar el Dilema Ético:} ¿Cuál es el conflicto real? ¿Deber vs. consecuencia? ¿Bien personal vs. bien común? ¿Lealtad vs. seguridad pública?
    \item \textbf{Aplicar Marcos Teóricos:} Debes ser capaz de ver un caso desde la perspectiva de Aristóteles (¿qué haría una persona virtuosa?), Kant (¿cuál es el deber universal?) y Mill (¿qué opción maximiza el bienestar general?).
    \item \textbf{Argumentación y Justificación:} La respuesta correcta no es solo una letra, es la justificación que la respalda. Debes poder explicar *por qué* una acción es ética o no, basándote en los principios estudiados.
\end{itemize}

%%%%%%%%%%%%%%%%%%%%%%%%%%%%%%%%%%%%%%%%%%%%%%%%%%%%%%%%%%%%%%%%%%%%%%%%%%%%%%%%
% SECCIÓN 2: TEMARIO ESTRUCTURADO
%%%%%%%%%%%%%%%%%%%%%%%%%%%%%%%%%%%%%%%%%%%%%%%%%%%%%%%%%%%%%%%%%%%%%%%%%%%%%%%%
\section{El Mapa Conceptual de la Ética}

\subsection{Fundamentos: ¿Qué es la Ética?}
\begin{itemize}
    \item \textbf{Concepto:} La ética es la rama de la filosofía que estudia la moral; es decir, lo que se considera bueno o malo, correcto o incorrecto. No busca responder "¿qué puedo hacer?", sino \textbf{"¿qué debo hacer?"} y, más profundamente, \textbf{"¿qué tipo de persona quiero ser?"}.
    \item \textbf{Realismo vs. Relativismo Moral:} El realismo sostiene que existen verdades morales objetivas (ej. "torturar es malo"). El relativismo argumenta que lo correcto o incorrecto depende del individuo o la cultura. Los códigos de ética profesional se basan en un realismo moral (hay acciones que son objetivamente incorrectas en la profesión).
    \item \textbf{Conciencia Moral (Kohlberg y Gilligan):} Se refiere al desarrollo de nuestra capacidad para razonar sobre lo que es correcto. Kohlberg propone etapas basadas en la justicia, mientras que Gilligan añade la "ética del cuidado", que valora las relaciones y la responsabilidad interpersonal.
\end{itemize}

\subsection{Las Grandes Corrientes Éticas (Las Herramientas de Análisis)}

\subsubsection{Ética de la Virtud (Aristóteles)}
\begin{itemize}
    \item \textbf{Idea Central:} El foco no está en el acto, sino en el \textbf{carácter} del agente. Una acción es correcta si es la que una persona virtuosa haría.
    \item \textbf{Pregunta Clave:} "¿Qué tipo de persona me convierte esta acción?"
    \item \textbf{Conceptos Fundamentales:}
        \begin{itemize}
            \item \textbf{Virtud (Areté):} La excelencia del carácter. No se nace virtuoso, se llega a serlo mediante la práctica y el hábito.
            \item \textbf{Justo Medio:} La virtud se encuentra en un punto intermedio entre dos vicios (un exceso y un defecto). Ej: La valentía es el justo medio entre la cobardía (defecto) y la temeridad (exceso).
            \item \textbf{Prudencia (Phronesis):} La virtud intelectual clave. Es la sabiduría práctica que nos permite discernir el justo medio en cada situación específica.
        \end{itemize}
    \nota{Usa Aristóteles cuando un caso evalúe el carácter de un profesional (honestidad, integridad, valentía) o cuando la decisión correcta no sea obvia y requiera de juicio práctico (prudencia).}
\end{itemize}

\subsubsection{Deontología (Immanuel Kant)}
\begin{itemize}
    \item \textbf{Idea Central:} La moralidad de una acción reside en el \textbf{deber} mismo, no en sus consecuencias. Hay reglas y principios que son universalmente válidos.
    \item \textbf{Pregunta Clave:} "¿Cuál es mi deber? ¿Puedo querer que mi acción se convierta en una ley universal?"
    \item \textbf{Conceptos Fundamentales:}
        \begin{itemize}
            \item \textbf{Imperativo Categórico:} Un mandato que se debe seguir incondicionalmente. Su formulación más famosa es: "Actúa solo según una máxima tal que puedas querer al mismo tiempo que se torne en ley universal". En simple: si no está bien que todos lo hagan, no está bien que tú lo hagas.
            \item \textbf{Actuar "por deber" vs. "conforme al deber":} Una acción tiene valor moral solo si se hace \textit{por} deber (la motivación es el deber mismo), no solo \textit{conforme} al deber (la acción es correcta, pero la motivación es egoísta, como el interés personal).
        \end{itemize}
    \nota{Piensa en Kant cuando un caso presente un conflicto entre una regla moral clara (ej. "no mentir", "no robar") y la posibilidad de obtener un buen resultado al romperla.}
\end{itemize}

\subsubsection{Utilitarismo (Bentham y Stuart Mill)}
\begin{itemize}
    \item \textbf{Idea Central:} La acción moralmente correcta es aquella que maximiza la felicidad o el bienestar general (\textbf{utilidad}) para el mayor número de personas.
    \item \textbf{Pregunta Clave:} "¿Qué acción producirá las mejores consecuencias para todos los afectados?"
    \item \textbf{Conceptos Fundamentales:}
        \begin{itemize}
            \item \textbf{Principio de la Mayor Felicidad:} Las acciones son correctas en la medida que tienden a promover la felicidad, e incorrectas si tienden a producir lo contrario.
            \item \textbf{Cálculo de Utilidad:} Se deben sopesar los placeres y dolores, beneficios y perjuicios de todos los involucrados. Es un enfoque consecuencialista.
        \end{itemize}
    \nota{El utilitarismo es la herramienta para casos que involucran la seguridad pública, decisiones de política o proyectos con impacto social. El dilema es calcular y comparar los beneficios y daños.}
\end{itemize}

\section{La Ética en la Práctica: El Código del Ingeniero en Chile}

Las teorías filosóficas nos dan el "porqué", pero el Código de Ética del Colegio de Ingenieros de Chile nos da el "cómo". Es el documento que traduce los grandes principios a deberes y responsabilidades concretas para la profesión en nuestro país. Para la prueba, no necesitas memorizar cada artículo, pero sí entender sus principios y estructura.

\subsection{Principios Fundamentales del Código}
El código se organiza en torno a las relaciones del ingeniero. Los deberes más importantes se pueden agrupar en:

\begin{itemize}
    \item \textbf{Deber Supremo con la Sociedad:} La obligación más alta es proteger la \textbf{vida, seguridad, salud y bienestar de la comunidad}, además del medio ambiente (Artículos d, B.4, E.1, E.2). Este deber prima sobre cualquier otro.
    \item \textbf{Deber con el Cliente o Empleador:} Implica \textbf{lealtad, confidencialidad y actuar como agente de confianza} (Artículos g, D.2). Sin embargo, esta lealtad tiene un límite claro.
    \item \textbf{Deber con la Profesión y los Colegas:} Requiere mantener el \textbf{prestigio de la profesión}, ser honesto, no competir de mala fe y dar crédito por el trabajo ajeno (Artículos b, h, C.4, C.5).
    \item \textbf{Deber con la Propia Competencia:} El ingeniero debe actuar solo en sus áreas de competencia, mantener su autonomía profesional y actualizar sus conocimientos (Artículos e, A.5, B.1, B.16).
\end{itemize}

\subsection{Cómo Analizar un Caso Usando el Código (Método de 4 Pasos)}
Frente a un caso práctico en la prueba, sigue estos pasos:

\begin{enumerate}
    \item \textbf{Identificar el Dilema:} ¿Cuál es el conflicto central? Usualmente es un choque entre deberes. Por ejemplo: Lealtad al empleador (que quiere ahorrar costos) vs. Deber con la seguridad pública (usar materiales de calidad).
    
    \item \textbf{Buscar los Artículos Pertinentes:} Piensa en las categorías anteriores. 
    \begin{itemize}
        \item ¿El caso involucra un riesgo para la gente? Busca en los deberes con la sociedad (Título II-B y V-E).
        \item ¿Hay un conflicto de intereses o un tema de confidencialidad? Revisa los deberes con el mandante (Título IV-D).
        \item ¿Se trata de plagio o de robar el mérito a un colega? Ve a las relaciones entre profesionales (Título III-C).
    \end{itemize}
    
    \item \textbf{Jerarquizar los Deberes:} El propio código establece una jerarquía. El artículo A.7, por ejemplo, resuelve el conflicto entre lealtad y autonomía, dando prioridad a la \textbf{autonomía profesional} y la integridad de los resultados. El deber con la seguridad pública siempre será el más importante.
    
    \item \textbf{Conectar con la Teoría Filosófica (Nivel Avanzado):} Justifica tu respuesta conectando el código con las corrientes éticas.
    \begin{itemize}
        \item El deber de proteger la seguridad pública (B.4) es una aplicación directa del \textbf{Utilitarismo} (minimizar el daño al mayor número de personas).
        \item El deber de ser veraz (B.3) o no mentir sobre las propias credenciales (B.11) refleja la \textbf{Deontología} de Kant (principios universales).
        \item La obligación de actuar con corrección, honor y prestigio (b, h) apela a la \textbf{Ética de la Virtud} de Aristóteles (el carácter del profesional).
    \end{itemize}
\end{enumerate}

\nota{El Código de Ética no es solo una lista de reglas, es una herramienta para el razonamiento. En caso de duda, el principio que casi siempre te guiará a la respuesta correcta es: \textbf{la protección de la vida y el bienestar de la comunidad está por encima de todo lo demás}.}


%%%%%%%%%%%%%%%%%%%%%%%%%%%%%%%%%%%%%%%%%%%%%%%%%%%%%%%%%%%%%%%%%%%%%%%%%%%%%%%%
% SECCIÓN DE EJERCICIOS (Ética)
%%%%%%%%%%%%%%%%%%%%%%%%%%%%%%%%%%%%%%%%%%%%%%%%%%%%%%%%%%%%%%%%%%%%%%%%%%%%%%%%
\newpage
\section{Ejercicios de Práctica Tipo Prueba}

\subsection{Ejercicio 1: Fundamentos de la Ética}
\textbf{Problema (FIL188-1-4-15):} ¿A cuál de las siguientes preguntas responde la Ética?
\begin{enumerate}[label=\alph*)]
    \item ¿En qué área de la ciencia o tecnología quiero desarrollarme?
    \item ¿Qué tipo de persona quiero ser?
    \item ¿Cómo puedo ser más eficiente en mi trabajo profesional?
    \item ¿Cómo puedo tener éxito en la vida?
\end{enumerate}

\subsection{Ejercicio 2: Aplicación de Deontología (Kant)}
\textbf{Problema (FIL188-1-1):} Una ingeniera que con su trabajo cumple sus deberes con la sociedad, pero que está motivada por su propio interés de obtener ingresos actúa:
\begin{enumerate}[label=\alph*)]
    \item Por el deber.
    \item Conforme al deber.
    \item De manera contraria al deber.
    \item Ninguna de las anteriores.
\end{enumerate}

\subsection{Ejercicio 3: Aplicación de Ética de la Virtud (Aristóteles)}
\textbf{Problema (FIL188-1-4):} La principal de las virtudes aristotélicas, que nos permite encontrar el medio entre el exceso y el defecto es:
\begin{enumerate}[label=\alph*)]
    \item La templanza
    \item La sabiduría
    \item La sensatez
    \item La prudencia
\end{enumerate}

\subsection{Ejercicio 4: Caso de Responsabilidad Profesional}
\textbf{Problema (FIL188-3-2):} El caso de los ingenieros de BART que, tras notar fallas de seguridad, insistieron en reportarlas al directorio a pesar de la inacción de su gerente y de ser despedidos, y que luego fueron premiados por el colegio de ingenieros. La razón que mejor debería explicar el premio es:
\begin{enumerate}[label=\alph*)]
    \item Ignorar las jerarquías si pienso que tengo la razón.
    \item Poner el mayor beneficio de los demás por sobre el bien individual.
    \item Tener las capacidades técnicas necesarias para identificar un problema.
    \item El cumplimiento del deber o poner el conocimiento al servicio de la seguridad.
\end{enumerate}

\subsection{Ejercicio 5: Aplicación del Código de Ética de Chile}
\textbf{Problema (FIL188-4-1):} De acuerdo con la disposición A.7 del Código de Ética: "Los ingenieros realizarán su trabajo con independencia... sin permitir que el interés de sus clientes o empleadores influyan en los resultados netamente profesionales...". ¿Existe un conflicto de deberes y cómo se resuelve?
\begin{enumerate}[label=\alph*)]
    \item No hay un conflicto de deberes.
    \item Sí hay un conflicto de deberes, pero queda al ingeniero resolver cuál deber tiene más peso.
    \item Sí hay un conflicto de deberes, y se resuelve prefiriendo el deber de lealtad al empleador.
    \item Sí hay un conflicto de deberes, y se resuelve prefiriendo el deber de autonomía profesional.
\end{enumerate}

\subsection{Ejercicio 6: Caso de Integridad Profesional}
\textbf{Problema (FIL188-4.3):} El caso de Fernando, el joven ingeniero que, presionado por el ritmo de trabajo y los bonos, comienza a entregar trabajos de mala calidad e incluso a copiar proyectos antiguos. De acuerdo al caso, Fernando actuó:
\begin{enumerate}[label=\alph*)]
    \item Bien, pues ayudó a que el grupo tuviera éxito.
    \item Bien, ya que trataba de obtener mejores ingresos.
    \item Mal, no poseer una competencia profesional no justifica caer en faltas a la ética.
    \item Bien, ya que muchos trabajadores que estuvieran en su lugar habrían hecho lo mismo.
\end{enumerate}

%%%%%%%%%%%%%%%%%%%%%%%%%%%%%%%%%%%%%%%%%%%%%%%%%%%%%%%%%%%%%%%%%%%%%%%%%%%%%%%%
% SECCIÓN DE SOLUCIONES (Ética)
%%%%%%%%%%%%%%%%%%%%%%%%%%%%%%%%%%%%%%%%%%%%%%%%%%%%%%%%%%%%%%%%%%%%%%%%%%%%%%%%
\newpage
\section{Soluciones Detalladas}

\subsection*{Solución Ejercicio 1}
\begin{solbox}
\textbf{Estrategia:} Identificar la pregunta fundamental que la ética intenta resolver. La ética no es una guía para el éxito o la eficiencia, sino para la formación del carácter y la moral.

\textbf{Análisis de Alternativas:}
\begin{itemize}
    \item a) y c) son preguntas sobre desarrollo profesional y técnico, no ético.
    \item d) es una pregunta sobre éxito personal, que puede o no estar alineado con la ética.
    \item b) "Qué tipo de persona quiero ser" va al corazón de la ética, especialmente de la ética de la virtud, que se enfoca en el desarrollo del carácter moral.
\end{itemize}
\textbf{Conclusión:} La ética se ocupa fundamentalmente de la pregunta sobre cómo debemos vivir y qué clase de persona debemos ser.

\textbf{Alternativa correcta: b)}
\end{solbox}

\subsection*{Solución Ejercicio 2}
\begin{solbox}
\textbf{Estrategia:} Aplicar la distinción clave de la deontología de Kant entre la motivación y la acción.

\textbf{Análisis Conceptual:}
\begin{itemize}
    \item La ingeniera \textbf{cumple sus deberes}, por lo tanto, su acción es externamente correcta. Esto descarta c).
    \item Su \textbf{motivación es el interés propio} (obtener ingresos), no el deber por el deber mismo.
    \item Según Kant, actuar \textbf{por el deber} significa que la única motivación es el respeto a la ley moral.
    \item Actuar \textbf{conforme al deber} significa que la acción coincide con lo que el deber exige, pero la motivación es otra (inclinación personal, interés, miedo, etc.).
\end{itemize}
\textbf{Conclusión:} La acción de la ingeniera es correcta en su resultado, pero su motivación no es puramente moral según Kant. Por lo tanto, actúa "conforme al deber".

\textbf{Alternativa correcta: b)}
\end{solbox}

\subsection*{Solución Ejercicio 3}
\begin{solbox}
\textbf{Estrategia:} Recordar el concepto central de la ética aristotélica para la toma de decisiones prácticas.

\textbf{Análisis Conceptual:} Aristóteles postula que la virtud moral es un "justo medio" entre dos extremos viciosos. La capacidad intelectual que nos permite identificar y elegir ese justo medio en situaciones concretas no es la sabiduría teórica (sophia) ni la templanza (una virtud moral específica), sino la \textbf{prudencia} o sabiduría práctica (phronesis).

\textbf{Conclusión:} La prudencia es la virtud clave que guía a las demás virtudes morales al identificar el camino correcto en cada contexto.

\textbf{Alternativa correcta: d)}
\end{solbox}

\subsection*{Solución Ejercicio 4}
\begin{solbox}
\textbf{Estrategia:} Identificar el principio ético superior que justifica la acción de los ingenieros, incluso a costa de su empleo.

\textbf{Análisis de Alternativas:}
\begin{itemize}
    \item a) es una mala interpretación. No se trata de "ignorar jerarquías" por capricho, sino por una razón de peso.
    \item b) es una formulación más utilitarista, y aunque es parcialmente cierta, la opción d) es más precisa desde la ética profesional.
    \item c) es una condición necesaria (sin su capacidad técnica no habrían visto el problema), pero no es la razón del premio ético. El premio es por lo que \textit{hicieron} con ese conocimiento.
    \item d) encapsula perfectamente el núcleo de la responsabilidad profesional: el deber de un ingeniero es usar su conocimiento para proteger la seguridad pública, un deber que prima sobre la lealtad a un empleador o el bienestar personal.
\end{itemize}
\textbf{Conclusión:} El premio reconoce que los ingenieros cumplieron con su deber profesional fundamental de anteponer la seguridad pública.

\textbf{Alternativa correcta: d)}
\end{solbox}

\subsection*{Solución Ejercicio 5}
\begin{solbox}
\textbf{Estrategia:} Analizar el texto del artículo A.7 para identificar los deberes presentes y cómo se jerarquizan.

\textbf{Análisis del Artículo:}
\begin{itemize}
    \item El artículo menciona dos deberes: ser leal al cliente/empleador ("independencia de todo interés que no sea el de su cliente"), y al mismo tiempo mantener la integridad de los resultados profesionales ("sin permitir que el interés de sus clientes... influyan en los resultados").
    \item Esto crea un \textbf{conflicto potencial}: ¿Qué pasa si el empleador quiere que alteres un resultado para su beneficio?
    \item La segunda parte del artículo resuelve el conflicto: "sin permitir que el interés [del empleador] influya en los resultados". Esto establece una jerarquía clara.
\end{itemize}
\textbf{Conclusión:} Existe un conflicto de deberes, y se resuelve explícitamente en el mismo artículo, dando prioridad a la integridad y la autonomía profesional por sobre la lealtad mal entendida.

\textbf{Alternativa correcta: d)}
\end{solbox}

\subsection*{Solución Ejercicio 6}
\begin{solbox}
\textbf{Estrategia:} Evaluar la acción de Fernando desde una perspectiva ética profesional, considerando la competencia, la honestidad y las consecuencias.

\textbf{Análisis de Alternativas:}
\begin{itemize}
    \item a) y b) son incorrectas. Justifican una acción antiética (engaño, plagio, trabajo de mala calidad) por sus resultados a corto plazo (éxito del grupo, ingresos). Esto es un razonamiento ético deficiente.
    \item d) es una falacia ("todos lo harían"). Que una conducta sea común no la hace éticamente correcta.
    \item c) identifica el problema central. Fernando enfrentaba un problema de \textbf{competencia} (no podía seguir el ritmo). En lugar de abordarlo de manera ética (pidiendo ayuda, formándose, siendo honesto sobre sus limitaciones), eligió un camino no ético (plagio, trabajo apresurado y de mala calidad) que violó la confianza de sus clientes y empleadores.
\end{itemize}
\textbf{Conclusión:} La falta de habilidad técnica o la presión laboral no son excusas para cometer faltas éticas como el plagio y el engaño.

\textbf{Alternativa correcta: c)}
\end{solbox}

\newpage

\end{document}
