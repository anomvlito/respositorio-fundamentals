\subsection{Economía (ICS1513)}
\begin{definicion}[title=Contenidos]
\begin{enumerate}
    \item[A.] Flujo de caja (por ejemplo, la equivalencia, tasa de retorno).
    \item[B.] Costo (por ejemplo, incremental, promedio, hundido, estimación).
    \item[C.] Análisis (por ejemplo, el punto de equilibrio, de costo-beneficio).
    \item[D.] La incertidumbre (por ejemplo, valor esperado y el riesgo).
\end{enumerate}
\end{definicion}

\begin{teorema}[title=Indicadores a evaluar (Números corresponden al correlativo del programa de cada curso)]
\begin{enumerate}
    \item[1.] Entender y aplicar los conceptos básicos del análisis económico y entender los problemas centrales que estudia la economía, tanto a nivel micro como macro.
    \item[2.] Analizar los elementos fundamentales que explican el comportamiento de los agentes, el rol de los mercados y las leyes de oferta y demanda.
    \item[4.] Entender y aplicar los conceptos básicos de flujo de caja, tasa de descuento, valor presente y tasa interna de retorno, asociados a un proyecto y ser capaz de calcularlos y aplicarlos en un proyecto.
\end{enumerate}
\end{teorema}

\subsection{Introducción a la Programación (IIC1103)}
\begin{definicion}[title=Contenidos]
\begin{enumerate}
    \item[1.] Terminología (tipos de memoria, CPU, velocidades de transmisión, internet).
    \item[3.] Programación.
\end{enumerate}
\end{definicion}

\begin{teorema}[title=Indicadores a evaluar (Números corresponden al correlativo del programa de cada curso)]
\begin{enumerate}
    \item[1.] Comprender conceptos básicos relativos a un programa computacional, tales como algoritmos, variables, expresiones, control de flujo, funciones, listas, strings, clases y objetos.
    \item[2.] Aplicar técnicas fundamentales para la resolución de diversos problemas con ayuda del computador.
    \item[3.] Aplicar el razonamiento algorítmico para generar la solución a un problema como una secuencia de pasos bien definidos, incluyendo pasos condicionales, repetición de pasos, llamadas a funciones, y recursión.
\end{enumerate}
\end{teorema}

\subsection{Hojas de Cálculo}
\begin{definicion}[title=Contenidos]
\begin{enumerate}
    \item[2.] Hojas de Cálculo: Manejo Nivel Básico.
\end{enumerate}
\end{definicion}

\begin{teorema}[title=Indicadores a evaluar (Números corresponden al correlativo del programa de cada curso)]
\begin{enumerate}
    \item Transversal a la formación.
\end{enumerate}
\end{teorema}

\subsection{Ética (FIL188)}
\begin{definicion}[title=Contenidos]
\begin{enumerate}
    \item[1.] Código de ética (sociedades profesionales y técnicas).
    \item[2.] Acuerdos y contratos.
    \item[3.] Legislación y ética.
    \item[4.] Responsabilidad Profesional.
    \item[5.] Cuestiones de protección pública.
\end{enumerate}
\end{definicion}

\begin{teorema}[title=Indicadores a evaluar (Números corresponden al correlativo del programa de cada curso)]
\begin{enumerate}
    \item[1.] Conocer los fundamentos y las principales corrientes éticas que subyacen a las decisiones morales del hombre, tales como:
    \begin{itemize}
        \item Relativismo, Subjetivismo y Realismo moral (Lawrence Kohlberg).
        \item La conciencia moral, y la ética del cuidado (Carol Gilligan).
        \item Utilitarismo (Jeremy Bentham y John Stuart Mill).
        \item Deontología (Immanuel Kant).
        \item Ética de la Virtud (Aristóteles).
        \item Ética de la Justicia (John Rawls).
    \end{itemize}
    \item[2.] Reflexionar sobre los criterios y principios para orientar la acción en el ámbito científico y tecnológico.
    \item[3.] Aplicar los principios de la ética a problemas específicos de la ciencia y la tecnología.
    \item[4.] Identificar los alcances de la legislación nacional (código de ética) vigente relacionada con la práctica de la ingeniería y saber cómo aplicarla en el ejercicio de la profesión.
\end{enumerate}
\end{teorema}
