\section*{MÓDULO 3: Ingeniería}

En este módulo se abordan tópicos transversales de la ingeniería: Economía, Programación, Hojas de Cálculo y Ética.

\begin{longtable}{|p{0.15\textwidth}|p{0.20\textwidth}|p{0.30\textwidth}|p{0.35\textwidth}|}
\hline
\rowcolor[HTML]{E6F0FF}
\textbf{Tópico} & \textbf{Curso} & \textbf{Contenidos} & \textbf{Indicadores a evaluar} \\ \hline
\endhead

\rowcolor[HTML]{FFFFFF}
Economía & 
ICS1513 \newline Introducción a la Economía & 
A. Flujo de caja (por ejemplo, la equivalencia, tasa de retorno). \newline
B. Costo (por ejemplo, incremental, promedio, hundido, estimación). \newline
C. Análisis (por ejemplo, el punto de equilibrio, de costo-beneficio). \newline
D. La incertidumbre (por ejemplo, valor esperado y el riesgo). & 
1. Entender y aplicar los conceptos básicos del análisis económico y entender los problemas centrales que estudia la economía, tanto a nivel micro como macro. \newline
2. Analizar los elementos fundamentales que explican el comportamiento de los agentes, el rol de los mercados y las leyes de oferta y demanda. \newline
4. Entender y aplicar los conceptos básicos de flujo de caja, tasa de descuento, valor presente y tasa interna de retorno, asociados a un proyecto y ser capaz de calcularlos y aplicarlos en un proyecto. \\ \hline

\rowcolor[HTML]{F9F9F9}
Introducción a la Programación & 
IIC1103 \newline Introducción a la Programación & 
1. Terminología (tipos de memoria, CPU, velocidades de transmisión, internet). \newline
3. Programación. & 
1. Comprender conceptos básicos relativos a un programa computacional, tales como algoritmos, variables, expresiones, control de flujo, funciones, listas, strings, clases y objetos. \newline
2. Aplicar técnicas fundamentales para la resolución de diversos problemas con ayuda del computador. \newline
3. Aplicar el razonamiento algorítmico para generar la solución a un problema como una secuencia de pasos bien definidos, incluyendo pasos condicionales, repetición de pasos, llamadas a funciones, y recursión. \\ \hline

\rowcolor[HTML]{FFFFFF}
Hojas de Cálculo & 
 & 
2. Hojas de Cálculo: Manejo Nivel Básico. & 
Transversal a la formación. \\ \hline

\rowcolor[HTML]{F9F9F9}
Ética & 
FIL188 \newline Ética para Ingenieros & 
1. Código de ética (sociedades profesionales y técnicas). \newline
2. Acuerdos y contratos. \newline
3. Legislación y ética. \newline
4. Responsabilidad Profesional. \newline
5. Cuestiones de protección pública. & 
1. Conocer los fundamentos y las principales corrientes éticas que subyacen a las decisiones morales del hombre, tales como: \newline
- Relativismo, Subjetivismo y Realismo moral (Lawrence Kohlberg). \newline
- La conciencia moral, y la ética del cuidado (Carol Gilligan). \newline
- Utilitarismo (Jeremy Bentham y John Stuart Mill). \newline
- Deontología (Immanuel Kant). \newline
- Ética de la Virtud (Aristóteles). \newline
- Ética de la Justicia (John Rawls). \newline
2. Reflexionar sobre los criterios y principios para orientar la acción en el ámbito científico y tecnológico. \newline
3. Aplicar los principios de la ética a problemas específicos de la ciencia y la tecnología. \newline
4. Identificar los alcances de la legislación nacional (código de ética) vigente relacionada con la práctica de la ingeniería y saber cómo aplicarla en el ejercicio de la profesión. \\ \hline

\end{longtable}
